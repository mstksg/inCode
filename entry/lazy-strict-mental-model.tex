\documentclass[]{article}
\usepackage{lmodern}
\usepackage{amssymb,amsmath}
\usepackage{ifxetex,ifluatex}
\usepackage{fixltx2e} % provides \textsubscript
\ifnum 0\ifxetex 1\fi\ifluatex 1\fi=0 % if pdftex
  \usepackage[T1]{fontenc}
  \usepackage[utf8]{inputenc}
\else % if luatex or xelatex
  \ifxetex
    \usepackage{mathspec}
    \usepackage{xltxtra,xunicode}
  \else
    \usepackage{fontspec}
  \fi
  \defaultfontfeatures{Mapping=tex-text,Scale=MatchLowercase}
  \newcommand{\euro}{€}
\fi
% use upquote if available, for straight quotes in verbatim environments
\IfFileExists{upquote.sty}{\usepackage{upquote}}{}
% use microtype if available
\IfFileExists{microtype.sty}{\usepackage{microtype}}{}
\usepackage[margin=1in]{geometry}
\ifxetex
  \usepackage[setpagesize=false, % page size defined by xetex
              unicode=false, % unicode breaks when used with xetex
              xetex]{hyperref}
\else
  \usepackage[unicode=true]{hyperref}
\fi
\hypersetup{breaklinks=true,
            bookmarks=true,
            pdfauthor={},
            pdftitle={Lazy Patterns, Strict Matches, and All That: My Mental Model of Haskell},
            colorlinks=true,
            citecolor=blue,
            urlcolor=blue,
            linkcolor=magenta,
            pdfborder={0 0 0}}
\urlstyle{same}  % don't use monospace font for urls
% Make links footnotes instead of hotlinks:
\renewcommand{\href}[2]{#2\footnote{\url{#1}}}
\setlength{\parindent}{0pt}
\setlength{\parskip}{6pt plus 2pt minus 1pt}
\setlength{\emergencystretch}{3em}  % prevent overfull lines
\setcounter{secnumdepth}{0}

\title{Lazy Patterns, Strict Matches, and All That: My Mental Model of Haskell}

\begin{document}
\maketitle

Evaluation \% Justin Le

\emph{Originally posted on
\textbf{\href{https://blog.jle.im/entry/lazy-strict-mental-model.html}{in
Code}}.}

\documentclass[]{}
\usepackage{lmodern}
\usepackage{amssymb,amsmath}
\usepackage{ifxetex,ifluatex}
\usepackage{fixltx2e} % provides \textsubscript
\ifnum 0\ifxetex 1\fi\ifluatex 1\fi=0 % if pdftex
  \usepackage[T1]{fontenc}
  \usepackage[utf8]{inputenc}
\else % if luatex or xelatex
  \ifxetex
    \usepackage{mathspec}
    \usepackage{xltxtra,xunicode}
  \else
    \usepackage{fontspec}
  \fi
  \defaultfontfeatures{Mapping=tex-text,Scale=MatchLowercase}
  \newcommand{\euro}{€}
\fi
% use upquote if available, for straight quotes in verbatim environments
\IfFileExists{upquote.sty}{\usepackage{upquote}}{}
% use microtype if available
\IfFileExists{microtype.sty}{\usepackage{microtype}}{}
\usepackage[margin=1in]{geometry}
\ifxetex
  \usepackage[setpagesize=false, % page size defined by xetex
              unicode=false, % unicode breaks when used with xetex
              xetex]{hyperref}
\else
  \usepackage[unicode=true]{hyperref}
\fi
\hypersetup{breaklinks=true,
            bookmarks=true,
            pdfauthor={},
            pdftitle={},
            colorlinks=true,
            citecolor=blue,
            urlcolor=blue,
            linkcolor=magenta,
            pdfborder={0 0 0}}
\urlstyle{same}  % don't use monospace font for urls
% Make links footnotes instead of hotlinks:
\renewcommand{\href}[2]{#2\footnote{\url{#1}}}
\setlength{\parindent}{0pt}
\setlength{\parskip}{6pt plus 2pt minus 1pt}
\setlength{\emergencystretch}{3em}  % prevent overfull lines
\setcounter{secnumdepth}{0}


\begin{document}

Every once in a while I get questions about "what does this `~` do in a
pattern", what strict pattern matching does, etc. And while you can understand
code like `case x of ~(a,b) -> ...` in terms of one-off special cases, I think
that having a good mental model of how evaluation with ADTs work in Haskell will
allow you to answer these questions with confidence, and without second-guessing
yourself.

This is less an article on formal semantics as it as an article trying to
explain some of the heuristics and mental models I've developed over my years of
writing Haskell to help me understand how things work. It's not a textbook, but
an experience report :)

## Let and Case

A lot of the distinction between lazy and strict matching can be illustrated by
comparing the semantics of `let` vs. `case`. (We're going to be ignoring type
checking here and assuming all values are nice monomorphic values).

For example, let's see the distinction between:

``` haskell
let (x, y) = undefined
in  True

case undefined of
    (x, y) -> True
```

What will either of these return, when you try to run it?

Small note --- here, we are using `undefined` as a way to directly observe
evaluation in Haskell. This is a common trick people use when discussing
evaluation in Haskell --- it works because `undefined :: a` can take on any
type, and any attempt to evaluate `undefined` can be immediately detected:

``` haskell
ghci> sum [1,2,undefined,100]
*** Exception: Prelude.undefined
ghci> sum (take 2 [1,2,undefined,100])
3
```

Think of it like laying a mine for poor old GHC to encounter. GHC tries to
evaluate things, but when it tries to evaluate `undefined`, it immediately
explodes and tells you that it has a boo-boo. Our investigation becomes reframed
as "is this mine ever tripped, or not?"

::: note
**Alternatives to `undefined`**

Admittedly, the idea that `undefined` can take on any type can seem a little
spooky -- so often I like to use something within the correct type to make
things feel a little more concrete. For example, instead of `undefined`, we can
write:

``` haskell
myTuple :: (Int, String)
myTuple
  | sum [1..] < 1000 = (3, "hi")
  | otherwise        = (5, "bye")

let (x, y) = myTuple
in  True

case myTuple of
    (x, y) -> True
```

This example should display all of the same nuances of using `undefined` --- the
only practical difference is that instead of seeing
`Exception: Prelude.undefined` if the value is ever evaluated, you'll instead
just see a program locking up if a value is evaluated.
:::

With that in mind, let's find the answer to our previous question:

``` haskell
ghci> let (x,y) = undefined in True
True
ghci> case undefined of (x,y) -> True
*** Exception: Prelude.undefined
```

So, there's a difference! What's going on here?

In my mind, this is how I see things:

1.  *let* is **aliasing**
2.  *case* is **control flow**

To apply this to what we just saw:

1.  *let* is just aliasing `x` and `y` to whatever is in the body. If you don't
    use the alias, it's no big deal, really. It's just like if you used
    `alias abc rm -rf /` in *bash*. Aliasing `abc` to `rm -rf /` doesn't
    actually erase your entire file system, it's only something that does it if
    you decide to use the alias.

2.  *case* here *decides* what to do based on what the input is. *If* you are
    given `(x, y)`, *then* return `True`. It's used to direct the flow of how a
    program executes, so it *has* to look at the value to see what to do.

So, the reason that the `case` version explodes when you attempt to evaluate it
is because it has to ask the question "Does this value match this pattern?" ---
and in the process of investigating it, it has to touch `undefined` and so sets
off the mine.

The *let* situation never has to answer the question "does this value match this
pattern?", and so never needs to investigate it and so doesn't explode.

### Option One of One

To someone new to Haskell (or unfamiliar with its semantics), it kind of seems
silly that `case` would "have" to check *which* constructor is being
used...since there is really only one option, `,`. Why can't `case` just know
that it *has* to be `,` and move on, without checking? Surely that would be more
sensible.

For example, let's look at a situation that is uncontroversial

``` haskell
ghci> case undefined of
        Left x -> False
        Right y -> True
*** Exception: Prelude.undefined
```

Of *course* that one has to explode, because it has to check if the `Either` is
`Left` or `Right`. But in the case of `(x, y)`, it's always going to be `,`, so
why can't it just "know" you are `(x, y)` since there isn't any other situation?

``` haskell
ghci> case undefined of
        (x, y) -> False
*** Exception: Prelude.undefined
```

We know the answer will be `False` no matter what, so why bother checking? Why
can't we just have it skip the check and return `False`?

There are a couple of ways to approach this answer...and I think all approaches
illustrate something important about the philosophy of writing Haskell.

1.  If you shift your question from "which constructor is it?" to "which
    constructor was it *made* with?", then it starts to make sense.

    Pattern matching isn't about asking "which constructor does this value
    match?", but rather "which constructor was this value *made* with?"

    So, it would be unwise to just have GHC skip the check. When we are doing

    ``` haskell
    case blahblah of
      (x, y) -> -- ...
    ```

    We don't *really* care "what is `blahblah`"? We care "how was `blahblah`
    made?"

    Under this light, we see that there actually are many ways to make a tuple
    value *without* ever using `(,)`:

    ``` haskell
    tup1 :: (Int, Bool)
    tup1 = undefined

    tup2 :: (Int, Bool)
    tup2 = go 0
      where
        go n = go (n + 1)

    tup3 :: (Int, Bool)
    tup3
      | even (sum [1..]) = (3, True)
      | otherwise        = undefined
    ```

    We can see here three perfectly valid values of type `(Int, Bool)` that are
    *not* constructed with `(,)`, or at least where the question is ambiguous
    and up in the air. So *skipping* checking the constructor here is skipping
    the real question -- not "what pattern does it match", but "what was it made
    with".

2.  Haskell `data` declarations with boxed fields all follow the same general
    schema in memory:

    ``` haskell
    data MyType = A Int Bool
                | B String Double
    ```

        +--------+------------------------+-------------------------+
        | A or B | pointer to first field | pointer to second field |
        +--------+------------------------+-------------------------+

    So, for example, a value like `A 3 True` would look like:

    ``` haskell
        +---+---+---+
    --> | A | o | o |
        +---+-|-+-|-+
          |   |
          v   v
          3   True
    ```

    For a type with a single constructor, that initial "which constructor" field
    is still there. This just makes things a little more consistent --- there is
    no special-case for a single-constructor situation.

    When considering sematics, there is no fundamental difference between a
    single-constructor type and a multi-constructor type. A single-constructor
    type is just an n-constructor type where n = 1. This means that a type like:

    ``` haskell
    data MyType = A Int Bool
    ```

    will still have the constructor byte at the front:

        +---+------------------------+-------------------------+
        | A | pointer to first field | pointer to second field |
        +---+------------------------+-------------------------+

To me, the underlying reason is *consistency*. A lot of beginners, when starting
Haskell, come to the impression that Haskell is *magical*: it just does the best
thing in every situation. For example, a lot of Haskell learners somehow all
independently arrive at the assumption that using the same function on the same
argument will only compute it once, and cache the result. After all, it's
possible (because all functions are pure, so calling the same function on the
same value will always return the same result), so why not have it magically be
true?

However, as you learn more Haskell, you start to realize that what might have
seemed magical at first is just the straightforward application of a few rules
that apply generally, with rare special-cases. That's because for a language to
be useful, it's not about *magic*, but *predictability*.

So, case matching on a single-constructor type should not be special-cased from
the mechanics of a two-constructor type. To check for a match, you always have
to peek at the constructor --- no matter how many there might be.

Another way you can look at it is that ADTs are stored in haskell as a
`(ConstructorTag, Fields)` tuple --- and all `data`-defined ADTs are stored in
that way, without special-case exceptions.

(Of course, the exception to this rule is `newtype`s, which follow different
rules from `data` in general)

And, of course, it's also possible to show "All tuples must match the `(x,y)`
pattern and be constructed with `(,)`" as false --- `undefined :: (Int, Bool)`
is an example of a tuple that isn't constructed with `(,)` and so doesn't match
the `(x, y)` pattern.

# Signoff

Hi, thanks for reading! You can reach me via email at <justin@jle.im>, or at
twitter at [\@mstk](https://twitter.com/mstk)! This post and all others are
published under the [CC-BY-NC-ND
3.0](https://creativecommons.org/licenses/by-nc-nd/3.0/) license. Corrections
and edits via pull request are welcome and encouraged at [the source
repository](https://github.com/mstksg/inCode).

If you feel inclined, or this post was particularly helpful for you, why not
consider [supporting me on Patreon](https://www.patreon.com/justinle/overview),
or a [BTC donation](bitcoin:3D7rmAYgbDnp4gp4rf22THsGt74fNucPDU)? :)

\end{document}

\end{document}
