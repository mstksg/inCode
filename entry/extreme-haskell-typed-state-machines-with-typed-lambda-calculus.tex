\documentclass[]{article}
\usepackage{lmodern}
\usepackage{amssymb,amsmath}
\usepackage{ifxetex,ifluatex}
\usepackage{fixltx2e} % provides \textsubscript
\ifnum 0\ifxetex 1\fi\ifluatex 1\fi=0 % if pdftex
  \usepackage[T1]{fontenc}
  \usepackage[utf8]{inputenc}
\else % if luatex or xelatex
  \ifxetex
    \usepackage{mathspec}
    \usepackage{xltxtra,xunicode}
  \else
    \usepackage{fontspec}
  \fi
  \defaultfontfeatures{Mapping=tex-text,Scale=MatchLowercase}
  \newcommand{\euro}{€}
\fi
% use upquote if available, for straight quotes in verbatim environments
\IfFileExists{upquote.sty}{\usepackage{upquote}}{}
% use microtype if available
\IfFileExists{microtype.sty}{\usepackage{microtype}}{}
\usepackage[margin=1in]{geometry}
\usepackage{color}
\usepackage{fancyvrb}
\newcommand{\VerbBar}{|}
\newcommand{\VERB}{\Verb[commandchars=\\\{\}]}
\DefineVerbatimEnvironment{Highlighting}{Verbatim}{commandchars=\\\{\}}
% Add ',fontsize=\small' for more characters per line
\newenvironment{Shaded}{}{}
\newcommand{\AlertTok}[1]{\textcolor[rgb]{1.00,0.00,0.00}{\textbf{#1}}}
\newcommand{\AnnotationTok}[1]{\textcolor[rgb]{0.38,0.63,0.69}{\textbf{\textit{#1}}}}
\newcommand{\AttributeTok}[1]{\textcolor[rgb]{0.49,0.56,0.16}{#1}}
\newcommand{\BaseNTok}[1]{\textcolor[rgb]{0.25,0.63,0.44}{#1}}
\newcommand{\BuiltInTok}[1]{\textcolor[rgb]{0.00,0.50,0.00}{#1}}
\newcommand{\CharTok}[1]{\textcolor[rgb]{0.25,0.44,0.63}{#1}}
\newcommand{\CommentTok}[1]{\textcolor[rgb]{0.38,0.63,0.69}{\textit{#1}}}
\newcommand{\CommentVarTok}[1]{\textcolor[rgb]{0.38,0.63,0.69}{\textbf{\textit{#1}}}}
\newcommand{\ConstantTok}[1]{\textcolor[rgb]{0.53,0.00,0.00}{#1}}
\newcommand{\ControlFlowTok}[1]{\textcolor[rgb]{0.00,0.44,0.13}{\textbf{#1}}}
\newcommand{\DataTypeTok}[1]{\textcolor[rgb]{0.56,0.13,0.00}{#1}}
\newcommand{\DecValTok}[1]{\textcolor[rgb]{0.25,0.63,0.44}{#1}}
\newcommand{\DocumentationTok}[1]{\textcolor[rgb]{0.73,0.13,0.13}{\textit{#1}}}
\newcommand{\ErrorTok}[1]{\textcolor[rgb]{1.00,0.00,0.00}{\textbf{#1}}}
\newcommand{\ExtensionTok}[1]{#1}
\newcommand{\FloatTok}[1]{\textcolor[rgb]{0.25,0.63,0.44}{#1}}
\newcommand{\FunctionTok}[1]{\textcolor[rgb]{0.02,0.16,0.49}{#1}}
\newcommand{\ImportTok}[1]{\textcolor[rgb]{0.00,0.50,0.00}{\textbf{#1}}}
\newcommand{\InformationTok}[1]{\textcolor[rgb]{0.38,0.63,0.69}{\textbf{\textit{#1}}}}
\newcommand{\KeywordTok}[1]{\textcolor[rgb]{0.00,0.44,0.13}{\textbf{#1}}}
\newcommand{\NormalTok}[1]{#1}
\newcommand{\OperatorTok}[1]{\textcolor[rgb]{0.40,0.40,0.40}{#1}}
\newcommand{\OtherTok}[1]{\textcolor[rgb]{0.00,0.44,0.13}{#1}}
\newcommand{\PreprocessorTok}[1]{\textcolor[rgb]{0.74,0.48,0.00}{#1}}
\newcommand{\RegionMarkerTok}[1]{#1}
\newcommand{\SpecialCharTok}[1]{\textcolor[rgb]{0.25,0.44,0.63}{#1}}
\newcommand{\SpecialStringTok}[1]{\textcolor[rgb]{0.73,0.40,0.53}{#1}}
\newcommand{\StringTok}[1]{\textcolor[rgb]{0.25,0.44,0.63}{#1}}
\newcommand{\VariableTok}[1]{\textcolor[rgb]{0.10,0.09,0.49}{#1}}
\newcommand{\VerbatimStringTok}[1]{\textcolor[rgb]{0.25,0.44,0.63}{#1}}
\newcommand{\WarningTok}[1]{\textcolor[rgb]{0.38,0.63,0.69}{\textbf{\textit{#1}}}}
\ifxetex
  \usepackage[setpagesize=false, % page size defined by xetex
              unicode=false, % unicode breaks when used with xetex
              xetex]{hyperref}
\else
  \usepackage[unicode=true]{hyperref}
\fi
\hypersetup{breaklinks=true,
            bookmarks=true,
            pdfauthor={Justin Le},
            pdftitle={Extreme Haskell: Typed State Machines with Typed Lambda Calculus},
            colorlinks=true,
            citecolor=blue,
            urlcolor=blue,
            linkcolor=magenta,
            pdfborder={0 0 0}}
\urlstyle{same}  % don't use monospace font for urls
% Make links footnotes instead of hotlinks:
\renewcommand{\href}[2]{#2\footnote{\url{#1}}}
\setlength{\parindent}{0pt}
\setlength{\parskip}{6pt plus 2pt minus 1pt}
\setlength{\emergencystretch}{3em}  % prevent overfull lines
\setcounter{secnumdepth}{0}

\title{Extreme Haskell: Typed State Machines with Typed Lambda Calculus}
\author{Justin Le}

\begin{document}
\maketitle

\emph{Originally posted on
\textbf{\href{https://blog.jle.im/entry/extreme-haskell-typed-state-machines-with-typed-lambda-calculus.html}{in
Code}}.}

\documentclass[]{}
\usepackage{lmodern}
\usepackage{amssymb,amsmath}
\usepackage{ifxetex,ifluatex}
\usepackage{fixltx2e} % provides \textsubscript
\ifnum 0\ifxetex 1\fi\ifluatex 1\fi=0 % if pdftex
  \usepackage[T1]{fontenc}
  \usepackage[utf8]{inputenc}
\else % if luatex or xelatex
  \ifxetex
    \usepackage{mathspec}
    \usepackage{xltxtra,xunicode}
  \else
    \usepackage{fontspec}
  \fi
  \defaultfontfeatures{Mapping=tex-text,Scale=MatchLowercase}
  \newcommand{\euro}{€}
\fi
% use upquote if available, for straight quotes in verbatim environments
\IfFileExists{upquote.sty}{\usepackage{upquote}}{}
% use microtype if available
\IfFileExists{microtype.sty}{\usepackage{microtype}}{}
\usepackage[margin=1in]{geometry}
\ifxetex
  \usepackage[setpagesize=false, % page size defined by xetex
              unicode=false, % unicode breaks when used with xetex
              xetex]{hyperref}
\else
  \usepackage[unicode=true]{hyperref}
\fi
\hypersetup{breaklinks=true,
            bookmarks=true,
            pdfauthor={},
            pdftitle={},
            colorlinks=true,
            citecolor=blue,
            urlcolor=blue,
            linkcolor=magenta,
            pdfborder={0 0 0}}
\urlstyle{same}  % don't use monospace font for urls
% Make links footnotes instead of hotlinks:
\renewcommand{\href}[2]{#2\footnote{\url{#1}}}
\setlength{\parindent}{0pt}
\setlength{\parskip}{6pt plus 2pt minus 1pt}
\setlength{\emergencystretch}{3em}  % prevent overfull lines
\setcounter{secnumdepth}{0}

\textbackslash begin\{document\}

I always say, inside every Haskeller there are two wolves, living on both ends
of the Haskell Fancy Code Spectrum (HFCS). Are you going to write ``simple
Haskell'', using basic GHC 2010 tools and writing universal Haskell that every
introductory course offers, trying to keep the code as immediately
understandable and accessible? Or are you going to pile in all of the Haskell
type system and evaluation tricks you can find and turn on all the extensions,
and go full fancy?

In my
\href{https://blog.jle.im/entry/levels-of-type-safety-haskell-lists.html}{Seven
Levels of Type Safety} post, I described different extremes of type safety and
fancy code. I talked about how writing effective code was finding the correct
compromise for the level of communication and safety you need.

But this is not that kind of blog post. This blog post is about what happens
when you say ``screw it, let's go full fancy''? Let's ignore the advice of the
great Kirk Lazarus. Let's go full fancy. Let's write code that is so
inscrutable, so much of a pain and torture to write, yet so \emph{undeniably
useful} that you can't help but try to throw it in every single thing you write
and will feel a gnawing emptiness in your soul until you do.

Here's one example: let's write a type-safe method to specify your program as a
series of states, with triggered transitions between them. A Type-Safe state
machine graph using a type-safe lambda calculus. We want to specify this in a
way that we can write once and it will

\begin{enumerate}
\def\labelenumi{\arabic{enumi}.}
\tightlist
\item
  Be interpretable in a type-safe way within Haskell
\item
  Be inspectable with visualizable control flow.
\item
  Be compilable to multiple actual backends, letting you run the same function
  under multiple implementations.
\end{enumerate}

This is all stuff I have been using in real life in my personal projects, where
I've needed to write a specification of an algorithm that I can simulate in
Haskell, generate graphical visualizations of, and also convert to multiple
(purescript, dhall, C dialects) to unify algorithms and formulas across
back-ends without writing them from scratch every time.

Once you go down this road, everything you ever write will feel woefully unsafe
and limited. And everything you will want to write will be woefully inscrutable
by normal humans and borderline unusable. But such is the curse we all bear.
Turn around now, you have been warned.

As Adam Neely asked in his \href{https://www.youtube.com/watch?v=U8dcFhF0Dlk}{AI
Music Video Essay}, if you had trained all of AI on pre-jazz music, could AI
have invented jazz? If you trained it on pre-80s hip-hop, could it have invented
80s hip-hop and its technological breakthroughs? If you trained AI on safe
Haskell code, could it invent the monstrosity we are about to explore in this
post?

All of the code here is
\href{https://github.com/mstksg/inCode/tree/master/code-samples/typed-sm-lc/flake.nix}{available
online}, and if you check out the repo and run \texttt{nix\ develop} you should
be able to load it all in ghci:

\begin{Shaded}
\begin{Highlighting}[]
\ExtensionTok{$}\NormalTok{ cd code{-}samples/typed{-}sm{-}lc}
\ExtensionTok{$}\NormalTok{ nix develop}
\ExtensionTok{$}\NormalTok{ ghci}
\ExtensionTok{ghci}\OperatorTok{\textgreater{}}\NormalTok{ :load ExprStage1.hs}
\end{Highlighting}
\end{Shaded}

\section{The Lambda Calculus}\label{the-lambda-calculus}

\subsection{A First Pass}\label{a-first-pass}

Let's derive a way to express an algorithm or expression in Haskell that can be
reified and analyzed within Haskell, and eventually be a form we can compile to
different backends, interpret in Haskell, or generate Graphviz visualizations
in.

One basic thing we can do is start with:

\begin{Shaded}
\begin{Highlighting}[]
\CommentTok{{-}{-} source: https://github.com/mstksg/inCode/tree/master/code{-}samples/typed{-}sm{-}lc/ExprStage1.hs\#L8{-}L22}

\KeywordTok{data} \DataTypeTok{Prim} \OtherTok{=} \DataTypeTok{PInt} \DataTypeTok{Int} \OperatorTok{|} \DataTypeTok{PBool} \DataTypeTok{Bool} \OperatorTok{|} \DataTypeTok{PString} \DataTypeTok{String}
  \KeywordTok{deriving}\NormalTok{ (}\DataTypeTok{Eq}\NormalTok{, }\DataTypeTok{Show}\NormalTok{)}

\KeywordTok{data} \DataTypeTok{Op} \OtherTok{=} \DataTypeTok{OPlus} \OperatorTok{|} \DataTypeTok{OTimes} \OperatorTok{|} \DataTypeTok{OLte} \OperatorTok{|} \DataTypeTok{OAnd}
  \KeywordTok{deriving}\NormalTok{ (}\DataTypeTok{Eq}\NormalTok{, }\DataTypeTok{Show}\NormalTok{)}

\KeywordTok{data} \DataTypeTok{Expr}
  \OtherTok{=} \DataTypeTok{EPrim} \DataTypeTok{Prim}
  \OperatorTok{|} \DataTypeTok{EVar} \DataTypeTok{String}
  \OperatorTok{|} \DataTypeTok{ELambda} \DataTypeTok{String} \DataTypeTok{Expr}
  \OperatorTok{|} \DataTypeTok{EApply} \DataTypeTok{Expr} \DataTypeTok{Expr}
  \OperatorTok{|} \DataTypeTok{EOp} \DataTypeTok{Op} \DataTypeTok{Expr} \DataTypeTok{Expr}
  \OperatorTok{|} \DataTypeTok{ERecord}\NormalTok{ (}\DataTypeTok{Map} \DataTypeTok{String} \DataTypeTok{Expr}\NormalTok{)}
  \OperatorTok{|} \DataTypeTok{EAccess} \DataTypeTok{Expr} \DataTypeTok{String}
  \KeywordTok{deriving}\NormalTok{ (}\DataTypeTok{Eq}\NormalTok{, }\DataTypeTok{Show}\NormalTok{)}
\end{Highlighting}
\end{Shaded}

And you can write \texttt{(\textbackslash{}x\ -\textgreater{}\ x\ *\ 3)\ 5} as:

\begin{Shaded}
\begin{Highlighting}[]
\CommentTok{{-}{-} source: https://github.com/mstksg/inCode/tree/master/code{-}samples/typed{-}sm{-}lc/ExprStage1.hs\#L24{-}L25}

\OtherTok{fifteen ::} \DataTypeTok{Expr}
\NormalTok{fifteen }\OtherTok{=} \DataTypeTok{ELambda} \StringTok{"x"}\NormalTok{ (}\DataTypeTok{EOp} \DataTypeTok{OTimes}\NormalTok{ (}\DataTypeTok{EVar} \StringTok{"x"}\NormalTok{) (}\DataTypeTok{EPrim}\NormalTok{ (}\DataTypeTok{PInt} \DecValTok{3}\NormalTok{))) }\OtherTok{\textasciigrave{}EApply\textasciigrave{}} \DataTypeTok{EPrim}\NormalTok{ (}\DataTypeTok{PInt} \DecValTok{5}\NormalTok{)}
\end{Highlighting}
\end{Shaded}

You can definitely easily render this in a graph, but what happens when you
write a Haskell interpreter? How can you ``evaluate'' this to 15, within
Haskell? What would the type even be?
\texttt{eval\ ::\ Expr\ -\textgreater{}\ Maybe\ Prim}? Maybe just
\texttt{normalize\ ::\ Expr\ -\textgreater{}\ Expr} and hope that the result is
\texttt{Prim}?

\begin{Shaded}
\begin{Highlighting}[]
\CommentTok{{-}{-} source: https://github.com/mstksg/inCode/tree/master/code{-}samples/typed{-}sm{-}lc/ExprStage1.hs\#L27{-}L54}

\OtherTok{normalize ::} \DataTypeTok{Map} \DataTypeTok{String} \DataTypeTok{Expr} \OtherTok{{-}\textgreater{}} \DataTypeTok{Expr} \OtherTok{{-}\textgreater{}} \DataTypeTok{Expr}
\NormalTok{normalize env }\OtherTok{=}\NormalTok{ \textbackslash{}}\KeywordTok{case}
  \DataTypeTok{EPrim}\NormalTok{ p }\OtherTok{{-}\textgreater{}} \DataTypeTok{EPrim}\NormalTok{ p}
  \DataTypeTok{EVar}\NormalTok{ v }\OtherTok{{-}\textgreater{}} \KeywordTok{case}\NormalTok{ M.lookup v env }\KeywordTok{of}
    \DataTypeTok{Nothing} \OtherTok{{-}\textgreater{}} \FunctionTok{error} \StringTok{"Variable not defined"}
    \DataTypeTok{Just}\NormalTok{ x }\OtherTok{{-}\textgreater{}}\NormalTok{ normalize env x}
  \DataTypeTok{ELambda}\NormalTok{ n x }\OtherTok{{-}\textgreater{}} \DataTypeTok{ELambda}\NormalTok{ n x}
  \DataTypeTok{EApply}\NormalTok{ f x }\OtherTok{{-}\textgreater{}} \KeywordTok{case}\NormalTok{ normalize env f }\KeywordTok{of}
    \DataTypeTok{ELambda}\NormalTok{ n u }\OtherTok{{-}\textgreater{}}\NormalTok{ normalize (M.insert n (normalize env x) env) u}
\NormalTok{    \_ }\OtherTok{{-}\textgreater{}} \FunctionTok{error} \StringTok{"Type error"}
  \DataTypeTok{EOp}\NormalTok{ o x y }\OtherTok{{-}\textgreater{}} \KeywordTok{case}\NormalTok{ (normalize env x, normalize env y) }\KeywordTok{of}
\NormalTok{    (}\DataTypeTok{EPrim}\NormalTok{ x\textquotesingle{}, }\DataTypeTok{EPrim}\NormalTok{ y\textquotesingle{}) }\OtherTok{{-}\textgreater{}} \KeywordTok{case}\NormalTok{ (x\textquotesingle{}, y\textquotesingle{}) }\KeywordTok{of}
\NormalTok{      (}\DataTypeTok{PInt}\NormalTok{ a, }\DataTypeTok{PInt}\NormalTok{ b) }\OtherTok{{-}\textgreater{}} \KeywordTok{case}\NormalTok{ o }\KeywordTok{of}
        \DataTypeTok{OPlus} \OtherTok{{-}\textgreater{}} \DataTypeTok{EPrim}\NormalTok{ (}\DataTypeTok{PInt}\NormalTok{ (a }\OperatorTok{+}\NormalTok{ b))}
        \DataTypeTok{OTimes} \OtherTok{{-}\textgreater{}} \DataTypeTok{EPrim}\NormalTok{ (}\DataTypeTok{PInt}\NormalTok{ (a }\OperatorTok{*}\NormalTok{ b))}
        \DataTypeTok{OLte} \OtherTok{{-}\textgreater{}} \DataTypeTok{EPrim}\NormalTok{ (}\DataTypeTok{PBool}\NormalTok{ (a }\OperatorTok{\textless{}=}\NormalTok{ b))}
        \DataTypeTok{OAnd} \OtherTok{{-}\textgreater{}} \FunctionTok{error} \StringTok{"Type error"}
\NormalTok{      (}\DataTypeTok{PBool}\NormalTok{ a, }\DataTypeTok{PBool}\NormalTok{ b) }\OtherTok{{-}\textgreater{}} \KeywordTok{case}\NormalTok{ o }\KeywordTok{of}
        \DataTypeTok{OAnd} \OtherTok{{-}\textgreater{}} \DataTypeTok{EPrim}\NormalTok{ (}\DataTypeTok{PBool}\NormalTok{ (a }\OperatorTok{\&\&}\NormalTok{ b))}
\NormalTok{        \_ }\OtherTok{{-}\textgreater{}} \FunctionTok{error} \StringTok{"Type error"}
\NormalTok{      \_ }\OtherTok{{-}\textgreater{}} \FunctionTok{error} \StringTok{"Type error"}
\NormalTok{    (x\textquotesingle{}, y\textquotesingle{}) }\OtherTok{{-}\textgreater{}} \DataTypeTok{EOp}\NormalTok{ o x\textquotesingle{} y\textquotesingle{}}
  \DataTypeTok{ERecord}\NormalTok{ xs }\OtherTok{{-}\textgreater{}} \DataTypeTok{ERecord}\NormalTok{ (M.map (normalize env) xs)}
  \DataTypeTok{EAccess}\NormalTok{ e k }\OtherTok{{-}\textgreater{}} \KeywordTok{case}\NormalTok{ normalize env e }\KeywordTok{of}
    \DataTypeTok{ERecord}\NormalTok{ xs }\OtherTok{{-}\textgreater{}} \KeywordTok{case}\NormalTok{ M.lookup k xs }\KeywordTok{of}
      \DataTypeTok{Just}\NormalTok{ v }\OtherTok{{-}\textgreater{}}\NormalTok{ normalize env v}
      \DataTypeTok{Nothing} \OtherTok{{-}\textgreater{}} \FunctionTok{error} \StringTok{"Field not found"}
\NormalTok{    \_ }\OtherTok{{-}\textgreater{}} \FunctionTok{error} \StringTok{"Type error"}
\end{Highlighting}
\end{Shaded}

This would properly evaluate:

\begin{Shaded}
\begin{Highlighting}[]
\NormalTok{ghci}\OperatorTok{\textgreater{}}\NormalTok{ normalize fifteen}
\DataTypeTok{EPrim}\NormalTok{ (}\DataTypeTok{PInt} \DecValTok{15}\NormalTok{)}
\end{Highlighting}
\end{Shaded}

Let's say this is 5\% fancy. We used recursive types, used \texttt{Map} to look
things up efficiently.

But this isn't type-safe\ldots we have undefined branches still. We could make
the entire thing monadic by returning \texttt{Maybe}:

\begin{Shaded}
\begin{Highlighting}[]
\CommentTok{{-}{-} source: https://github.com/mstksg/inCode/tree/master/code{-}samples/typed{-}sm{-}lc/ExprStage2.hs\#L27{-}L55}

\OtherTok{normalize ::} \DataTypeTok{Map} \DataTypeTok{String} \DataTypeTok{Expr} \OtherTok{{-}\textgreater{}} \DataTypeTok{Expr} \OtherTok{{-}\textgreater{}} \DataTypeTok{Maybe} \DataTypeTok{Expr}
\NormalTok{normalize env }\OtherTok{=}\NormalTok{ \textbackslash{}}\KeywordTok{case}
  \DataTypeTok{EPrim}\NormalTok{ p }\OtherTok{{-}\textgreater{}} \FunctionTok{pure}\NormalTok{ (}\DataTypeTok{EPrim}\NormalTok{ p)}
  \DataTypeTok{EVar}\NormalTok{ v }\OtherTok{{-}\textgreater{}}\NormalTok{ M.lookup v env }\OperatorTok{\textgreater{}\textgreater{}=}\NormalTok{ normalize env}
  \DataTypeTok{ELambda}\NormalTok{ n x }\OtherTok{{-}\textgreater{}} \FunctionTok{pure}\NormalTok{ (}\DataTypeTok{ELambda}\NormalTok{ n x)}
  \DataTypeTok{EApply}\NormalTok{ f x }\OtherTok{{-}\textgreater{}}\NormalTok{ normalize env f }\OperatorTok{\textgreater{}\textgreater{}=}\NormalTok{ \textbackslash{}}\KeywordTok{case}
    \DataTypeTok{ELambda}\NormalTok{ n u }\OtherTok{{-}\textgreater{}} \KeywordTok{do}
\NormalTok{      x\textquotesingle{} }\OtherTok{\textless{}{-}}\NormalTok{ normalize env x}
\NormalTok{      normalize (M.insert n x\textquotesingle{} env) u}
\NormalTok{    f\textquotesingle{} }\OtherTok{{-}\textgreater{}} \DataTypeTok{EApply}\NormalTok{ f\textquotesingle{} }\OperatorTok{\textless{}$\textgreater{}}\NormalTok{ normalize env x}
  \DataTypeTok{EOp}\NormalTok{ o x y }\OtherTok{{-}\textgreater{}} \KeywordTok{do}
\NormalTok{    u }\OtherTok{\textless{}{-}}\NormalTok{ normalize env x}
\NormalTok{    v }\OtherTok{\textless{}{-}}\NormalTok{ normalize env y}
    \KeywordTok{case}\NormalTok{ (u, v) }\KeywordTok{of}
\NormalTok{      (}\DataTypeTok{EPrim}\NormalTok{ x\textquotesingle{}, }\DataTypeTok{EPrim}\NormalTok{ y\textquotesingle{}) }\OtherTok{{-}\textgreater{}} \KeywordTok{case}\NormalTok{ (x\textquotesingle{}, y\textquotesingle{}) }\KeywordTok{of}
\NormalTok{        (}\DataTypeTok{PInt}\NormalTok{ a, }\DataTypeTok{PInt}\NormalTok{ b) }\OtherTok{{-}\textgreater{}} \KeywordTok{case}\NormalTok{ o }\KeywordTok{of}
          \DataTypeTok{OPlus} \OtherTok{{-}\textgreater{}} \FunctionTok{pure}\NormalTok{ (}\DataTypeTok{EPrim}\NormalTok{ (}\DataTypeTok{PInt}\NormalTok{ (a }\OperatorTok{+}\NormalTok{ b)))}
          \DataTypeTok{OTimes} \OtherTok{{-}\textgreater{}} \FunctionTok{pure}\NormalTok{ (}\DataTypeTok{EPrim}\NormalTok{ (}\DataTypeTok{PInt}\NormalTok{ (a }\OperatorTok{*}\NormalTok{ b)))}
          \DataTypeTok{OLte} \OtherTok{{-}\textgreater{}} \FunctionTok{pure}\NormalTok{ (}\DataTypeTok{EPrim}\NormalTok{ (}\DataTypeTok{PBool}\NormalTok{ (a }\OperatorTok{\textless{}=}\NormalTok{ b)))}
          \DataTypeTok{OAnd} \OtherTok{{-}\textgreater{}} \DataTypeTok{Nothing}
\NormalTok{        (}\DataTypeTok{PBool}\NormalTok{ a, }\DataTypeTok{PBool}\NormalTok{ b) }\OtherTok{{-}\textgreater{}} \KeywordTok{case}\NormalTok{ o }\KeywordTok{of}
          \DataTypeTok{OAnd} \OtherTok{{-}\textgreater{}} \FunctionTok{pure}\NormalTok{ (}\DataTypeTok{EPrim}\NormalTok{ (}\DataTypeTok{PBool}\NormalTok{ (a }\OperatorTok{\&\&}\NormalTok{ b)))}
\NormalTok{          \_ }\OtherTok{{-}\textgreater{}} \DataTypeTok{Nothing}
\NormalTok{        \_ }\OtherTok{{-}\textgreater{}} \DataTypeTok{Nothing}
\NormalTok{      (x\textquotesingle{}, y\textquotesingle{}) }\OtherTok{{-}\textgreater{}} \FunctionTok{pure} \OperatorTok{$} \DataTypeTok{EOp}\NormalTok{ o x\textquotesingle{} y\textquotesingle{}}
  \DataTypeTok{ERecord}\NormalTok{ xs }\OtherTok{{-}\textgreater{}} \DataTypeTok{ERecord} \OperatorTok{\textless{}$\textgreater{}} \FunctionTok{traverse}\NormalTok{ (normalize env) xs}
  \DataTypeTok{EAccess}\NormalTok{ e k }\OtherTok{{-}\textgreater{}} \KeywordTok{do}
    \DataTypeTok{ERecord}\NormalTok{ xs }\OtherTok{\textless{}{-}}\NormalTok{ normalize env e}
\NormalTok{    M.lookup k xs}
\end{Highlighting}
\end{Shaded}

This kind of works if you remember to thread everything through \texttt{Maybe}
(or \texttt{Either}) or what have you. But this is not ideal. You should be able
to know, at compile-time, that your \texttt{Expr} is valid. After all, you want
to be able to create one ``valid'' \texttt{Expr}, and run it at every context.
It's utterly useless to you if every single time you used an \texttt{Expr}, you
had to manually handle the \texttt{Nothing} case. Your diagram generator, your
Haskell runner, your code generator, will always be in \texttt{Either} even
though you know your \texttt{Expr} is valid, via tests or something.

This is maybe 10\% fancy. We used \texttt{Maybe}/\texttt{Either} to prevent
runtime exceptions, but didn't actually get rid of any runtime \emph{errors}.

No, this is not okay and unacceptable. We should be able to verify in the types
if an \texttt{Expr} is valid.

\subsection{First Layer of Types}\label{first-layer-of-types}

The next step you'll see in posts online is to add a phantom index type to
\texttt{Expr}:

\begin{Shaded}
\begin{Highlighting}[]
\CommentTok{{-}{-} source: https://github.com/mstksg/inCode/tree/master/code{-}samples/typed{-}sm{-}lc/ExprStage3.hs\#L14{-}L48}

\KeywordTok{type} \KeywordTok{data} \DataTypeTok{Ty}
  \OtherTok{=} \DataTypeTok{TInt}
  \OperatorTok{|} \DataTypeTok{TBool}
  \OperatorTok{|} \DataTypeTok{TString}
  \OperatorTok{|} \DataTypeTok{TRecord}
  \OperatorTok{|} \DataTypeTok{Ty} \OperatorTok{:{-}\textgreater{}} \DataTypeTok{Ty}

\KeywordTok{data} \DataTypeTok{STy}\OtherTok{ ::} \DataTypeTok{Ty} \OtherTok{{-}\textgreater{}} \DataTypeTok{Type} \KeywordTok{where}
  \DataTypeTok{STInt}\OtherTok{ ::} \DataTypeTok{STy} \DataTypeTok{TInt}
  \DataTypeTok{STBool}\OtherTok{ ::} \DataTypeTok{STy} \DataTypeTok{TBool}
  \DataTypeTok{STString}\OtherTok{ ::} \DataTypeTok{STy} \DataTypeTok{TString}
  \DataTypeTok{STRecord}\OtherTok{ ::} \DataTypeTok{STy} \DataTypeTok{TRecord}
  \DataTypeTok{STFun}\OtherTok{ ::} \DataTypeTok{STy}\NormalTok{ a }\OtherTok{{-}\textgreater{}} \DataTypeTok{STy}\NormalTok{ b }\OtherTok{{-}\textgreater{}} \DataTypeTok{STy}\NormalTok{ (a }\OperatorTok{:{-}\textgreater{}}\NormalTok{ b)}

\KeywordTok{data} \DataTypeTok{Prim}\OtherTok{ ::} \DataTypeTok{Ty} \OtherTok{{-}\textgreater{}} \DataTypeTok{Type} \KeywordTok{where}
  \DataTypeTok{PInt}\OtherTok{ ::} \DataTypeTok{Int} \OtherTok{{-}\textgreater{}} \DataTypeTok{Prim} \DataTypeTok{TInt}
  \DataTypeTok{PBool}\OtherTok{ ::} \DataTypeTok{Bool} \OtherTok{{-}\textgreater{}} \DataTypeTok{Prim} \DataTypeTok{TBool}
  \DataTypeTok{PString}\OtherTok{ ::} \DataTypeTok{String} \OtherTok{{-}\textgreater{}} \DataTypeTok{Prim} \DataTypeTok{TString}

\KeywordTok{data} \DataTypeTok{Op}\OtherTok{ ::} \DataTypeTok{Ty} \OtherTok{{-}\textgreater{}} \DataTypeTok{Ty} \OtherTok{{-}\textgreater{}} \DataTypeTok{Ty} \OtherTok{{-}\textgreater{}} \DataTypeTok{Type} \KeywordTok{where}
  \DataTypeTok{OPlus}\OtherTok{ ::} \DataTypeTok{Op} \DataTypeTok{TInt} \DataTypeTok{TInt} \DataTypeTok{TInt}
  \DataTypeTok{OTimes}\OtherTok{ ::} \DataTypeTok{Op} \DataTypeTok{TInt} \DataTypeTok{TInt} \DataTypeTok{TInt}
  \DataTypeTok{OLte}\OtherTok{ ::} \DataTypeTok{Op} \DataTypeTok{TInt} \DataTypeTok{TInt} \DataTypeTok{TBool}
  \DataTypeTok{OAnd}\OtherTok{ ::} \DataTypeTok{Op} \DataTypeTok{TBool} \DataTypeTok{TBool} \DataTypeTok{TBool}

\KeywordTok{data} \DataTypeTok{Expr}\OtherTok{ ::} \DataTypeTok{Ty} \OtherTok{{-}\textgreater{}} \DataTypeTok{Type} \KeywordTok{where}
  \DataTypeTok{EPrim}\OtherTok{ ::} \DataTypeTok{Prim}\NormalTok{ t }\OtherTok{{-}\textgreater{}} \DataTypeTok{Expr}\NormalTok{ t}
  \DataTypeTok{EVar}\OtherTok{ ::} \DataTypeTok{STy}\NormalTok{ t }\OtherTok{{-}\textgreater{}} \DataTypeTok{String} \OtherTok{{-}\textgreater{}} \DataTypeTok{Expr}\NormalTok{ t}
  \DataTypeTok{ELambda}\OtherTok{ ::} \DataTypeTok{STy}\NormalTok{ a }\OtherTok{{-}\textgreater{}} \DataTypeTok{String} \OtherTok{{-}\textgreater{}} \DataTypeTok{Expr}\NormalTok{ b }\OtherTok{{-}\textgreater{}} \DataTypeTok{Expr}\NormalTok{ (a }\OperatorTok{:{-}\textgreater{}}\NormalTok{ b)}
  \DataTypeTok{EApply}\OtherTok{ ::} \DataTypeTok{Expr}\NormalTok{ (a }\OperatorTok{:{-}\textgreater{}}\NormalTok{ b) }\OtherTok{{-}\textgreater{}} \DataTypeTok{Expr}\NormalTok{ a }\OtherTok{{-}\textgreater{}} \DataTypeTok{Expr}\NormalTok{ b}
  \DataTypeTok{EOp}\OtherTok{ ::} \DataTypeTok{Op}\NormalTok{ a b c }\OtherTok{{-}\textgreater{}} \DataTypeTok{Expr}\NormalTok{ a }\OtherTok{{-}\textgreater{}} \DataTypeTok{Expr}\NormalTok{ b }\OtherTok{{-}\textgreater{}} \DataTypeTok{Expr}\NormalTok{ c}
  \DataTypeTok{ERecord}\OtherTok{ ::} \DataTypeTok{Map} \DataTypeTok{String} \DataTypeTok{SomeExpr} \OtherTok{{-}\textgreater{}} \DataTypeTok{Expr} \DataTypeTok{TRecord}
  \DataTypeTok{EAccess}\OtherTok{ ::} \DataTypeTok{STy}\NormalTok{ t }\OtherTok{{-}\textgreater{}} \DataTypeTok{Expr} \DataTypeTok{TRecord} \OtherTok{{-}\textgreater{}} \DataTypeTok{String} \OtherTok{{-}\textgreater{}} \DataTypeTok{Expr}\NormalTok{ t}
\end{Highlighting}
\end{Shaded}

Here we use \texttt{-XTypeData} to define a data kind, \texttt{Ty} is a kind
with types \texttt{TInt\ ::\ Ty}, \texttt{TBool\ ::\ Ty}, etc.

So now \texttt{Expr\ a} evaluates to an \texttt{a}, which is either our domain's
\texttt{Int}, our domain's \texttt{Bool}, or our domain's \texttt{String}. At
least, now, it is impossible to create an \texttt{Expr} that doesn't type check:

\begin{Shaded}
\begin{Highlighting}[]
\CommentTok{{-}{-} source: https://github.com/mstksg/inCode/tree/master/code{-}samples/typed{-}sm{-}lc/ExprStage3.hs\#L50{-}L54}

\OtherTok{fifteen ::} \DataTypeTok{Expr} \DataTypeTok{TInt}
\NormalTok{fifteen }\OtherTok{=}
  \DataTypeTok{EApply}
\NormalTok{    (}\DataTypeTok{ELambda} \DataTypeTok{STInt} \StringTok{"x"}\NormalTok{ (}\DataTypeTok{EOp} \DataTypeTok{OTimes}\NormalTok{ (}\DataTypeTok{EVar} \DataTypeTok{STInt} \StringTok{"x"}\NormalTok{) (}\DataTypeTok{EPrim}\NormalTok{ (}\DataTypeTok{PInt} \DecValTok{3}\NormalTok{))))}
\NormalTok{    (}\DataTypeTok{EPrim}\NormalTok{ (}\DataTypeTok{PInt} \DecValTok{5}\NormalTok{))}
\end{Highlighting}
\end{Shaded}

We also need a
\href{https://blog.jle.im/entries/series/+introduction-to-singletons.html}{singleton}
for our \texttt{Ty} type, \texttt{STy}\ldots this makes a whole lot of things
simpler. Usually when you have a data kind, you can try to avoid singletons but
a lot of times you're just delaying the inevitable. In this case our lambda is
``typed'', \texttt{ELambda\ STInt\ "x"}, so it binds a variable of type
\texttt{Int} with name \texttt{x}.

Overall, I'll say this is about 50\% fancy. Anything with GADTs will be a
significant bump. But you might see the problem here: \texttt{EVar\ STInt\ "x"}.
\texttt{x} might not be defined, and it also might not have the correct type.
Soooo yes, we still have issues here.

But now at least we can write \texttt{eval}:

\begin{Shaded}
\begin{Highlighting}[]
\CommentTok{{-}{-} source: https://github.com/mstksg/inCode/tree/master/code{-}samples/typed{-}sm{-}lc/ExprStage3.hs\#L56{-}L132}

\KeywordTok{data} \DataTypeTok{EValue}\OtherTok{ ::} \DataTypeTok{Ty} \OtherTok{{-}\textgreater{}} \DataTypeTok{Type} \KeywordTok{where}
  \DataTypeTok{EVInt}\OtherTok{ ::} \DataTypeTok{Int} \OtherTok{{-}\textgreater{}} \DataTypeTok{EValue} \DataTypeTok{TInt}
  \DataTypeTok{EVBool}\OtherTok{ ::} \DataTypeTok{Bool} \OtherTok{{-}\textgreater{}} \DataTypeTok{EValue} \DataTypeTok{TBool}
  \DataTypeTok{EVString}\OtherTok{ ::} \DataTypeTok{String} \OtherTok{{-}\textgreater{}} \DataTypeTok{EValue} \DataTypeTok{TString}
  \DataTypeTok{EVRecord}\OtherTok{ ::} \DataTypeTok{Map} \DataTypeTok{String} \DataTypeTok{SomeValue} \OtherTok{{-}\textgreater{}} \DataTypeTok{EValue} \DataTypeTok{TRecord}
  \DataTypeTok{EVFun}\OtherTok{ ::}\NormalTok{ (}\DataTypeTok{EValue}\NormalTok{ a }\OtherTok{{-}\textgreater{}} \DataTypeTok{Maybe}\NormalTok{ (}\DataTypeTok{EValue}\NormalTok{ b)) }\OtherTok{{-}\textgreater{}} \DataTypeTok{EValue}\NormalTok{ (a }\OperatorTok{:{-}\textgreater{}}\NormalTok{ b)}

\KeywordTok{data} \DataTypeTok{SomeValue} \KeywordTok{where}
  \DataTypeTok{SomeValue}\OtherTok{ ::} \DataTypeTok{STy}\NormalTok{ t }\OtherTok{{-}\textgreater{}} \DataTypeTok{EValue}\NormalTok{ t }\OtherTok{{-}\textgreater{}} \DataTypeTok{SomeValue}

\OtherTok{sameTy ::} \DataTypeTok{STy}\NormalTok{ a }\OtherTok{{-}\textgreater{}} \DataTypeTok{STy}\NormalTok{ b }\OtherTok{{-}\textgreater{}} \DataTypeTok{Maybe}\NormalTok{ (a }\OperatorTok{:\textasciitilde{}:}\NormalTok{ b)}
\NormalTok{sameTy }\OtherTok{=}\NormalTok{ \textbackslash{}}\KeywordTok{case}
  \DataTypeTok{STInt} \OtherTok{{-}\textgreater{}}\NormalTok{ \textbackslash{}}\KeywordTok{case} \DataTypeTok{STInt} \OtherTok{{-}\textgreater{}} \DataTypeTok{Just} \DataTypeTok{Refl}\NormalTok{; \_ }\OtherTok{{-}\textgreater{}} \DataTypeTok{Nothing}
  \DataTypeTok{STBool} \OtherTok{{-}\textgreater{}}\NormalTok{ \textbackslash{}}\KeywordTok{case} \DataTypeTok{STBool} \OtherTok{{-}\textgreater{}} \DataTypeTok{Just} \DataTypeTok{Refl}\NormalTok{; \_ }\OtherTok{{-}\textgreater{}} \DataTypeTok{Nothing}
  \DataTypeTok{STString} \OtherTok{{-}\textgreater{}}\NormalTok{ \textbackslash{}}\KeywordTok{case} \DataTypeTok{STString} \OtherTok{{-}\textgreater{}} \DataTypeTok{Just} \DataTypeTok{Refl}\NormalTok{; \_ }\OtherTok{{-}\textgreater{}} \DataTypeTok{Nothing}
  \DataTypeTok{STRecord} \OtherTok{{-}\textgreater{}}\NormalTok{ \textbackslash{}}\KeywordTok{case} \DataTypeTok{STRecord} \OtherTok{{-}\textgreater{}} \DataTypeTok{Just} \DataTypeTok{Refl}\NormalTok{; \_ }\OtherTok{{-}\textgreater{}} \DataTypeTok{Nothing}
  \DataTypeTok{STFun}\NormalTok{ a b }\OtherTok{{-}\textgreater{}}\NormalTok{ \textbackslash{}}\KeywordTok{case}
    \DataTypeTok{STFun}\NormalTok{ c d }\OtherTok{{-}\textgreater{}} \KeywordTok{do}
      \DataTypeTok{Refl} \OtherTok{\textless{}{-}}\NormalTok{ sameTy a c}
      \DataTypeTok{Refl} \OtherTok{\textless{}{-}}\NormalTok{ sameTy b d}
      \DataTypeTok{Just} \DataTypeTok{Refl}
\NormalTok{    \_ }\OtherTok{{-}\textgreater{}} \DataTypeTok{Nothing}

\OtherTok{eval ::} \DataTypeTok{Map} \DataTypeTok{String} \DataTypeTok{SomeValue} \OtherTok{{-}\textgreater{}} \DataTypeTok{Expr}\NormalTok{ t }\OtherTok{{-}\textgreater{}} \DataTypeTok{Maybe}\NormalTok{ (}\DataTypeTok{EValue}\NormalTok{ t)}
\NormalTok{eval env }\OtherTok{=}\NormalTok{ \textbackslash{}}\KeywordTok{case}
  \DataTypeTok{EPrim}\NormalTok{ (}\DataTypeTok{PInt}\NormalTok{ n) }\OtherTok{{-}\textgreater{}} \FunctionTok{pure}\NormalTok{ (}\DataTypeTok{EVInt}\NormalTok{ n)}
  \DataTypeTok{EPrim}\NormalTok{ (}\DataTypeTok{PBool}\NormalTok{ b) }\OtherTok{{-}\textgreater{}} \FunctionTok{pure}\NormalTok{ (}\DataTypeTok{EVBool}\NormalTok{ b)}
  \DataTypeTok{EPrim}\NormalTok{ (}\DataTypeTok{PString}\NormalTok{ s) }\OtherTok{{-}\textgreater{}} \FunctionTok{pure}\NormalTok{ (}\DataTypeTok{EVString}\NormalTok{ s)}
  \DataTypeTok{EVar}\NormalTok{ t v }\OtherTok{{-}\textgreater{}} \KeywordTok{do}
    \DataTypeTok{SomeValue}\NormalTok{ t\textquotesingle{} v\textquotesingle{} }\OtherTok{\textless{}{-}}\NormalTok{ M.lookup v env}
    \DataTypeTok{Refl} \OtherTok{\textless{}{-}}\NormalTok{ sameTy t t\textquotesingle{}}
    \FunctionTok{pure}\NormalTok{ v\textquotesingle{}}
  \DataTypeTok{ELambda}\NormalTok{ ta n body }\OtherTok{{-}\textgreater{}}
    \FunctionTok{pure} \OperatorTok{$} \DataTypeTok{EVFun} \OperatorTok{$}\NormalTok{ \textbackslash{}x }\OtherTok{{-}\textgreater{}}\NormalTok{ eval (M.insert n (}\DataTypeTok{SomeValue}\NormalTok{ ta x) env) body}
  \DataTypeTok{EApply}\NormalTok{ f x }\OtherTok{{-}\textgreater{}} \KeywordTok{do}
    \DataTypeTok{EVFun}\NormalTok{ g }\OtherTok{\textless{}{-}}\NormalTok{ eval env f}
\NormalTok{    x\textquotesingle{} }\OtherTok{\textless{}{-}}\NormalTok{ eval env x}
\NormalTok{    g x\textquotesingle{}}
  \DataTypeTok{EOp}\NormalTok{ o x y }\OtherTok{{-}\textgreater{}} \KeywordTok{case}\NormalTok{ o }\KeywordTok{of}
    \DataTypeTok{OPlus} \OtherTok{{-}\textgreater{}} \KeywordTok{do}
      \DataTypeTok{EVInt}\NormalTok{ a }\OtherTok{\textless{}{-}}\NormalTok{ eval env x}
      \DataTypeTok{EVInt}\NormalTok{ b }\OtherTok{\textless{}{-}}\NormalTok{ eval env y}
      \FunctionTok{pure}\NormalTok{ (}\DataTypeTok{EVInt}\NormalTok{ (a }\OperatorTok{+}\NormalTok{ b))}
    \DataTypeTok{OTimes} \OtherTok{{-}\textgreater{}} \KeywordTok{do}
      \DataTypeTok{EVInt}\NormalTok{ a }\OtherTok{\textless{}{-}}\NormalTok{ eval env x}
      \DataTypeTok{EVInt}\NormalTok{ b }\OtherTok{\textless{}{-}}\NormalTok{ eval env y}
      \FunctionTok{pure}\NormalTok{ (}\DataTypeTok{EVInt}\NormalTok{ (a }\OperatorTok{*}\NormalTok{ b))}
    \DataTypeTok{OLte} \OtherTok{{-}\textgreater{}} \KeywordTok{do}
      \DataTypeTok{EVInt}\NormalTok{ a }\OtherTok{\textless{}{-}}\NormalTok{ eval env x}
      \DataTypeTok{EVInt}\NormalTok{ b }\OtherTok{\textless{}{-}}\NormalTok{ eval env y}
      \FunctionTok{pure}\NormalTok{ (}\DataTypeTok{EVBool}\NormalTok{ (a }\OperatorTok{\textless{}=}\NormalTok{ b))}
    \DataTypeTok{OAnd} \OtherTok{{-}\textgreater{}} \KeywordTok{do}
      \DataTypeTok{EVBool}\NormalTok{ a }\OtherTok{\textless{}{-}}\NormalTok{ eval env x}
      \DataTypeTok{EVBool}\NormalTok{ b }\OtherTok{\textless{}{-}}\NormalTok{ eval env y}
      \FunctionTok{pure}\NormalTok{ (}\DataTypeTok{EVBool}\NormalTok{ (a }\OperatorTok{\&\&}\NormalTok{ b))}
  \DataTypeTok{ERecord}\NormalTok{ xs }\OtherTok{{-}\textgreater{}}
    \DataTypeTok{EVRecord} \OperatorTok{\textless{}$\textgreater{}} \FunctionTok{traverse}\NormalTok{ evalField xs}
  \DataTypeTok{EAccess}\NormalTok{ t e k }\OtherTok{{-}\textgreater{}} \KeywordTok{do}
    \DataTypeTok{EVRecord}\NormalTok{ xs }\OtherTok{\textless{}{-}}\NormalTok{ eval env e}
    \DataTypeTok{SomeValue}\NormalTok{ t\textquotesingle{} v\textquotesingle{} }\OtherTok{\textless{}{-}}\NormalTok{ M.lookup k xs}
    \DataTypeTok{Refl} \OtherTok{\textless{}{-}}\NormalTok{ sameTy t t\textquotesingle{}}
    \FunctionTok{pure}\NormalTok{ v\textquotesingle{}}
  \KeywordTok{where}
\NormalTok{    evalField (}\DataTypeTok{SomeExpr}\NormalTok{ t v) }\OtherTok{=} \KeywordTok{do}
\NormalTok{      v\textquotesingle{} }\OtherTok{\textless{}{-}}\NormalTok{ eval env v}
      \FunctionTok{pure}\NormalTok{ (}\DataTypeTok{SomeValue}\NormalTok{ t v\textquotesingle{})}
\end{Highlighting}
\end{Shaded}

What did we gain here? We have a type-safe \texttt{eval} now that will create a
value of the type we want. But we still have the same errors when looking at
variables: variables can still not be defined, or be defined as the wrong type.

So, again, we cannot create an \texttt{Expr} that must be sensible and
well-formed to compile. We still have to deal with \emph{most} of the same
errors. This is noble, but clearly not good enough. We have to go deeper.

\subsection{Type-Safe Environments}\label{type-safe-environments}

In order to have \texttt{Var} be type-safe, the environment itself needs to be a
part of the \texttt{Expr} type, and you should only be able to use \texttt{Var}
if the \texttt{Expr} enforces it. \texttt{ELambda} would, therefore, introduce
the new variable to the environment.

We'll have:

\begin{Shaded}
\begin{Highlighting}[]
\CommentTok{{-}{-} source: https://github.com/mstksg/inCode/tree/master/code{-}samples/typed{-}sm{-}lc/ExprStage4.hs\#L44{-}L59}

\KeywordTok{data} \DataTypeTok{Expr}\OtherTok{ ::}\NormalTok{ [(}\DataTypeTok{Symbol}\NormalTok{, }\DataTypeTok{Ty}\NormalTok{)] }\OtherTok{{-}\textgreater{}} \DataTypeTok{Ty} \OtherTok{{-}\textgreater{}} \DataTypeTok{Type} \KeywordTok{where}

\KeywordTok{type}\NormalTok{ (}\OperatorTok{:::}\NormalTok{) l a }\OtherTok{=}\NormalTok{ \textquotesingle{}(l, a)}
\end{Highlighting}
\end{Shaded}

So a value of type
\texttt{Expr\ \textquotesingle{}{[}"x"\ :::\ TInt,\ "y"\ :::\ TBool{]}} is an
expression with free variables \texttt{x} of type \texttt{Int} and a \texttt{y}
of type \texttt{Bool}.

\texttt{ELambda} would therefore take a \texttt{Expr} with a free variable and
turn it into an \texttt{Expr} of a function type: (and \texttt{KnownSymbol}
instance so that we can debug print the variable name)

\begin{Shaded}
\begin{Highlighting}[]
\CommentTok{{-}{-} source: https://github.com/mstksg/inCode/tree/master/code{-}samples/typed{-}sm{-}lc/ExprStage4.hs\#L62{-}L66}

  \DataTypeTok{ELambda}\OtherTok{ ::} \DataTypeTok{KnownSymbol}\NormalTok{ n }\OtherTok{=\textgreater{}} \DataTypeTok{Expr}\NormalTok{ (n }\OperatorTok{:::}\NormalTok{ a \textquotesingle{}}\OperatorTok{:}\NormalTok{ vs) b }\OtherTok{{-}\textgreater{}} \DataTypeTok{Expr}\NormalTok{ vs (a }\OperatorTok{:{-}\textgreater{}}\NormalTok{ b)}
  \DataTypeTok{EApply}\OtherTok{ ::} \DataTypeTok{Expr}\NormalTok{ vs (a }\OperatorTok{:{-}\textgreater{}}\NormalTok{ b) }\OtherTok{{-}\textgreater{}} \DataTypeTok{Expr}\NormalTok{ vs a }\OtherTok{{-}\textgreater{}} \DataTypeTok{Expr}\NormalTok{ vs b}
  \DataTypeTok{EOp}\OtherTok{ ::} \DataTypeTok{Op}\NormalTok{ a b c }\OtherTok{{-}\textgreater{}} \DataTypeTok{Expr}\NormalTok{ vs a }\OtherTok{{-}\textgreater{}} \DataTypeTok{Expr}\NormalTok{ vs b }\OtherTok{{-}\textgreater{}} \DataTypeTok{Expr}\NormalTok{ vs c}
  \DataTypeTok{ERecord}\OtherTok{ ::} \DataTypeTok{Rec}\NormalTok{ (}\DataTypeTok{ExprField}\NormalTok{ vs) as }\OtherTok{{-}\textgreater{}} \DataTypeTok{Expr}\NormalTok{ vs (}\DataTypeTok{TRecord}\NormalTok{ as)}
  \DataTypeTok{EAccess}\OtherTok{ ::} \DataTypeTok{KnownSymbol}\NormalTok{ l }\OtherTok{=\textgreater{}} \DataTypeTok{Expr}\NormalTok{ vs (}\DataTypeTok{TRecord}\NormalTok{ as) }\OtherTok{{-}\textgreater{}} \DataTypeTok{Index}\NormalTok{ as (l }\OperatorTok{:::}\NormalTok{ a) }\OtherTok{{-}\textgreater{}} \DataTypeTok{Expr}\NormalTok{ vs a}
\end{Highlighting}
\end{Shaded}

So how do we implement \texttt{Var}? We have to gate it on whether or not the
free variable is available in the environment. For that, we can use
\texttt{Index}:

\begin{Shaded}
\begin{Highlighting}[]
\CommentTok{{-}{-} source: https://github.com/mstksg/inCode/tree/master/code{-}samples/typed{-}sm{-}lc/ExprStage4.hs\#L31{-}L33}

\KeywordTok{data} \DataTypeTok{Index}\OtherTok{ ::}\NormalTok{ [k] }\OtherTok{{-}\textgreater{}}\NormalTok{ k }\OtherTok{{-}\textgreater{}} \DataTypeTok{Type} \KeywordTok{where}
  \DataTypeTok{IZ}\OtherTok{ ::} \DataTypeTok{Index}\NormalTok{ (x \textquotesingle{}}\OperatorTok{:}\NormalTok{ xs) x}
  \DataTypeTok{IS}\OtherTok{ ::} \DataTypeTok{Index}\NormalTok{ xs x }\OtherTok{{-}\textgreater{}} \DataTypeTok{Index}\NormalTok{ (y \textquotesingle{}}\OperatorTok{:}\NormalTok{ xs) x}
\end{Highlighting}
\end{Shaded}

I have this in {[}functor-products{]}{[}{]}, but it's also \texttt{CoRec\ Proxy}
from {[}vinyl{]}{[}{]} or \texttt{NP\ Proxy} from {[}sop-core{]}{[}{]}. You can
reason about this like axioms: \texttt{Index\ as\ a} means that \texttt{a} is an
item in \texttt{as}, which is either \texttt{a} being at the start of the list
(\texttt{IZ}) or \texttt{a} being within the tail (\texttt{IS}).

{[}functor-products{]}{[}{]}:
https://hackage.haskell.org/package/functor-products {[}vinyl{]}:
https://hackage.haskell.org/package/vinyl {[}sop-core{]}:
https://hackage.haskell.org/package/sop-core

For example, we have values
\texttt{IZ\ ::\ Index\ \textquotesingle{}{[}a,b,c{]}\ a},
\texttt{IS\ IZ\ ::\ Index\ \textquotesingle{}{[}a,b,c{]}\ b}, and
\texttt{IS\ (IS\ IZ)\ ::\ Index\ \textquotesingle{}{[}a,b,c{]}\ c}. So, if we
require \texttt{Var} to take an \texttt{Index}, we require it to indicate
something that \emph{is} inside the \texttt{Expr}'s free variable list and at
that given index:

\begin{Shaded}
\begin{Highlighting}[]
\CommentTok{{-}{-} source: https://github.com/mstksg/inCode/tree/master/code{-}samples/typed{-}sm{-}lc/ExprStage4.hs\#L61{-}L66}

  \DataTypeTok{EVar}\OtherTok{ ::} \DataTypeTok{Index}\NormalTok{ vs (n }\OperatorTok{:::}\NormalTok{ t) }\OtherTok{{-}\textgreater{}} \DataTypeTok{Expr}\NormalTok{ vs t}
  \DataTypeTok{ELambda}\OtherTok{ ::} \DataTypeTok{KnownSymbol}\NormalTok{ n }\OtherTok{=\textgreater{}} \DataTypeTok{Expr}\NormalTok{ (n }\OperatorTok{:::}\NormalTok{ a \textquotesingle{}}\OperatorTok{:}\NormalTok{ vs) b }\OtherTok{{-}\textgreater{}} \DataTypeTok{Expr}\NormalTok{ vs (a }\OperatorTok{:{-}\textgreater{}}\NormalTok{ b)}
  \DataTypeTok{EApply}\OtherTok{ ::} \DataTypeTok{Expr}\NormalTok{ vs (a }\OperatorTok{:{-}\textgreater{}}\NormalTok{ b) }\OtherTok{{-}\textgreater{}} \DataTypeTok{Expr}\NormalTok{ vs a }\OtherTok{{-}\textgreater{}} \DataTypeTok{Expr}\NormalTok{ vs b}
  \DataTypeTok{EOp}\OtherTok{ ::} \DataTypeTok{Op}\NormalTok{ a b c }\OtherTok{{-}\textgreater{}} \DataTypeTok{Expr}\NormalTok{ vs a }\OtherTok{{-}\textgreater{}} \DataTypeTok{Expr}\NormalTok{ vs b }\OtherTok{{-}\textgreater{}} \DataTypeTok{Expr}\NormalTok{ vs c}
  \DataTypeTok{ERecord}\OtherTok{ ::} \DataTypeTok{Rec}\NormalTok{ (}\DataTypeTok{ExprField}\NormalTok{ vs) as }\OtherTok{{-}\textgreater{}} \DataTypeTok{Expr}\NormalTok{ vs (}\DataTypeTok{TRecord}\NormalTok{ as)}
  \DataTypeTok{EAccess}\OtherTok{ ::} \DataTypeTok{KnownSymbol}\NormalTok{ l }\OtherTok{=\textgreater{}} \DataTypeTok{Expr}\NormalTok{ vs (}\DataTypeTok{TRecord}\NormalTok{ as) }\OtherTok{{-}\textgreater{}} \DataTypeTok{Index}\NormalTok{ as (l }\OperatorTok{:::}\NormalTok{ a) }\OtherTok{{-}\textgreater{}} \DataTypeTok{Expr}\NormalTok{ vs a}
\end{Highlighting}
\end{Shaded}

So it is legal to have
\texttt{EVar\ IZ\ ::\ Expr\ \textquotesingle{}{[}"x"\ :::\ TInt,\ "y"\ :::\ TBool{]}\ TInt},
and also it is automatically inferred to be a \texttt{TInt}. But we could
\emph{not} write \texttt{EVar\ IZ\ ::\ Expr\ \textquotesingle{}{[}{]}\ TInt}.

\begin{Shaded}
\begin{Highlighting}[]
\CommentTok{{-}{-} source: https://github.com/mstksg/inCode/tree/master/code{-}samples/typed{-}sm{-}lc/ExprStage4.hs\#L59{-}L88}

\KeywordTok{data} \DataTypeTok{Expr}\OtherTok{ ::}\NormalTok{ [(}\DataTypeTok{Symbol}\NormalTok{, }\DataTypeTok{Ty}\NormalTok{)] }\OtherTok{{-}\textgreater{}} \DataTypeTok{Ty} \OtherTok{{-}\textgreater{}} \DataTypeTok{Type} \KeywordTok{where}
  \DataTypeTok{EPrim}\OtherTok{ ::} \DataTypeTok{Prim}\NormalTok{ t }\OtherTok{{-}\textgreater{}} \DataTypeTok{Expr}\NormalTok{ vs t}
  \DataTypeTok{EVar}\OtherTok{ ::} \DataTypeTok{Index}\NormalTok{ vs (n }\OperatorTok{:::}\NormalTok{ t) }\OtherTok{{-}\textgreater{}} \DataTypeTok{Expr}\NormalTok{ vs t}
  \DataTypeTok{ELambda}\OtherTok{ ::} \DataTypeTok{KnownSymbol}\NormalTok{ n }\OtherTok{=\textgreater{}} \DataTypeTok{Expr}\NormalTok{ (n }\OperatorTok{:::}\NormalTok{ a \textquotesingle{}}\OperatorTok{:}\NormalTok{ vs) b }\OtherTok{{-}\textgreater{}} \DataTypeTok{Expr}\NormalTok{ vs (a }\OperatorTok{:{-}\textgreater{}}\NormalTok{ b)}
  \DataTypeTok{EApply}\OtherTok{ ::} \DataTypeTok{Expr}\NormalTok{ vs (a }\OperatorTok{:{-}\textgreater{}}\NormalTok{ b) }\OtherTok{{-}\textgreater{}} \DataTypeTok{Expr}\NormalTok{ vs a }\OtherTok{{-}\textgreater{}} \DataTypeTok{Expr}\NormalTok{ vs b}
  \DataTypeTok{EOp}\OtherTok{ ::} \DataTypeTok{Op}\NormalTok{ a b c }\OtherTok{{-}\textgreater{}} \DataTypeTok{Expr}\NormalTok{ vs a }\OtherTok{{-}\textgreater{}} \DataTypeTok{Expr}\NormalTok{ vs b }\OtherTok{{-}\textgreater{}} \DataTypeTok{Expr}\NormalTok{ vs c}
  \DataTypeTok{ERecord}\OtherTok{ ::} \DataTypeTok{Rec}\NormalTok{ (}\DataTypeTok{ExprField}\NormalTok{ vs) as }\OtherTok{{-}\textgreater{}} \DataTypeTok{Expr}\NormalTok{ vs (}\DataTypeTok{TRecord}\NormalTok{ as)}
  \DataTypeTok{EAccess}\OtherTok{ ::} \DataTypeTok{KnownSymbol}\NormalTok{ l }\OtherTok{=\textgreater{}} \DataTypeTok{Expr}\NormalTok{ vs (}\DataTypeTok{TRecord}\NormalTok{ as) }\OtherTok{{-}\textgreater{}} \DataTypeTok{Index}\NormalTok{ as (l }\OperatorTok{:::}\NormalTok{ a) }\OtherTok{{-}\textgreater{}} \DataTypeTok{Expr}\NormalTok{ vs a}

\OtherTok{eLambda ::} \KeywordTok{forall}\NormalTok{ n }\OtherTok{{-}\textgreater{}} \DataTypeTok{KnownSymbol}\NormalTok{ n }\OtherTok{=\textgreater{}} \DataTypeTok{Expr}\NormalTok{ (n }\OperatorTok{:::}\NormalTok{ a \textquotesingle{}}\OperatorTok{:}\NormalTok{ vs) b }\OtherTok{{-}\textgreater{}} \DataTypeTok{Expr}\NormalTok{ vs (a }\OperatorTok{:{-}\textgreater{}}\NormalTok{ b)}
\NormalTok{eLambda n x }\OtherTok{=} \DataTypeTok{ELambda} \OperatorTok{@}\NormalTok{n x}

\OtherTok{fifteen ::} \DataTypeTok{Expr}\NormalTok{ \textquotesingle{}[] }\DataTypeTok{TInt}
\NormalTok{fifteen }\OtherTok{=}
  \DataTypeTok{EApply}
\NormalTok{    (eLambda }\StringTok{"x"}\NormalTok{ (}\DataTypeTok{EOp} \DataTypeTok{OTimes}\NormalTok{ (}\DataTypeTok{EVar} \DataTypeTok{IZ}\NormalTok{) (}\DataTypeTok{EPrim}\NormalTok{ (}\DataTypeTok{PInt} \DecValTok{3}\NormalTok{))))}
\NormalTok{    (}\DataTypeTok{EPrim}\NormalTok{ (}\DataTypeTok{PInt} \DecValTok{5}\NormalTok{))}
\end{Highlighting}
\end{Shaded}

In GHC 9.12 we can write \texttt{eLambda} using \texttt{RequiredTypeArguments}
and so can pass the type variable as a string literal, \texttt{eLambda\ "x"} but
we can't yet put this directly in \texttt{ELambda} for some reason.

Note that this is sometimes done using straight
\href{https://en.wikipedia.org/wiki/De_Bruijn_index}{De Bruijn indices}:
\texttt{Expr\ ::\ {[}Ty{]}\ -\textgreater{}\ Type}, so we don't use any names
but just the direct index, but the point of this exercise is to be
\emph{borderline} unbearable to write, and \emph{not} to be \emph{actually}
unbearable to write.

To actually write \emph{eval} now, we need to have a type-safe environment to
store these variables, and for this we can use \texttt{Rec} (from
{[}vinyl{]}{[}{]}) or \texttt{NP} from {[}sop-core{]}:

\begin{Shaded}
\begin{Highlighting}[]
\CommentTok{{-}{-} source: https://github.com/mstksg/inCode/tree/master/code{-}samples/typed{-}sm{-}lc/ExprStage4.hs\#L27{-}L118}

\KeywordTok{data} \DataTypeTok{Rec}\OtherTok{ ::}\NormalTok{ (k }\OtherTok{{-}\textgreater{}} \DataTypeTok{Type}\NormalTok{) }\OtherTok{{-}\textgreater{}}\NormalTok{ [k] }\OtherTok{{-}\textgreater{}} \DataTypeTok{Type} \KeywordTok{where}
  \DataTypeTok{RNil}\OtherTok{ ::} \DataTypeTok{Rec}\NormalTok{ f \textquotesingle{}[]}
\OtherTok{  (:\&) ::}\NormalTok{ f x }\OtherTok{{-}\textgreater{}} \DataTypeTok{Rec}\NormalTok{ f xs }\OtherTok{{-}\textgreater{}} \DataTypeTok{Rec}\NormalTok{ f (x \textquotesingle{}}\OperatorTok{:}\NormalTok{ xs)}

\OtherTok{indexRec ::} \DataTypeTok{Index}\NormalTok{ xs x }\OtherTok{{-}\textgreater{}} \DataTypeTok{Rec}\NormalTok{ f xs }\OtherTok{{-}\textgreater{}}\NormalTok{ f x}
\NormalTok{indexRec }\OtherTok{=}\NormalTok{ \textbackslash{}}\KeywordTok{case}
  \DataTypeTok{IZ} \OtherTok{{-}\textgreater{}}\NormalTok{ \textbackslash{}(x }\OperatorTok{:\&}\NormalTok{ \_) }\OtherTok{{-}\textgreater{}}\NormalTok{ x}
  \DataTypeTok{IS}\NormalTok{ i }\OtherTok{{-}\textgreater{}}\NormalTok{ \textbackslash{}(\_ }\OperatorTok{:\&}\NormalTok{ xs) }\OtherTok{{-}\textgreater{}}\NormalTok{ indexRec i xs}

\OtherTok{eval ::} \DataTypeTok{Rec} \DataTypeTok{EValueField}\NormalTok{ vs }\OtherTok{{-}\textgreater{}} \DataTypeTok{Expr}\NormalTok{ vs t }\OtherTok{{-}\textgreater{}} \DataTypeTok{EValue}\NormalTok{ t}
\NormalTok{eval env }\OtherTok{=}\NormalTok{ \textbackslash{}}\KeywordTok{case}
  \DataTypeTok{EPrim}\NormalTok{ (}\DataTypeTok{PInt}\NormalTok{ n) }\OtherTok{{-}\textgreater{}} \DataTypeTok{EVInt}\NormalTok{ n}
  \DataTypeTok{EPrim}\NormalTok{ (}\DataTypeTok{PBool}\NormalTok{ b) }\OtherTok{{-}\textgreater{}} \DataTypeTok{EVBool}\NormalTok{ b}
  \DataTypeTok{EPrim}\NormalTok{ (}\DataTypeTok{PString}\NormalTok{ s) }\OtherTok{{-}\textgreater{}} \DataTypeTok{EVString}\NormalTok{ s}
  \DataTypeTok{EVar}\NormalTok{ i }\OtherTok{{-}\textgreater{}} \KeywordTok{case}\NormalTok{ indexRec i env }\KeywordTok{of}
    \DataTypeTok{EVField}\NormalTok{ v }\OtherTok{{-}\textgreater{}}\NormalTok{ v}
  \DataTypeTok{ELambda}\NormalTok{ body }\OtherTok{{-}\textgreater{}}
    \DataTypeTok{EVFun} \OperatorTok{$}\NormalTok{ \textbackslash{}x }\OtherTok{{-}\textgreater{}}\NormalTok{ eval (}\DataTypeTok{EVField}\NormalTok{ x }\OperatorTok{:\&}\NormalTok{ env) body}
  \DataTypeTok{EApply}\NormalTok{ f x }\OtherTok{{-}\textgreater{}} \KeywordTok{case}\NormalTok{ eval env f }\KeywordTok{of}
    \DataTypeTok{EVFun}\NormalTok{ g }\OtherTok{{-}\textgreater{}}\NormalTok{ g (eval env x)}
  \DataTypeTok{EOp}\NormalTok{ o x y }\OtherTok{{-}\textgreater{}} \KeywordTok{case}\NormalTok{ (o, eval env x, eval env y) }\KeywordTok{of}
\NormalTok{    (}\DataTypeTok{OPlus}\NormalTok{, }\DataTypeTok{EVInt}\NormalTok{ a, }\DataTypeTok{EVInt}\NormalTok{ b) }\OtherTok{{-}\textgreater{}} \DataTypeTok{EVInt}\NormalTok{ (a }\OperatorTok{+}\NormalTok{ b)}
\NormalTok{    (}\DataTypeTok{OTimes}\NormalTok{, }\DataTypeTok{EVInt}\NormalTok{ a, }\DataTypeTok{EVInt}\NormalTok{ b) }\OtherTok{{-}\textgreater{}} \DataTypeTok{EVInt}\NormalTok{ (a }\OperatorTok{*}\NormalTok{ b)}
\NormalTok{    (}\DataTypeTok{OLte}\NormalTok{, }\DataTypeTok{EVInt}\NormalTok{ a, }\DataTypeTok{EVInt}\NormalTok{ b) }\OtherTok{{-}\textgreater{}} \DataTypeTok{EVBool}\NormalTok{ (a }\OperatorTok{\textless{}=}\NormalTok{ b)}
\NormalTok{    (}\DataTypeTok{OAnd}\NormalTok{, }\DataTypeTok{EVBool}\NormalTok{ a, }\DataTypeTok{EVBool}\NormalTok{ b) }\OtherTok{{-}\textgreater{}} \DataTypeTok{EVBool}\NormalTok{ (a }\OperatorTok{\&\&}\NormalTok{ b)}
  \DataTypeTok{ERecord}\NormalTok{ xs }\OtherTok{{-}\textgreater{}}
    \DataTypeTok{EVRecord} \OperatorTok{$}\NormalTok{ mapRec (\textbackslash{}(}\DataTypeTok{EField}\NormalTok{ x) }\OtherTok{{-}\textgreater{}} \DataTypeTok{EVField}\NormalTok{ (eval env x)) xs}
  \DataTypeTok{EAccess}\NormalTok{ e i }\OtherTok{{-}\textgreater{}} \KeywordTok{case}\NormalTok{ eval env e }\KeywordTok{of}
    \DataTypeTok{EVRecord}\NormalTok{ xs }\OtherTok{{-}\textgreater{}} \KeywordTok{case}\NormalTok{ indexRec i xs }\KeywordTok{of}
      \DataTypeTok{EVField}\NormalTok{ v }\OtherTok{{-}\textgreater{}}\NormalTok{ v}
\end{Highlighting}
\end{Shaded}

\section{The State Machine}\label{the-state-machine}

\section{Signoff}\label{signoff}

Hi, thanks for reading! You can reach me via email at
\href{mailto:justin@jle.im}{\nolinkurl{justin@jle.im}}, or at twitter at
\href{https://twitter.com/mstk}{@mstk}! This post and all others are published
under the \href{https://creativecommons.org/licenses/by-nc-nd/3.0/}{CC-BY-NC-ND
3.0} license. Corrections and edits via pull request are welcome and encouraged
at \href{https://github.com/mstksg/inCode}{the source repository}.

If you feel inclined, or this post was particularly helpful for you, why not
consider \href{https://www.patreon.com/justinle/overview}{supporting me on
Patreon}, or a \href{bitcoin:3D7rmAYgbDnp4gp4rf22THsGt74fNucPDU}{BTC donation}?
:)

\textbackslash end\{document\}

\end{document}
