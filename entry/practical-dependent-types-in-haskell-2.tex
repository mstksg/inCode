\documentclass[]{article}
\usepackage{lmodern}
\usepackage{amssymb,amsmath}
\usepackage{ifxetex,ifluatex}
\usepackage{fixltx2e} % provides \textsubscript
\ifnum 0\ifxetex 1\fi\ifluatex 1\fi=0 % if pdftex
  \usepackage[T1]{fontenc}
  \usepackage[utf8]{inputenc}
\else % if luatex or xelatex
  \ifxetex
    \usepackage{mathspec}
    \usepackage{xltxtra,xunicode}
  \else
    \usepackage{fontspec}
  \fi
  \defaultfontfeatures{Mapping=tex-text,Scale=MatchLowercase}
  \newcommand{\euro}{€}
\fi
% use upquote if available, for straight quotes in verbatim environments
\IfFileExists{upquote.sty}{\usepackage{upquote}}{}
% use microtype if available
\IfFileExists{microtype.sty}{\usepackage{microtype}}{}
\usepackage[margin=1in]{geometry}
\ifxetex
  \usepackage[setpagesize=false, % page size defined by xetex
              unicode=false, % unicode breaks when used with xetex
              xetex]{hyperref}
\else
  \usepackage[unicode=true]{hyperref}
\fi
\hypersetup{breaklinks=true,
            bookmarks=true,
            pdfauthor={},
            pdftitle={Practical Dependent Types in Haskell 2: Existential Neural Networks and Types at},
            colorlinks=true,
            citecolor=blue,
            urlcolor=blue,
            linkcolor=magenta,
            pdfborder={0 0 0}}
\urlstyle{same}  % don't use monospace font for urls
% Make links footnotes instead of hotlinks:
\renewcommand{\href}[2]{#2\footnote{\url{#1}}}
\setlength{\parindent}{0pt}
\setlength{\parskip}{6pt plus 2pt minus 1pt}
\setlength{\emergencystretch}{3em}  % prevent overfull lines
\setcounter{secnumdepth}{0}

\title{Practical Dependent Types in Haskell 2: Existential Neural Networks and
Types at}

\begin{document}
\maketitle

Runtime \% Justin Le \% June 30, 2016

\emph{Originally posted on
\textbf{\href{https://blog.jle.im/entry/practical-dependent-types-in-haskell-2.html}{in
Code}}.}

\documentclass[]{}
\usepackage{lmodern}
\usepackage{amssymb,amsmath}
\usepackage{ifxetex,ifluatex}
\usepackage{fixltx2e} % provides \textsubscript
\ifnum 0\ifxetex 1\fi\ifluatex 1\fi=0 % if pdftex
  \usepackage[T1]{fontenc}
  \usepackage[utf8]{inputenc}
\else % if luatex or xelatex
  \ifxetex
    \usepackage{mathspec}
    \usepackage{xltxtra,xunicode}
  \else
    \usepackage{fontspec}
  \fi
  \defaultfontfeatures{Mapping=tex-text,Scale=MatchLowercase}
  \newcommand{\euro}{€}
\fi
% use upquote if available, for straight quotes in verbatim environments
\IfFileExists{upquote.sty}{\usepackage{upquote}}{}
% use microtype if available
\IfFileExists{microtype.sty}{\usepackage{microtype}}{}
\usepackage[margin=1in]{geometry}
\ifxetex
  \usepackage[setpagesize=false, % page size defined by xetex
              unicode=false, % unicode breaks when used with xetex
              xetex]{hyperref}
\else
  \usepackage[unicode=true]{hyperref}
\fi
\hypersetup{breaklinks=true,
            bookmarks=true,
            pdfauthor={},
            pdftitle={},
            colorlinks=true,
            citecolor=blue,
            urlcolor=blue,
            linkcolor=magenta,
            pdfborder={0 0 0}}
\urlstyle{same}  % don't use monospace font for urls
% Make links footnotes instead of hotlinks:
\renewcommand{\href}[2]{#2\footnote{\url{#1}}}
\setlength{\parindent}{0pt}
\setlength{\parskip}{6pt plus 2pt minus 1pt}
\setlength{\emergencystretch}{3em}  % prevent overfull lines
\setcounter{secnumdepth}{0}


\begin{document}

We're back to continue on [our
journey](https://blog.jle.im/entries/series/+practical-dependent-types-in-haskell.html)
in using dependent types to write type-safe neural networks! In [Part
1](https://blog.jle.im/entry/practical-dependent-types-in-haskell-1.html), we
wrote things out in normal, untyped Haskell, and looked at red flags and general
design principles that nudged us in the direction of adding dependent types to
our program. We learned to appreciate what dependent types offered in terms of
guiding us in writing our code, helping the compiler check our correctness,
providing a better interface for users, and more.

We also learned how to use singletons to work around some of Haskell's
fundamental limitations to let us "pattern match" on the structure of types, and
how to use typeclasses to generate singletons reflecting the structure of types
we are dealing with.

(If you read [Part
1](https://blog.jle.im/entry/practical-dependent-types-in-haskell-1.html)
*before* the singletons section was re-written to use the
[singletons](https://hackage.haskell.org/package/singletons) library, [here's a
link to the
section](https://blog.jle.im/entry/practical-dependent-types-in-haskell-1.html#singletons-and-induction)
in specific. This tutorial will assume familiarity with what is discussed
there!)

All of what we've dealt with so far has essentially been with types that are
fixed at compile-time. All the networks we've made have had "static" types, with
their sizes in their types indicated directly in the source code. In this post,
we're going to dive into the world of types that *depend* on factors unknown
until runtime, and see how dependent types in a strongly typed language like
Haskell helps us write safer, more correct, and more maintainable code.

This post was written for GHC 8 on stackage snapshot
[nightly-2016-06-28](https://www.stackage.org/nightly-2016-06-28), but should
work with GHC 7.10 for the most part. All of the set-up instructions and caveats
(like the *singletons-2.0.1* bug affecting GHC 7.10 users and the unreleased
*hmatrix* version) are the same as for [part 1's
setup](https://blog.jle.im/entry/practical-dependent-types-in-haskell-1.html#setup).

All of the code in this post is [downloadable as a standalone source
file](https://github.com/mstksg/inCode/tree/master/code-samples/dependent-haskell/NetworkTyped2.hs)
so you can follow along!

A fair disclosure: a lot of this post doesn't actually directly deal with
machine learning or neural networks. Most of it will be learning general
principles for working with dependent types through implementing things you'd
want to do with neural networks. More stuff with ML will come in the next posts!

## Types at Runtime

Recall the type we had for our neural networks:

``` haskell
ghci> :k Network
Network :: Nat -> [Nat] -> Nat -> *
```

They're of the form `Network i hs o`, where `i` is the size of the input vector
it expects, `hs` is the list of hidden layer sizes, and `o` is the size of the
output vector it produces. Something of type `Network 10 '[6, 4] 3` is a network
with 10 input nodes, two input layers of size 6 and 4, and 3 output nodes.

This is great and all, but there's an severe limitation to this: Haskell is a
statically typed language, right? So doesn't this mean that using a network
requires that you know the entire structure of the network at compile-time?

It's conceivable that you might be able to have the input and output sizes known
at compile-time, but it's possible that you don't care or know your hidden layer
structure. You might load it from a configuration file, or have it depend on
user input. You might even want to receive one over a network channel without
knowing what the internal structure is. But can a type really depend on things
that you can't know until runtime?

To illustrate more clearly:

``` haskell
main :: IO ()
main = do
    putStrLn "What hidden layer structure do you want?"
    hs  <- readLn    :: IO [Integer]
    net <- randomNet :: IO (Network 10 ??? 3)   -- what is ???
    -- ...?
```

We *want* to put `hs` there where `???` is, but...`???` has to be a type (of
kind `[Nat]`). `hs` is a value (of type `[Integer]`). It's clear here that the
*type* of our network depends on something we can't write down or decide until
runtime.

### An Existential Crisis

There are two main ways to go about solving this issue in Haskell. We'll look at
both, and then see that they are really actually just two styles of doing the
same thing.

#### Types hiding behind constructors

Now, having the entire structure of your neural network in the type is nice and
all for cool tricks like `randomNet`...but do you *really* want to work with
this directly? After all, from the user's perspective, the user really only ever
needs to know `i` and `o`: What vectors the network *expects* and what vectors
the network *outputs*. In the end, a (feed-forward) Neural Network is really
just a fancy `R i -> R o`.

Remember, the main benefits of having the entire structure in the type was to
help us *implement* our functions more safely, with the compiler's help, and
also for cute return type polymorphism tricks like `randomNet` and `getNet` and
some stronger documentation. The *user* of the network really only benefits from
the second two types of benefits.

One practical downside of having the structure in the type is that you can't
store them in the same list or data structure. A `Network 10 '[5,3] 1` won't
share a list with a `Network 10 '[5,2] 1`, despite having the same
inputs/outputs (and API).

Imagine that we had written a `Network` type that *didn't* have the internal
structure in the type ---

``` haskell
data OpaqueNet i o
```

Recall that our issue earlier was that we had to write `Network i ??? o`, but we
had no idea what to put in for `???`. But, if we worked with an `OpaqueNet i o`,
we wouldn't even care! We wouldn't have to tell GHC what the internal structure
is.

I'd actually argue that `OpaqueNet` might often be the more useful type to offer
to your users (or to use yourself), because it only exposes the types that are
*relevant* to its usage/API. You can store them in a list or MVar ---
`[OpaqueNet 10 3]` and `MVar (OpaqueNet 10 3)`, serialize/deserialize them
without knowing their internal structure in advance
(`loadNet :: FilePath -> IO (OpaqueNet 10 3)`), etc. (if you wanted to load a
`Network`, you would need to know exactly what internal structure was stored, in
advance). Though `Network` is a much easier type to *implement*, `OpaqueNet` is
a often a more ideal type to *use*.

We can implement our vision for `OpaqueNet` as an "existential" wrapper over
`Network`, actually:

``` haskell
-- source: https://github.com/mstksg/inCode/tree/master/code-samples/dependent-haskell/NetworkTyped2.hs#L110-L111

data OpaqueNet :: Nat -> Nat -> * where
    ONet :: Network i hs o -> OpaqueNet i o
```

So, if you have `net :: Network 6 '[10,6,3] 2`, you can create
`ONet net :: OpaqueNet 6 2`. When you use the `ONet` constructor, the structure
of the hidden layers disappears from the type!

We can use the network inside by *pattern matching* on `ONet`:

``` haskell
-- source: https://github.com/mstksg/inCode/tree/master/code-samples/dependent-haskell/NetworkTyped2.hs#L113-L125

runOpaqueNet :: (KnownNat i, KnownNat o)
             => OpaqueNet i o
             -> R i
             -> R o
runOpaqueNet (ONet n) x = runNet n x

numHiddens :: OpaqueNet i o -> Int
numHiddens (ONet n) = go n
  where
    go :: Network i hs o -> Int
    go = \case
        O _      -> 0
        _ :&~ n' -> 1 + go n'
```

With the *ScopedTypeVariables* extension, we can even bring `hs` back into
scope, as in:

``` haskell
case oN of
  ONet (n :: Network i hs o) -> ...
```

This pattern is sometimes called the **dependent pair**, because pattern
matching on `ONet` yields the hidden **existentially quantified** type (`hs`)
and also a value whose type is based on it (`Network i hs o`). It's like `hs`
"paired" with `Network i hs o`. Pattern match on the results to give both the
type (`hs`) *and* the data structure. (To make this more explicit, we could have
implemented it as `ONet :: Sing hs -> Network i hs o -> OpaqueNet i o`)

And here's the key to making this all work: once you pattern match on `ONet`,
you have to handle the `hs` in a *completely polymorphic way*. You're not
allowed to assume anything about `hs`...you have to provide a completely
parametrically polymorphic way of dealing with it!

For example, this function is completely *not* ok:

``` haskell
bad :: OpaqueNet i o -> Network i hs o
bad (ONet n) = n            -- nope, not ok at all.
```

Why not? Well, a type signature like `OpaqueNet i o -> Network i hs o` means
that the *caller* can decide what `hs` can be --- just like
`read :: Read a => String -> a`, where the caller decides what `a` is.

Of course, this isn't the case in the way we've written the function...the
function can only return a *specific* `hs` (namely, the `hs` of the network that
`ONet` hides). The *caller* has to accommodate whatever is inside `ONet`.

#### The Universal and the Existential

We just brushed here on something at the heart of using existential types in
Haskell: the issue of who has the power to decide what the types will be
instantiated as. Most polymorphic functions you work with in Haskell are
"universally qualified". For example, for a function like

``` haskell
map :: (a -> b) -> [a] -> [b]
```

`a` and `b` are universally quantified, which means that the person who *uses*
`map` gets to decide what `a` and `b` are. To be more explicit, that type
signature can be written as:

``` haskell
map :: forall a b. (a -> b) -> [a] -> [b]
```

This means that `map` is defined in a way that will work for *any* `a` and `b`
that the *caller* wants. As a caller, you can request:

``` haskell
map :: (Int -> Bool)    -> [Int]    -> [Bool]
map :: (Double -> Void) -> [Double] -> [Void]
map :: (String -> (Bool -> Char)) -> [String] -> [Bool -> Char]
```

Or anything else!

Consequentially, the function has to be implemented in a way that will work for
*any* `a` and `b`. The function's implementation has the burden of being
flexible enough to handle whatever the caller asks for.

But, for a function like:

``` haskell
foo :: [Int] -> OpaqueNet i o
```

While the caller can choose what `i` and `o` are, the *function* gets to choose
what `hs` (in the hidden `Network i hs o`) is. If I want to *use* the thing that
`foo` returns...then *I* have to be flexible. *I* have the burden of being
flexible enough to handle whatever `hs` the *function* returns.

In summary:

-   For universally quantified types, the *caller* chooses the type being
    instanced, and the *function's implementation* has to accommodate any
    choice.

-   For existentially quantified types, the *function's implementation* chooses
    the type being instanced, and the *caller* has to accommodate any choice.

Indeed, we saw earlier that if we ever wanted to *use* the `Network i hs o`
inside the `OpaqueNet i o`, we were forced to deal with it in a parametrically
polymorphic way. We had to be able to handle *any* `hs` that the `ONet` could
throw at us!

#### A familiar friend

I called `OpaqueNet i o` a "dependent pair" earlier, pairing `hs` with
`Network i hs o`. But there's another common term for it: a **dependent sum**.

People familiar with Haskell might recognize that "sum types" are `Either`-like
types that can be one thing or another. Sum types are one of the first things
you learn about in Haskell --- heck, even `Maybe a` is the sum of `a` and `()`.
Dependent pairs/existential types actually are very similar to `Either`/sum
types, in spirit, and it might help to see the parallel so that you can see that
they're nothing scary, and that the fundamentals/intuition of working with
existential types in Haskell is no different than working with `Either`!

If I had:

``` haskell
foo :: String -> Either Int Bool
```

I have to handle the result for both the case where I get an `Int` and the case
where I get a `Bool`. The *function* gets to pick what type I have to handle
(`Int` or `Bool`), and *I* have to adapt to whatever it returns. Sound familiar?
In fact, you can even imagine that `OpaqueNet i o` as being just a infinite
*Either* over `'[]`, `'[1]`, `'[1,2]`, etc.[^1]

Remember that the basic way of handling an `Either` and figuring out what the
type of the value is inside is through *pattern matching* on it. You can't know
if an `Either Int Bool` contains an `Int` or `Bool` until you pattern match.
But, once you do, all is revealed, and GHC lets you take advantage of knowing
the type.

For `OpaqueNet i o`, it's the same! You don't know the actual type of the
`Network i hs o` it contains until you *pattern match* on the network (This
time, it's a "dependent pattern match"). Once you pattern match on it, all is
revealed...and GHC lets you take advantage of knowing the type!

### Reification

For simplicity, let's re-write `randomNet` the more sensible way --- with the
explicit singleton input style:

``` haskell
-- source: https://github.com/mstksg/inCode/tree/master/code-samples/dependent-haskell/NetworkTyped2.hs#L79-L87

randomNet' :: forall m i hs o. (MonadRandom m, KnownNat i, KnownNat o)
           => Sing hs -> m (Network i hs o)
randomNet' = \case
    SNil            ->     O <$> randomWeights
    SNat `SCons` ss -> (:&~) <$> randomWeights <*> randomNet' ss

randomNet :: forall m i hs o. (MonadRandom m, KnownNat i, SingI hs, KnownNat o)
          => m (Network i hs o)
randomNet = randomNet' sing
```

We use `sing :: SingI hs => Sing hs` to go call the `Sing hs ->`-style function
from the `SingI hs =>` one.[^2]

Now, we still need to somehow get our list of integers to the type level so that
we can create a `Network i hs o` to stuff into our `ONet`. For that, the
*singletons* library offers the necessary tooling. It gives us `SomeSing`, which
is a lot like our `OpaqueNet` above, wrapping the `Sing a` inside an existential
data constructor. `toSing` takes the term-level value (for us, an `[Integer]`)
and returns a `SomeSing` wrapping the type-level value (for us, a `[Nat]`). When
we pattern match on the `SomeSing` constructor, we get `a` in scope!

As of *singletons-2.2* and GHC 8[^3], `SomeSing` is implemented as:

``` haskell
data SomeSing :: * -> * where
    SomeSing :: Sing (a :: k) -> SomeSing k
```

And you have:

``` haskell
foo :: SomeSing Bool
foo = SomeSing STrue

bar :: SomeSing Nat
bar = SomeSing (SNat :: Sing 10)
```

Pattern matching looks like:

``` haskell
main :: IO ()
main = do
    putStrLn "How many cats do you own?"
    c <- readLn :: IO Integer
    case toSing c of
      SomeSing (SNat :: Sing n) -> -- ...
```

Now, inside the case statement branch (the `...`), we have *type* `n :: Nat` in
scope! And by pattern matching on the `SNat` constructor, we also have a
`KnownNat n` instance (As discussed in [previous
part](https://blog.jle.im/entry/practical-dependent-types-in-haskell-1.html#on-typeclasses-and-dictionaries)).

`toSing` works using a simple typeclass mechanism with an associated type whose
job is to connect the types of values with the kinds of their singletons. It
associates `Bool` (the type) with `Bool` (the kind), `Integer` (the type) with
`Nat` (the kind), `[Integer]` (the type) with `[Nat]` (the kind), etc., and it
does it with simple applications of type families (here's a [nice tutorial on
type families](https://ocharles.org.uk/blog/posts/2014-12-12-type-families.html)
courtesy of Oliver Charles, as a refresher). With it, we can convert any normal
value `x` of type `a` to a singleton representing type `x` with kind `a`.

We now have enough to write our `randomONet`:

``` haskell
-- source: https://github.com/mstksg/inCode/tree/master/code-samples/dependent-haskell/NetworkTyped2.hs#L127-L131

randomONet :: (MonadRandom m, KnownNat i, KnownNat o)
           => [Integer]
           -> m (OpaqueNet i o)
randomONet hs = case toSing hs of
                  SomeSing ss -> ONet <$> randomNet' ss
```

This process of bringing a term-level value into the type level is known in
Haskell as **reification**. With this, our original goal is (finally) within
reach:

``` haskell
-- source: https://github.com/mstksg/inCode/tree/master/code-samples/dependent-haskell/NetworkTyped2.hs#L205-L213

main :: IO ()
main = do
    putStrLn "What hidden layer structure do you want?"
    hs <- readLn
    n  <- randomONet hs
    case n of
      ONet (net :: Network 10 hs 3) -> do
        print net
        -- blah blah stuff with our dynamically generated net
```

#### The Boundary

With the power of existentially quantified types (like in `SomeSing`), we
essentially gained the ability to work with types that depend on runtime
results.

In a way, you can consider the `toSing` and the `SomeSing` as our "boundary"
between the "untyped world" and the "typed world". This layer (and the process
of reification) cleanly separates the two.

This boundary can be thought of as a lot like the boundary we talk about between
"pure" functions and values and "impure" (IO, etc.) ones. People say to always
write as much of your program as possible in the "pure" world --- to separate
and pull out as much logic as you can to be pure logic. That's one of the first
things you learn about as a Haskell programmer: how to separate logic that *can*
be pure from logic that is "impure" (IO, etc.), and then finally combine them at
the very end, as late as possible.

It's easy to think that just because the final program is going to "be in IO in
the end anyway", there isn't any point in separating out pure and impure parts
of your program logic. But we know that we gain separation of concerns, the
increased ability to reason with your code and analyze what it does, the
compiler's ability to check what you write, the limitation of implementations,
etc.

You can think of the general philosophy of working with typed/untyped worlds as
being the same thing. You try to write as much of your program as possible in
the "typed" world, like we did in Part 1. Take advantage of the increased
ability to reason with your code, parametric polymorphism helping you *write*
your code, limit your implementations, nab you compiler help, etc. All of those
are benefits of working in the typed world.

Then, write what you absolutely must in your "untyped" world, such as dealing
with values that pop up at runtime like the `[Integer]` above.

Finally, at the very end, *unite* them at the boundary. Pass the control
football from the untyped world to the typed world!

The great part about this all is that GHC and the type system is there at every
step holding your hand, guiding you as you implement your programs and making
sure everything is type-safe and fits together! (This, after all, is why
dependently typed programming with dynamically generated types is *not* the same
thing as "*dynamically* typed programming"!)

### Continuation-Based Existentials

There's another way in Haskell that we work with existential types that can be
more natural to use in a lot of cases. Remember that when we pattern match on an
existential data type, we have to work with the values in the constructor in a
parametrically polymorphic way. For example, if we had:

``` haskell
oNetToFoo :: OpaqueNet i o -> Foo
oNetToFoo (ONet n) = f n
```

`f` has to take a `Network i hs o` but deal with it in a way that works *for
all* `hs`. It can't be written for *only* `'[5]` or *only* `'[6,3]`...it has to
work for *any* `hs`. That's the whole "existential vs. universal quantification"
thing we just talked about.

Well, we could really also just skip the constructor altogether and represent an
existential type as something *taking* the continuation `f` and giving it what
it needs.

``` haskell
-- source: https://github.com/mstksg/inCode/tree/master/code-samples/dependent-haskell/NetworkTyped2.hs#L154-L154

type OpaqueNet' i o r = (forall hs. Network i hs o -> r) -> r
```

"Tell me how you would make an `r` if you had a `Network i hs o` (that works for
any `hs`) and I'll make it for you!"

(This takes advantage of Rank-N types. If you're unfamiliar with them, Gregor
Riegler has a [nice
tutorial](http://sleepomeno.github.io/blog/2014/02/12/Explaining-Haskell-RankNTypes-for-all/)
on the subject.)

*Using* these types is very similar to using the constructor-style ones:

``` haskell
-- source: https://github.com/mstksg/inCode/tree/master/code-samples/dependent-haskell/NetworkTyped2.hs#L159-L176

runOpaqueNet' :: (KnownNat i, KnownNat o)
              => OpaqueNet' i o (R o)
              -> R i
              -> R o
runOpaqueNet' oN x = oN (\n -> runNet n x)
--            :: ((forall hs. Network i hs o -> R o) -> R o)
--            -> R i
--            -> R o

numHiddens' :: OpaqueNet' i o Int -> Int
numHiddens' oN = oN go
  where
    go :: Network i hs o -> Int
    go = \case
        O _      -> 0
        _ :&~ n' -> 1 + go n'
--          :: ((forall hs. Network i hs o -> Int) -> Int)
--          -> Int
```

This "continuation transformation" is formally known as **skolemization**.[^4]

We can "wrap" a `Network i hs o` into an `OpaqueNet' i o r`:

``` haskell
-- source: https://github.com/mstksg/inCode/tree/master/code-samples/dependent-haskell/NetworkTyped2.hs#L156-L157

oNet' :: Network i hs o -> OpaqueNet' i o r
oNet' n = \f -> f n
```

Let's write a version of `randomONet` that returns a continuation-style
existential:

``` haskell
withRandomONet' :: (MonadRandom m, KnownNat i, KnownNat o)
                => [Integer]
                -> (forall hs. Network i hs o -> m r)
                -> m r
--         aka, => [Integer]
--              -> OpaqueNet' i o (m r)
withRandomONet' hs f = case toSing hs of
                         SomeSing ss -> do
                           net <- randomNet' ss
                           f net
```

But, hey, because we're skolemizing everything, let's do it with the skolemized
version of `toSing`/`SomeSing`, `withSomeSing`:

``` haskell
-- a version of `toSing` that returns a skolemized `SomeSing`
withSomeSing :: [Integer]
             -> (forall (hs :: [Nat]). Sing hs -> r)
             -> r
```

Because why not? Skolemize all the things!

``` haskell
-- source: https://github.com/mstksg/inCode/tree/master/code-samples/dependent-haskell/NetworkTyped2.hs#L178-L186

withRandomONet' :: (MonadRandom m, KnownNat i, KnownNat o)
                => [Integer]
                -> (forall hs. Network i hs o -> m r)
                -> m r
--         aka, => [Integer]
--              -> OpaqueNet' i o (m r)
withRandomONet' hs f = withSomeSing hs $ \ss -> do
                         net <- randomNet' ss
                         f net
```

We can use it to do the same things we used the constructor-based existential
for, as well...and, in a way, it actually seems (oddly) more natural.

``` haskell
-- source: https://github.com/mstksg/inCode/tree/master/code-samples/dependent-haskell/NetworkTyped2.hs#L215-L221

main' :: IO ()
main' = do
    putStrLn "What hidden layer structure do you want?"
    hs <- readLn
    withRandomONet' hs $ \(net :: Network 10 hs 3) -> do
      print net
      -- blah blah stuff with our dynamically generated net
```

Like the case statement pattern match represented the lexical "wall"/"boundary"
between the untyped and typed world when using constructor-style existentials,
the `... $ \net -> ...` can be thought of the "wall" for the continuation-style
existentials.

## A Tale of Two Styles

We've just discussed two ways of doing the same thing. Two styles of
representing/working with existential types. The two are equivalent, in that you
can always "convert" between one or the other, but the choice of which one you
use/reach for/offer can make a difference in code clarity. After working with
both styles a lot (sometimes, libraries only offer one style), you start to get
a feel for which one you like more in which situations. In the end, I don't
think there are any hard or fast rules. Just use whichever one you feel is more
readable!

That being said, here are some general Pros and Cons that I've noticed over the
years:

-   Most obviously, continuation-style doesn't require you to define a throwaway
    data type/constructor. While new types are cheap in Haskell, they force your
    users to learn a new set of types and constructors for every single
    existential type you return. If you or the library you're writing
    uses/returns a *lot* of different existentially qualified types, all those
    extra dumb wrappers are a huge hassle.

-   Continuation-style existentials are in general smoother to use than
    constructor-style ones when functions *return* existentials. Especially if
    you intend to immediately use them, continuation-style basically saves you
    an extraneous pattern match.

-   When you have to use several existentials at once, continuation-style is
    much better because each nested existential doesn't force another level of
    indentation:

    ``` haskell
    foo = withSomeSing x $ \sx ->
          withSomeSing y $ \sy ->
          withSomeSing z $ \sz ->
            -- ...
    ```

    vs.

    ``` haskell
    foo = case toSing x of
            SomeSing sx ->
              case toSing y of
                SomeSing sy ->
                  case toSing z of
                    SomeSing sz ->
                      -- ...
    ```

    Every time you nest a case statement, you actually waste *two* levels of
    indentation, which can be annoying even at 2-space indentation. But you
    don't need *any* to nest in the continuation style!

-   If you're working monadically, though, you can take advantage of do notation
    and *ScopedTypeVariables* for a nicer style that doesn't require any nesting
    at all:

    ``` haskell
    main = do
        ONet n1 <- randomONet [7,5,3] :: IO (OpaqueNet 10 1)
        ONet n2 <- randomONet [5,5,5] :: IO (OpaqueNet 10 1)
        ONet n3 <- randomONet [5,4,3] :: IO (OpaqueNet 10 1)
        hs <- readLn
        ONet (n4 :: Network 10 hs 1) <- randomONet hs
        -- ...
    ```

    Which is arguably nicer than

    ``` haskell
    main = withRandomONet' [7,5,3] $ \n1 ->
           withRandomONet' [5,5,5] $ \n2 ->
           withRandomONet' [5,4,3] $ \n3 -> do
             hs <- readLn
             withRandomONet' hs $ \(n4 :: Network 10 hs 1) -> do
               -- ...
    ```

    A lot of libraries return existentials in `Maybe`'s ([base is
    guilty](http://hackage.haskell.org/package/base-4.9.0.0/docs/GHC-TypeLits.html#v:someNatVal)),
    so this trick can be useful for those, too!

    This trick is less useful for functions like `toSing` where things are *not*
    returned in a monad. You could wrap it in Identity, but that's kind of
    silly:

    ``` haskell
    foo = runIdentity $ do
            SomeSing sx <- Identity $ toSing x
            SomeSing sy <- Identity $ toSing y
            SomeSing sz <- Identity $ toSing z
            return $ -- ...
    ```

-   Constructor-style is necessary for writing typeclass instances. You can't
    write a `Show` instance for `(forall hs. Network i hs o -> r) -> r`, but you
    can write one for `OpaqueNet i o`. We'll also be writing `Binary` instances
    later for serialization/deserialization, and it all only works in
    constructor-style.

-   Haskell doesn't allow you to use Rank-N types as arguments to type
    constructors, so you can have `[OpaqueNet i o]`, but *not*
    `[OpaqueNet' i     o r]` or `[(forall hs. Network i hs o -> r) -> r]`. You
    can have `MVar     (OpaqueNet i o)`, but not
    `MVar ((forall hs. Network i hs o -> r) -> r)`. The latter are known as
    *impredicative* types, which are a big no-no in GHC Haskell. Don't even go
    there! The constructor style is necessary in these situations.

    If the type constructor is a Monad, you can get away with a ContT-style
    skolemization, like `(forall hs. Network i hs o -> [r]) -> [r]` and
    `(forall hs. Network i hs o -> IO r) -> IO r`. But this doesn't work for
    `MVar` and other useful type constructors you might want to put `OpaqueNet`
    in.

-   When writing functions that *take* existentials as inputs, the
    constructor-style is arguably more natural. But barely.

    For example, we wrote a function to find the number of hidden layers in a
    network earlier:

    ``` haskell
    numHiddens :: OpaqueNet i o -> Int
    ```

    But the continuation-style version has a slightly messier type:

    ``` haskell
    numHiddens' :: ((forall hs. Network i hs o -> Int) -> Int)
                -> Int
    ```

    Even with with the type synonym, it's still a little awkward:

    ``` haskell
    numHiddens' :: OpaqueNet' i o Int -> Int
    ```

    This is why you'll encounter many more functions *returning*
    continuation-style existentials in libraries than *taking* them, for the
    most part.

These are just general principles, not hard fast rules. This list is nowhere
near exhaustive and reflects my current progress in my journey towards a
dependently typed lifestyle. If you come back in a month, you might see more
things listed here!

All said, I do find myself very happy when I see that a library I'm using offers
*both* styles for me to use. And I've been known to submit PR's to a library to
have it offer one style or another, if it's lacking.

Be judicious. If you're writing a library, don't spam it with too many throwaway
constructors. After a while, you'll begin to intuitively see which style shines
in which situations! (And, in some case, there might not even be a definitive
"better" style to use.)

## Serializing Networks

To drive things home, let's apply what we learned about existential types and
reification to another simple application: serialization.

### Recap on the Binary Library

Serializing networks of *known* size --- whose sizes are statically in their
types --- is pretty straightforward, and its ease is one of the often-quoted
advantages of having sizes in your types.[^5] I'm going to be using the
*[binary](https://hackage.haskell.org/package/binary)* library, which offers a
very standard typeclass-based approach for serializing and deserializing data.
There are a lot of tutorials online (and I even [wrote a small
one](https://blog.jle.im/entry/streaming-huffman-compression-in-haskell-part-2-binary.html)
myself a few years ago), but a very high-level view is that the library offers
monads (`Get`, `Put`) for describing serialization schemes and also a typeclass
used to provide serialization instructions for different types.

In practice, we usually don't write our own instances from scratch. Instead, we
use GHC's generics features to give us instances for free:

``` haskell
-- source: https://github.com/mstksg/inCode/tree/master/code-samples/dependent-haskell/NetworkTyped2.hs#L25-L30

data Weights i o = W { wBiases :: !(R o)
                     , wNodes  :: !(L o i)
                     }
  deriving (Show, Generic)

instance (KnownNat i, KnownNat o) => Binary (Weights i o)
```

For simple types like `Weights`, which simply contain serializable things, the
*binary* library is smart enough to write your instances automatically for you!
This gives us `get` and `put`:

``` haskell
get :: (KnownNat i, KnownNat o)
    => Get (Weights i o)

put :: (KnownNat i, KnownNat o)
    => Weights i o
    -> Put
```

However, for GADTs like `Network`, we have to things manually.

#### Serializing `Network`

Taking advantage of having the entire structure in the type, `put` is simple:

``` haskell
-- source: https://github.com/mstksg/inCode/tree/master/code-samples/dependent-haskell/NetworkTyped2.hs#L89-L94

putNet :: (KnownNat i, KnownNat o)
       => Network i hs o
       -> Put
putNet = \case
    O w     -> put w
    w :&~ n -> put w *> putNet n
```

If it's an `O w`, just serialize the `w`. If it's a `w :&~ net`, serialize the
`w` then the rest of the `net`. Normally, we might have to put a "flag" to tell
what constructor we serializing, so that the deserializer can know what
constructor to deserialize at every step. But for `Network`, we don't have to:

``` haskell
-- source: https://github.com/mstksg/inCode/tree/master/code-samples/dependent-haskell/NetworkTyped2.hs#L96-L101

getNet :: forall i hs o. (KnownNat i, KnownNat o)
       => Sing hs
       -> Get (Network i hs o)
getNet = \case
    SNil            ->     O <$> get
    SNat `SCons` ss -> (:&~) <$> get <*> getNet ss
```

`getNet` doesn't need flags because we already *know* how many `:&~` layers to
expect *just from the type*. If we want to deserialize a
`Network 5 '[10,6,3] 2`, we *know* we want three `(:&~)`'s and one `O`.

`getNet` is written similarly to how we wrote
[`randomNet'`](https://github.com/mstksg/inCode/tree/master/code-samples/dependent-haskell/NetworkTyped2.hs#L79-L83).
We "pattern match" on `hs` (using singletons) to get the constructors we are
expecting to deserialize and just follow what the singleton's structure tells
us.

To write a `Binary` instance for `Network`, we can't have `get` take a `Sing hs`
input --- that'd change the arity/type of the function. We have to switch to
`SingI`-style had have their `Binary` instances require a `SingI hs` constraint.

``` haskell
-- source: https://github.com/mstksg/inCode/tree/master/code-samples/dependent-haskell/NetworkTyped2.hs#L103-L105

instance (KnownNat i, SingI hs, KnownNat o) => Binary (Network i hs o) where
    put = putNet
    get = getNet sing
```

### Serializating `OpaqueNet`

It's arguably much more useful to serialize/deserialize `OpaqueNet`s. Between
different iterations of your program, you might have the same inputs/outputs,
but want to try out different internal structures. You'd want to store them and
access them uniformly, or send them over a network without requiring the
receiver to know the internal structure beforehand. Remember, you can't even
*load* a `Network i hs o` without knowing its complete structure!

Because the complete structure of the network is not in the type, we need to
encode it as a flag in the binary serialization so that the deserializer will
know what constructors to expect and deserialize. We can write a simple function
to get the `[Integer]` of a network's structure:

``` haskell
-- source: https://github.com/mstksg/inCode/tree/master/code-samples/dependent-haskell/NetworkTyped2.hs#L72-L77

hiddenStruct :: Network i hs o -> [Integer]
hiddenStruct = \case
    O _    -> []
    _ :&~ (n' :: Network h hs' o)
           -> natVal (Proxy @h)
            : hiddenStruct n'
```

Recall that `natVal :: KnownNat n => Proxy n -> Integer` returns the value-level
`Integer` corresponding to the type-level `n :: Nat`. (I'm also using GHC 8's
fancy *TypeApplications* syntax, and `Proxy @h` is the same as
`Proxy :: Proxy h`).

`natVal` and `hiddenStruct` are kind of interesting --- they take type-level
information (`n`, `hs`) and turns them into term-level values (`Integer`s,
`[Integer]`s). They are the opposites of our reification functions (like
`toSing`). Going from the "type level" to the "value level" is known in Haskell
as **reflection**, and is the dual concept of reification. (The *singletons*
library offers reflectors for all of its singletons, as `fromSing`.)

That's all we need!

``` haskell
-- source: https://github.com/mstksg/inCode/tree/master/code-samples/dependent-haskell/NetworkTyped2.hs#L133-L138

putONet :: (KnownNat i, KnownNat o)
        => OpaqueNet i o
        -> Put
putONet (ONet net) = do
    put (hiddenStruct net)
    putNet net
```

"Put the structure (as a binary flag), and then put the network itself."

Now, to deserialize, we want to *load* the list of `Integer`s and reify it back
to the type level to know what type of network we're expecting to load:

``` haskell
-- source: https://github.com/mstksg/inCode/tree/master/code-samples/dependent-haskell/NetworkTyped2.hs#L140-L145

getONet :: (KnownNat i, KnownNat o)
        => Get (OpaqueNet i o)
getONet = do
    hs <- get
    withSomeSing hs $ \ss ->
      ONet <$> getNet ss
```

We load our flag, reify it, and once we're back in the typed land again, we can
do our normal business. Isn't it nice that GHC is also there at every step to
make sure we make the transition safely?

Our final instance:

``` haskell
-- source: https://github.com/mstksg/inCode/tree/master/code-samples/dependent-haskell/NetworkTyped2.hs#L147-L149

instance (KnownNat i, KnownNat o) => Binary (OpaqueNet i o) where
    put = putONet
    get = getONet
```

And, of course, we used the constructor-style existential this whole time
instead of the continuation-style one because we can't directly write typeclass
instances for the latter.

## An Existence For All

We've learned about how to "cross" from the untyped world to the typed world and
bring about contexts involving types that can depend on runtime factors. To me,
this is really the point where dependently typed programming starts --- when you
start having to work with types that depend on run-time factors.

We've already been able to reap a lot of benefits. All of the type safety we
discovered in the last part is now available to us in a fully dynamic world, as
well. We also learned the advantages of *separating* the typed world from the
untyped world and how the compiler helps us make the transition safely.

But really, this is all just the *start* of dependently typed programming. This
is where things *really* start to get fun.

Stepping into this new world can be disorienting at first. There's a lot of
unexpected things that come up when we start working more with these fancy new
types. We have to deal with types coming from different sources, convince the
type system about their properties and relationships between them, and deal with
a whole bunch of other concerns that just don't happen when you program only at
the value level. But don't worry! Like all things, it will more naturally with
practice.

Now that we have existential types and run-time types out of the way, come back
for the next post in the series, where we start to have the *real* fun! :D

#### Exercises

Here are some fun exercises you can try, if you want to test your understanding!
Links are to the solutions.

1.  Implement
    [`putONet'`](https://github.com/mstksg/inCode/tree/master/code-samples/dependent-haskell/NetworkTyped2.hs#L188-L193)
    and
    [`getONet'`](https://github.com/mstksg/inCode/tree/master/code-samples/dependent-haskell/NetworkTyped2.hs#L195-L203)
    using the continuation-style existentials, instead.

2.  Work with an existential wrapper over the *entire* network structure (inputs
    and outputs, too):

    ``` haskell
    data SomeNet where
        SNet :: (KnownNat i, KnownNat o)
             => Network i hs o
             -> SomeNet
    ```

    (We need the `KnownNat` constraints because of type erasure, to recover the
    original input/output dimensions back once we pattern match)

    And write:

    -   A function to [convert `SomeNet`s to
        `OpaqueNet`s](https://github.com/mstksg/inCode/tree/master/code-samples/dependent-haskell/NetworkTyped2.hs#L231-L234).
        Return the `OpaqueNet` with existentially quantified `i` and `o` in
        continuation-style. (You can write a data type to return it in
        constructor-style, too, for funsies.)

    -   [`randomSNet`](https://github.com/mstksg/inCode/tree/master/code-samples/dependent-haskell/NetworkTyped2.hs#L236-L245),
        returning `m SomeNet`.

    -   While you're at it, write it to return [a random continuation-style
        `SomeNet`,
        too](https://github.com/mstksg/inCode/tree/master/code-samples/dependent-haskell/NetworkTyped2.hs#L247-L258)!
        (See the type of
        [`withRandomONet'`](https://github.com/mstksg/inCode/tree/master/code-samples/dependent-haskell/NetworkTyped2.hs#L178-L186)
        for reference on how to write the type)

    -   The [binary
        instance](https://github.com/mstksg/inCode/tree/master/code-samples/dependent-haskell/NetworkTyped2.hs#L260-L274)
        for `SomeNet`.

        Hint: Remember `natVal :: KnownNat n => Proxy n -> Integer`!

    Hint: Remember that `toSomeSing` also works for `Integer`s, to get `Sing`s
    for `Nat`s, too!

# Signoff

Hi, thanks for reading! You can reach me via email at <justin@jle.im>, or at
twitter at [\@mstk](https://twitter.com/mstk)! This post and all others are
published under the [CC-BY-NC-ND
3.0](https://creativecommons.org/licenses/by-nc-nd/3.0/) license. Corrections
and edits via pull request are welcome and encouraged at [the source
repository](https://github.com/mstksg/inCode).

If you feel inclined, or this post was particularly helpful for you, why not
consider [supporting me on Patreon](https://www.patreon.com/justinle/overview),
or a [BTC donation](bitcoin:3D7rmAYgbDnp4gp4rf22THsGt74fNucPDU)? :)

[^1]: A bit of a stretch, because the set of all `[Nat]`s is non-enumerable and
    uncountable, but hopefully you get the picture!

[^2]: Recall that I recommend (personally, and subjectively) a style where your
    external API functions and typeclass instances are implemented in
    `SingI a =>` style, and your internal ones in `Sing a ->` style. This lets
    all of your internal functions fit together more nicely (`Sing a ->` style
    tends to be easier to write in, especially if you stay in it the entire
    time, because `Sing`s are normal first-class values, unlike those global and
    magical typeclasses) while at the same time removing the burden of calling
    with explicit singletons from people using the functionality externally.

[^3]: In older versions of singletons, before GHC 8 and *TypeInType*, we had to
    implement it using "kind proxies". Don't worry if you're following along in
    7.10; the basic usage of `SomeSing` is essentially identical either way.

[^4]: Skolemization is probably one of the coolest words you'll encounter when
    learning/using Haskell, and sometimes just knowing that you're "skolemizing"
    something makes you feel cooler. Thank you [Thoralf
    Skolem](https://en.wikipedia.org/wiki/Thoralf_Skolem). If you ever see a
    "rigid, skolem" error in GHC, you can thank him for that too! He is also the
    inspiration behind my decision to name my first-born son Thoralf. (My second
    son's name will be Curry)

[^5]: It even lets you write `Storable` instances!


\end{document}

\end{document}
