\documentclass[]{article}
\usepackage{lmodern}
\usepackage{amssymb,amsmath}
\usepackage{ifxetex,ifluatex}
\usepackage{fixltx2e} % provides \textsubscript
\ifnum 0\ifxetex 1\fi\ifluatex 1\fi=0 % if pdftex
  \usepackage[T1]{fontenc}
  \usepackage[utf8]{inputenc}
\else % if luatex or xelatex
  \ifxetex
    \usepackage{mathspec}
    \usepackage{xltxtra,xunicode}
  \else
    \usepackage{fontspec}
  \fi
  \defaultfontfeatures{Mapping=tex-text,Scale=MatchLowercase}
  \newcommand{\euro}{€}
\fi
% use upquote if available, for straight quotes in verbatim environments
\IfFileExists{upquote.sty}{\usepackage{upquote}}{}
% use microtype if available
\IfFileExists{microtype.sty}{\usepackage{microtype}}{}
\usepackage[margin=1in]{geometry}
\ifxetex
  \usepackage[setpagesize=false, % page size defined by xetex
              unicode=false, % unicode breaks when used with xetex
              xetex]{hyperref}
\else
  \usepackage[unicode=true]{hyperref}
\fi
\hypersetup{breaklinks=true,
            bookmarks=true,
            pdfauthor={Justin Le},
            pdftitle={Subtyping in Haskell},
            colorlinks=true,
            citecolor=blue,
            urlcolor=blue,
            linkcolor=magenta,
            pdfborder={0 0 0}}
\urlstyle{same}  % don't use monospace font for urls
% Make links footnotes instead of hotlinks:
\renewcommand{\href}[2]{#2\footnote{\url{#1}}}
\setlength{\parindent}{0pt}
\setlength{\parskip}{6pt plus 2pt minus 1pt}
\setlength{\emergencystretch}{3em}  % prevent overfull lines
\setcounter{secnumdepth}{0}

\title{Subtyping in Haskell}
\author{Justin Le}

\begin{document}
\maketitle

\emph{Originally posted on
\textbf{\href{https://blog.jle.im/entry/subtyping-in-haskell.html}{in Code}}.}

\documentclass[]{}
\usepackage{lmodern}
\usepackage{amssymb,amsmath}
\usepackage{ifxetex,ifluatex}
\usepackage{fixltx2e} % provides \textsubscript
\ifnum 0\ifxetex 1\fi\ifluatex 1\fi=0 % if pdftex
  \usepackage[T1]{fontenc}
  \usepackage[utf8]{inputenc}
\else % if luatex or xelatex
  \ifxetex
    \usepackage{mathspec}
    \usepackage{xltxtra,xunicode}
  \else
    \usepackage{fontspec}
  \fi
  \defaultfontfeatures{Mapping=tex-text,Scale=MatchLowercase}
  \newcommand{\euro}{€}
\fi
% use upquote if available, for straight quotes in verbatim environments
\IfFileExists{upquote.sty}{\usepackage{upquote}}{}
% use microtype if available
\IfFileExists{microtype.sty}{\usepackage{microtype}}{}
\usepackage[margin=1in]{geometry}
\ifxetex
  \usepackage[setpagesize=false, % page size defined by xetex
              unicode=false, % unicode breaks when used with xetex
              xetex]{hyperref}
\else
  \usepackage[unicode=true]{hyperref}
\fi
\hypersetup{breaklinks=true,
            bookmarks=true,
            pdfauthor={},
            pdftitle={},
            colorlinks=true,
            citecolor=blue,
            urlcolor=blue,
            linkcolor=magenta,
            pdfborder={0 0 0}}
\urlstyle{same}  % don't use monospace font for urls
% Make links footnotes instead of hotlinks:
\renewcommand{\href}[2]{#2\footnote{\url{#1}}}
\setlength{\parindent}{0pt}
\setlength{\parskip}{6pt plus 2pt minus 1pt}
\setlength{\emergencystretch}{3em}  % prevent overfull lines
\setcounter{secnumdepth}{0}


\begin{document}

It is often said that Haskell does not have subtyping. While it is indeed true
that Haskell doesn't have *ad-hoc* subtyping, you can achieve something similar
with Haskell, RankN types, and some key choice of data types or typeclasses.
And, in many situations, you can build programs around it!

As a simple example, let's redesign the API of the *Control.Monad.Trans.State*
module from the *transformers* package. Here, they define

``` haskell
data StateT s m a = StateT (s -> m (a, s))

runStateT  :: StateT s m a -> s -> m (a, s)
evalStateT :: StateT s m a -> s -> m a
execStateT :: StateT s m a -> s -> m s

get    ::                  StateT s m s
put    :: s             -> StateT s m ()
modify :: (s -> s)      -> StateT s m ()
state  :: (s -> (a, s)) -> StateT s m a

type State s = StateT s Identity

runState  :: State s a  -> s -> (a, s)
evalState :: State s a -> s -> a
execState :: State s a -> s -> s
```

# Signoff

Hi, thanks for reading! You can reach me via email at <justin@jle.im>, or at
twitter at [\@mstk](https://twitter.com/mstk)! This post and all others are
published under the [CC-BY-NC-ND
3.0](https://creativecommons.org/licenses/by-nc-nd/3.0/) license. Corrections
and edits via pull request are welcome and encouraged at [the source
repository](https://github.com/mstksg/inCode).

If you feel inclined, or this post was particularly helpful for you, why not
consider [supporting me on Patreon](https://www.patreon.com/justinle/overview),
or a [BTC donation](bitcoin:3D7rmAYgbDnp4gp4rf22THsGt74fNucPDU)? :)

\end{document}

\end{document}
