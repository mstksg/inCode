\documentclass[]{article}
\usepackage{lmodern}
\usepackage{amssymb,amsmath}
\usepackage{ifxetex,ifluatex}
\usepackage{fixltx2e} % provides \textsubscript
\ifnum 0\ifxetex 1\fi\ifluatex 1\fi=0 % if pdftex
  \usepackage[T1]{fontenc}
  \usepackage[utf8]{inputenc}
\else % if luatex or xelatex
  \ifxetex
    \usepackage{mathspec}
    \usepackage{xltxtra,xunicode}
  \else
    \usepackage{fontspec}
  \fi
  \defaultfontfeatures{Mapping=tex-text,Scale=MatchLowercase}
  \newcommand{\euro}{€}
\fi
% use upquote if available, for straight quotes in verbatim environments
\IfFileExists{upquote.sty}{\usepackage{upquote}}{}
% use microtype if available
\IfFileExists{microtype.sty}{\usepackage{microtype}}{}
\usepackage[margin=1in]{geometry}
\usepackage{color}
\usepackage{fancyvrb}
\newcommand{\VerbBar}{|}
\newcommand{\VERB}{\Verb[commandchars=\\\{\}]}
\DefineVerbatimEnvironment{Highlighting}{Verbatim}{commandchars=\\\{\}}
% Add ',fontsize=\small' for more characters per line
\newenvironment{Shaded}{}{}
\newcommand{\AlertTok}[1]{\textcolor[rgb]{1.00,0.00,0.00}{\textbf{#1}}}
\newcommand{\AnnotationTok}[1]{\textcolor[rgb]{0.38,0.63,0.69}{\textbf{\textit{#1}}}}
\newcommand{\AttributeTok}[1]{\textcolor[rgb]{0.49,0.56,0.16}{#1}}
\newcommand{\BaseNTok}[1]{\textcolor[rgb]{0.25,0.63,0.44}{#1}}
\newcommand{\BuiltInTok}[1]{\textcolor[rgb]{0.00,0.50,0.00}{#1}}
\newcommand{\CharTok}[1]{\textcolor[rgb]{0.25,0.44,0.63}{#1}}
\newcommand{\CommentTok}[1]{\textcolor[rgb]{0.38,0.63,0.69}{\textit{#1}}}
\newcommand{\CommentVarTok}[1]{\textcolor[rgb]{0.38,0.63,0.69}{\textbf{\textit{#1}}}}
\newcommand{\ConstantTok}[1]{\textcolor[rgb]{0.53,0.00,0.00}{#1}}
\newcommand{\ControlFlowTok}[1]{\textcolor[rgb]{0.00,0.44,0.13}{\textbf{#1}}}
\newcommand{\DataTypeTok}[1]{\textcolor[rgb]{0.56,0.13,0.00}{#1}}
\newcommand{\DecValTok}[1]{\textcolor[rgb]{0.25,0.63,0.44}{#1}}
\newcommand{\DocumentationTok}[1]{\textcolor[rgb]{0.73,0.13,0.13}{\textit{#1}}}
\newcommand{\ErrorTok}[1]{\textcolor[rgb]{1.00,0.00,0.00}{\textbf{#1}}}
\newcommand{\ExtensionTok}[1]{#1}
\newcommand{\FloatTok}[1]{\textcolor[rgb]{0.25,0.63,0.44}{#1}}
\newcommand{\FunctionTok}[1]{\textcolor[rgb]{0.02,0.16,0.49}{#1}}
\newcommand{\ImportTok}[1]{\textcolor[rgb]{0.00,0.50,0.00}{\textbf{#1}}}
\newcommand{\InformationTok}[1]{\textcolor[rgb]{0.38,0.63,0.69}{\textbf{\textit{#1}}}}
\newcommand{\KeywordTok}[1]{\textcolor[rgb]{0.00,0.44,0.13}{\textbf{#1}}}
\newcommand{\NormalTok}[1]{#1}
\newcommand{\OperatorTok}[1]{\textcolor[rgb]{0.40,0.40,0.40}{#1}}
\newcommand{\OtherTok}[1]{\textcolor[rgb]{0.00,0.44,0.13}{#1}}
\newcommand{\PreprocessorTok}[1]{\textcolor[rgb]{0.74,0.48,0.00}{#1}}
\newcommand{\RegionMarkerTok}[1]{#1}
\newcommand{\SpecialCharTok}[1]{\textcolor[rgb]{0.25,0.44,0.63}{#1}}
\newcommand{\SpecialStringTok}[1]{\textcolor[rgb]{0.73,0.40,0.53}{#1}}
\newcommand{\StringTok}[1]{\textcolor[rgb]{0.25,0.44,0.63}{#1}}
\newcommand{\VariableTok}[1]{\textcolor[rgb]{0.10,0.09,0.49}{#1}}
\newcommand{\VerbatimStringTok}[1]{\textcolor[rgb]{0.25,0.44,0.63}{#1}}
\newcommand{\WarningTok}[1]{\textcolor[rgb]{0.38,0.63,0.69}{\textbf{\textit{#1}}}}
\usepackage{graphicx}
\makeatletter
\def\maxwidth{\ifdim\Gin@nat@width>\linewidth\linewidth\else\Gin@nat@width\fi}
\def\maxheight{\ifdim\Gin@nat@height>\textheight\textheight\else\Gin@nat@height\fi}
\makeatother
% Scale images if necessary, so that they will not overflow the page
% margins by default, and it is still possible to overwrite the defaults
% using explicit options in \includegraphics[width, height, ...]{}
\setkeys{Gin}{width=\maxwidth,height=\maxheight,keepaspectratio}
\ifxetex
  \usepackage[setpagesize=false, % page size defined by xetex
              unicode=false, % unicode breaks when used with xetex
              xetex]{hyperref}
\else
  \usepackage[unicode=true]{hyperref}
\fi
\hypersetup{breaklinks=true,
            bookmarks=true,
            pdfauthor={Justin Le},
            pdftitle={``Five Point Haskell'' Part 3: Limited Atonement},
            colorlinks=true,
            citecolor=blue,
            urlcolor=blue,
            linkcolor=magenta,
            pdfborder={0 0 0}}
\urlstyle{same}  % don't use monospace font for urls
% Make links footnotes instead of hotlinks:
\renewcommand{\href}[2]{#2\footnote{\url{#1}}}
\setlength{\parindent}{0pt}
\setlength{\parskip}{6pt plus 2pt minus 1pt}
\setlength{\emergencystretch}{3em}  % prevent overfull lines
\setcounter{secnumdepth}{0}

\title{``Five Point Haskell'' Part 3: Limited Atonement}
\author{Justin Le}

\begin{document}
\maketitle

\emph{Originally posted on
\textbf{\href{https://blog.jle.im/entry/five-point-haskell-part-3-limited-atonement.html}{in
Code}}.}

\documentclass[]{}
\usepackage{lmodern}
\usepackage{amssymb,amsmath}
\usepackage{ifxetex,ifluatex}
\usepackage{fixltx2e} % provides \textsubscript
\ifnum 0\ifxetex 1\fi\ifluatex 1\fi=0 % if pdftex
  \usepackage[T1]{fontenc}
  \usepackage[utf8]{inputenc}
\else % if luatex or xelatex
  \ifxetex
    \usepackage{mathspec}
    \usepackage{xltxtra,xunicode}
  \else
    \usepackage{fontspec}
  \fi
  \defaultfontfeatures{Mapping=tex-text,Scale=MatchLowercase}
  \newcommand{\euro}{€}
\fi
% use upquote if available, for straight quotes in verbatim environments
\IfFileExists{upquote.sty}{\usepackage{upquote}}{}
% use microtype if available
\IfFileExists{microtype.sty}{\usepackage{microtype}}{}
\usepackage[margin=1in]{geometry}
\ifxetex
  \usepackage[setpagesize=false, % page size defined by xetex
              unicode=false, % unicode breaks when used with xetex
              xetex]{hyperref}
\else
  \usepackage[unicode=true]{hyperref}
\fi
\hypersetup{breaklinks=true,
            bookmarks=true,
            pdfauthor={},
            pdftitle={},
            colorlinks=true,
            citecolor=blue,
            urlcolor=blue,
            linkcolor=magenta,
            pdfborder={0 0 0}}
\urlstyle{same}  % don't use monospace font for urls
% Make links footnotes instead of hotlinks:
\renewcommand{\href}[2]{#2\footnote{\url{#1}}}
\setlength{\parindent}{0pt}
\setlength{\parskip}{6pt plus 2pt minus 1pt}
\setlength{\emergencystretch}{3em}  % prevent overfull lines
\setcounter{secnumdepth}{0}

\textbackslash begin\{document\}

Hi! We're in Part 3 of
\emph{\href{https://blog.jle.im/entries/series/+five-point-haskell.html}{Five-Point
Haskell}}! I've been trying to build a framework to describe how I write
maintainable and effective Haskell and also highlight anti-principles I reject,
and this the third part of that framework.

In
\href{https://blog.jle.im/entry/five-point-haskell-part-1-total-depravity.html}{Total
Depravity}, we talked about how the failure of mental context windows is always
only a matter of time, and how to use types defensively in that light. In
\href{https://blog.jle.im/entry/five-point-haskell-part-2-unconditional-election.html}{Unconditional
Election}, we talked about how mathematical properties ensure the behavior of
our instantiations regardless of any foreseen merit of the implementations.

Of course, real code doesn't just live in an ivory tower of perfect, clean pure
abstractions. In some sense, it's where all of the theory meets practice and
finally becomes useful. The goal of Haskell isn't universal purity: it's about
the correct balance between the impure and pure, and how the existence of one
enriches the other. This is \textbf{Limited Atonement}.

\begin{quote}
Limited Atonement: Every domain has a clean line between what is pure and what
is not. Declarations of purity are perfectly effective. Impurity is not a
failure, but an intentional boundary that gives purity its very meaning.

Therefore, in every domain, find that beautiful line. Be intentional: let how
you treat the impure give meaning to the limited yet definite purity.
\end{quote}

\section{The Case for Purity}\label{the-case-for-purity}

Before we can start diving into the nuances, let's actually remind ourselves why
it's important to put purity in the type system in the first place. After all,
this isn't exactly a widely accepted ``value'' in the wider programming world.

If you've encountered Haskell in popular culture, it might have been
\href{https://xkcd.com/1312/}{in this xkcd comic}:

\begin{figure}
\centering
\pandocbounded{\includegraphics[keepaspectratio]{/img/entries/five-point-haskell/xkcd-1312-haskell.png}}
\caption{xkcd --- Haskell}
\end{figure}

There's some truth to it, but there's also some truth to the quote that
``Haskell is the best imperative language'': we understand that in order to
conquer the beast, we must first \emph{name} it. If your code has no way to tell
the difference between code that does IO and code that does not, you really have
no hope in reasoning with \emph{anything}.

If you have some familiarity with the topic (or have read my
\href{https://blog.jle.im/entry/first-class-statements.html}{past}
\href{https://blog.jle.im/entry/the-compromiseless-reconciliation-of-i-o-and-purity.html}{posts}),
\emph{technically} we can look at IO in Haskell as still ``pure'' in the sense
that we are purely constructing an IO action. That is,
\texttt{putStrLn\ ::\ String\ -\textgreater{}\ IO\ ()} purely returns the same
\texttt{IO\ ()} action every single time.
\texttt{readIORef\ ::\ IORef\ a\ -\textgreater{}\ IO\ a} returns the same
\texttt{IO\ a} action every time, even if the \texttt{a} itself might be
different every time you execute it.

While \emph{technically} correct, this framing is somewhat facetious (an
accusation I fully accept) and doesn't really get to the point of what impurity
really is. It's not about pure or impure \emph{functions}
(\texttt{-\textgreater{}}), it's about pure or impure \emph{logic}, and being
able to represent that in the type system.

Remember in
\href{https://blog.jle.im/entry/five-point-haskell-part-2-unconditional-election.html}{Part
2}, we compared and contrasted similar type signatures in Haskell and Java

\begin{Shaded}
\begin{Highlighting}[]
\OtherTok{foo ::} \KeywordTok{forall}\NormalTok{ a}\OperatorTok{.}\NormalTok{ a }\OtherTok{{-}\textgreater{}}\NormalTok{ a}
\end{Highlighting}
\end{Shaded}

\begin{Shaded}
\begin{Highlighting}[]
\DataTypeTok{static} \OperatorTok{\textless{}}\NormalTok{T}\OperatorTok{\textgreater{}}\NormalTok{ T }\FunctionTok{foo}\OperatorTok{(}\NormalTok{T x}\OperatorTok{)}
\end{Highlighting}
\end{Shaded}

In order to even \emph{start talking} about all the cool things about parametric
polymorphism, etc., we had to add the caveat ``Assume no IO, assume no mutation
of arguments''.

Once we put them side-by-side:

\begin{Shaded}
\begin{Highlighting}[]
\OtherTok{mkString ::} \DataTypeTok{Int} \OtherTok{{-}\textgreater{}} \DataTypeTok{String}
\OtherTok{mkStringIO ::} \DataTypeTok{Int} \OtherTok{{-}\textgreater{}} \DataTypeTok{IO} \DataTypeTok{String}
\end{Highlighting}
\end{Shaded}

Instantly we understand that the first function must:

\begin{enumerate}
\def\labelenumi{\arabic{enumi}.}
\tightlist
\item
  Always return the same string for any \texttt{Int}
\item
  Not perform any IO during the computation
\item
  Cannot modify the environment
\end{enumerate}

whereas the second has no such guarantee. We have no such expressiveness in

\begin{Shaded}
\begin{Highlighting}[]
\DataTypeTok{static} \BuiltInTok{String} \FunctionTok{mkString}\OperatorTok{(}\BuiltInTok{Integer}\NormalTok{ x}\OperatorTok{)}
\end{Highlighting}
\end{Shaded}

which could presumably call out to the network, use a random generator to
produce the string, or modify some internal counter every time it is called.

On one hand, you might end up needlessly introducing overhead and complexity in
your language, akin to the
\href{https://journal.stuffwithstuff.com/2015/02/01/what-color-is-your-function/}{colored
function debate}. Sneaking in IO in a deeply nested function requires completely
propagating this to every caller up to the very top.

But on the other hand\ldots maybe it \emph{should}? If some deeply nested
sub-function call in your function does IO, this seems exactly the situation you
\emph{would} want such a change to be flagged by the compiler. The fact that
impurity must bubble up is a feature, not a bug.

One day in traffic school, I remember a student asking the teacher a question,
``If there is nobody at the 4-way stop sign intersection, do I still have to
stop?''

The teacher answered, ``My child, if you don't see the other person at a 4-way
stop, that is exactly the situation you \emph{should} be stopping in.''

And, so what? Why do we care in the first place, if the purpose of our function
is to do IO anyway?

Well:

\begin{itemize}
\item
  You should care about unmarked IO if you care about global variables. IO is
  the ultimate unrestricted global variable: any \texttt{IO} action can freely
  modify the process environment or the file system: it can call \texttt{setEnv}
  or \texttt{lookupEnv} against global variables that can be access across your
  entire process.
\item
  You should care about IO if you care about memoization: we can safely cache
  \texttt{mkString\ 42} and \texttt{mkString\ 67} forever, without worrying that
  every time we access them they should have been different numbers.
\item
  You should care about unmarked IO if you care about safe refactoring and
  common subexpression elimination. Consider populating data with
  \texttt{mkString}:

\begin{Shaded}
\begin{Highlighting}[]
\OtherTok{mkUser ::} \DataTypeTok{Int} \OtherTok{{-}\textgreater{}} \DataTypeTok{User}
\NormalTok{mkUser n }\OtherTok{=} \DataTypeTok{User}\NormalTok{ \{ name }\OtherTok{=}\NormalTok{ mkString n, ident }\OtherTok{=}\NormalTok{ mkString n \}}
\end{Highlighting}
\end{Shaded}

  This \emph{should} be the same as:

\begin{Shaded}
\begin{Highlighting}[]
\OtherTok{mkUser ::} \DataTypeTok{Int} \OtherTok{{-}\textgreater{}} \DataTypeTok{User}
\NormalTok{mkUser n }\OtherTok{=} \DataTypeTok{User}\NormalTok{ \{ name }\OtherTok{=}\NormalTok{ nameAndIdent, ident }\OtherTok{=}\NormalTok{ nameAndIdent \}}
  \KeywordTok{where}
\NormalTok{    nameAndIdent }\OtherTok{=}\NormalTok{ mkString n}
\end{Highlighting}
\end{Shaded}

  In many cases, this could be more performant and save space. You might only
  have to allocate only a single heap object instead of multiple. However, this
  is \emph{not} a valid program optimization if \texttt{mkString} could do IO!
  Calling it twice could give different results, or affect the environment in
  different ways, than calling it once!
\item
  You should care about unmarked IO if you care about laziness. Granted, this
  requires a desire for laziness in the first place (so, Ocaml users, you're off
  the hook). But if you can imagine:

\begin{Shaded}
\begin{Highlighting}[]
\OtherTok{mkUser ::} \DataTypeTok{Int} \OtherTok{{-}\textgreater{}} \DataTypeTok{User}
\NormalTok{mkUser n }\OtherTok{=} \DataTypeTok{User}\NormalTok{ \{ name }\OtherTok{=}\NormalTok{ b, ident }\OtherTok{=}\NormalTok{ a \}}
  \KeywordTok{where}
\NormalTok{    a }\OtherTok{=}\NormalTok{ mkString n}
\NormalTok{    b }\OtherTok{=}\NormalTok{ mkString (n }\OperatorTok{+} \DecValTok{1}\NormalTok{)}
\NormalTok{    c }\OtherTok{=}\NormalTok{ mkString (n }\OperatorTok{+} \DecValTok{3}\NormalTok{)}
\end{Highlighting}
\end{Shaded}

  What\ldots what order are those \texttt{mkString}s called, in a lazy language?
  If there was unmarked IO, the order \emph{does} matter. If there wasn't, they
  \emph{can't}. And \texttt{c} could not even be freely discarded if there was
  unmarked IO.
\item
  You should care about IO if you care about concurrency. Imagine parallel
  mapping over multiple numbers:

\begin{Shaded}
\begin{Highlighting}[]
\OtherTok{myStrings ::}\NormalTok{ [}\DataTypeTok{String}\NormalTok{]}
\NormalTok{myStrings }\OtherTok{=}\NormalTok{ parMap mkString [}\DecValTok{1}\OperatorTok{..}\DecValTok{100}\NormalTok{]}
\end{Highlighting}
\end{Shaded}

  If \texttt{mkString} had unmarked IO and accessed locks or mutexes, this could
  easily be a race condition. But it doesn't, so we can guarantee no race
  conditions. We can also be assured that the order in which we schedule our
  threads will have no affect on the result.
\item
  You should care about unmarked IO if you care about testing. Because it states
  no external dependency, you can test \texttt{mkString} without requiring any
  sandboxing, isolation, or extra interleaving interactions with other
  functions.
\end{itemize}

\section{The Real Worlds}\label{the-real-worlds}

\subsection{IO}\label{io}

\subsection{ST}\label{st}

\subsection{STM}\label{stm}

\subsection{Scoped Environments}\label{scoped-environments}

\section{The Simulated Worlds}\label{the-simulated-worlds}

\subsection{Pure Short Circuiting}\label{pure-short-circuiting}

\subsection{Reader}\label{reader}

\subsection{State}\label{state}

\section{The Bespoke World}\label{the-bespoke-world}

\subsection{Free}\label{free}

\subsection{Tagless Final}\label{tagless-final}

\subsection{Extensible Effects}\label{extensible-effects}

\subsection{Wait, did I just write a Monad
Tutorial?}\label{wait-did-i-just-write-a-monad-tutorial}

\section{Signoff}\label{signoff}

Hi, thanks for reading! You can reach me via email at
\href{mailto:justin@jle.im}{\nolinkurl{justin@jle.im}}, or at twitter at
\href{https://twitter.com/mstk}{@mstk}! This post and all others are published
under the \href{https://creativecommons.org/licenses/by-nc-nd/3.0/}{CC-BY-NC-ND
3.0} license. Corrections and edits via pull request are welcome and encouraged
at \href{https://github.com/mstksg/inCode}{the source repository}.

If you feel inclined, or this post was particularly helpful for you, why not
consider \href{https://www.patreon.com/justinle/overview}{supporting me on
Patreon}, or a \href{bitcoin:3D7rmAYgbDnp4gp4rf22THsGt74fNucPDU}{BTC donation}?
:)

\textbackslash end\{document\}

\end{document}
