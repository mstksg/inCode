\documentclass[]{article}
\usepackage{lmodern}
\usepackage{amssymb,amsmath}
\usepackage{ifxetex,ifluatex}
\usepackage{fixltx2e} % provides \textsubscript
\ifnum 0\ifxetex 1\fi\ifluatex 1\fi=0 % if pdftex
  \usepackage[T1]{fontenc}
  \usepackage[utf8]{inputenc}
\else % if luatex or xelatex
  \ifxetex
    \usepackage{mathspec}
    \usepackage{xltxtra,xunicode}
  \else
    \usepackage{fontspec}
  \fi
  \defaultfontfeatures{Mapping=tex-text,Scale=MatchLowercase}
  \newcommand{\euro}{€}
\fi
% use upquote if available, for straight quotes in verbatim environments
\IfFileExists{upquote.sty}{\usepackage{upquote}}{}
% use microtype if available
\IfFileExists{microtype.sty}{\usepackage{microtype}}{}
\usepackage[margin=1in]{geometry}
\usepackage{color}
\usepackage{fancyvrb}
\newcommand{\VerbBar}{|}
\newcommand{\VERB}{\Verb[commandchars=\\\{\}]}
\DefineVerbatimEnvironment{Highlighting}{Verbatim}{commandchars=\\\{\}}
% Add ',fontsize=\small' for more characters per line
\newenvironment{Shaded}{}{}
\newcommand{\AlertTok}[1]{\textcolor[rgb]{1.00,0.00,0.00}{\textbf{#1}}}
\newcommand{\AnnotationTok}[1]{\textcolor[rgb]{0.38,0.63,0.69}{\textbf{\textit{#1}}}}
\newcommand{\AttributeTok}[1]{\textcolor[rgb]{0.49,0.56,0.16}{#1}}
\newcommand{\BaseNTok}[1]{\textcolor[rgb]{0.25,0.63,0.44}{#1}}
\newcommand{\BuiltInTok}[1]{\textcolor[rgb]{0.00,0.50,0.00}{#1}}
\newcommand{\CharTok}[1]{\textcolor[rgb]{0.25,0.44,0.63}{#1}}
\newcommand{\CommentTok}[1]{\textcolor[rgb]{0.38,0.63,0.69}{\textit{#1}}}
\newcommand{\CommentVarTok}[1]{\textcolor[rgb]{0.38,0.63,0.69}{\textbf{\textit{#1}}}}
\newcommand{\ConstantTok}[1]{\textcolor[rgb]{0.53,0.00,0.00}{#1}}
\newcommand{\ControlFlowTok}[1]{\textcolor[rgb]{0.00,0.44,0.13}{\textbf{#1}}}
\newcommand{\DataTypeTok}[1]{\textcolor[rgb]{0.56,0.13,0.00}{#1}}
\newcommand{\DecValTok}[1]{\textcolor[rgb]{0.25,0.63,0.44}{#1}}
\newcommand{\DocumentationTok}[1]{\textcolor[rgb]{0.73,0.13,0.13}{\textit{#1}}}
\newcommand{\ErrorTok}[1]{\textcolor[rgb]{1.00,0.00,0.00}{\textbf{#1}}}
\newcommand{\ExtensionTok}[1]{#1}
\newcommand{\FloatTok}[1]{\textcolor[rgb]{0.25,0.63,0.44}{#1}}
\newcommand{\FunctionTok}[1]{\textcolor[rgb]{0.02,0.16,0.49}{#1}}
\newcommand{\ImportTok}[1]{\textcolor[rgb]{0.00,0.50,0.00}{\textbf{#1}}}
\newcommand{\InformationTok}[1]{\textcolor[rgb]{0.38,0.63,0.69}{\textbf{\textit{#1}}}}
\newcommand{\KeywordTok}[1]{\textcolor[rgb]{0.00,0.44,0.13}{\textbf{#1}}}
\newcommand{\NormalTok}[1]{#1}
\newcommand{\OperatorTok}[1]{\textcolor[rgb]{0.40,0.40,0.40}{#1}}
\newcommand{\OtherTok}[1]{\textcolor[rgb]{0.00,0.44,0.13}{#1}}
\newcommand{\PreprocessorTok}[1]{\textcolor[rgb]{0.74,0.48,0.00}{#1}}
\newcommand{\RegionMarkerTok}[1]{#1}
\newcommand{\SpecialCharTok}[1]{\textcolor[rgb]{0.25,0.44,0.63}{#1}}
\newcommand{\SpecialStringTok}[1]{\textcolor[rgb]{0.73,0.40,0.53}{#1}}
\newcommand{\StringTok}[1]{\textcolor[rgb]{0.25,0.44,0.63}{#1}}
\newcommand{\VariableTok}[1]{\textcolor[rgb]{0.10,0.09,0.49}{#1}}
\newcommand{\VerbatimStringTok}[1]{\textcolor[rgb]{0.25,0.44,0.63}{#1}}
\newcommand{\WarningTok}[1]{\textcolor[rgb]{0.38,0.63,0.69}{\textbf{\textit{#1}}}}
\ifxetex
  \usepackage[setpagesize=false, % page size defined by xetex
              unicode=false, % unicode breaks when used with xetex
              xetex]{hyperref}
\else
  \usepackage[unicode=true]{hyperref}
\fi
\hypersetup{breaklinks=true,
            bookmarks=true,
            pdfauthor={},
            pdftitle={Introduction to Singletons (Part 3)},
            colorlinks=true,
            citecolor=blue,
            urlcolor=blue,
            linkcolor=magenta,
            pdfborder={0 0 0}}
\urlstyle{same}  % don't use monospace font for urls
% Make links footnotes instead of hotlinks:
\renewcommand{\href}[2]{#2\footnote{\url{#1}}}
\setlength{\parindent}{0pt}
\setlength{\parskip}{6pt plus 2pt minus 1pt}
\setlength{\emergencystretch}{3em}  % prevent overfull lines
\setcounter{secnumdepth}{0}

\title{Introduction to Singletons (Part 3)}

\begin{document}
\maketitle

\% Justin Le \% October 1, 2018

\emph{Originally posted on
\textbf{\href{https://blog.jle.im/entry/introduction-to-singletons-3.html}{in
Code}}.}

Welcome back! This article is part 3 of our journey through the \emph{singleton
design pattern}, and the great
\emph{\href{http://hackage.haskell.org/package/singletons}{singletons}} library!

This post will be a continuation of
\href{https://blog.jle.im/entry/introduction-to-singletons-1.html}{Part 1} and
\href{https://blog.jle.im/entry/introduction-to-singletons-2.html}{Part 2}, so
if you haven't read those first, now would be a good time to pause and do so
(and also try to complete the exercises). Today we will be expanding on the
ideas in those posts by working with more complex ways to \emph{restrict
functions} based on types. Like the previous posts, we will start by writing
things ``by hand'', and then jumping into the singletons library and seeing how
the framework gives you tools to work with these ideas in a smoother way.

The first half of today's post will introduce a new application and design
pattern that the usage of singletons greatly enhances. The second part of
today's post deals directly with the lifting of functions to the type level,
which is made practical by the usage of singletons and the \emph{singletons}
library.

Code in this post is built on \emph{GHC 8.6.1} with the
\emph{\href{https://www.stackage.org/nightly-2018-09-29}{nightly-2018-09-29}}
snapshot (so, \emph{singletons-2.5}). However, unless noted, all of the code
should still work with \emph{GHC 8.4} and \emph{singletons-2.4}. Again, you can
download the source for this file
\href{https://github.com/mstksg/inCode/tree/master/code-samples/singletons/Door3.hs}{here},
and, if \emph{stack} is installed, you can drop into a ghci session with all of
the bindings in scope executing it:

\begin{Shaded}
\begin{Highlighting}[]
\ExtensionTok{$}\NormalTok{ ./Door3.hs}
\end{Highlighting}
\end{Shaded}

\section{Review}\label{review}

In the first post we looked at the \texttt{Door} type, indexed with a phantom
type of kind \texttt{DoorState}.

\begin{Shaded}
\begin{Highlighting}[]
\CommentTok{{-}{-} source: https://github.com/mstksg/inCode/tree/master/code{-}samples/singletons/Door3.hs\#L29{-}L38}

\OperatorTok{$}\NormalTok{(singletons [d|}
\NormalTok{  data DoorState = Opened | Closed | Locked}
\NormalTok{    deriving (Show, Eq)}
\NormalTok{  |])}

\KeywordTok{data} \DataTypeTok{Door}\OtherTok{ ::} \DataTypeTok{DoorState} \OtherTok{{-}\textgreater{}} \DataTypeTok{Type} \KeywordTok{where}
    \DataTypeTok{UnsafeMkDoor}\OtherTok{ ::}\NormalTok{ \{}\OtherTok{ doorMaterial ::} \DataTypeTok{String}\NormalTok{ \} }\OtherTok{{-}\textgreater{}} \DataTypeTok{Door}\NormalTok{ s}

\OtherTok{mkDoor ::} \DataTypeTok{Sing}\NormalTok{ s }\OtherTok{{-}\textgreater{}} \DataTypeTok{String} \OtherTok{{-}\textgreater{}} \DataTypeTok{Door}\NormalTok{ s}
\NormalTok{mkDoor \_ }\OtherTok{=} \DataTypeTok{UnsafeMkDoor}
\end{Highlighting}
\end{Shaded}

This gives us (at least) three distinct types
\texttt{Door\ \textquotesingle{}Opened},
\texttt{Door\ \textquotesingle{}Closed}, and
\texttt{Door\ \textquotesingle{}Locked}, which can be used to represent opened,
closed, and locked doors, respectively. We talked in previous posts about how we
can use this for for a lot of enat tings, including enforcing type-safety,
talking about how inputs relate to outputs, and uniting functions polymorphic on
all door states.

Then we talked about situations where we want to ``not care'' about the door
status in the type system, or when we want to return a door with a state that is
not known statically, and must be determined dynamically at runtime. After going
through many ``analogous'' and equivalent type, we arrived at the existential
wrapper \texttt{SomeDoor}:

\begin{Shaded}
\begin{Highlighting}[]
\CommentTok{{-}{-} source: https://github.com/mstksg/inCode/tree/master/code{-}samples/singletons/Door3.hs\#L40{-}L45}

\KeywordTok{data} \DataTypeTok{SomeDoor}\OtherTok{ ::} \DataTypeTok{Type} \KeywordTok{where}
    \DataTypeTok{MkSomeDoor}\OtherTok{ ::} \DataTypeTok{Sing}\NormalTok{ s }\OtherTok{{-}\textgreater{}} \DataTypeTok{Door}\NormalTok{ s }\OtherTok{{-}\textgreater{}} \DataTypeTok{SomeDoor}

\OtherTok{mkSomeDoor ::} \DataTypeTok{DoorState} \OtherTok{{-}\textgreater{}} \DataTypeTok{String} \OtherTok{{-}\textgreater{}} \DataTypeTok{SomeDoor}
\NormalTok{mkSomeDoor ds mat }\OtherTok{=}\NormalTok{ withSomeSing ds }\OperatorTok{$}\NormalTok{ \textbackslash{}dsSing }\OtherTok{{-}\textgreater{}}
    \DataTypeTok{MkSomeDoor}\NormalTok{ dsSing (mkDoor dsSing mat)}
\end{Highlighting}
\end{Shaded}

(We must be careful to pack the \texttt{Sing\ s} with the \texttt{Door\ s}, so
that we can pattern match at runtime to determine what the original \texttt{s}
was.)

For the rest of this post, \texttt{SomeDoor} will essentially be used as a
stand-in for a \texttt{Door\ s} that we do not know the state (the \texttt{s})
of until runtime, because to use a \texttt{SomeDoor}, we pattern-match at
runtime. In general you'll encounter types at runtime in a variety of different
situations (discussed more deeply in
\href{https://blog.jle.im/entry/introduction-to-singletons-2.html}{Part 2}), but
\texttt{SomeDoor} is a nice nugget that we can examine to demonstrate more
general points.

\section{A Need for More Expressive
Restrictions}\label{a-need-for-more-expressive-restrictions}

Let's write a function that ``knocks'' on a door in IO:

\begin{Shaded}
\begin{Highlighting}[]
\OtherTok{knock ::} \DataTypeTok{Door}\NormalTok{ s }\OtherTok{{-}\textgreater{}} \DataTypeTok{IO}\NormalTok{ ()}
\NormalTok{knock d }\OtherTok{=} \FunctionTok{putStrLn} \OperatorTok{$} \StringTok{"Knock knock on "} \OperatorTok{++}\NormalTok{ doorMaterial d }\OperatorTok{++} \StringTok{" door!"}
\end{Highlighting}
\end{Shaded}

Hm. This doesn't feel right. We can't knock on an opened door..can we? Is there
a way we can restrict this function to only work on \emph{non-opened} doors? Or,
more generally, is there a way to be more expressive in the manner in which we
can restrict functions?

There are a couple of ways of doing this --- we're going to look at two possible
ways that singletons and the \emph{singletons} library help with. Both of these
methods allow us to write dependently typed functions that are ``type-safe'' in
more expressive ways than before.

Note that we'll be exploring ways that are ``generalizable'' --- to different
types of restrictions that might be more complicated than just ``cannot be
\texttt{\textquotesingle{}Opened}''.

\section{Dependently Typed Proofs}\label{dependently-typed-proofs}

To look at our first way of tackling this restriction problem, we're going to
explore a fun \emph{new application} of singletons and DataKinds.

This new application is the usage of the dependently-typed ``proof'' to prove
that an operation is legal. \emph{Proofs} (in the dependently
typed/constructivist/Curry-Howard sense) are witnesses to some type-level
\emph{predicate} or proposition.

A \textbf{value-level predicate} in Haskell is (generally) a function of type
\texttt{a\ -\textgreater{}\ Bool}. Given a value of type \texttt{a}, if the
function returns \texttt{True}, then the predicate is satisfied. If it returns
\texttt{False}, it is not.

A \textbf{type-level predicate} is (generally) a type constructor of kind
\texttt{k\ -\textgreater{}\ Type}. Given a type of kind \texttt{k}, if \emph{a
value exists of that type} (or, if a value can be constructed), then the
predicate is satisfied. If no value exists, it is not. That value, if it exists,
is called a \emph{witness} or a \emph{proof}.\footnote{All of this is ignoring
  the ``bottom'' value that is an occupant of every type in Haskell. We can use
  bottom to subvert pretty much all proofs in Haskell, unfortunately, so the
  discussion from this point forward assumes we are talking about a subset of
  Haskell where all values are non-bottom and all functions are total.}

We can define a predicate
\texttt{Knockable\ ::\ DoorState\ -\textgreater{}\ Type} as a GADT that only has
values if given \texttt{\textquotesingle{}Closed} and
\texttt{\textquotesingle{}Locked}, but not \texttt{\textquotesingle{}Opened}:

\begin{Shaded}
\begin{Highlighting}[]
\CommentTok{{-}{-} source: https://github.com/mstksg/inCode/tree/master/code{-}samples/singletons/Door3.hs\#L47{-}L49}

\KeywordTok{data} \DataTypeTok{Knockable}\OtherTok{ ::} \DataTypeTok{DoorState} \OtherTok{{-}\textgreater{}} \DataTypeTok{Type} \KeywordTok{where}
    \DataTypeTok{KnockClosed}\OtherTok{ ::} \DataTypeTok{Knockable} \DataTypeTok{\textquotesingle{}Closed}
    \DataTypeTok{KnockLocked}\OtherTok{ ::} \DataTypeTok{Knockable} \DataTypeTok{\textquotesingle{}Locked}
\end{Highlighting}
\end{Shaded}

Now, we have a value of type \texttt{Knockable\ \textquotesingle{}Closed} and
\texttt{Knockable\ \textquotesingle{}Locked} (\texttt{KnockClosed} and
\texttt{KnockLocked}, respectively), but no value of type
\texttt{Knockable\ \textquotesingle{}Opened}. How can we use this?

Well, we can make a version of \texttt{knock} that requires a proof that
\texttt{s} is \texttt{Knockable}:

\begin{Shaded}
\begin{Highlighting}[]
\CommentTok{{-}{-} source: https://github.com/mstksg/inCode/tree/master/code{-}samples/singletons/Door3.hs\#L51{-}L52}

\OtherTok{knock ::} \DataTypeTok{Knockable}\NormalTok{ s }\OtherTok{{-}\textgreater{}} \DataTypeTok{Door}\NormalTok{ s }\OtherTok{{-}\textgreater{}} \DataTypeTok{IO}\NormalTok{ ()}
\NormalTok{knock \_ d }\OtherTok{=} \FunctionTok{putStrLn} \OperatorTok{$} \StringTok{"Knock knock on "} \OperatorTok{++}\NormalTok{ doorMaterial d }\OperatorTok{++} \StringTok{" door!"}
\end{Highlighting}
\end{Shaded}

\texttt{knock} can now only be called with \texttt{Closed} and \texttt{Locked}
doors --- do you see why? There is no way to call \texttt{knock} with
\texttt{s\ \textasciitilde{}\ \textquotesingle{}Opened}\ldots because there is
no way to pass a value of \texttt{Knockable\ \textquotesingle{}Opened}. No such
value exists! There's no compiler error because it's ``not even wrong''!

\begin{Shaded}
\begin{Highlighting}[]
\NormalTok{ghci}\OperatorTok{\textgreater{}}\NormalTok{ knock }\DataTypeTok{KnockClosed}\NormalTok{ (mkDoor }\DataTypeTok{SClosed} \StringTok{"Birch"}\NormalTok{)}
\DataTypeTok{Knock}\NormalTok{ knock on }\DataTypeTok{Birch}\NormalTok{ door}\OperatorTok{!}
\end{Highlighting}
\end{Shaded}

This works well if we want to do things at compile-time

\subsection{Let the compiler prove it for
you}\label{let-the-compiler-prove-it-for-you}

We can even make it more seamless to use by auto-generating proofs at
compile-time, with a general class like \texttt{Auto}:

\begin{Shaded}
\begin{Highlighting}[]
\CommentTok{{-}{-} source: https://github.com/mstksg/inCode/tree/master/code{-}samples/singletons/Door3.hs\#L54{-}L61}

\KeywordTok{class} \DataTypeTok{Proved}\NormalTok{ p a }\KeywordTok{where}
\OtherTok{    auto ::}\NormalTok{ p a}

\KeywordTok{instance} \DataTypeTok{Proved} \DataTypeTok{Knockable} \DataTypeTok{\textquotesingle{}Closed} \KeywordTok{where}
\NormalTok{    auto }\OtherTok{=} \DataTypeTok{KnockClosed}

\KeywordTok{instance} \DataTypeTok{Proved} \DataTypeTok{Knockable} \DataTypeTok{\textquotesingle{}Locked} \KeywordTok{where}
\NormalTok{    auto }\OtherTok{=} \DataTypeTok{KnockLocked}
\end{Highlighting}
\end{Shaded}

\begin{Shaded}
\begin{Highlighting}[]
\NormalTok{ghci}\OperatorTok{\textgreater{}}\NormalTok{ knock auto (mkDoor }\DataTypeTok{SClosed} \StringTok{"Acacia"}\NormalTok{)}
\DataTypeTok{Knock}\NormalTok{ knock on }\DataTypeTok{Acacia}\NormalTok{ door}\OperatorTok{!}

\NormalTok{ghci}\OperatorTok{\textgreater{}}\NormalTok{ knock auto (mkDoor }\DataTypeTok{SOpened} \StringTok{"Jungle"}\NormalTok{)}
\DataTypeTok{COMPILER} \DataTypeTok{ERROR}\OperatorTok{!!} \DataTypeTok{COMPILER} \DataTypeTok{ERROR}\OperatorTok{!!}
\end{Highlighting}
\end{Shaded}

Such a typeclass exists in libraries like
\emph{\href{http://hackage.haskell.org/package/type-combinators}{type-combinators}}
(called \texttt{Known}) and
\href{http://hackage.haskell.org/package/decidable}{decidable} (called
\texttt{Auto}). In dependently typed languages like Idris, \texttt{auto} is
actually a built-in language keyword that does this automatically!

\subsection{Decidable Predicates}\label{decidable-predicates}

However, all of this only works if you know what \texttt{s} is at compile-time.
What if you don't? What if you are retrieving \texttt{s} at runtime (like from a
\texttt{SomeDoor} or \texttt{withSomeSing}), or you are forced to handle all
possible \texttt{s}s?

To do this, we're going to take advantage of a property of some predicates
called ``decidability''. We say that a predicate is \emph{decidable} if, for any
input type, we can say whether or not the predicate is satisfiable.

We say that a predicate \texttt{P} in Haskell is \emph{decidable} if we can
always prove, for any input, if the predicate holds or does not hold.
Concretely, it means that we can write a total function:

\begin{Shaded}
\begin{Highlighting}[]
\NormalTok{decidePred}
\OtherTok{    ::} \DataTypeTok{Sing}\NormalTok{ x               }\CommentTok{{-}{-} \^{} given a type}
    \OtherTok{{-}\textgreater{}} \DataTypeTok{Decision}\NormalTok{ (}\DataTypeTok{P}\NormalTok{ x)       }\CommentTok{{-}{-} \^{} return a decision}
\end{Highlighting}
\end{Shaded}

Where:

\begin{Shaded}
\begin{Highlighting}[]
\KeywordTok{data} \DataTypeTok{Decision}\NormalTok{ a }\OtherTok{=} \DataTypeTok{Proved}\NormalTok{ a                  }\CommentTok{{-}{-} \^{} a value of a exists}
                \OperatorTok{|} \DataTypeTok{Disproved}\NormalTok{ (}\DataTypeTok{Refuted}\NormalTok{ a)     }\CommentTok{{-}{-} \^{} a value of a cannot exist}

\CommentTok{{-}{-} | The data type with no values}
\KeywordTok{data} \DataTypeTok{Void}

\CommentTok{{-}{-} | \textquotesingle{}a\textquotesingle{} cannot exist.  Commonly also called \textasciigrave{}Not\textasciigrave{}}
\KeywordTok{type} \DataTypeTok{Refuted}\NormalTok{ a }\OtherTok{=}\NormalTok{ a }\OtherTok{{-}\textgreater{}} \DataTypeTok{Void}
\end{Highlighting}
\end{Shaded}

\texttt{Decision\ a} is like a \texttt{Maybe\ a}, except instead of
\texttt{Nothing}, we include a proof that the predicate is \emph{not} true.

The \texttt{a\ -\textgreater{}\ Void} idiom (often called \texttt{Not\ a}, or
\texttt{Refuted\ a}) is type we use in Haskell and other languages to represent
the fact that it is impossible to construct a value of type \texttt{a}. That's
because if you could, then you could give it to an
\texttt{a\ -\textgreater{}\ Void} to get a value of type \texttt{Void}, which is
impossible to have. So, if a possible function \texttt{a\ -\textgreater{}\ Void}
exists, it necessarily means that a value of type \texttt{a} cannot exist.

It's a lot to handle all at once, so let's look at an example. Is
\texttt{Knockable} a decidable predicate? Yes!

We need to write a function:

\begin{Shaded}
\begin{Highlighting}[]
\OtherTok{isKnockable ::} \DataTypeTok{Sing}\NormalTok{ s }\OtherTok{{-}\textgreater{}} \DataTypeTok{Decision}\NormalTok{ (}\DataTypeTok{Knockable}\NormalTok{ s)}
\end{Highlighting}
\end{Shaded}

I recommend taking a moment and trying to implement this yourself. Remember to
enable \texttt{-Werror=incomplete-patterns}\footnote{Thanks to
  \href{https://www.reddit.com/r/haskell/comments/9kkbci/introduction_to_singletons_part_3_dependently/e70nc7k/}{Darwin226
  on reddit} for this tip!} (or at least \texttt{-Wall}) to make sure you're
handling all potential pattern matching cases.

Are you ready? Here's a solution:

\begin{Shaded}
\begin{Highlighting}[]
\CommentTok{{-}{-} source: https://github.com/mstksg/inCode/tree/master/code{-}samples/singletons/Door3.hs\#L63{-}L67}

\OtherTok{isKnockable ::} \DataTypeTok{Sing}\NormalTok{ s }\OtherTok{{-}\textgreater{}} \DataTypeTok{Decision}\NormalTok{ (}\DataTypeTok{Knockable}\NormalTok{ s)}
\NormalTok{isKnockable }\OtherTok{=}\NormalTok{ \textbackslash{}}\KeywordTok{case}
    \DataTypeTok{SOpened} \OtherTok{{-}\textgreater{}} \DataTypeTok{Disproved} \OperatorTok{$}\NormalTok{ \textbackslash{}}\KeywordTok{case}\NormalTok{ \{\}    }\CommentTok{{-}{-} s \textasciitilde{} \textquotesingle{}Opened}
    \DataTypeTok{SClosed} \OtherTok{{-}\textgreater{}} \DataTypeTok{Proved} \DataTypeTok{KnockClosed}      \CommentTok{{-}{-} s \textasciitilde{} \textquotesingle{}Closed}
    \DataTypeTok{SLocked} \OtherTok{{-}\textgreater{}} \DataTypeTok{Proved} \DataTypeTok{KnockLocked}      \CommentTok{{-}{-} s \textasciitilde{} \textquotesingle{}Locked}
\end{Highlighting}
\end{Shaded}

This definition should seem pretty straightforward for the \texttt{SClosed} and
\texttt{SLocked} branches.
\texttt{isKnockable\ SClosed\ ::\ Decision\ (Knockable\ \textquotesingle{}Closed)},
we give \texttt{Proved\ KnockClosed}, which gives us just that!

However,
\texttt{isKnockable\ SOpened\ ::\ Decision\ (Knockable\ \textquotesingle{}Opened)}.
We can't use \texttt{Proved\ ::\ a\ -\textgreater{}\ Decision\ a}, because no
such value of type \texttt{Knockable\ \textquotesingle{}Opened} exists. So, we
have to say that we \emph{disprove} it: we have to prove to GHC that no such
type could possibly exist. We do this by providing a function of type
\texttt{Refuted\ (Knockable\ \textquotesingle{}Opened)}, or type
\texttt{Knockable\ \textquotesingle{}Opened\ -\textgreater{}\ Void}.

We can write it like this:

\begin{Shaded}
\begin{Highlighting}[]
\CommentTok{{-}{-} source: https://github.com/mstksg/inCode/tree/master/code{-}samples/singletons/Door3.hs\#L69{-}L70}

\OtherTok{disproveOpened ::} \DataTypeTok{Knockable} \DataTypeTok{\textquotesingle{}Opened} \OtherTok{{-}\textgreater{}} \DataTypeTok{Void}
\NormalTok{disproveOpened k }\OtherTok{=} \KeywordTok{case}\NormalTok{ k }\KeywordTok{of}\NormalTok{ \{\}             }\CommentTok{{-}{-} empty pattern match}
\end{Highlighting}
\end{Shaded}

And we're good to go!

How does this work?

Well, remember, we have to pattern match on the possible inputs. However, we
can't use any of the ``legal'' patterns:

\begin{Shaded}
\begin{Highlighting}[]
\OtherTok{disproveOpened ::} \DataTypeTok{Knockable} \DataTypeTok{\textquotesingle{}Opened} \OtherTok{{-}\textgreater{}} \DataTypeTok{Void}
\NormalTok{disproveOpened k }\OtherTok{=} \KeywordTok{case}\NormalTok{ k }\KeywordTok{of}
    \DataTypeTok{KnockClosed} \OtherTok{{-}\textgreater{}} \OperatorTok{...}    \CommentTok{{-}{-} not a valid pattern, since it\textquotesingle{}s \textasciigrave{}Knockable \textquotesingle{}Closed\textasciigrave{}}
    \DataTypeTok{KnockLocked} \OtherTok{{-}\textgreater{}} \OperatorTok{...}    \CommentTok{{-}{-} not a valid pattern, since it\textquotesingle{}s \textasciigrave{}Knockable \textquotesingle{}Locked\textasciigrave{}}
\end{Highlighting}
\end{Shaded}

If you include either of those patterns, GHC will complain. So, there is no
valid pattern to match on\ldots{} so
\texttt{disproveOpened\ =\ \textbackslash{}case\ \{\}} is enough to write the
function \texttt{Knockable\ \textquotesingle{}Opened\ -\textgreater{}\ Void},
since there is no constructor for a value of type
\texttt{Knockable\ \textquotesingle{}Opened} to match on. This only works
because \texttt{disproveOpened} is a \textbf{complete pattern match}, and
therefore total.

We can use this decision function, finally, to handle an arbitrary \texttt{Door}
whose status we not know until runtime:

\begin{Shaded}
\begin{Highlighting}[]
\CommentTok{{-}{-} source: https://github.com/mstksg/inCode/tree/master/code{-}samples/singletons/Door3.hs\#L72{-}L77}

\NormalTok{knockSomeDoor}
\OtherTok{    ::} \DataTypeTok{SomeDoor}     \CommentTok{{-}{-} \^{} status not known until you pattern match at runtime}
    \OtherTok{{-}\textgreater{}} \DataTypeTok{IO}\NormalTok{ ()}
\NormalTok{knockSomeDoor (}\DataTypeTok{MkSomeDoor}\NormalTok{ s d) }\OtherTok{=} \KeywordTok{case}\NormalTok{ isKnockable s }\KeywordTok{of}
    \DataTypeTok{Proved}\NormalTok{ k    }\OtherTok{{-}\textgreater{}}\NormalTok{ knock k d}
    \DataTypeTok{Disproved}\NormalTok{ \_ }\OtherTok{{-}\textgreater{}} \FunctionTok{putStrLn} \StringTok{"No knocking allowed!"}
\end{Highlighting}
\end{Shaded}

While arguably less useful than one provable predicates, a typeclass for
decidable predicates is also possible; the aptly-named
\href{http://hackage.haskell.org/package/decidable}{decidable} package offers
such a typeclass, called \texttt{Decidable}!

\subsection{Decision}\label{decision}

The definition of the \texttt{Decision} data type might be surprising if you're
first seeing it. You want to prove something\ldots so why would you care about
the case where it's ``not true''? Why not just have something like
\texttt{Maybe}, where you have
\texttt{data\ Decision\ a\ =\ Proved\ a\ \textbar{}\ Disproved}?

In other words, why do we care about proving both true or false, when it looks
like we only ever use the true situation? After all, we ignore the
\texttt{Refuted\ (Knockable\ s)} in our implementation of
\texttt{knockSomeDoor}.

One answer is that we \emph{do} use the contents of \texttt{Disproved} in
practice. In \texttt{knockSomeDoor}, we matched on \texttt{Disproved\ \_} and
threw away the counter-proof\ldots however, as we see later in the exercises,
there are situations where the contents of \texttt{Disproved} are used.

However, a deeper answer to me is that it keeps the author of the function
accountable. You can't just say ``this predicate isn't true''\ldots you have to
\emph{earn} it. And often, the act of trying to earn your disproof (or not being
able to) helps you iron out bad assumptions you've made.

For example, if we used \texttt{Maybe} instead of \texttt{Decision}, we could
write:

\begin{Shaded}
\begin{Highlighting}[]
\OtherTok{isKnockable ::} \DataTypeTok{Sing}\NormalTok{ s }\OtherTok{{-}\textgreater{}} \DataTypeTok{Maybe}\NormalTok{ s}
\NormalTok{isKnockable }\OtherTok{=}\NormalTok{ \textbackslash{}}\KeywordTok{case}
    \DataTypeTok{SOpened} \OtherTok{{-}\textgreater{}} \DataTypeTok{Nothing}
    \DataTypeTok{SClosed} \OtherTok{{-}\textgreater{}} \DataTypeTok{Nothing}
    \DataTypeTok{SLocked} \OtherTok{{-}\textgreater{}} \DataTypeTok{Just} \DataTypeTok{KnockLocked}
\end{Highlighting}
\end{Shaded}

We might falsely claim that \texttt{SClosed} is not knockable. So, if the user
of our bad \texttt{isKnockable} gets \texttt{Nothing}, they don't know if their
input is not knockable or knockable\ldots they know \emph{nothing} about the
knockability status of \texttt{\textquotesingle{}Opened} or
\texttt{\textquotesingle{}Closed}.

However, we can't write this bad implementation with \texttt{Decision}:

\begin{Shaded}
\begin{Highlighting}[]
\OtherTok{isKnockable ::} \DataTypeTok{Sing}\NormalTok{ s }\OtherTok{{-}\textgreater{}} \DataTypeTok{Decision}\NormalTok{ s}
\NormalTok{isKnockable }\OtherTok{=}\NormalTok{ \textbackslash{}}\KeywordTok{case}
    \DataTypeTok{SOpened} \OtherTok{{-}\textgreater{}} \DataTypeTok{Disproved} \OperatorTok{$}\NormalTok{ \textbackslash{}}\KeywordTok{case}\NormalTok{ \{\}}
    \DataTypeTok{SClosed} \OtherTok{{-}\textgreater{}} \DataTypeTok{Disproved} \OperatorTok{$} \CommentTok{{-}{-} ????}
    \DataTypeTok{SLocked} \OtherTok{{-}\textgreater{}} \DataTypeTok{Proved} \DataTypeTok{KnockLocked}
\end{Highlighting}
\end{Shaded}

There is no valid thing you can put in the \texttt{????}! That's because you
need to write a function of type
\texttt{Knocked\ \textquotesingle{}Closed\ -\textgreater{}\ Void}\ldots but no
such (total or non-partial) function exists. We can't write
\texttt{\textbackslash{}case\ \{\}}, because that's an incomplete pattern match
--- it's missing a match on the \texttt{KnockClosed} pattern.

Note also that this is why it's very important to always have
\texttt{-Werror=incomplete-patterns} (or at least \texttt{-Wall} --- warn all)
on when writing dependently typed proofs, to ensure that GHC warns you when your
pattern matches are incomplete and you know your proof is invalid.

In the examples, we see two more non-trivial examples of decision functions
(\texttt{and\ p\ q} and \texttt{or\ p\ q}) that are impossible to implement
incorrectly, due to the structure of the predicates.

\subsection{Perspective on Proofs}\label{perspective-on-proofs}

We just briefly touched on a very simple version of a dependently typed proof,
and how to ``prove'' properties.

If you have heard things about dependently typed programming before, you might
have heard that a lot of it involves ``proving properties about your programs''
and ``forcing you to provide proofs for all of your actions''. The idea of a
``proof'' might seem a bit scary and ``mathematical'' to those coming from a
software development world.

However, as we just saw, working with proofs and decisions of proofs can be as
simple as a couple lines of GADTs and dependent pattern matches.

So, when we see a function like:

\begin{Shaded}
\begin{Highlighting}[]
\OtherTok{knock ::} \DataTypeTok{Knockable}\NormalTok{ s }\OtherTok{{-}\textgreater{}} \DataTypeTok{Door}\NormalTok{ s }\OtherTok{{-}\textgreater{}} \DataTypeTok{IO}\NormalTok{ ()}
\end{Highlighting}
\end{Shaded}

We can read the type signature as: ``Knocking requires both a \texttt{Door\ s}
and a \emph{proof} that the door's status is \texttt{Knockable}''. It makes it
impossible for us to run \texttt{knock} on a status that is not
\texttt{Knockable}, like, say, \texttt{\textquotesingle{}Opened}.

In this light, the role of a proof is like a ``key'' that a type (like
\texttt{\textquotesingle{}Closed}) must provide to ``unlock'' functions like
\texttt{knock}.\footnote{Sorry to mix up similar metaphors like this! Definitely
  not intentional :)} A \emph{decision function} is a function to generate these
proofs (or prove that they are impossible) for given types.

On one level, you can think of proofs as ``compiler tricks'', or things that
exist only to appease the compiler. In fact, compilers of languages that
encourage heavy usage of proofs (like Agda, Coq, Idris) actually implement
something called \emph{proof erasure}. That is, in those languages, values like
\texttt{KnockClosed} and \texttt{KnockLocked} might never exist at runtime,
since they never actually \emph{do} anything at runtime. They only exist as ways
to limit or enable specific programs from compiling, and serve no purpose after
compilation. GHC Haskell does not implement proof erasure at the time of this
post (current GHC version 8.6), but if proofs like this become commonplace, you
might be reading this during a time where GHC Haskell erases proofs like
\texttt{Knockable} witnesses!\footnote{Note, however, that we are a little lucky
  in our case. In the case of our implementation of \texttt{knock}, we match on
  a wildcard pattern, so the input proof is never evaluated.}

\subsection{The Role of Singletons}\label{the-role-of-singletons}

Proofs themselves might not play a role at run-time, but generating/deciding
them with types requires being able to pattern match and work with \emph{types}
at run-time. Because of this, singletons play an important practical role in
working with proofs in Haskell.

After all, remember the type of our decision function:

\begin{Shaded}
\begin{Highlighting}[]
\OtherTok{isKnockable ::} \DataTypeTok{Sing}\NormalTok{ a }\OtherTok{{-}\textgreater{}} \DataTypeTok{Decision}\NormalTok{ (}\DataTypeTok{Knockable}\NormalTok{ a)}
\end{Highlighting}
\end{Shaded}

The \texttt{Sing} allows \texttt{isKnockable} to pattern match and inspect the
\emph{type} \texttt{a} to create your proof.

In this light, the \emph{singletons} library provides many tools for working
with proofs and decisions. In fact, the entire \emph{Data.Singletons.Decide}
module is dedicated to working with proofs and decisions. It provides the
\texttt{Decision} data type and \texttt{Refuted} type synonym, both featured
above.

It also re-exports a particularly useful predicate from \emph{base},
\emph{propositional equality}:

\begin{Shaded}
\begin{Highlighting}[]
\KeywordTok{data}\OtherTok{ (:\textasciitilde{}:) ::}\NormalTok{ k }\OtherTok{{-}\textgreater{}}\NormalTok{ k }\OtherTok{{-}\textgreater{}} \DataTypeTok{Type} \KeywordTok{where}
    \DataTypeTok{Refl}\OtherTok{ ::}\NormalTok{ a }\OperatorTok{:\textasciitilde{}:}\NormalTok{ a}
\end{Highlighting}
\end{Shaded}

Like how \texttt{Knockable} is a predicate that a given status is ``knockable'',
\texttt{(\textquotesingle{}Blah\ :\textasciitilde{}:)} is a predicate that a
given type is \emph{equal to} \texttt{\textquotesingle{}Blah}. A value of type
\texttt{Knockable\ s} is a proof that \texttt{s} is knockable, and a value of
type \texttt{\textquotesingle{}Blah\ :\textasciitilde{}:\ a} is a proof that
\texttt{a} is \emph{equal to} \texttt{\textquotesingle{}Blah}.

To see how, note the constructors that it allows. Remember that we limit
\texttt{Knockable\ s} to only having ``knockable'' \texttt{s} by only allowing
two constructors, so we can only construct valid values. The same thing happens
here -- \texttt{(\textquotesingle{}Blah\ :\textasciitilde{}:)} only has
\emph{one single constructor}:
\texttt{Refl\ ::\ \textquotesingle{}Blah\ :\textasciitilde{}:\ \textquotesingle{}Blah}.
The only valid constructor is one where the left hand side is equal to the right
hand side. I like to use \texttt{Refl} with type application syntax, like
\texttt{Refl\ @\textquotesingle{}Blah}, so it's always clear exactly what we are
saying is the same.

It also offers the ``kindclass'' \texttt{SDecide}, which provides \emph{decision
functions} for the \texttt{(a\ :\textasciitilde{}:)} predicate:

\begin{Shaded}
\begin{Highlighting}[]
\KeywordTok{class} \DataTypeTok{SDecide}\NormalTok{ k }\KeywordTok{where}
\OtherTok{    (\%\textasciitilde{}) ::} \DataTypeTok{Sing}\NormalTok{ (}\OtherTok{a ::}\NormalTok{ k)}
         \OtherTok{{-}\textgreater{}} \DataTypeTok{Sing}\NormalTok{ (}\OtherTok{b ::}\NormalTok{ k)}
         \OtherTok{{-}\textgreater{}} \DataTypeTok{Decision}\NormalTok{ (a }\OperatorTok{:\textasciitilde{}:}\NormalTok{ b)}
\end{Highlighting}
\end{Shaded}

For example, \texttt{Bool} is an instance of \texttt{SDecide}, so we have a
function:

\begin{Shaded}
\begin{Highlighting}[]
\NormalTok{(}\DataTypeTok{STrue} \OperatorTok{\%\textasciitilde{}}\NormalTok{)}\OtherTok{ ::} \DataTypeTok{Sing}\NormalTok{ b }\OtherTok{{-}\textgreater{}} \DataTypeTok{Decision}\NormalTok{ (}\DataTypeTok{\textquotesingle{}True} \OperatorTok{:\textasciitilde{}:}\NormalTok{ b)}
\end{Highlighting}
\end{Shaded}

which is a decision function to check if \texttt{b} is equal to
\texttt{\textquotesingle{}True}. You can sort of imagine \texttt{SDecide} as a
type-level \texttt{Eq} typeclass, but for ``type equality''.

\section{Type Level Functions}\label{type-level-functions}

We're now going to look at a different method useful for restricting how we can
call functions. Something we can do is define a type that expresses
knockable-or-not-knockable, as a value:\footnote{Really, we could just use
  \texttt{Bool} instead of defining a \texttt{Pass} type. We're just going
  through a new type for the sake of example, and it can be useful because a
  type like \texttt{Pass} might potentially have even more constructors!}

\begin{Shaded}
\begin{Highlighting}[]
\OperatorTok{$}\NormalTok{(singletons [d|}
\NormalTok{  data Pass = Obstruct | Allow}
\NormalTok{  |])}
\end{Highlighting}
\end{Shaded}

And we can write a \emph{type-level function} (implemented as \emph{type
family}) from \texttt{DoorState} to a \texttt{Pass}:

\begin{Shaded}
\begin{Highlighting}[]
\KeywordTok{type} \KeywordTok{family} \DataTypeTok{StatePass}\NormalTok{ (}\OtherTok{s ::} \DataTypeTok{DoorState}\NormalTok{)}\OtherTok{ ::} \DataTypeTok{Pass} \KeywordTok{where}
    \DataTypeTok{StatePass} \DataTypeTok{\textquotesingle{}Opened} \OtherTok{=} \DataTypeTok{\textquotesingle{}Allow}
    \DataTypeTok{StatePass} \DataTypeTok{\textquotesingle{}Closed} \OtherTok{=} \DataTypeTok{\textquotesingle{}Obstruct}
    \DataTypeTok{StatePass} \DataTypeTok{\textquotesingle{}Locked} \OtherTok{=} \DataTypeTok{\textquotesingle{}Obstruct}
\end{Highlighting}
\end{Shaded}

We've briefly touched on type families before (in talking about
\texttt{SingKind}), but, as a quick review: type families act a bit like
type-level functions. They take types as input arguments and return types in
return.

We can inspect how type families are applied by using the \texttt{:kind!}
command in ghci:

\begin{Shaded}
\begin{Highlighting}[]
\NormalTok{ghci}\OperatorTok{\textgreater{}} \OperatorTok{:}\NormalTok{kind}\OperatorTok{!} \DataTypeTok{StatePass} \DataTypeTok{\textquotesingle{}Opened}
\DataTypeTok{\textquotesingle{}Allow}
\NormalTok{ghci}\OperatorTok{\textgreater{}} \OperatorTok{:}\NormalTok{kind}\OperatorTok{!} \DataTypeTok{StatePass} \DataTypeTok{\textquotesingle{}Closed}
\DataTypeTok{\textquotesingle{}Obstruct}
\end{Highlighting}
\end{Shaded}

(Note that \texttt{:kind!} is different from \texttt{:kind}: \texttt{:kind} is
the ghci command to report the kind of a type expression, and \texttt{:kind!} is
the ghci command to evaluate type families in type expressions.)

Like type synonyms, type families can't be partially applied (``unsaturated'').
They only ever make sense in ``fully applied'' (or ``saturated'') form, with all
arguments given syntactically.

Armed with this type family, we can write a new version of \texttt{knock}:

\begin{Shaded}
\begin{Highlighting}[]
\CommentTok{{-}{-} source: https://github.com/mstksg/inCode/tree/master/code{-}samples/singletons/Door3.hs\#L89{-}L90}

\OtherTok{knockP ::}\NormalTok{ (}\DataTypeTok{StatePass}\NormalTok{ s }\OperatorTok{\textasciitilde{}} \DataTypeTok{\textquotesingle{}Obstruct}\NormalTok{) }\OtherTok{=\textgreater{}} \DataTypeTok{Door}\NormalTok{ s }\OtherTok{{-}\textgreater{}} \DataTypeTok{IO}\NormalTok{ ()}
\NormalTok{knockP d }\OtherTok{=} \FunctionTok{putStrLn} \OperatorTok{$} \StringTok{"Knock knock on "} \OperatorTok{++}\NormalTok{ doorMaterial d }\OperatorTok{++} \StringTok{" door!"}
\end{Highlighting}
\end{Shaded}

\texttt{a\ \textasciitilde{}\ b} is a constraint for \emph{type equality}. This
constraint means that calling \texttt{knock} requires that \texttt{StatePass\ s}
is \emph{equal to} (or unifies with) \texttt{\textquotesingle{}Obstruct}. So, if
we attempt to call \texttt{knock} with a \texttt{\textquotesingle{}Locked} door,
then because \texttt{StatePass\ \textquotesingle{}Locked} is
\texttt{\textquotesingle{}Obstruct}, the constraint is satisfied and everyone is
happy. If we attempt to call \texttt{knock} with an
\texttt{\textquotesingle{}Opened} door,
\texttt{StatePass\ \textquotesingle{}Opened} is
\texttt{\textquotesingle{}Allow}, so the constraint is not satisfied and
everyone is sad.

\begin{Shaded}
\begin{Highlighting}[]
\NormalTok{ghci}\OperatorTok{\textgreater{}} \KeywordTok{let}\NormalTok{ door1 }\OtherTok{=}\NormalTok{ mkDoor }\DataTypeTok{SClosed} \StringTok{"Oak"}
\NormalTok{ghci}\OperatorTok{\textgreater{}} \KeywordTok{let}\NormalTok{ door2 }\OtherTok{=}\NormalTok{ mkDoor }\DataTypeTok{SOpened} \StringTok{"Spruce"}
\NormalTok{ghci}\OperatorTok{\textgreater{}}\NormalTok{ knock door1}
\CommentTok{{-}{-} Knock knock on Oak door!}
\NormalTok{ghci}\OperatorTok{\textgreater{}}\NormalTok{ knock door2}
\DataTypeTok{COMPILE} \DataTypeTok{ERROR}\OperatorTok{!}
\CommentTok{{-}{-}     • Couldn\textquotesingle{}t match type ‘\textquotesingle{}Allow’ with ‘\textquotesingle{}Obstruct’}
\CommentTok{{-}{-}             arising from a use of ‘knock’}
\end{Highlighting}
\end{Shaded}

\subsection{Deciding at Runtime}\label{deciding-at-runtime}

One nice thing is that, if we know \texttt{s} at compile-time, we can call this
function without having to pass any manual proofs. However, we have to deal with
the same issue as before: what happens if we don't know \texttt{s} until
runtime? How do we prove to the compiler that \texttt{Passable\ s} is
\texttt{\textquotesingle{}Allow}?

Remember that type families take \emph{types} as inputs, so we can't write:

\begin{Shaded}
\begin{Highlighting}[]
\OtherTok{knockSomeDoor ::} \DataTypeTok{SomeDoor} \OtherTok{{-}\textgreater{}} \DataTypeTok{IO}\NormalTok{ ()}
\NormalTok{knockSomeDoor (}\DataTypeTok{MkSomeDoor}\NormalTok{ s d) }\OtherTok{=}
    \KeywordTok{case} \DataTypeTok{StatePass}\NormalTok{ s }\KeywordTok{of}
      \CommentTok{{-}{-} ...}
\end{Highlighting}
\end{Shaded}

because \texttt{s}, a value, can't be given to \texttt{StatePass}.

What we really want to do is pass \texttt{s} (the singleton representing a type)
to \texttt{StatePass} (the type family). And then, we want to match on the
\emph{resulting type}, so we can decide what to do based on the result.

If you think about this predicament long enough, you might begin to see a
solution. Essentially, we want a function that takes a \emph{singleton} of
\texttt{s}, and return a \emph{singleton} of \texttt{StatePass\ s}.

What we want, in the end, is a \emph{mirror} of the type-level function \emph{at
the value level}. We need to write a function of type
\texttt{Sing\ s\ -\textgreater{}\ Sing\ (StatePass\ s)}: given a singleton of a
type, return a singleton of the type family applied to the type.

\begin{Shaded}
\begin{Highlighting}[]
\KeywordTok{type} \KeywordTok{family} \DataTypeTok{StatePass}\NormalTok{ (}\OtherTok{s ::} \DataTypeTok{DoorState}\NormalTok{)}\OtherTok{ ::} \DataTypeTok{Pass} \KeywordTok{where}
    \DataTypeTok{StatePass} \DataTypeTok{\textquotesingle{}Opened} \OtherTok{=} \DataTypeTok{\textquotesingle{}Allow}
    \DataTypeTok{StatePass} \DataTypeTok{\textquotesingle{}Closed} \OtherTok{=} \DataTypeTok{\textquotesingle{}Obstruct}
    \DataTypeTok{StatePass} \DataTypeTok{\textquotesingle{}Locked} \OtherTok{=} \DataTypeTok{\textquotesingle{}Obstruct}

\OtherTok{sStatePass ::} \DataTypeTok{Sing}\NormalTok{ s }\OtherTok{{-}\textgreater{}} \DataTypeTok{Sing}\NormalTok{ (}\DataTypeTok{StatePass}\NormalTok{ s)}
\NormalTok{sStatePass }\OtherTok{=}\NormalTok{ \textbackslash{}}\KeywordTok{case}
    \DataTypeTok{SOpened} \OtherTok{{-}\textgreater{}} \DataTypeTok{SAllow}
    \DataTypeTok{SClosed} \OtherTok{{-}\textgreater{}} \DataTypeTok{SObstruct}
    \DataTypeTok{SLocked} \OtherTok{{-}\textgreater{}} \DataTypeTok{SObstruct}
\end{Highlighting}
\end{Shaded}

We have to be very careful with how we define \texttt{sStatePass}, because GHC
isn't too smart. It'll reject any definition that isn't structurally identical
to the type family it's mirroring.

With our new tool, we can now write:

\begin{Shaded}
\begin{Highlighting}[]
\CommentTok{{-}{-} source: https://github.com/mstksg/inCode/tree/master/code{-}samples/singletons/Door3.hs\#L92{-}L97}

\NormalTok{knockSomeDoorP}
\OtherTok{    ::} \DataTypeTok{SomeDoor}     \CommentTok{{-}{-} \^{} status not known until you pattern match at runtime}
    \OtherTok{{-}\textgreater{}} \DataTypeTok{IO}\NormalTok{ ()}
\NormalTok{knockSomeDoorP (}\DataTypeTok{MkSomeDoor}\NormalTok{ s d) }\OtherTok{=} \KeywordTok{case}\NormalTok{ sStatePass s }\KeywordTok{of}
    \DataTypeTok{SObstruct} \OtherTok{{-}\textgreater{}}\NormalTok{ knockP d                        }\CommentTok{{-}{-} \^{} \textasciigrave{}StatePass s \textasciitilde{} \textquotesingle{}Obstruct\textasciigrave{}}
    \DataTypeTok{SAllow}    \OtherTok{{-}\textgreater{}} \FunctionTok{putStrLn} \StringTok{"No knocking allowed!"} \CommentTok{{-}{-} \^{} \textasciigrave{}StatePass s \textasciitilde{} \textquotesingle{}Allow\textasciigrave{}}
\end{Highlighting}
\end{Shaded}

First we use \texttt{sStatePass\ s} to check the ``pass'' of the \texttt{s}.
Then, we match on the \texttt{Pass}: if it's
\texttt{\textquotesingle{}Obstruct}, like the type signature of \texttt{knock}
requires, we can run \texttt{knock}. If not, then we cannot!

\subsection{Singletons Library to the
Rescue}\label{singletons-library-to-the-rescue}

At the high level, we defined a ``function'' on types (\texttt{StatePass}),
using type families.

And, just like we have to define singletons (\texttt{SOpened}, \texttt{SClosed},
etc.) at the value level to mirror what is happening at the type level, we also
have to define \emph{singleton functions} (\texttt{sStatePass}) at the value
level to mirror what is happening at the type level.

Defining singletons for our types is a tedious and mechanical process. Defining
singletonized functions for our type families is also similarly tedious and
mechanical. This is where the \emph{singletons} library comes in: it provides us
Template Haskell tools to automatically define type families and their
associated singleton functions:

\begin{Shaded}
\begin{Highlighting}[]
\OperatorTok{$}\NormalTok{(singletons [d|}
\NormalTok{  statePass :: DoorState {-}\textgreater{} Pass}
\NormalTok{  statePass Opened = Allow}
\NormalTok{  statePass Closed = Obstruct}
\NormalTok{  statePass Locked = Obstruct}
\NormalTok{  |])}
\end{Highlighting}
\end{Shaded}

The above declaration would normally declare only the value-level function
\texttt{statePass} with the type \texttt{DoorSate\ -\textgreater{}\ Pass}.

However, with singleton's template haskell, this also generates:\footnote{In the
  spirit of full disclosure, the Template Haskell also generates some other
  things (known as \emph{defunctionalization symbols}), which we will be talking
  about in the next part of this series.}

\begin{itemize}
\tightlist
\item
  The \emph{type family} \texttt{StatePass\ (s\ ::\ DoorState)\ ::\ Pass}, like
  we defined above
\item
  The \emph{singleton function} \texttt{sStatePass}, with the type
  \texttt{Sing\ s\ -\textgreater{}\ Sing\ \ \ \ \ (StatePass\ s)}, like we
  defined above.
\end{itemize}

The naming convention for functions with non-symbol names takes a function like
\texttt{myFunction} and generates the type family \texttt{MyFunction} and the
singleton function \texttt{sMyFunction}.

The naming convention for functions with symbolic names (operators) takes an
operator like \texttt{++} and generates the type family \texttt{++} (keeping the
identical name) and the singleton function \texttt{\%++}.\footnote{Note that
  this is a change since \emph{singletons-2.4}. In previous versions,
  \texttt{++} would generate the type family \texttt{:++} and the singleton
  function \texttt{\%:++}.}

\section{A Comparison}\label{a-comparison}

We went over two methods of using phantom types with the singleton library and
dependent types to restrict how certain functions can be called, on a more
non-trivial level.

Our first method was leveraging ``dependently typed proofs''. These are useful
because they are constructed to exploit the ``structure'' of the types you
create. Essentially, we create a data type (predicate) in a way so that it is
impossible to create an ``invalid'' proof. And, often, if we write our proofs in
a clever enough way, we can actually use and combine proofs to generate new
proofs. (More examples in the exercises)

Personally, I find this to be the source of a lot of the ``fun'' of dependently
typed programming --- our proofs become first class values, and if we define
them in a nice enough way, we can use manipulate them to create new proofs. (A
full exploration of this is well beyond the scope of this post)

However, in practice, carefully constructing predicates and proofs (ones more
complicated than the one we just looked at) requires some up-front cost in
thinking about how to best express your predicate, and is sometimes not
straightforward.

I consider the second method, using type-level functions, to be the more
``mechanical'' way, with less upfront cost in thinking time. For the most part,
if you can write a normal term-level function (something that most Haskell
programmers are comfortable doing), you can write a type-level function. This is
even made simpler with singletons --- you can just write your term-level
relationship as a normal function, and you can now just directly use your
function at the type level.

In fact, consider if there were more than two \texttt{Pass} (maybe allow,
obstruct, or partial?). In that case, we can easily restrict a function based on
the \texttt{Pass} being equal to any of the three or more by using the
\texttt{\textasciitilde{}} constraint. Using the dependently typed proof
version, though, we would have to create a new GADT for each situation.

In a way, type-level functions deliver on the promise of blurring the line
between type and value. Our term-level functions are now type-level functions!
We just need to remember to switch our capitalizations!

But this strength is also its weakness. Remember that the problem of normal
term-level functions was that they are potentially ``incorrect'', and not
directly verifiable. So, if you just lift your potentially incorrect term-level
functions to the type level\ldots what you get is potentially incorrect
type-level functions! You get the \emph{same} logic errors. Really, writing
type-level functions (unsurprisingly) brings all of the error-proneness of
writing at the term-level.

In contrast, if you use dependently typed proofs correctly, these proofs can
\emph{compose}, and GHC can check that \emph{these proofs compose correctly}, or
that the compositions of your proofs are also valid proofs. That's because this
is enforced at the \emph{structural level}. (We'll look at some examples in the
exercises) GHC can't do that directly with functions; it can't check that the
composition of functions gives correct answers.

These two approaches aren't necessarily mutually exclusive, and you often might
mix the two. It's good to understand the trade-offs in up-front cost,
expressiveness, and correctness! But, however way you play, the
\emph{singletons} library is here to make our life easier.

\section{Singleton Library Functions}\label{singleton-library-functions}

As we have seen, working with type-level functions with singletons involves at
least two parts --- the type family working on the type-level values, and the
singleton functions mirroring the type family, working on the term-level
singletons.

The singletons library offers template haskell to make working with these things
pretty seamless. In fact, a good portion of Prelude and base is promoted and
exported by singletons!

You can find most of these in the \emph{Data.Singletons.Prelude} module
namespace. So, with singletons, you get functions like:

\begin{Shaded}
\begin{Highlighting}[]
\FunctionTok{fst}\OtherTok{ ::}\NormalTok{ (a, b) }\OtherTok{{-}\textgreater{}}\NormalTok{ a}

\KeywordTok{type} \KeywordTok{family} \DataTypeTok{Fst}\NormalTok{ (}\OtherTok{t ::}\NormalTok{ (a, b))}\OtherTok{ ::}\NormalTok{ a}

\OtherTok{sFst ::} \DataTypeTok{Sing}\NormalTok{ t }\OtherTok{{-}\textgreater{}} \DataTypeTok{Sing}\NormalTok{ (}\DataTypeTok{Fst}\NormalTok{ t)}
\end{Highlighting}
\end{Shaded}

and

\begin{Shaded}
\begin{Highlighting}[]
\OtherTok{isLeft ::} \DataTypeTok{Either}\NormalTok{ a b }\OtherTok{{-}\textgreater{}} \DataTypeTok{Bool}

\KeywordTok{type} \KeywordTok{family} \DataTypeTok{IsLeft}\NormalTok{ (}\OtherTok{t ::} \DataTypeTok{Either}\NormalTok{ a b)}\OtherTok{ ::} \DataTypeTok{Bool}

\OtherTok{sIsLeft ::} \DataTypeTok{Sing}\NormalTok{ t }\OtherTok{{-}\textgreater{}} \DataTypeTok{Sing}\NormalTok{ (}\DataTypeTok{IsLeft}\NormalTok{ t)}
\end{Highlighting}
\end{Shaded}

and

\begin{Shaded}
\begin{Highlighting}[]
\OtherTok{(++) ::}\NormalTok{ [a] }\OtherTok{{-}\textgreater{}}\NormalTok{ [a] }\OtherTok{{-}\textgreater{}}\NormalTok{ [a]}

\KeywordTok{type} \KeywordTok{family}\NormalTok{ (}\OtherTok{xs ::}\NormalTok{ [a]) }\OperatorTok{++}\NormalTok{ (}\OtherTok{ys ::}\NormalTok{ [a])}\OtherTok{ ::}\NormalTok{ [a]}

\OtherTok{(\%++) ::} \DataTypeTok{Sing}\NormalTok{ xs }\OtherTok{{-}\textgreater{}} \DataTypeTok{Sing}\NormalTok{ ys }\OtherTok{{-}\textgreater{}} \DataTypeTok{Sing}\NormalTok{ (xs }\OperatorTok{++}\NormalTok{ ys)}
\end{Highlighting}
\end{Shaded}

\subsection{Promoted Typeclasses}\label{promoted-typeclasses}

But, how can we promote functions like \texttt{(==)} and \texttt{max}, which are
typeclass-polymorphic?

With kindclasses (typeclasses for kinds), of course!

Let's remember what we need for these promoted functions to work: the type
families, and the singleton functions.

The \emph{singletons} library handles this by providing each of these in a
separate typeclass. Let's look at the humble \texttt{Eq} typeclass as an
example:

\begin{Shaded}
\begin{Highlighting}[]
\KeywordTok{class} \DataTypeTok{Eq}\NormalTok{ a }\KeywordTok{where}
\OtherTok{    (==) ::}\NormalTok{ a }\OtherTok{{-}\textgreater{}}\NormalTok{ a }\OtherTok{{-}\textgreater{}} \DataTypeTok{Bool}
\OtherTok{    (/=) ::}\NormalTok{ a }\OtherTok{{-}\textgreater{}}\NormalTok{ a }\OtherTok{{-}\textgreater{}} \DataTypeTok{Bool}
\end{Highlighting}
\end{Shaded}

The \emph{singletons} library promotes this as:

\begin{Shaded}
\begin{Highlighting}[]
\KeywordTok{class} \DataTypeTok{PEq}\NormalTok{ a }\KeywordTok{where}
    \KeywordTok{type}\NormalTok{ (}\OtherTok{x ::}\NormalTok{ a) }\OperatorTok{==}\NormalTok{ (}\OtherTok{y ::}\NormalTok{ a)}\OtherTok{ ::} \DataTypeTok{Bool}       \CommentTok{{-}{-} \^{} associated type / type family}
    \KeywordTok{type}\NormalTok{ (}\OtherTok{x ::}\NormalTok{ a) }\OperatorTok{/=}\NormalTok{ (}\OtherTok{y ::}\NormalTok{ a)}\OtherTok{ ::} \DataTypeTok{Bool}

\KeywordTok{class} \DataTypeTok{SEq}\NormalTok{ a }\KeywordTok{where}
\OtherTok{    (\%==) ::} \DataTypeTok{Sing}\NormalTok{ (}\OtherTok{x ::}\NormalTok{ a) }\OtherTok{{-}\textgreater{}} \DataTypeTok{Sing}\NormalTok{ (}\OtherTok{y ::}\NormalTok{ a) }\OtherTok{{-}\textgreater{}} \DataTypeTok{Sing}\NormalTok{ (x }\OperatorTok{==}\NormalTok{ y)}
\OtherTok{    (\%/=) ::} \DataTypeTok{Sing}\NormalTok{ (}\OtherTok{x ::}\NormalTok{ a) }\OtherTok{{-}\textgreater{}} \DataTypeTok{Sing}\NormalTok{ (}\OtherTok{y ::}\NormalTok{ a) }\OtherTok{{-}\textgreater{}} \DataTypeTok{Sing}\NormalTok{ (x }\OperatorTok{/=}\NormalTok{ y)}
\end{Highlighting}
\end{Shaded}

The naming convention is to just add \texttt{P} for the ``promoted'' type family
functions, and \texttt{S} for the singleton functions.

In fact, you can even promote your own custom typeclasses:

\begin{Shaded}
\begin{Highlighting}[]
\OperatorTok{$}\NormalTok{(singletons [d|}
\NormalTok{  class MyClass a where}
\NormalTok{    myFunc :: a {-}\textgreater{} a}
\NormalTok{  |])}
\end{Highlighting}
\end{Shaded}

This would create:

\begin{enumerate}
\def\labelenumi{\arabic{enumi}.}
\tightlist
\item
  The \emph{typeclass} \texttt{MyClass} with method
  \texttt{myFunc\ ::\ MyClass\ a\ =\textgreater{}\ a\ -\textgreater{}\ a}
\item
  The \emph{promoted typeclass} \texttt{PMyClass} with associated type/type
  family \texttt{MyFunc\ (x\ ::\ a)\ ::\ a}
\item
  The \emph{singletonized} typeclass \texttt{SMyClass} with method
  \texttt{sMyFunc\ ::\ Sing\ x\ -\textgreater{}\ \ \ \ \ Sing\ (MyFunc\ x)}.
\end{enumerate}

\subsection{Automatically Promoting
Instances}\label{automatically-promoting-instances}

The \emph{singletons} library is smart enough to automatically promote
instances, as well, including derived ones!

\begin{Shaded}
\begin{Highlighting}[]
\OperatorTok{$}\NormalTok{(singletons [d|}
\NormalTok{  data Pass = Obstruct | Allow}

\NormalTok{  instance Eq Pass where}
\NormalTok{      Obstruct == Obstruct = True}
\NormalTok{      Obstruct == Allow    = False}
\NormalTok{      Allow    == Obstruct = False}
\NormalTok{      Allow    == Allow    = True}

\NormalTok{      Obstruct /= Obstruct = True}
\NormalTok{      Obstruct /= Allow    = False}
\NormalTok{      Allow    /= Obstruct = False}
\NormalTok{      Allow    /= Allow    = True}
\NormalTok{  |])}
\end{Highlighting}
\end{Shaded}

This automatically also generates \texttt{PEq} and \texttt{SEq} instances for
\texttt{Pass}:

\begin{Shaded}
\begin{Highlighting}[]
\NormalTok{ghci}\OperatorTok{\textgreater{}} \OperatorTok{:}\NormalTok{kind}\OperatorTok{!} \DataTypeTok{\textquotesingle{}Obstruct} \OperatorTok{==} \DataTypeTok{\textquotesingle{}Obstruct}
\DataTypeTok{\textquotesingle{}True}
\NormalTok{ghci}\OperatorTok{\textgreater{}} \DataTypeTok{SAllow} \OperatorTok{\%==} \DataTypeTok{SObstruct}
\DataTypeTok{SFalse}
\end{Highlighting}
\end{Shaded}

But, you can also just write:

\begin{Shaded}
\begin{Highlighting}[]
\OperatorTok{$}\NormalTok{(singletons [d|}
\NormalTok{  data Pass = Obstruct | Allow}
\NormalTok{    deriving (Show, Eq, Ord)}
\NormalTok{  |])}
\end{Highlighting}
\end{Shaded}

And this works as well!

\begin{Shaded}
\begin{Highlighting}[]
\NormalTok{ghci}\OperatorTok{\textgreater{}} \OperatorTok{:}\NormalTok{kind}\OperatorTok{!} \DataTypeTok{Show\_} \DataTypeTok{\textquotesingle{}Obstruct}      \CommentTok{{-}{-} is named Show\_ to not conflict with prelude Show}
\StringTok{"Obstruct"}
\NormalTok{ghci}\OperatorTok{\textgreater{}}\NormalTok{ sMax }\DataTypeTok{SObstruct} \DataTypeTok{SAllow}
\DataTypeTok{SAllow}
\end{Highlighting}
\end{Shaded}

\section{Next Steps}\label{next-steps}

In this article we tackled the problem of more expressive ways to
\emph{restrict} the ways users can manipulate our data types. We talked about
``dependently typed proofs'' (a staple tool of dependently typed programming)
and about ``type level functions'' (a familiar friend in a new context), their
trade-offs, and how the \emph{singletons} library provides tools to make working
with both easier.

When we first looked at the idea of phantom type parameters, using them to
\emph{restrict} how functions are called was definitely one of the promises I
made. By now, this promise has hopefully been fully realized.

However, the \emph{other} promise we made about the usefulness of phantom type
parameters is that we can use them be more expressive in what our functions do.
One huge benefit of using phantom types in this way is that we can express how
our input values relate to our output values in ways that we couldn't before.
(as a simple example, we had previously written
\texttt{closeDoor\ ::\ Door\ \textquotesingle{}Opened\ -\textgreater{}\ Door\ \textquotesingle{}Closed},
which we know closes a door just by looking at its type)

This goes beyond simple restrictions, and we will begin discussing this in the
next post! We'll explore using type-level functions to express more non-trivial
and complex relationships, and also talk about code re-use using
\emph{higher-order functions} via singleton's defunctionalization system.

That's it for now --- check out the exercises, and feel free to ask any
questions in the comments, or in freenode \texttt{\#haskell}, where I idle as
\emph{jle`}!

\section{Exercises}\label{exercises}

Here are some exercises to help cement your understanding of the concepts here!
Feel free to start from
\href{https://github.com/mstksg/inCode/tree/master/code-samples/singletons/Door3.hs}{the
sample source code}; it contains all of the solutions, but you can delete
everything after the comment \texttt{-\/-\ Exercises} if you wish to start on
your own!

\textbf{Remember to enable \texttt{-Werror=incomplete-patterns} or
\texttt{-Wall}} to ensure that all of your functions are total! None of these
implementations should require any incomplete pattern matches!

\begin{enumerate}
\def\labelenumi{\arabic{enumi}.}
\item
  We talk about predicates as type constructors with type
  \texttt{k\ -\textgreater{}\ Type}. This fits a lot of things we've seen before
  (all instances of \texttt{Functor}, for example), but some predicates are more
  interesting than others.

  What is the interpretation of \texttt{SDoorState} as a predicate? (remember,
  \texttt{SDoorState\ s} is the type synonym for
  \texttt{Sing\ (s\ ::\ DoorState)}) What ``traditional'' (that is,
  \texttt{a\ -\textgreater{}\ Bool}) predicate does it correspond to?

  What is the type of its \emph{decision function}? Can you implement it?

  Solution available
  \href{https://github.com/mstksg/inCode/tree/master/code-samples/singletons/Door3.hs\#L104-L109}{here}!
\item
  Now let's practice working with predicates, singletons, and negation via
  \texttt{Refuted} together.

  You may have heard of the principle of ``double negation'', where \emph{not
  (not p)} implies \emph{p}. So, we should be able to say that
  \texttt{Refuted\ (Refuted\ (Knockable\ s))} implies
  \texttt{Knockable\ s}.\footnote{Double negation is not true in general, but it
    is true in the case that our predicate is \emph{decidable}. That's because
    \texttt{Decision\ a} is essentially a witness to the
    \href{https://en.wikipedia.org/wiki/Law_of_excluded_middle}{excluded middle}
    for that specific predicate, from which double negation can be derived.} If
  something is not ``not knockable'', then it must be knockable, right?

  Try writing \texttt{refuteRefuteKnockable} to verify this principle --- at
  least for the \texttt{Knockable} predicate.

\begin{Shaded}
\begin{Highlighting}[]
\CommentTok{{-}{-} source: https://github.com/mstksg/inCode/tree/master/code{-}samples/singletons/Door3.hs\#L112{-}L115}

\NormalTok{refuteRefuteKnockable}
\OtherTok{    ::} \KeywordTok{forall}\NormalTok{ s}\OperatorTok{.} \DataTypeTok{SingI}\NormalTok{ s}
    \OtherTok{=\textgreater{}} \DataTypeTok{Refuted}\NormalTok{ (}\DataTypeTok{Refuted}\NormalTok{ (}\DataTypeTok{Knockable}\NormalTok{ s))}
    \OtherTok{{-}\textgreater{}} \DataTypeTok{Knockable}\NormalTok{ s}
\end{Highlighting}
\end{Shaded}

  While not required, I recommend using \texttt{isKnockable} and writing your
  implementation in terms of it! Use \texttt{sing} to give \texttt{isKnockable}
  the singleton it needs.

  Solution available
  \href{https://github.com/mstksg/inCode/tree/master/code-samples/singletons/Door3.hs\#L112-L119}{here}!

  \emph{Hint:} You might find \texttt{absurd} (from \emph{Data.Void}) helpful:

\begin{Shaded}
\begin{Highlighting}[]
\OtherTok{absurd ::} \KeywordTok{forall}\NormalTok{ a}\OperatorTok{.} \DataTypeTok{Void} \OtherTok{{-}\textgreater{}}\NormalTok{ a}
\end{Highlighting}
\end{Shaded}

  If you have a \texttt{Void}, you can make a value of any type!\footnote{It's
    the good ol'
    \href{https://en.wikipedia.org/wiki/Principle_of_explosion}{Principle of
    Explosion}}
\item
  (This next one is fairly difficult compared to the others, and is only
  tangentially related to singletons, so feel free to skip it!)

  Type-level predicates are logical constructs, so we should be able to define
  concepts like ``and'' and ``or'' with them.

  \begin{enumerate}
  \def\labelenumii{\alph{enumii}.}
  \item
    Define a predicate constructor \texttt{And} that takes two predicates and
    returns a new predicate. This new predicate is true (aka, has an inhabitant)
    if and only if the two original predicates are true (aka, have inhabitants)

\begin{Shaded}
\begin{Highlighting}[]
\CommentTok{{-}{-} source: https://github.com/mstksg/inCode/tree/master/code{-}samples/singletons/Door3.hs\#L122{-}L122}

\KeywordTok{data} \DataTypeTok{And}\OtherTok{ ::}\NormalTok{ (k }\OtherTok{{-}\textgreater{}} \DataTypeTok{Type}\NormalTok{) }\OtherTok{{-}\textgreater{}}\NormalTok{ (k }\OtherTok{{-}\textgreater{}} \DataTypeTok{Type}\NormalTok{) }\OtherTok{{-}\textgreater{}}\NormalTok{ (k }\OtherTok{{-}\textgreater{}} \DataTypeTok{Type}\NormalTok{) }\KeywordTok{where}
\end{Highlighting}
\end{Shaded}
  \item
    Define a predicate constructor \texttt{Or} that takes two predicates and
    returns a new predicate. This new predicate is true (aka, has an inhabitant)
    if and only if at least one of the two original predicates are true (aka,
    have inhabitants)

\begin{Shaded}
\begin{Highlighting}[]
\CommentTok{{-}{-} source: https://github.com/mstksg/inCode/tree/master/code{-}samples/singletons/Door3.hs\#L125{-}L125}

\KeywordTok{data} \DataTypeTok{Or}\OtherTok{ ::}\NormalTok{ (k }\OtherTok{{-}\textgreater{}} \DataTypeTok{Type}\NormalTok{) }\OtherTok{{-}\textgreater{}}\NormalTok{ (k }\OtherTok{{-}\textgreater{}} \DataTypeTok{Type}\NormalTok{) }\OtherTok{{-}\textgreater{}}\NormalTok{ (k }\OtherTok{{-}\textgreater{}} \DataTypeTok{Type}\NormalTok{) }\KeywordTok{where}
\end{Highlighting}
\end{Shaded}

    There are potentially multiple non-trivial variations of this type.

    Do \texttt{And} and \texttt{Or} look similar to any types you might have
    encountered in the past? Maybe, perhaps, similiar to types that are a part
    of basic beginner Haskell concepts?
  \item
    Maybe surprisingly, \texttt{And\ p\ q} and \texttt{Or\ p\ q} are decidable
    if \texttt{p} and \texttt{q} are. Can we write the decision functions?

\begin{Shaded}
\begin{Highlighting}[]
\CommentTok{{-}{-} source: https://github.com/mstksg/inCode/tree/master/code{-}samples/singletons/Door3.hs\#L129{-}L144}

\NormalTok{decideAnd}
\OtherTok{    ::}\NormalTok{ (}\KeywordTok{forall}\NormalTok{ x}\OperatorTok{.} \DataTypeTok{Sing}\NormalTok{ x }\OtherTok{{-}\textgreater{}} \DataTypeTok{Decision}\NormalTok{ (p x))}
    \OtherTok{{-}\textgreater{}}\NormalTok{ (}\KeywordTok{forall}\NormalTok{ x}\OperatorTok{.} \DataTypeTok{Sing}\NormalTok{ x }\OtherTok{{-}\textgreater{}} \DataTypeTok{Decision}\NormalTok{ (q x))}
    \OtherTok{{-}\textgreater{}} \DataTypeTok{Sing}\NormalTok{ a}
    \OtherTok{{-}\textgreater{}} \DataTypeTok{Decision}\NormalTok{ (}\DataTypeTok{And}\NormalTok{ p q a)}

\NormalTok{decideOr}
\OtherTok{    ::}\NormalTok{ (}\KeywordTok{forall}\NormalTok{ x}\OperatorTok{.} \DataTypeTok{Sing}\NormalTok{ x }\OtherTok{{-}\textgreater{}} \DataTypeTok{Decision}\NormalTok{ (p x))}
    \OtherTok{{-}\textgreater{}}\NormalTok{ (}\KeywordTok{forall}\NormalTok{ x}\OperatorTok{.} \DataTypeTok{Sing}\NormalTok{ x }\OtherTok{{-}\textgreater{}} \DataTypeTok{Decision}\NormalTok{ (q x))}
    \OtherTok{{-}\textgreater{}} \DataTypeTok{Sing}\NormalTok{ a}
    \OtherTok{{-}\textgreater{}} \DataTypeTok{Decision}\NormalTok{ (}\DataTypeTok{Or}\NormalTok{ p q a)}
\end{Highlighting}
\end{Shaded}

    These functions actually demonstrate, I feel, why \texttt{Decision} having
    both a \texttt{Proved\ a} and \texttt{Disproved\ (Refuted\ a)} branch is
    very useful. This is because, if you wrote the \emph{structure} of
    \texttt{And} and \texttt{Or} correctly, it's \emph{impossible} to
    incorrectly define \texttt{decideAnd} and \texttt{decideOr}. You can't
    accidentally say false when it's true, or true when it's false --- your
    implementation is guarunteed correct.
  \item
    Now let's use \texttt{And} and \texttt{Or} to prove some useful facts about
    \texttt{Knockable} and
    \texttt{(\textquotesingle{}Opened\ :\textasciitilde{}:)}. We know that it's
    impossible for something to be both \texttt{Knockable} \emph{and}
    \texttt{(\textquotesingle{}Opened\ :\textasciitilde{}:)} (that is, both
    knockable \emph{and} equal to \texttt{\textquotesingle{}Opened}). Write such
    a witness:

\begin{Shaded}
\begin{Highlighting}[]
\CommentTok{{-}{-} source: https://github.com/mstksg/inCode/tree/master/code{-}samples/singletons/Door3.hs\#L153{-}L155}

\NormalTok{knockableNotOpened}
\OtherTok{    ::} \KeywordTok{forall}\NormalTok{ s}\OperatorTok{.} \DataTypeTok{SingI}\NormalTok{ s}
    \OtherTok{=\textgreater{}} \DataTypeTok{Refuted}\NormalTok{ (}\DataTypeTok{And} \DataTypeTok{Knockable}\NormalTok{ ((}\OperatorTok{:\textasciitilde{}:}\NormalTok{) }\DataTypeTok{\textquotesingle{}Opened}\NormalTok{) s)}
\end{Highlighting}
\end{Shaded}

    We also know that a given \texttt{DoorState} is either \texttt{Knockable} or
    \texttt{(\textquotesingle{}Opened\ :\textasciitilde{}:)} --- at least one of
    these is always true. Write such a witness:

\begin{Shaded}
\begin{Highlighting}[]
\CommentTok{{-}{-} source: https://github.com/mstksg/inCode/tree/master/code{-}samples/singletons/Door3.hs\#L160{-}L162}

\NormalTok{knockableOrOpened}
\OtherTok{    ::} \KeywordTok{forall}\NormalTok{ s}\OperatorTok{.} \DataTypeTok{SingI}\NormalTok{ s}
    \OtherTok{=\textgreater{}} \DataTypeTok{Or} \DataTypeTok{Knockable}\NormalTok{ ((}\OperatorTok{:\textasciitilde{}:}\NormalTok{) }\DataTypeTok{\textquotesingle{}Opened}\NormalTok{) s}
\end{Highlighting}
\end{Shaded}
  \end{enumerate}

  Solutions available
  \href{https://github.com/mstksg/inCode/tree/master/code-samples/singletons/Door3.hs\#L122-L166}{here}!
\item
  Instead of creating an entire \texttt{Knocked} type, we could have just said
  ``as long as the door is not \texttt{\textquotesingle{}Opened}, you can
  knock''. This means we could write \texttt{knock} as:

\begin{Shaded}
\begin{Highlighting}[]
\OtherTok{knock ::} \DataTypeTok{Refuted}\NormalTok{ (s }\OperatorTok{:\textasciitilde{}:} \DataTypeTok{\textquotesingle{}Opened}\NormalTok{) }\OtherTok{{-}\textgreater{}} \DataTypeTok{Door}\NormalTok{ s }\OtherTok{{-}\textgreater{}} \DataTypeTok{IO}\NormalTok{ ()}
\end{Highlighting}
\end{Shaded}

  Which we must pass a proof that \texttt{s} is not equal to
  \texttt{\textquotesingle{}Opened} in order to open our door.

  Is this really the same thing? Is
  \texttt{Refuted\ (s\ :\textasciitilde{}:\ \textquotesingle{}Opened)} the same
  thing as \texttt{Knockable\ s}?

  Let's try to say that the two things are the same! Write the following
  functions to show that
  \texttt{Refuted\ (s\ :\textasciitilde{}:\ \textquotesingle{}Opened)} is the
  same logical predicate as \texttt{Knockable\ s}!

\begin{Shaded}
\begin{Highlighting}[]
\CommentTok{{-}{-} source: https://github.com/mstksg/inCode/tree/master/code{-}samples/singletons/Door3.hs\#L169{-}L180}

\NormalTok{knockedRefute}
\OtherTok{    ::} \KeywordTok{forall}\NormalTok{ s}\OperatorTok{.} \DataTypeTok{SingI}\NormalTok{ s}
    \OtherTok{=\textgreater{}} \DataTypeTok{Knockable}\NormalTok{ s}
    \OtherTok{{-}\textgreater{}} \DataTypeTok{Refuted}\NormalTok{ (s }\OperatorTok{:\textasciitilde{}:} \DataTypeTok{\textquotesingle{}Opened}\NormalTok{)}

\NormalTok{refuteKnocked}
\OtherTok{    ::} \KeywordTok{forall}\NormalTok{ s}\OperatorTok{.} \DataTypeTok{SingI}\NormalTok{ s}
    \OtherTok{=\textgreater{}} \DataTypeTok{Refuted}\NormalTok{ (s }\OperatorTok{:\textasciitilde{}:} \DataTypeTok{\textquotesingle{}Opened}\NormalTok{)}
    \OtherTok{{-}\textgreater{}} \DataTypeTok{Knockable}\NormalTok{ s}
\end{Highlighting}
\end{Shaded}

  Solution available
  \href{https://github.com/mstksg/inCode/tree/master/code-samples/singletons/Door3.hs\#L169-L186}{here}!

  \emph{Note:} \texttt{knockedRefute} is fairly straightforward, but
  \texttt{refuteKnocked} is definitely trickier, so don't be discouraged!

  \emph{Hint:} See the note about \texttt{absurd} from Exercise 2!
\item
  On our type level function version of \texttt{knock}, we wrote, with a
  constraint:

\begin{Shaded}
\begin{Highlighting}[]
\OtherTok{knock ::}\NormalTok{ (}\DataTypeTok{StatePass}\NormalTok{ s }\OperatorTok{\textasciitilde{}} \DataTypeTok{\textquotesingle{}Obstruct}\NormalTok{) }\OtherTok{=\textgreater{}} \DataTypeTok{Door}\NormalTok{ s }\OtherTok{{-}\textgreater{}} \DataTypeTok{IO}\NormalTok{ ()}
\NormalTok{knock d }\OtherTok{=} \FunctionTok{putStrLn} \OperatorTok{$} \StringTok{"Knock knock on "} \OperatorTok{++}\NormalTok{ doorMaterial d }\OperatorTok{++} \StringTok{" door!"}
\end{Highlighting}
\end{Shaded}

  We can muddy the waters a bit, for fun, by having this take a proof of the
  constraint instead:

\begin{Shaded}
\begin{Highlighting}[]
\CommentTok{{-}{-} source: https://github.com/mstksg/inCode/tree/master/code{-}samples/singletons/Door3.hs\#L189{-}L190}

\OtherTok{knockRefl ::}\NormalTok{ (}\DataTypeTok{StatePass}\NormalTok{ s }\OperatorTok{:\textasciitilde{}:} \DataTypeTok{\textquotesingle{}Obstruct}\NormalTok{) }\OtherTok{{-}\textgreater{}} \DataTypeTok{Door}\NormalTok{ s }\OtherTok{{-}\textgreater{}} \DataTypeTok{IO}\NormalTok{ ()}
\NormalTok{knockRefl \_ d }\OtherTok{=} \FunctionTok{putStrLn} \OperatorTok{$} \StringTok{"Knock knock on "} \OperatorTok{++}\NormalTok{ doorMaterial d }\OperatorTok{++} \StringTok{" door!"}
\end{Highlighting}
\end{Shaded}

  Rewrite a version of \texttt{knockSomeDoor} in terms of \texttt{knockRefl},
  called \texttt{knockSomeDoorRefl}:

\begin{Shaded}
\begin{Highlighting}[]
\CommentTok{{-}{-} source: https://github.com/mstksg/inCode/tree/master/code{-}samples/singletons/Door3.hs\#L192{-}L195}

\NormalTok{knockSomeDoorRefl}
\OtherTok{    ::} \DataTypeTok{SomeDoor}
    \OtherTok{{-}\textgreater{}} \DataTypeTok{IO}\NormalTok{ ()}
\NormalTok{knockSomeDoorRefl (}\DataTypeTok{MkSomeDoor}\NormalTok{ s d) }\OtherTok{=}
\end{Highlighting}
\end{Shaded}

  Remember not to use \texttt{knock}!

  Solution available
  \href{https://github.com/mstksg/inCode/tree/master/code-samples/singletons/Door3.hs\#L189-L198}{here}.

  Assume that \texttt{DoorState} has an instance of \texttt{SDecide}, so you can
  use \texttt{(\%\textasciitilde{})}. This should be derived automatically as
  long as you derive \texttt{Eq}:

\begin{Shaded}
\begin{Highlighting}[]
\OperatorTok{$}\NormalTok{(singletons [d|}
\NormalTok{  data DoorState = Opened | Closed | Locked}
\NormalTok{    deriving (Show, Eq)}
\NormalTok{  |])}
\end{Highlighting}
\end{Shaded}
\item
  With the function that inverts \texttt{Pass}:

\begin{Shaded}
\begin{Highlighting}[]
\OperatorTok{$}\NormalTok{(singletons [d|}
\NormalTok{  invertPass :: Pass {-}\textgreater{} Pass}
\NormalTok{  invertPass Obstruct = Allow}
\NormalTok{  invertPass Allow    = Obstruct}
\NormalTok{|])}
\end{Highlighting}
\end{Shaded}

  Implement \texttt{knock} in a way that lets you knock if \texttt{invertPass}
  is \texttt{Allow}:

\begin{Shaded}
\begin{Highlighting}[]
\CommentTok{{-}{-} source: https://github.com/mstksg/inCode/tree/master/code{-}samples/singletons/Door3.hs\#L207{-}L208}

\OtherTok{knockInv ::}\NormalTok{ (}\DataTypeTok{InvertPass}\NormalTok{ (}\DataTypeTok{StatePass}\NormalTok{ s) }\OperatorTok{\textasciitilde{}} \DataTypeTok{\textquotesingle{}Allow}\NormalTok{) }\OtherTok{=\textgreater{}} \DataTypeTok{Door}\NormalTok{ s }\OtherTok{{-}\textgreater{}} \DataTypeTok{IO}\NormalTok{ ()}
\NormalTok{knockInv d }\OtherTok{=} \FunctionTok{putStrLn} \OperatorTok{$} \StringTok{"Knock knock on "} \OperatorTok{++}\NormalTok{ doorMaterial d }\OperatorTok{++} \StringTok{" door!"}
\end{Highlighting}
\end{Shaded}

  And write \texttt{knockSomeDoor} in terms of it:

\begin{Shaded}
\begin{Highlighting}[]
\CommentTok{{-}{-} source: https://github.com/mstksg/inCode/tree/master/code{-}samples/singletons/Door3.hs\#L210{-}L213}

\NormalTok{knockSomeDoorInv}
\OtherTok{    ::} \DataTypeTok{SomeDoor}
    \OtherTok{{-}\textgreater{}} \DataTypeTok{IO}\NormalTok{ ()}
\NormalTok{knockSomeDoorInv (}\DataTypeTok{MkSomeDoor}\NormalTok{ s d) }\OtherTok{=}
\end{Highlighting}
\end{Shaded}

  Remember again to implement it in terms of \texttt{knockInv}, \emph{not}
  \texttt{knock}.

  Solution available
  \href{https://github.com/mstksg/inCode/tree/master/code-samples/singletons/Door3.hs\#L207-L216}{here}!
\item
  Let's work with a toy typeclass called \texttt{Cycle}, based on \texttt{Enum}

\begin{Shaded}
\begin{Highlighting}[]
\OperatorTok{$}\NormalTok{(singletons [d|}
\NormalTok{  class Cycle a where}
\NormalTok{    next :: a {-}\textgreater{} a}
\NormalTok{    prev :: a {-}\textgreater{} a}
\NormalTok{  |])}
\end{Highlighting}
\end{Shaded}

  \texttt{next} is like \texttt{succ}, but loops over to the first item after
  the last constructor. \texttt{prev} is like \texttt{pred}, but loops over to
  the last item if pred-ing the first item

\begin{Shaded}
\begin{Highlighting}[]
\CommentTok{{-}{-} source: https://github.com/mstksg/inCode/tree/master/code{-}samples/singletons/Door3.hs\#L225{-}L232}

\KeywordTok{instance} \DataTypeTok{Cycle} \DataTypeTok{DoorState} \KeywordTok{where}
\NormalTok{    next }\DataTypeTok{Opened} \OtherTok{=} \DataTypeTok{Closed}
\NormalTok{    next }\DataTypeTok{Closed} \OtherTok{=} \DataTypeTok{Locked}
\NormalTok{    next }\DataTypeTok{Locked} \OtherTok{=} \DataTypeTok{Opened}

\NormalTok{    prev }\DataTypeTok{Opened} \OtherTok{=} \DataTypeTok{Locked}
\NormalTok{    prev }\DataTypeTok{Closed} \OtherTok{=} \DataTypeTok{Opened}
\NormalTok{    prev }\DataTypeTok{Locked} \OtherTok{=} \DataTypeTok{Closed}
\end{Highlighting}
\end{Shaded}

  Can you manually promote this instance for \texttt{DoorState} to the type
  level?

\begin{Shaded}
\begin{Highlighting}[]
\CommentTok{{-}{-} source: https://github.com/mstksg/inCode/tree/master/code{-}samples/singletons/Door3.hs\#L234{-}L243}

\KeywordTok{instance} \DataTypeTok{PCycle} \DataTypeTok{DoorState} \KeywordTok{where}

\KeywordTok{instance} \DataTypeTok{SCycle} \DataTypeTok{DoorState} \KeywordTok{where}
\end{Highlighting}
\end{Shaded}

  Solution available
  \href{https://github.com/mstksg/inCode/tree/master/code-samples/singletons/Door3.hs\#L225-L252}{here}!
\end{enumerate}

\section{Special Thanks}\label{special-thanks}

I am very humbled to be supported by an amazing community, who make it possible
for me to devote time to researching and writing these posts. Very special
thanks to my two supporters at the ``Amazing'' level on
\href{https://www.patreon.com/justinle/overview}{patreon}, Sam Stites and Josh
Vera! :)

Thanks also to \href{https://twitter.com/KozRoss}{Koz Ross} for helping
proofread this post!

\section{Signoff}\label{signoff}

Hi, thanks for reading! You can reach me via email at
\href{mailto:justin@jle.im}{\nolinkurl{justin@jle.im}}, or at twitter at
\href{https://twitter.com/mstk}{@mstk}! This post and all others are published
under the \href{https://creativecommons.org/licenses/by-nc-nd/3.0/}{CC-BY-NC-ND
3.0} license. Corrections and edits via pull request are welcome and encouraged
at \href{https://github.com/mstksg/inCode}{the source repository}.

If you feel inclined, or this post was particularly helpful for you, why not
consider \href{https://www.patreon.com/justinle/overview}{supporting me on
Patreon}, or a \href{bitcoin:3D7rmAYgbDnp4gp4rf22THsGt74fNucPDU}{BTC donation}?
:)

\end{document}
