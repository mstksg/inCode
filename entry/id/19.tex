\documentclass[]{article}
\usepackage{lmodern}
\usepackage{amssymb,amsmath}
\usepackage{ifxetex,ifluatex}
\usepackage{fixltx2e} % provides \textsubscript
\ifnum 0\ifxetex 1\fi\ifluatex 1\fi=0 % if pdftex
  \usepackage[T1]{fontenc}
  \usepackage[utf8]{inputenc}
\else % if luatex or xelatex
  \ifxetex
    \usepackage{mathspec}
    \usepackage{xltxtra,xunicode}
  \else
    \usepackage{fontspec}
  \fi
  \defaultfontfeatures{Mapping=tex-text,Scale=MatchLowercase}
  \newcommand{\euro}{€}
\fi
% use upquote if available, for straight quotes in verbatim environments
\IfFileExists{upquote.sty}{\usepackage{upquote}}{}
% use microtype if available
\IfFileExists{microtype.sty}{\usepackage{microtype}}{}
\usepackage[margin=1in]{geometry}
\ifxetex
  \usepackage[setpagesize=false, % page size defined by xetex
              unicode=false, % unicode breaks when used with xetex
              xetex]{hyperref}
\else
  \usepackage[unicode=true]{hyperref}
\fi
\hypersetup{breaklinks=true,
            bookmarks=true,
            pdfauthor={Justin Le},
            pdftitle={Blog engine updates: Markdown Preprocessor \& Fay Scripts},
            colorlinks=true,
            citecolor=blue,
            urlcolor=blue,
            linkcolor=magenta,
            pdfborder={0 0 0}}
\urlstyle{same}  % don't use monospace font for urls
% Make links footnotes instead of hotlinks:
\renewcommand{\href}[2]{#2\footnote{\url{#1}}}
\setlength{\parindent}{0pt}
\setlength{\parskip}{6pt plus 2pt minus 1pt}
\setlength{\emergencystretch}{3em}  % prevent overfull lines
\setcounter{secnumdepth}{0}

\title{Blog engine updates: Markdown Preprocessor \& Fay Scripts}
\author{Justin Le}
\date{January 27, 2014}

\begin{document}
\maketitle

\emph{Originally posted on
\textbf{\href{https://blog.jle.im/entry/blog-engine-updates-markdown-preprocessor-fay-scripts.html}{in
Code}}.}

\documentclass[]{}
\usepackage{lmodern}
\usepackage{amssymb,amsmath}
\usepackage{ifxetex,ifluatex}
\usepackage{fixltx2e} % provides \textsubscript
\ifnum 0\ifxetex 1\fi\ifluatex 1\fi=0 % if pdftex
  \usepackage[T1]{fontenc}
  \usepackage[utf8]{inputenc}
\else % if luatex or xelatex
  \ifxetex
    \usepackage{mathspec}
    \usepackage{xltxtra,xunicode}
  \else
    \usepackage{fontspec}
  \fi
  \defaultfontfeatures{Mapping=tex-text,Scale=MatchLowercase}
  \newcommand{\euro}{€}
\fi
% use upquote if available, for straight quotes in verbatim environments
\IfFileExists{upquote.sty}{\usepackage{upquote}}{}
% use microtype if available
\IfFileExists{microtype.sty}{\usepackage{microtype}}{}
\usepackage[margin=1in]{geometry}
\ifxetex
  \usepackage[setpagesize=false, % page size defined by xetex
              unicode=false, % unicode breaks when used with xetex
              xetex]{hyperref}
\else
  \usepackage[unicode=true]{hyperref}
\fi
\hypersetup{breaklinks=true,
            bookmarks=true,
            pdfauthor={},
            pdftitle={},
            colorlinks=true,
            citecolor=blue,
            urlcolor=blue,
            linkcolor=magenta,
            pdfborder={0 0 0}}
\urlstyle{same}  % don't use monospace font for urls
% Make links footnotes instead of hotlinks:
\renewcommand{\href}[2]{#2\footnote{\url{#1}}}
\setlength{\parindent}{0pt}
\setlength{\parskip}{6pt plus 2pt minus 1pt}
\setlength{\emergencystretch}{3em}  % prevent overfull lines
\setcounter{secnumdepth}{0}


\begin{document}

I spent some time over the past week writing a preprocessor for the entry copy
markdowns and getting Fay to deploy some simple scripts.

The need for a preprocessor was sparked by a post I'm writing that sort of
necessitated the features. I write [all of my
posts](https://github.com/mstksg/inCode/tree/master/copy/entries) in markdown,
and it all integrated well with the preprocessor. In addition I needed some
javascript scripting to make the preprocessor actions worthwhile, so I buckled
down and wrestled with getting Fay to work in a production environment. So I
guess this post is to show off some new features of the blog engine?

## Demonstration

Here it is in action:

    > !!!monad-plus/WolfGoatCabbage.hs "findSolutions ::" "makeMove ::" wolf-goat-cabbage

yields:

``` haskell
-- source: https://github.com/mstksg/inCode/tree/master/code-samples/monad-plus/WolfGoatCabbage.hs#L28-L45
-- interactive: https://www.fpcomplete.com/user/jle/wolf-goat-cabbage

findSolutions :: Int -> [Plan]
findSolutions n = do
    p <- makeNMoves
    guard $ isSolution p
    return p
    where
        makeNMoves = iterate (>>= makeMove) (return startingPlan) !! n

makeMove :: Plan -> [Plan]
makeMove p = do
    next <- MoveThe <$> [Farmer ..]
    guard       $ moveLegal p next
    guard . not $ moveRedundant p next
    let
        p' = p ++ [next]
    guard $ safePlan p'
    return p'
```

(If you're reading this on the actual website, mouse over or click them to see
the full effect, or if nothing is happening, try a hard refresh --- CTRL+SHIFT+R
in Chrome --- to clear out the cache)

## Code quoting/linking preprocessor

So I find myself writing a lot of sample code for my posts, and then later
copying and pasting them over to the
[code-samples](https://github.com/mstksg/inCode/tree/master/code-samples)
directory on the github in order to allow people to download them...and then
later awkwardly putting a link saying "download these here!" afterwards and
taking up space. Also linking to a live
[FPComplete](https://www.fpcomplete.com/) version is a bit awkward too, right
after the block.

I was rather inspired by the interface on the code blocks for [luite's
blog](http://weblog.luite.com/wordpress/?p=127), where every relevant code block
has a little link box on the top right hand corner linking to the source and a
working/running example.

So I wrote a Haskell preprocessor to take in a specification of a code file and
a what blocks in the code file to load, and then load it into the markdown file
before it is processed by pandoc.

The syntax is:

    > !!!path/to/code "keyword" "limited"n live_link

Where "keyword" is the text in the line to match for, `n` is the number of lines
after the keyword to display (if left off, it takes the next "block", or the
next continuous piece of code before a new non-indented line), and `live_link`
is a link to the live/interactive version on FPComplete.

### Reflections

So...writing the parser for the syntax specification was pretty easy due to
parsec and parser combinators:

``` haskell
-- source: https://github.com/mstksg/inCode/tree/master/code-samples/source/EntryPP.hs#L32-L127

data SampleSpec = SampleSpec  { sSpecFile       :: FilePath
                              , _sSpecLive      :: Maybe String
                              , _sSpecKeywords  :: [(String,Maybe Int)]
                              } deriving (Show)

sampleSpec :: Parser SampleSpec
sampleSpec = do
    filePath <- noSpaces <?> "sample filePath"
    spaces
    keywords <- many $ do
      keyword <- char '"' *> manyTill anyChar (char '"') <?> "keyword"
      keylimit <- optionMaybe (read <$> many1 digit <?> "keyword limit")
      spaces
      return (keyword,keylimit)

    live <- optionMaybe noSpaces <?> "live url"
    let
      live' = mfilter (not . null) live

    return $ SampleSpec filePath live' keywords
  where
    noSpaces = manyTill anyChar (space <|> ' ' <$ eof)
```

The code to actually find the right code block to paste was complicated and
horrifying at first, but after I sat down and really sorted out the logic, it
wasn't too bad. Still, it isn't the cleanest code in the world and I wonder how
I could have made it better, either with a Haskell library or even another
language.

This all left two little comment lines before the source code insert with the
link to the source and interactive versions.

All that was left was a front-end script to get the comments and turn them into
floating divs.

## Fay

Ah okay, here was the fun part.

Fay is actually pretty fun to use. And while it was perhaps complete overkill to
use the entire Fay runtime and build system for just a simple script, but I was
pretty inspired by [ocharles's post on
fay](http://ocharles.org.uk/blog/posts/2013-12-23-24-days-of-hackage-fay.html)
and I thought this would be a good time to get to learn it.

So there was a lot of cognitive friction going in, and trying to really get in
the groove took a few days. There was also a rather unhelpful error message
involving the ffi that I was able to bring up to the maintainers and be a part
of getting the fix working.

I converted as much of my current scripts as I could to fay. There was one that
I couldn't --- a function call to a library that required a javascript object of
function callbacks, and I couldn't really get that to work cleanly and I decided
it wasn't worth the effort for now --- maybe another day. If anything I could
re-write the entire library (a Table of Contents generator) myself some day.

### Reflections

#### fay-jquery

Here is a characteristic example of fay code with
[fay-jquery](http://hackage.haskell.org/package/fay-jquery-0.6.0.2) (0.6.0.2):

``` haskell
-- source: https://github.com/mstksg/inCode/tree/master/code-samples/source/entry.hs#L45-L54

appendTopLinks :: Fay ()
appendTopLinks = do
  mainContent <- select ".main-content"
  headings <- childrenMatching "h2,h3,h4,h5" mainContent
  J.append topLink headings
  topLinks <- select ".top-link"
  click (scrollTo 400) topLinks
  return ()
  where
    topLink = "<a href='#title' class='top-link'>top</a>"
```

As you can see, some of the method calls in fay-jquery seem a bit backwards...I
had to resist the urge to write things like

``` haskell
container `append` contained
container `childrenMatching` ".contained"
```

Which matches the JQuery calling model:

``` javascript
container.append(contained);
container.children('.contained');
```

Unfortunately, this doesn't work, and you're supposed to reverse the order of
the parameters. I guess it is more Haskell-y in a way, to be able to play with
partial application and do something like

``` haskell
let
  appendIt = append container
in
  appendIt contained1
  appendIt contained2
  appendIt contained3
```

So I guess that's okay.

However, something I was less understanding of was the ordering for event
binding and loops, which needed the handlers *before* the object being binded.

``` haskell
flip click header $ \_ -> do
  toggled <- readFayRef sourceToggled
  if toggled
    then sHide sourceInfo
    else unhide sourceInfo
  modifyFayRef' sourceToggled Prelude.not
```

This one kind of bucks the convention that methods like `append` maintain...and
also needs those annoying `flips` to have easy anonymous callbacks. I don't want
to have to name every little thing. Oh well. Maybe there is a good justification
here? I just don't see it. But then again, there is a reason why we have both
`mapM` and `forM` in base.

Other than that, the fay-jquery library is a pretty good example of how to
interface seamlessly with JQuery from Fay. Sometimes, though, the dynamic nature
of JQuery (implicit lists, dynamic type of returns, etc) was a little
unsettling...but that's the nature of JQuery. Perhaps working directly with the
DOM would alleviate this --- there's
[fay-dom](http://hackage.haskell.org/package/fay-dom) out there, but I didn't
get a chance to give it a try.

#### Deploying fay

Deploying fay ain't all too bad. I [deploy
binaries](http://blog.jle.im/entry/deploying-medium-to-large-haskell-apps-to-heroku),
however, so I was unable to ever process fay on my limited-access production
server because it requires `ghc-pkg` (installed under `/usr/local/bin`) among
other things...I probably could have gotten this to work, but I did not have the
proper skills. You also need to provide the binaries and headers in `share` for
all of your fay libraries in order to use them when compiling to javascript. So
while this isn't so bad if you have the whole Haskell Platform and are compiling
on your production server, I had to pre-compile my fay "binaries" before
pushing...just like I have to pre-compile my regular binaries, interestingly
enough.

Of course, the fay javascript files were a bit larger than the normal javascript
ones. Not too significantly, though, only about 80x. This actually puts them
however at around the size of my image files (\~100KB)...this is slightly
worrisome, but I don't really stress too much about one image, so I guess I
shouldn't stress too much about this either. Not ideal, but what else could I
expect?

## Future stuff

Hopefully I'm able to make [that javascript
call](http://blog.jle.im/source/code-samples/source/entry_toc.js#L4-21) on fay
one day, without having to rewrite the entire library in Fay (although it might
be a fun exercise).

If anyone knows how I can do this, I'd really appreciate any help!

I'd also in the future like to make my preprocessor a bit more robust and also
take more languages to determine the right comment syntax. But...I probably
wouldn't do this until the need actually arises :)

# Signoff

Hi, thanks for reading! You can reach me via email at <justin@jle.im>, or at
twitter at [\@mstk](https://twitter.com/mstk)! This post and all others are
published under the [CC-BY-NC-ND
3.0](https://creativecommons.org/licenses/by-nc-nd/3.0/) license. Corrections
and edits via pull request are welcome and encouraged at [the source
repository](https://github.com/mstksg/inCode).

If you feel inclined, or this post was particularly helpful for you, why not
consider [supporting me on Patreon](https://www.patreon.com/justinle/overview),
or a [BTC donation](bitcoin:3D7rmAYgbDnp4gp4rf22THsGt74fNucPDU)? :)

\end{document}

\end{document}
