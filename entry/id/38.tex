\documentclass[]{article}
\usepackage{lmodern}
\usepackage{amssymb,amsmath}
\usepackage{ifxetex,ifluatex}
\usepackage{fixltx2e} % provides \textsubscript
\ifnum 0\ifxetex 1\fi\ifluatex 1\fi=0 % if pdftex
  \usepackage[T1]{fontenc}
  \usepackage[utf8]{inputenc}
\else % if luatex or xelatex
  \ifxetex
    \usepackage{mathspec}
    \usepackage{xltxtra,xunicode}
  \else
    \usepackage{fontspec}
  \fi
  \defaultfontfeatures{Mapping=tex-text,Scale=MatchLowercase}
  \newcommand{\euro}{€}
\fi
% use upquote if available, for straight quotes in verbatim environments
\IfFileExists{upquote.sty}{\usepackage{upquote}}{}
% use microtype if available
\IfFileExists{microtype.sty}{\usepackage{microtype}}{}
\usepackage[margin=1in]{geometry}
\ifxetex
  \usepackage[setpagesize=false, % page size defined by xetex
              unicode=false, % unicode breaks when used with xetex
              xetex]{hyperref}
\else
  \usepackage[unicode=true]{hyperref}
\fi
\hypersetup{breaklinks=true,
            bookmarks=true,
            pdfauthor={Justin Le},
            pdftitle={Fixed-Length Vector Types in Haskell, 2015},
            colorlinks=true,
            citecolor=blue,
            urlcolor=blue,
            linkcolor=magenta,
            pdfborder={0 0 0}}
\urlstyle{same}  % don't use monospace font for urls
% Make links footnotes instead of hotlinks:
\renewcommand{\href}[2]{#2\footnote{\url{#1}}}
\setlength{\parindent}{0pt}
\setlength{\parskip}{6pt plus 2pt minus 1pt}
\setlength{\emergencystretch}{3em}  % prevent overfull lines
\setcounter{secnumdepth}{0}

\title{Fixed-Length Vector Types in Haskell, 2015}
\author{Justin Le}
\date{May 5, 2015}

\begin{document}
\maketitle

\emph{Originally posted on
\textbf{\href{https://blog.jle.im/entry/fixed-length-vector-types-in-haskell-2015.html}{in
Code}}.}

\documentclass[]{}
\usepackage{lmodern}
\usepackage{amssymb,amsmath}
\usepackage{ifxetex,ifluatex}
\usepackage{fixltx2e} % provides \textsubscript
\ifnum 0\ifxetex 1\fi\ifluatex 1\fi=0 % if pdftex
  \usepackage[T1]{fontenc}
  \usepackage[utf8]{inputenc}
\else % if luatex or xelatex
  \ifxetex
    \usepackage{mathspec}
    \usepackage{xltxtra,xunicode}
  \else
    \usepackage{fontspec}
  \fi
  \defaultfontfeatures{Mapping=tex-text,Scale=MatchLowercase}
  \newcommand{\euro}{€}
\fi
% use upquote if available, for straight quotes in verbatim environments
\IfFileExists{upquote.sty}{\usepackage{upquote}}{}
% use microtype if available
\IfFileExists{microtype.sty}{\usepackage{microtype}}{}
\usepackage[margin=1in]{geometry}
\ifxetex
  \usepackage[setpagesize=false, % page size defined by xetex
              unicode=false, % unicode breaks when used with xetex
              xetex]{hyperref}
\else
  \usepackage[unicode=true]{hyperref}
\fi
\hypersetup{breaklinks=true,
            bookmarks=true,
            pdfauthor={},
            pdftitle={},
            colorlinks=true,
            citecolor=blue,
            urlcolor=blue,
            linkcolor=magenta,
            pdfborder={0 0 0}}
\urlstyle{same}  % don't use monospace font for urls
% Make links footnotes instead of hotlinks:
\renewcommand{\href}[2]{#2\footnote{\url{#1}}}
\setlength{\parindent}{0pt}
\setlength{\parskip}{6pt plus 2pt minus 1pt}
\setlength{\emergencystretch}{3em}  % prevent overfull lines
\setcounter{secnumdepth}{0}


\begin{document}

***Update***: This post was written by me when I was just starting to learn
about type-level things in Haskell, and reflects my own inexperience at the time
of writing it. I have [released an
update](https://blog.jle.im/entry/fixed-length-vector-types-in-haskell.html),
which presents what I hope to be an introduction that is more grounded in modern
Haskell and dependent type idioms.

## Original Article (written in 2015)

Fixed-length vector types (vector types that indicate the length of the vector
in the type itself) are one of the more straightforward applications of the
"super-Haskell" GHC type extensions. There's a lot of magic you can do with
GHC's advanced type mechanisms, but I think fixed length vectors are a good
first step to beginning to understand several extensions, including
(potentially):

-   ConstraintKinds
-   DataKinds
-   GADTs
-   KindSignatures
-   TypeFamilies
-   TypeOperators
-   OverloadedLists

And using type system plugins. (And of course the usual `UndecidableInstances`
etc.) We'll be discussing two different ways to implement this --- using
type-level nats, and using the *GHC.TypeLits* model to actually be able to use
numeric literals in your types. These things are seen in the wild like with the
popular
*[linear](http://hackage.haskell.org/package/linear-1.18.0.1/docs/Linear-V.html)*
package's `V` type.

There are a few great tutorials/writeups on this topic, but many of them are
from the time before we had some of these extensions, or only discuss a few. I
hope to provide a nice comprehensive look about the tools available today to
really approach this topic. That being said, I am no expert myself, so I would
appreciate any tips/edits/suggestions for things that I've missed or done
not-the-best :) This post has a lot of open questions that I'm sure people who
know more about this than me can answer.

Most of the code in this article can be [downloaded and tried
out](https://github.com/mstksg/inCode/blob/master/code-samples/fixvec), so
follow along if you want!

## The Idea

The basic idea is we'll have a type:

``` haskell
Vec n a
```

Which is a vector with items of type `a`, whose length is somehow encoded in the
`n`. We'll then discuss ways to do useful operations on this, as if it were a
list.

`n` can really only be a certain "kind" of thing --- a type that encodes a
length. We can represent this by giving it a "kind signature":

``` haskell
data Vec :: Nat -> * -> *
```

Which says that our `Vec` type constructor takes two arguments: something of
kind `Nat` (so it can't be any type...it has to be a type of kind `Nat`),
something of kind `*` (the "normal" kind, of things that have values, like
`Int`, `Maybe Bool`, etc.), and returns something of kind `*` (our vector
itself).

## Using DataKinds for Type-Level Nats

(The code in this section for this type is [available
online](https://github.com/mstksg/inCode/tree/master/code-samples/fixvec/FVTypeNats.hs),
if you wanted to play along!)

There are a couple of ways to find something for that `n` `Nat` kind, and one
way is to use the simple inductive `Nat`:

``` haskell
-- source: https://github.com/mstksg/inCode/tree/master/code-samples/fixvec/FVTypeNats.hs#L26-L27

data Nat = Z | S Nat
         deriving Show
```

You might have seen this type before...it gives us value-level natural numbers,
where `Z` is zero, `S Z` is one, `S (S Z)` is two, `S (S (S Z))` is three, etc.
So if we had something of type `Nat`, it could represent any natural number.
This declaration gives you:

-   A type `Nat`
-   A value constructor `Z :: Nat`
-   A value constructor `S :: Nat -> Nat`

However, with the *DataKinds* extension, when you define this, you also define
some extra fancy things. You also define a *kind* `Nat`! More specifically, you
get:

-   A kind `Nat`
-   A type `Z :: Nat` (`Z`, of *kind* `Nat`)
-   A type constructor `S :: Nat -> Nat` (`S`, which takes something of kind
    `Nat`, and returns a new thing of kind `Nat`)

(Note that, to be principled, GHC would prefer us to use `'Z` and `'S` when we
are referring to the *types*, and this is how it'll print them out in error
messages. But we're going to run with this for now...mostly for aesthetic
reasons)

We can check this out in GHCi:

``` haskell
ghci> :set -XDataKinds
ghci> data Nat = Z | S Nat
ghci> :k Z
Nat
ghci> :k S Z
Nat
ghci> :k S (S Z)
Nat
```

So now we have a *type* that can encode numbers. Something of type `Z`
represents zero...something of type `S Z` represents 1...something of type
`S (S Z)` represents two.

Note that you can't ever have anything like `S Bool`...that doesn't work,
because `Bool` is of kind `*`, but `S` expects only `Nat`s.

Now we can make our `Vec` data type, with the *GADTs* extension, or "generalized
algebraic data types":

``` haskell
-- source: https://github.com/mstksg/inCode/tree/master/code-samples/fixvec/FVTypeNats.hs#L37-L44

data Vec :: Nat -> * -> * where
    Nil  :: Vec Z a
    (:#) :: a -> Vec n a -> Vec (S n) a

infixr 5 :#

deriving instance Show a => Show (Vec n a)
deriving instance Eq a => Eq (Vec n a)
```

If you've never seen GADTs before, think of it as a way of declaring a type by
giving the type of your constructors instead of just the normal boring form.
It's nothing too crazy...it's basically like defining `Maybe` as:

``` haskell
data Maybe :: * -> * where
    Nothing :: Maybe a
    Just    :: a -> Maybe a
```

instead of

``` haskell
data Maybe a = Nothing | Just a
```

In both cases, they create constructors of type `Nothing :: Maybe a` and
`Just :: a -> Maybe a` anyway...so the GADT form just gives us a way to state it
explicitly.

Oh, we also used the *KindSignatures* extension to be able to give a kind
signature to `Vec`...this is important because we want to make sure the first
argument has to be a `Nat`. That is, we can't have anything silly like
`Vec Bool Int`. We also have to put a separate *StandaloneDeriving*-extension
standalone deriving clause instead of just having `deriving Show` because `Vec`
isn't a type that can be expressed in "normal Haskell".

Note that our type is basically like a list:

``` haskell
data [] :: * -> * where
    []  :: [a]
    (:) :: a -> [a] -> [a]
```

Except now our type constructor actually has a new `Nat`

This means that, because of type erasure, everything "runtime" on our new type
is basically going to be identical to `[]` (not considering compiler tricks).
In-memory, this new type is essentially exactly `[]`, but its type has an extra
tag that is erased at compile-time.

Okay, let's define some useful methods:

``` haskell
-- source: https://github.com/mstksg/inCode/tree/master/code-samples/fixvec/FVTypeNats.hs#L93-L97

headV :: Vec (S n) a -> a
headV (x :# _)  = x

tailV :: Vec (S n) a -> Vec n a
tailV (_ :# xs) = xs
```

Ah, the classic `head`/`tail` duo from the days pre-dating Haskell. `head` and
`tail` are somewhat of a sore spot or wart in Haskell's list API[^1], because
they're *partial functions*. You tell people all about how Haskell is great
because it can prevent run-time errors by ensuring completeness and having the
type system enforce null-pointer checks...but then you go ahead and put unsafe
functions that throw errors for empty lists anyways in Prelude.

But here...this will never happen! We can only use `headV` and `tailV` on
non-empty lists...it won't typecheck for empty lists. Do you see why?

It's because all empty lists are of type `Vec Z a`. But `headV` and `tailV` only
take things of *type* `Vec (S n) a`, for any `Nat` `n`. So, if you ever try to
use it on an empty list, it won't even compile! No more pesky runtime bugs.
`headV` and `tailV` are safe and will never crash at runtime!

Note that the return type of `tailV` is a vector of a length one less than the
given vector. `tailV :: Vec (S Z) a -> Vec Z a`, for instance...or
`tailV :: Vec (S (S Z)) a -> Vec (S Z) a`. Just like we want!

If you tried implementing this yourself, you might notice that you actually get
an *error* from GHC if you even *try* to handle the `Nil` case for `tailV` or
`headV`. GHC will know when you've handled all possible cases, and get mad at
you if you try to handle a case that doesn't even make sense!

### Type families and appending

We can also "append" vectors. But we need a way to add `Nat`s together first.
For that, we can use a type family, using the *TypeFamilies* extension (with
`TypeOperators`):

``` haskell
-- source: https://github.com/mstksg/inCode/tree/master/code-samples/fixvec/FVTypeNats.hs#L29-L31

type family (x :: Nat) + (y :: Nat) where
    'Z   + y = y
    'S x + y = 'S (x + y)
```

A "type family" is like a type level function. Compare this to defining `(+)` on
the value level to the `Nat` *data* type:

``` haskell
-- source: https://github.com/mstksg/inCode/tree/master/code-samples/fixvec/FVTypeNats.hs#L33-L35

(+#) :: Nat -> Nat -> Nat       -- types!
Z   +# y = y
S x +# y = S (x +# y)
```

Basically, we're defining a new type-level function `(+)` on two types `x` and
`y`, both of kind `Nat`...and the result is their "sum". Convince yourself that
this "addition" is actually addition. Now, let's use it for `appendV`:

``` haskell
-- source: https://github.com/mstksg/inCode/tree/master/code-samples/fixvec/FVTypeNats.hs#L99-L101

appendV :: Vec n a -> Vec m a -> Vec (n + m) a
appendV Nil       ys = ys
appendV (x :# xs) ys = x :# appendV xs ys
```

``` haskell
ghci> let v1 = 1 :# 2 :# 3 :# Nil
ghci> let v2 = 0 :# 1 :# Nil
ghci> v1 `appendV` v2
1 :# 2 :# 3 :# 0 :# 1 :# Nil
ghci> :t v1 `appendV` v2
v1 `appendV` v2 :: Vec (S (S (S (S (S Z))) Int
```

### Generating

It'd be nice to have type-safe methods of *generating* these things,
too...functions like `iterate`, or `enumFrom`. One of the ways to do this is by
using a typeclass. (Available in a [separate
file](https://github.com/mstksg/inCode/tree/master/code-samples/fixvec/Unfoldable.hs)
to try out).

``` haskell
-- source: https://github.com/mstksg/inCode/tree/master/code-samples/fixvec/Unfoldable.hs#L7-L8

class Unfoldable v where
    unfold :: (b -> (a, b)) -> b -> v a
```

We're going to call `v` an `Unfoldable` if you can build a `v` from an
"unfolding function" and an "initial state". Run the function on the initial
value and get the first item and a new state. Run the function on the new state
and get the second item and the next state.

The list instance should make it more clear:

``` haskell
-- source: https://github.com/mstksg/inCode/tree/master/code-samples/fixvec/Unfoldable.hs#L11-L13

instance Unfoldable [] where
    unfold f x0 = let (y, x1) = f x0
                  in  y : unfold f x1
```

``` haskell
ghci> take 5 $ unfold (\x -> (x `mod` 3 == 2, x^2 - 1)) 2
[True, False, True, False, True]
```

Note that we can have an instance for any fixed-length vector type...where the
thing "cuts off" after it's filled the entire vector:

``` haskell
-- source: https://github.com/mstksg/inCode/tree/master/code-samples/fixvec/FVTypeNats.hs#L46-L51

instance Unfoldable (Vec Z) where
    unfold _ _ = Nil

instance Unfoldable (Vec n) => Unfoldable (Vec (S n)) where
    unfold f x0 = let (y, x1) = f x0
                  in  y :# unfold f x1
```

Take a moment to think about what these instances are doing.

You can create a `Vec Z a` from an unfolding function pretty easily, because the
only thing with type `Vec Z a` is `Nil`. So just ignore the function/initial
state and return `Nil`.

The instance for `Vec (S n)` is slightly more involved. To make a `Vec (S n) a`,
you need an `a` and a `Vec n a`. You can get the `a` from the unfolding
function...but where will you get the `Vec n a` from? Well, you can use `unfold`
to make a `Vec n a`! But that only makes sense if `Vec n` is an `Unfoldable`.

So, that's why in the instance for `Vec (S n)`, we constrain that `Vec n` must
also be an `Unfoldable`. We make our result by using our function to create an
`a` and `unfold` to create a `Vec n a` (provided `Vec n` is an `Unfoldable`).

Note that this style of declaration looks a lot like induction. We define our
instance for zero...and then we say, "if `n` is an instance, then so is `S n`".
Induction!

Let's see this in action.

``` haskell
-- source: https://github.com/mstksg/inCode/tree/master/code-samples/fixvec/Unfoldable.hs#L15-L24

replicateU :: Unfoldable v => a -> v a
replicateU = unfold (\x -> (x, x))

iterateU :: Unfoldable v => (a -> a) -> a -> v a
iterateU f = unfold (\x -> (x, f x))

fromListMaybes :: Unfoldable v => [a] -> v (Maybe a)
fromListMaybes = unfold $ \l -> case l of
                                  []   -> (Nothing, [])
                                  x:xs -> (Just x , xs)
```

``` haskell
ghci> replicateU 'a'       :: Vec (S (S (S Z))) Char
'a' :# 'a' :# 'a' :# Nil
ghci> replicateU 'a'       :: Vec Z Char
Nil
ghci> iterateU succ 1      :: Vec (S (S (S (S Z)))) Int
1 :# 2 :# 3 :# 4 :# Nil
ghci> fromListMaybes [1,2] :: Vec (S (S (S Z))) (Maybe Int)
Just 1 :# Just 2 :# Nothing :# Nil
ghci> tailV (iterateU succ 1 :: Vec (S Z) Int)
Nil
```

Note that `replicateU` doesn't need to take in an `Int` parameter, like the on
in Prelude, to say how many items to have. It just replicates enough to fill the
entire vector we want!

### Common Typeclasses

We can go in and implement common typeclasses, too. All the ones you'd expect.

We can actually use the *DeriveFunctor* extension to write a `Functor` instance,
but let's write one on our own just for learning purposes:

``` haskell
-- source: https://github.com/mstksg/inCode/tree/master/code-samples/fixvec/FVTypeNats.hs#L53-L55

instance Functor (Vec n) where
    fmap _ Nil       = Nil
    fmap f (x :# xs) = f x :# fmap f xs
```

For `Applicative`, it isn't so simple. The Applicative instance is going to be
the "ZipList" instance...so we have to be able to make a `pure` that depends on
the type, and a `(<*>)` that depends on the type, too.

``` haskell
-- source: https://github.com/mstksg/inCode/tree/master/code-samples/fixvec/FVTypeNats.hs#L57-L63

instance Applicative (Vec Z) where
    pure _    = Nil
    Nil <*> _ = Nil

instance Applicative (Vec n) => Applicative (Vec (S n)) where
    pure x = x :# pure x
    (f :# fs) <*> (x :# xs) = f x :# (fs <*> xs)
```

For `Vec Z`, it's just `Nil`. For `Vec (S n)`...for pure, you need `x :#`
something...and that something has to be a `Vec n a`. That's just `pure` for
`Vec n`! Remember, we can't assume that `Vec n` is an `Applicative` just because
`Vec (S n)` is. So we need to add a constraint, that `Vec n` an Applicative.
Induction, again!

For `(<*>)`, we can get the first item easily, it's just `f x`. But for the next
item, we need a `Vec n a`. Luckily...we have exactly that with the `(<*>)` for
`Vec n`!

Remember, at the end, we're saying "We have an `Applicative` instance for *any*
type `Vec n`". The instance for `Vec Z` has `pure _ = Nil`. The instance for
`Vec (S Z)` has `pure x = x :# Nil`. The instance for `Vec (S (S Z))` has
`pure x = x :# x :# Nil`, etc. etc.

``` haskell
ghci> fmap (*2) (1 :# 2 :# 3 :# Nil)
2 :# 4 :# 6 :# Nil
ghci> pure 10 :: Vec (S (S Z)) Int
10 :# 10 :# Nil         -- like replicateV!
ghci> liftA2 (+) (1 :# 2 :# 3 :# Nil) (100 :# 201 :# 302 :# Nil)
101 :# 203 :# 305 :# Nil
```

I'll leave the `Monad` instance as an exercise, but it's in the source files for
this post. `join` for this instance should be a "diagonal" --- the first item of
the first vector, the second item of the second vector, the third item of the
third vector, etc.

We can define `Foldable` and `Traversable` the same way. Like for `Functor`, GHC
can derive these with *DeriveFoldable* and *DeriveTraversable*...but we'll do it
again here just to demonstrate.

``` haskell
-- source: https://github.com/mstksg/inCode/tree/master/code-samples/fixvec/FVTypeNats.hs#L65-L75

instance Foldable (Vec Z) where
    foldMap _ Nil = mempty

instance Foldable (Vec n) => Foldable (Vec (S n)) where
    foldMap f (x :# xs) = f x <> foldMap f xs

instance Traversable (Vec Z) where
    traverse _ Nil = pure Nil

instance Traversable (Vec n) => Traversable (Vec (S n)) where
    traverse f (x :# xs) = liftA2 (:#) (f x) (traverse f xs)
```

Note that we can only use `foldMap f xs` on `xs :: Vec n a`, if `Vec n` is a
`Foldable`. So that's why we add that constraint.

Again,
`liftA2 (:#) :: Applicative f => f a -> f (Vec n a) -> f (Vec (S n) a)`...so
this only makes sense if `traverse f s` gives us a `Vec n a`. So we have to add
that as a constraint.

``` haskell
ghci> toList $ 1 :# 2 :# 3 :# Nil
[1,2,3]
ghci> traverse Identity $ 1 :# 2 :# 3 :# Nil
Identity (1 :# 2 :# 3 :# Nil)
ghci> sequence_ $ putStrLn "hello" :# putStrLn "world" :# Nil
"hello"
"world"
ghci> sequence $ Just 1 :# Just 2 :# Nil
Just (1 :# 2 :# Nil)
ghci> sequence $ Just 1 :# Nothing :# Nil
Nothing
```

`Traversable` of course opens a whole lot of doors. For example, we can write a
"safe `fromList`":

``` haskell
-- source: https://github.com/mstksg/inCode/tree/master/code-samples/fixvec/Unfoldable.hs#L26-L27

fromListU :: (Unfoldable v, Traversable v) => [a] -> Maybe (v a)
fromListU = sequence . fromListMaybes
```

``` haskell
ghci> fromListU [1,2,3] :: Maybe (Vec (S Z) Int)
Just (1 :# Nil)
ghci> fromListU [1,2,3] :: Maybe (Vec (S (S (S Z))) Int)
Just (1 :# 2 :# 3 :# Nil)
ghci> fromListU [1,2,3] :: Maybe (Vec (S (S (S (S Z)))) Int)
Nothing
```

And, if you're on GHC 7.8+, you have access to the *OverloadedLists* language
extension, where you can interpret list literals as if they were other
structures.

We've already already implemented both `fromList` and `toList`, in a way,
already, so this should be a breeze. The only trick you might see is that the
`IsList` typeclass asks for a type family to return the *type of the element in
the container* from the container type.

``` haskell
-- source: https://github.com/mstksg/inCode/tree/master/code-samples/fixvec/FVTypeNats.hs#L86-L91

instance (Unfoldable (Vec n), Traversable (Vec n)) => L.IsList (Vec n a) where
    type Item (Vec n a) = a
    fromList xs = case fromListU xs of
                    Nothing -> error "Demanded vector from a list that was too short."
                    Just ys -> ys
    toList      = Data.Foldable.toList
```

``` haskell
ghci> :set -XOverloadedLists
ghci> [1,2,3] :: Vec (S (S Z)) Int
1 :# 2 :# Nil
ghci> [1,2,3] :: Vec (S (S (S (S Z)))) Int
*** Exception: Demanded vector from a list that was too short.
ghci> [1,3..] :: Vec (S (S (S (S Z)))) Int
1 :# 3 :# 5 :# 7 :# Nil
```

Neat! All of the benefits of list literals that *OverloadedLists* offers is now
available to us.[^2] Unfortunately, you now open yourself up to runtime errors,
so...it's actually a really bad idea for safety purposes unless you stick to
only using it with infinite lists or are very disciplined. (Unless you really
want to use list syntax, `fromListU` is probably a safer choice for finite
lists!)

### Indexing

It'd be nice to be able to index into these, of course. For type-safe indexing,
we can take advantage of a trick using the `Proxy` type.

Many might remember having to get a `TypeRep` for a `Typeable` instance by doing
something like `typeOf (undefined :: IO Double)`. That's because
`typeOf :: Typeable a => a -> TypeRep`. If you wanted to get the `typeRep` for
an `IO Double` using `typeOf`, you have to pass in an `IO Double`. But if you
don't have one at hand, you can just use `undefined` with a type annotation.
It's a bit of a dirty hack, but it works because `typeOf` doesn't care about the
first argument's value...just its type.

These days, we like to be a bit less embarrassing and use something called
`Proxy`:

``` haskell
data Proxy a = Proxy
```

`Proxy a` is a bit like `()`. It only has one constructor, and doesn't take any
arguments. But we can use the type signature to "pass in types" to functions, as
"arguments".

We have a couple of options here. One is to make a typeclass for type level nats
to turn them into an `Integer` or a value-level `Nat`, and then do an "unsafe
indexing" after verifying, through types, that the index is smaller than the
length.

However, this is a little bit silly because we're just doing an unsafe indexing
in the end anyway, so the compiler can't help us at all. Wouldn't it be nice if
we could get the compiler on our side and write a *real* safe index?

There are many ways to approach this problem, but one way is to make a specific
`Index` typeclass: (or make another typeclass like `Take`, and write `index` in
terms of it)

``` haskell
-- source: https://github.com/mstksg/inCode/tree/master/code-samples/fixvec/FVTypeNats.hs#L77-L78

class Index (n :: Nat) v where
    index :: Proxy n -> v a -> a
```

Here, we can say that `n` and `v` are instances of `Index n v` if and only if
you can safely (totally) index into `v a` at index `n`. That is, if every value
of type `v a` ever has an index at `n`, a `Nat`. (By the way, we need
*MultiParamTypeClasses* to be able to make a type class with two parameters)

So, `n ~ S Z` and `v ~ Vec (S (S Z)) a` has an instance, because you can get the
$n = 1$ element (the second element) from *any* value of type `Vec (S (S Z)) a`
(a length-two vector).

But `n ~ S Z` and `v ~ Vec (S Z) a` does *not*. There are actually *no* length-1
vectors that have a $1$ index (second element).

Note that we use the `Proxy` trick we discussed, so that we can indicate somehow
what index we really want. It is a trick that basically allows us to pass a
*type* (`S Z`, `S (S Z)`, etc.) as a "value".

Let's write our instances --- but only the instances that *make sense*.

``` haskell
-- source: https://github.com/mstksg/inCode/tree/master/code-samples/fixvec/FVTypeNats.hs#L80-L84

instance Index Z (Vec (S n)) where
    index _ (x :# _) = x

instance forall n m. Index n (Vec m) => Index (S n) (Vec (S m)) where
    index _ (_ :# xs) = index (Proxy :: Proxy n) xs
```

The first case instance makes sense. We can definitely index at index `Z` (zero)
of *any* `Vec (S n) a` --- the only thing we can't index `Z` into is `Vec Z a`.
So, if our vector is of length 1 or higher, we can index at position 0.

The second case says that, if we can index into `n` of a `Vec m a`, then of
course we can index into an `S n` of a `Vec (S m) a`. To index into `S n` of a
`Vec (S m) a`, all we need to do is index into `n` of the `Vec m a` tail!

We have to use the *ScopedTypeVariables* extension to enable us to use, with the
`forall` statement, the `n` in our instance when we are writing our type for
`Proxy`. If we didn't, the `n` in `Proxy n` in our `index` definition would be
considered unrelated by GHC to the `n` in the instance statement,
`Index (S n) (Vec (S m))`. (This is the only reason we need the `forall`)

In any case, note the similarity of this algorithm to the actual indexing
function on lists:

``` haskell
0 !! (x:_ ) = x
n !! (_:xs) = (n - 1) !! xs
```

trying it out...

``` haskell
ghci> index (Proxy :: Proxy (S (S Z))) (1 :# 2 :# 3 :# Nil)
3
ghci> index (Proxy :: Proxy (S (S Z))) (1 :# 2 :# Nil)
*** Compile error!
```

It's an error, but remember, it's a *compiler* error, that happens before any
code is ever even run! No more indexing errors at runtime! Kiss your days of
hunting segfault errors in C goodbye!

::: note
**Aside**

This is something I haven't really been able to find a good answer too. But
notice that we actually could have written a "bad" instance of the second
instance of `Index`:

``` haskell
instance Index (S n) (Vec (S m)) where
    index _ (x :# _) = x
```

And this compiles fine...but gives the wrong behavior, or at least the behavior
we don't want!

Does anybody know a way to state the type of `Index` or `index` in a way that
implementations like this are impossible?

There's a "fundamental" problem here, it seems, because we can't really demand
or specify anything by the return type, like we could in the other examples. In
the other examples, we sort of restricted the implementation by choosing our
return type carefully...but for here, it's just `a`. I'd love to hear if anyone
has any thoughts on this.
:::

You might notice that it's a bit of a plain to write `S (S (S (S Z)))`, etc.,
especially for large numbers. And I wouldn't even think about writing it for the
hundreds.

We'll "fix" this in the next section. However, even before this, you actually
can generate these "automatically" with template haskell, using techniques from
[Functional Pearls: Implicit
Configurations](http://www.cs.rutgers.edu/~ccshan/prepose/prepose.pdf), and the
[linear](http://hackage.haskell.org/package/linear-1.18.0.1/docs/Linear-V.html)
package does just this. (This path slipped my mind before I posted because I
didn't really consider template Haskell, and I think I'll edit in a section here
soon). With this in mind, I still don't really consider Template Haskell an
optimal or clean approach :)

## Using TypeLits and Type Checker Plugins

(This next section uses code that is [also available
online](https://github.com/mstksg/inCode/tree/master/code-samples/fixvec/FVTypeLits.hs),
as well!)

Using a custom `Nat` kind and *DataKinds* is nice and all, but it's a bit of a
hassle to express large numbers like 100, 1000, etc. However, as of GHC 7.8,
we've had the ability to actually *use* numeric (integer) literals in our types.
Instead of writing `S (S Z)`, we can write `2`.

GHC can't yet quite work with that well by default. It has trouble proving
statements about variables, like `(n + 1) ~ (1 + n)` (that `n + 1` is "the same
as" `1 + n`). Fortunately for us, since GHC 7.10, we have a way to "extend" the
type checker with custom plugins that *can* prove things like this for us. (Note
that this `+` is the one from `GHC.TypeLits`...not the one we defined earlier.)

The
*[ghc-typelits-natnormalise](https://hackage.haskell.org/package/ghc-typelits-natnormalise)*
package is a package providing such a plugin. We can have GHC use it to extend
its type checking by passing in `-fplugin GHC.TypeLits.Normalise` when we
execute our code, or by adding a pragma:

``` haskell
-- source: https://github.com/mstksg/inCode/tree/master/code-samples/fixvec/FVTypeLits.hs#L14-L14

{-# OPTIONS_GHC -fplugin GHC.TypeLits.Normalise #-}
```

to the top of our file, along with our `LANGUAGE` pragmas. (Assuming, of course,
a GHC 7.10+)

``` haskell
ghci> :set -XDataKinds -XTypeOperators -XTypeFamilies
ghci> import GHC.TypeLits
ghci> Proxy :: ((n + 1) ~ (1 + n)) => Proxy n
*** Compile error: Cannot match `1 + n` with `n + 1`
ghci> :set -fplugin GHC.TypeLits.Normalise
ghci> Proxy :: ((n + 1) ~ (1 + n)) => Proxy n
Proxy   -- success!
```

GHC now uses the plugin to prove that the two are really equivalent.

If you wanted to play along or try out the code samples, I recommend you use a
sandbox:

``` bash
# in directory of your choice
$ cabal sandbox init
$ cabal install ghc-typelits-natnormalise
$ cabal exec bash
# now the package is in scope, when you use ghci or runghc
```

With that in mind, let's start restating everything in terms of *TypeLits* and
see what it gains us.

``` haskell
-- source: https://github.com/mstksg/inCode/tree/master/code-samples/fixvec/FVTypeLits.hs#L33-L40

data Vec :: Nat -> * -> * where
    Nil  :: Vec 0 a
    (:#) :: a -> Vec (n - 1) a -> Vec n a

infixr 5 :#

deriving instance Show a => Show (Vec n a)
deriving instance Eq a => Eq (Vec n a)
```

A little nicer, right? `Nil` is a `Vec 0 a`, and `x :# xs` is an element with a
`Vec (n - 1) a`, which overall is a `Vec n a`. Let's go over everything again to
see how it'd look in the new regime. (Note that the kind of the type number
literals is also called `Nat`...unrelated to our `Nat` we used before.)

## A new look

First of all, we're going to have to define *TypeLit* comparison operators, as
they aren't built in in a useful way.

We have the type family (remember those?) `CmpNat x y`, which returns an
`Ordering` (`LT`, `EQ`, or `GT`) type (of kind `Ordering`, using
*DataKinds*...lifting a type and its value constructors to a kind and its
types), which is provided and defined for us by GHC in `GHC.TypeLits`.

So defining a `x > y` constraint is pretty straightforward:

``` haskell
-- source: https://github.com/mstksg/inCode/tree/master/code-samples/fixvec/FVTypeLits.hs#L31-L31

type x > y = CmpNat x y ~ 'GT
```

Note that we need the *ConstraintKinds* extension for this to work, as `1 > 2`
is now a *constraint*, of kind `Constraint`.

Given this, let's do our favorite list functions, `headV` and `tailV`:

``` haskell
-- source: https://github.com/mstksg/inCode/tree/master/code-samples/fixvec/FVTypeLits.hs#L89-L93

headV :: (n > 0) => Vec n a -> a
headV (x :# _)  = x

tailV :: (n > 0) => Vec n a -> Vec (n - 1) a
tailV (_ :# xs) = xs
```

Magnificent!

``` haskell
ghci> headV (Nil :: Vec 0 ())
-- Error!  Cannot unite 'EQ with 'GT
```

Neat! The error, remember, is at *compile time*, and not at runtime. If we ever
tried to do an unsafe head, our code wouldn't even *compile*! The error message
comes from the fact that we need $n > 0$, but we have $n = 0$ instead. We have
`EQ`, but we need `GT`.

There is one problem here, though --- GHC gives us a warning for not pattern
matching on `Nil`. But, if we do try to pattern match on `Nil`, we get a type
error, like the same one we got when using our custom type nats. I think this is
probably something that a plugin or sufficiently smart `CmpNat` might be able to
handle...but I'm not totally sure. Right now, the best thing I can think of is
just to do a wildcard match, `headV _ = error "What?"`, knowing that that case
will never be reached if your program compiles successfully.

Moving on, we see that we don't even have to do any extra work to define our own
type family `x + y`...because `GHC.TypeLits` already defines it for us! So, we
can instantly write....

``` haskell
-- source: https://github.com/mstksg/inCode/tree/master/code-samples/fixvec/FVTypeLits.hs#L95-L97

appendV :: Vec n a -> Vec m a -> Vec (n + m) a
appendV Nil       ys = ys
appendV (x :# xs) ys = x :# appendV xs ys
```

``` haskell
ghci> let v1 = 1 :# 2 :# 3 :# Nil
ghci> let v2 = iterateU succ 0 :: Vec 2 Int
ghci> v1 `appendV` v2
1 :# 2 :# 3 :# 0 :# 1 :# Nil
ghci> :t v1 `appendV` v2 :: Vec 5 Int
v1 `appendV` v2 :: Vec 5 Int
```

And our list generating typeclasses ---

``` haskell
-- source: https://github.com/mstksg/inCode/tree/master/code-samples/fixvec/FVTypeLits.hs#L42-L47

instance Unfoldable (Vec 0) where
    unfold _ _ = Nil

instance (Unfoldable (Vec (n - 1)), n > 0) => Unfoldable (Vec n) where
    unfold f x0 = let (y, x1) = f x0
                  in  y :# unfold f x1
```

The translation is pretty mechanical, but I think that this new formulation
looks...really nice, and really powerful. "If you can build a list from $n - 1$
and $n > 0$, then you can build a list for $n$!

Note that because our definitions of `replicateU`, `iterateU`, and
`fromListMaybes` was polymorphic over all `Unfoldable`, we can actually re-use
them from before:

``` haskell
ghci> iterateU succ 1 :: Vec 3 int
1 :# 2 :# 3 :# Nil
ghci> iterateU succ 1 :: Vec 10 Int
1 :# 2 :# 3 :# 4 :# 5 :# 6 :# 7 :# 8 :# 9 :# 10 :# Nil
ghci> replicateU 'a' :: Vec 4 Char
'a' :# 'a' :# 'a' :# 'a' :# Nil
```

The actual types are much nicer, too --- we can write `Vec 10 Int` instead of
`Vec (S (S (S (S (S (S (S (S (S (S Z)))))))))) Int` without resorting to
template haskell.

Going through all of our other typeclasses/functions and making the
adjustments... (remembering that we can also derive `Functor`, `Traversable`,
and `Foldable` using GHC)

``` haskell
-- source: https://github.com/mstksg/inCode/tree/master/code-samples/fixvec/FVTypeLits.hs#L49-L87

instance Functor (Vec n) where
    fmap _ Nil       = Nil
    fmap f (x :# xs) = f x :# fmap f xs

instance Applicative (Vec 0) where
    pure _    = Nil
    Nil <*> _ = Nil

instance (Applicative (Vec (n - 1)), n > 0) => Applicative (Vec n) where
    pure x = x :# pure x
    (f :# fs) <*> (x :# xs) = f x :# (fs <*> xs)

instance Foldable (Vec 0) where
    foldMap _ Nil = mempty

instance (Foldable (Vec (n - 1)), n > 0) => Foldable (Vec n) where
    foldMap f (x :# xs) = f x <> foldMap f xs

instance Traversable (Vec 0) where
    traverse _ Nil = pure Nil

instance (Traversable (Vec (n - 1)), n > 0) => Traversable (Vec n) where
    traverse f (x :# xs) = liftA2 (:#) (f x) (traverse f xs)

class Index (n :: Nat) v where
    index :: Proxy n -> v a -> a

instance (m > 0) => Index 0 (Vec m) where
    index _ (x :# _) = x

instance forall n m. (Index (n - 1) (Vec (m - 1)), n > 0, m > 0) => Index n (Vec m) where
    index _ (_ :# xs) = index (Proxy :: Proxy (n - 1)) xs

instance (Unfoldable (Vec n), Traversable (Vec n)) => L.IsList (Vec n a) where
    type Item (Vec n a) = a
    fromList xs = case fromListU xs of
                    Nothing -> error "Demanded vector from a list that was too short."
                    Just ys -> ys
    toList      = Data.Foldable.toList
```

(Remember, we use the `forall` here with *ScopedTypeVariables* to be able to say
that the `n` in the type signature is the same `n` that is in the type of
`Proxy`)

``` haskell
ghci> fromListU [1,2,3,4] :: Vec 10 Int
Nothing
ghci> fromListU [1,2,3,4] :: Vec 3 Int
Just (1 :# 2 :# 3 :# Nil)
ghci> index (Proxy :: Proxy 2) (1 :# 2 :# 3 :# Nil)
3
ghci> index (Proxy :: Proxy 2) (1 :# 2 :# Nil)
*** Type Error: Couldn't match 'EQ with 'GT
ghci> :set -XOverloadedLists
ghci> [1,2,3] :: Vec 2 Int
1 :# 2 :# Nil
ghci> [1,2,3] :: Vec 4 Int
*** Exception: Demanded vector from a list that was too short.
ghci> [1,3..] :: Vec 5 Int
1 :# 3 :# 5 :# 7 :# 9 :# Nil
```

I think, overall, this formulation gives a much nicer interface. Being able to
just write $10$ is pretty powerful. The usage with *OverloadedLists* is pretty
clean, too, especially when you can do things like `[1,3..] :: Vec 10 Int` and
take full advantage of list syntax and succinct vector types. (Minding your
runtime errors, of course)

However, you do again get the problem that GHC is not able to do real
completeness checking and asks for the `Nil` cases still of everything...but
adding a `Nil` case will cause a type error. The only solution is to add a `_`
wildcard chase, but...again, this isn't quite satisfactory.[^3] If anybody has a
way to get around this, I'd love to know :)

## Alternative Underlying Representations

Recall that our `Vec` was basically identically the normal list type, with an
extra field in the type. Due to type erasure, the two are represented exactly
the same in memory. So we have $O(n)$ appends, $O(n)$ indexing, etc. Our type is
essentially equal to

``` haskell
newtype Vec :: Nat -> * -> * where
    VecList :: [a] -> Vec n a
```

For this type, though, we'd need to use "smart constructors" and extractors
instead of `1 :# 2 :# Nil` etc.

We could, however, chose a more efficient type, like `Vector` from the
*[vector](http://hackage.haskell.org/package/vector-0.10.12.2/docs/Data-Vector.html#t:Vector)*
package:

``` haskell
newtype Vec :: Nat -> * -> * where
    VecVector :: Vector a -> Vec n a
```

And, if you made sure to wrap everything with smart constructors, you now have
*type safe* $O(1)$ random indexing!

(This is representation is similar to the one used by the
*[linear](http://hackage.haskell.org/package/linear-1.18.0.1/docs/Linear-V.html)*
package.)

## More Operations

One really weird quirk with this is that many functions you'd normally write
using pattern matching you'd now might start writing using typeclasses. One
example would be our implementation of indexing, using an `IndexV` typeclass.

A bunch of one-shot typeclasses is sort of unideal, as typeclasses are sort of
ugly and non-first-class. Ideally you'd only have a few typeclasses for as
generic an interface as possible, and then be able to do everything from those.
Sometimes this just isn't practical. I did mention one way around it, which was
to make a typeclass to "reify" or turn your type into actual data, and then
manipulate your data in an "unsafe" way knowing that the type checker checked
that the data matched.

We'll demonstrate with `SomeNat` from `GHC.TypeLits`, but you can also make our
own for our inductive `Nat` type we used in the first half, too.

If we use our "wrapped `Vector` approach", we can just do:

``` haskell
newtype Vec :: Nat -> * -> * where
    Vec :: Vector a -> Vec n a

index :: (KnownNat n, m > n) => Proxy n -> Vec m a -> a
index p (Vec v) = v ! fromInteger (natVal p)
```

That is, `index` internally uses `(!)`, an unsafe operator...but only after we
assure properly that it's safe to use by stating `m > n` in the constraint. We
can be sure that GHC will catch any instance where someone tries to index into a
`Vec m a` whose `m` is *not* greater than the index desired.

The rest is up to you, though --- to prove that indexing into a number smaller
than `m` will always provide an answer. We have to make sure our smart
constructors are okay and that `(!)` behaves like we think it does.

## Singletons

Another answer to these sort of ad-hoc typeclasses is to use techniques
involving singletons. Going all into how to use singletons to work with these is
an article on its own...luckily, this article has already been written as [Part
1: Dependent Types in
Haskell](https://www.fpcomplete.com/user/konn/prove-your-haskell-for-great-safety/dependent-types-in-haskell)
by Hiromi ISHII. A major advantage is that you replace typeclasses with type
families and more parameterized types. You'll have to work with an understanding
of how singletons work, and accept using some template haskell to generate
singleton types for your data types (or write them yourself!). But it's a
powerful way to bring something like dependent types into Haskell, and there's
already a lot of infrastructure of support on it on hackage and in the haskell
dev ecosystem in general. I recommend looking at the linked article!

## Conclusion

Hopefully you'll see that we are able to apply the full type-safety of the
Haskell compiler to our programs regarding lists by encoding the length of the
list in its type and limiting its operations by specifically typed functions and
choice of instances. I also hope that you've been able to become familiar with
seeing a lot of GHC's basic type extensions in real applications :)

Feel free to [download and
run](https://github.com/mstksg/inCode/blob/master/code-samples/fixvec) any of
the samples

Please let me know if I got anything wrong, or if there are any techniques that
I should mention here that are out and in the wild today :)

# Signoff

Hi, thanks for reading! You can reach me via email at <justin@jle.im>, or at
twitter at [\@mstk](https://twitter.com/mstk)! This post and all others are
published under the [CC-BY-NC-ND
3.0](https://creativecommons.org/licenses/by-nc-nd/3.0/) license. Corrections
and edits via pull request are welcome and encouraged at [the source
repository](https://github.com/mstksg/inCode).

If you feel inclined, or this post was particularly helpful for you, why not
consider [supporting me on Patreon](https://www.patreon.com/justinle/overview),
or a [BTC donation](bitcoin:3D7rmAYgbDnp4gp4rf22THsGt74fNucPDU)? :)

[^1]: Can we get them out of Prelude? Please? :)

[^2]: By the way, the GHC wiki seems to claim that [using *OverloadedLists* this
    way is
    impossible](https://ghc.haskell.org/trac/ghc/wiki/OverloadedLists#Length-indexedobservedVectors).
    Anyone know what's going on here? Did we move fast and break everything?

[^3]: Interestingly enough, I think this is something where you could have the
    best of both situations with the Template Haskell method. But I'd hope for
    something that works on the beautiful TypeLits :'(


\end{document}

\end{document}
