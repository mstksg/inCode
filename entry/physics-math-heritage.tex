\documentclass[]{article}
\usepackage{lmodern}
\usepackage{amssymb,amsmath}
\usepackage{ifxetex,ifluatex}
\usepackage{fixltx2e} % provides \textsubscript
\ifnum 0\ifxetex 1\fi\ifluatex 1\fi=0 % if pdftex
  \usepackage[T1]{fontenc}
  \usepackage[utf8]{inputenc}
\else % if luatex or xelatex
  \ifxetex
    \usepackage{mathspec}
    \usepackage{xltxtra,xunicode}
  \else
    \usepackage{fontspec}
  \fi
  \defaultfontfeatures{Mapping=tex-text,Scale=MatchLowercase}
  \newcommand{\euro}{€}
\fi
% use upquote if available, for straight quotes in verbatim environments
\IfFileExists{upquote.sty}{\usepackage{upquote}}{}
% use microtype if available
\IfFileExists{microtype.sty}{\usepackage{microtype}}{}
\usepackage[margin=1in]{geometry}
\ifxetex
  \usepackage[setpagesize=false, % page size defined by xetex
              unicode=false, % unicode breaks when used with xetex
              xetex]{hyperref}
\else
  \usepackage[unicode=true]{hyperref}
\fi
\hypersetup{breaklinks=true,
            bookmarks=true,
            pdfauthor={Justin Le},
            pdftitle={My Physics and Math Heritage},
            colorlinks=true,
            citecolor=blue,
            urlcolor=blue,
            linkcolor=magenta,
            pdfborder={0 0 0}}
\urlstyle{same}  % don't use monospace font for urls
% Make links footnotes instead of hotlinks:
\renewcommand{\href}[2]{#2\footnote{\url{#1}}}
\setlength{\parindent}{0pt}
\setlength{\parskip}{6pt plus 2pt minus 1pt}
\setlength{\emergencystretch}{3em}  % prevent overfull lines
\setcounter{secnumdepth}{0}

\title{My Physics and Math Heritage}
\author{Justin Le}
\date{September 8, 2024}

\begin{document}
\maketitle

\emph{Originally posted on
\textbf{\href{https://blog.jle.im/entry/physics-math-heritage.html}{in Code}}.}

\documentclass[]{}
\usepackage{lmodern}
\usepackage{amssymb,amsmath}
\usepackage{ifxetex,ifluatex}
\usepackage{fixltx2e} % provides \textsubscript
\ifnum 0\ifxetex 1\fi\ifluatex 1\fi=0 % if pdftex
  \usepackage[T1]{fontenc}
  \usepackage[utf8]{inputenc}
\else % if luatex or xelatex
  \ifxetex
    \usepackage{mathspec}
    \usepackage{xltxtra,xunicode}
  \else
    \usepackage{fontspec}
  \fi
  \defaultfontfeatures{Mapping=tex-text,Scale=MatchLowercase}
  \newcommand{\euro}{€}
\fi
% use upquote if available, for straight quotes in verbatim environments
\IfFileExists{upquote.sty}{\usepackage{upquote}}{}
% use microtype if available
\IfFileExists{microtype.sty}{\usepackage{microtype}}{}
\usepackage[margin=1in]{geometry}
\ifxetex
  \usepackage[setpagesize=false, % page size defined by xetex
              unicode=false, % unicode breaks when used with xetex
              xetex]{hyperref}
\else
  \usepackage[unicode=true]{hyperref}
\fi
\hypersetup{breaklinks=true,
            bookmarks=true,
            pdfauthor={},
            pdftitle={},
            colorlinks=true,
            citecolor=blue,
            urlcolor=blue,
            linkcolor=magenta,
            pdfborder={0 0 0}}
\urlstyle{same}  % don't use monospace font for urls
% Make links footnotes instead of hotlinks:
\renewcommand{\href}[2]{#2\footnote{\url{#1}}}
\setlength{\parindent}{0pt}
\setlength{\parskip}{6pt plus 2pt minus 1pt}
\setlength{\emergencystretch}{3em}  % prevent overfull lines
\setcounter{secnumdepth}{0}


\begin{document}

This is just a "personal life update" kind of post, but I recently found out a
couple of cool things about my academic history that I thought were neat enough
to write down so that I don't forget them.

## Oppenheimer

When the [Christopher Nolan
Biopic](https://en.wikipedia.org/wiki/Oppenheimer_(film)) about the life of [J.
Robert Oppenheimer](https://en.wikipedia.org/wiki/J._Robert_Oppenheimer) was
about to come out, it was billed as an "Avengers of Physics", where every major
physicist working in the US early and middle 20th century would be featured. I
had a thought tracing my "academic family tree" to see if my PhD advisor's
advisor's advisor's advisor's was involved in any of the major physics projects
depicted in the movie, to see if I could spot them portrayed in the movie as a
nice personal connection.

If you're not familiar with the concept, the relationship between a PhD
candidate and their doctoral advisor is a very personal and individual one: they
personally direct and guide the candidate's research and thesis. To an extent,
they are like an academic parent.

I was able to [find my academic family tree](https://academictree.org/physics/)
and, to my surprise, my academic lineage actually traces directly back to a key
figure in the movie!

-   My advisor, [Hesham
    El-Askary](https://www.chapman.edu/our-faculty/hesham-el-askary), received
    his PhD under the advisory of [Menas
    Kafatos](https://en.wikipedia.org/wiki/Menas_Kafatos) at George Mason
    university
-   Dr. Kafatos received his PhD under the advisory of [Philip
    Morrison](https://en.wikipedia.org/wiki/Philip_Morrison) at the
    Massachusetts Institute of Technology.
-   Dr. Morrison received his PhD in 1940 at University of California, Berkeley
    under the advisory of none other than [J. Robert
    Oppenheimer](https://en.wikipedia.org/wiki/J._Robert_Oppenheimer) himself!

So, I started this out on a quest to figure out if I was "academically
descended" from anyone in the movie, and I ended up finding out I was
Oppenheimer's advisee's advisee's advisee's advisee! I ended up being able to
watch the movie and identify my great-great-grand advisor no problem, and I
think even my great-grand advisor. A fun little unexpected surprise and a cool
personal connection to a movie that I enjoyed a lot.

## Erdos

As an employee at Google, you can customize your directory page with "badges",
which are little personalized accomplishments or achievements, usually unrelated
to any actual work you do. I noticed that some people had an "Erdos Number N"
badge (1, 2, 3, etc.). I had never given any thought into my own personal Erdos
number (it was probably really high, in my mind) but I thought maybe I could
look into it in order to get a shiny worthless badge.

In academia, [Paul Erdos](https://en.wikipedia.org/wiki/Paul_Erd%C5%91s) is
someone who wrote [so many papers and collaborated with so many
people](https://users.renyi.hu/~p_erdos/Erdos.html) that it became a joking
"non-accomplishment" to say that you wrote a paper with him. Then after a while
it became an joking non-accomplishment to say that you wrote a paper with
someone who wrote a paper with him (because, who hasn't?). And then it became an
even more joking more non-accomplishment to say you had an Erdos Number of 3
(you wrote a paper with someone who wrote a paper with someone who wrote a paper
with Dr. Erdos).

Anyway I just wanted to get that badge so I tried to figure it out. It turns my
most direct trace through:

1.  I co-authored "[Application of recurrent neural networks for drought
    projections in
    California](https://www.sciencedirect.com/science/article/abs/pii/S0169809517300157)"
    with Daniele C. Struppa.
2.  Dr. Struppa co-authored "[Applications of commutative and computational
    algebra to partial differential
    equations](https://www.researchgate.net/publication/2505592_Applications_Of_Commutative_And_Computational_Algebra_To_Partial_Differential_Equations)"
    with William W. Adams.
3.  Dr. Adams co-authored "[Non-Archimedian analytic functions taking the same
    values at the same
    points](https://projecteuclid.org/journals/illinois-journal-of-mathematics/volume-15/issue-3/Non-Archimedian-analytic-functions-taking-the-same-values-at-the/10.1215/ijm/1256052610.full)"
    with Ernst G. Straus.
4.  Dr. Straus collaborated with many people, including Einstein, Graham,
    Goldberg, and 20 papers with Erdos.

So I guess my Erdos number is 4? The median number for mathematicians today
seems to be 5, so it's just one step above that. Not really a note-worthy
accomplishment, but still neat enough that I want a place to put the work
tracking this down the next time I am curious again.

Anyways I submitted the information above and they gave me that sweet Edros 4
badge! It was nice to have for about a month before quitting the company.

## That's It

Thanks for reading and I hope you have a nice rest of your day!

# Signoff

Hi, thanks for reading! You can reach me via email at <justin@jle.im>, or at
twitter at [\@mstk](https://twitter.com/mstk)! This post and all others are
published under the [CC-BY-NC-ND
3.0](https://creativecommons.org/licenses/by-nc-nd/3.0/) license. Corrections
and edits via pull request are welcome and encouraged at [the source
repository](https://github.com/mstksg/inCode).

If you feel inclined, or this post was particularly helpful for you, why not
consider [supporting me on Patreon](https://www.patreon.com/justinle/overview),
or a [BTC donation](bitcoin:3D7rmAYgbDnp4gp4rf22THsGt74fNucPDU)? :)

\end{document}

\end{document}
