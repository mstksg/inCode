\documentclass[]{article}
\usepackage{lmodern}
\usepackage{amssymb,amsmath}
\usepackage{ifxetex,ifluatex}
\usepackage{fixltx2e} % provides \textsubscript
\ifnum 0\ifxetex 1\fi\ifluatex 1\fi=0 % if pdftex
  \usepackage[T1]{fontenc}
  \usepackage[utf8]{inputenc}
\else % if luatex or xelatex
  \ifxetex
    \usepackage{mathspec}
    \usepackage{xltxtra,xunicode}
  \else
    \usepackage{fontspec}
  \fi
  \defaultfontfeatures{Mapping=tex-text,Scale=MatchLowercase}
  \newcommand{\euro}{€}
\fi
% use upquote if available, for straight quotes in verbatim environments
\IfFileExists{upquote.sty}{\usepackage{upquote}}{}
% use microtype if available
\IfFileExists{microtype.sty}{\usepackage{microtype}}{}
\usepackage[margin=1in]{geometry}
\ifxetex
  \usepackage[setpagesize=false, % page size defined by xetex
              unicode=false, % unicode breaks when used with xetex
              xetex]{hyperref}
\else
  \usepackage[unicode=true]{hyperref}
\fi
\hypersetup{breaklinks=true,
            bookmarks=true,
            pdfauthor={Justin Le},
            pdftitle={Starting a Patreon},
            colorlinks=true,
            citecolor=blue,
            urlcolor=blue,
            linkcolor=magenta,
            pdfborder={0 0 0}}
\urlstyle{same}  % don't use monospace font for urls
% Make links footnotes instead of hotlinks:
\renewcommand{\href}[2]{#2\footnote{\url{#1}}}
\setlength{\parindent}{0pt}
\setlength{\parskip}{6pt plus 2pt minus 1pt}
\setlength{\emergencystretch}{3em}  % prevent overfull lines
\setcounter{secnumdepth}{0}

\title{Starting a Patreon}
\author{Justin Le}
\date{May 30, 2018}

\begin{document}
\maketitle

\emph{Originally posted on
\textbf{\href{https://blog.jle.im/entry/starting-a-patreon.html}{in Code}}.}

\documentclass[]{}
\usepackage{lmodern}
\usepackage{amssymb,amsmath}
\usepackage{ifxetex,ifluatex}
\usepackage{fixltx2e} % provides \textsubscript
\ifnum 0\ifxetex 1\fi\ifluatex 1\fi=0 % if pdftex
  \usepackage[T1]{fontenc}
  \usepackage[utf8]{inputenc}
\else % if luatex or xelatex
  \ifxetex
    \usepackage{mathspec}
    \usepackage{xltxtra,xunicode}
  \else
    \usepackage{fontspec}
  \fi
  \defaultfontfeatures{Mapping=tex-text,Scale=MatchLowercase}
  \newcommand{\euro}{€}
\fi
% use upquote if available, for straight quotes in verbatim environments
\IfFileExists{upquote.sty}{\usepackage{upquote}}{}
% use microtype if available
\IfFileExists{microtype.sty}{\usepackage{microtype}}{}
\usepackage[margin=1in]{geometry}
\ifxetex
  \usepackage[setpagesize=false, % page size defined by xetex
              unicode=false, % unicode breaks when used with xetex
              xetex]{hyperref}
\else
  \usepackage[unicode=true]{hyperref}
\fi
\hypersetup{breaklinks=true,
            bookmarks=true,
            pdfauthor={},
            pdftitle={},
            colorlinks=true,
            citecolor=blue,
            urlcolor=blue,
            linkcolor=magenta,
            pdfborder={0 0 0}}
\urlstyle{same}  % don't use monospace font for urls
% Make links footnotes instead of hotlinks:
\renewcommand{\href}[2]{#2\footnote{\url{#1}}}
\setlength{\parindent}{0pt}
\setlength{\parskip}{6pt plus 2pt minus 1pt}
\setlength{\emergencystretch}{3em}  % prevent overfull lines
\setcounter{secnumdepth}{0}


\begin{document}

*Short version:* I've [started a Patreon
page](https://www.patreon.com/justinle/overview) for those who want to donate or
help me sustain my writing in my post-doctoral life! :D

*Long version:* Hi all! Hope you can excuse the two non-technical and personal
posts in a row :) To give some context on this announcement, here's some news
--- I graduated\*!

![Graduating\* from a Doctoral Program at Chapman
University](/img/entries/patreon/graduation.jpg "I graduated!")

\*Well, not exactly :) I haven't finished my dissertation and defended my thesis
yet (but expect to in the upcoming months); Chapman University was generous
enough to allow me to walk in this past Spring's ceremony.

I have a lot of excitement about my future after my defense and official
commencement. However, with this excitement comes a lot of uncertainty.

As a student I have been able to devote time on-and-off again to this blog,
which has become a passion of mine. I've even managed to find time to
[live-stream](https://www.twitch.tv/mstksg "Twitch") some coding projects
occasionally! When I started this blog almost five years ago, I only wanted to
write down what I was learning to help myself remember, and to help others going
through the same thing as me. I never expected the warm reception from the
Haskell and Programming community. Every once in a while I meet people who tell
me personally that an article of mine had inspired them to get into functional
programming, or had been the final straw to helping them get over a conceptual
hurdle. It's moments like these that really make everything worth it. And, as a
researcher in a field dominated by imperative and dynamically typed programming,
watching and encouraging the growth of functional programming has brought me a
lot of joy.

In the upcoming months, however, I know that my circumstances will be changing.
Without the cushion of research and grant funding, I know I will start having to
make some more serious decisions balancing my time.

But in the face of all this uncertainty, one thing I do know for sure is that I
want to start taking my writing and educational passion projects more seriously
:)

In this light, I'm excited to announce that I am starting a **[Patreon
page](https://www.patreon.com/justinle/overview)**!

Patreon is a service that allows people to support their favorite content
producers. Users can choose to "become patrons" of content creators by pledging
to support them monetarily on a per-month or per-creation basis.

Through your support, you are telling me that:

-   The content I have written has helped or impacted you in some way
-   You appreciate that what I write adds value to the functional programming
    and Haskell community
-   You want be a part of this journey of helping and educating the FP and
    Haskell community
-   You want to provide support that allows me to take my writing, educational
    efforts, and live-coding more seriously in my post-doctoral life

In addition, different levels of support have some fun little rewards, like
being featured on the homepage, footer, or entries on this blog as a supporter
who makes all of this possible. There are also some rewards involving access to
regular in-depth code reviews I can give you, and long-term community goals
include the start of a live-coding series.

Thank you so much for joining me on this journey --- whether you support me
monetarily, or just through kind words. I am grateful for your encouragement and
support in all forms, and I am genuinely looking forward to this next chapter of
my life!

\end{document}

\end{document}
