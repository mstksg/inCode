\documentclass[]{article}
\usepackage{lmodern}
\usepackage{amssymb,amsmath}
\usepackage{ifxetex,ifluatex}
\usepackage{fixltx2e} % provides \textsubscript
\ifnum 0\ifxetex 1\fi\ifluatex 1\fi=0 % if pdftex
  \usepackage[T1]{fontenc}
  \usepackage[utf8]{inputenc}
\else % if luatex or xelatex
  \ifxetex
    \usepackage{mathspec}
    \usepackage{xltxtra,xunicode}
  \else
    \usepackage{fontspec}
  \fi
  \defaultfontfeatures{Mapping=tex-text,Scale=MatchLowercase}
  \newcommand{\euro}{€}
\fi
% use upquote if available, for straight quotes in verbatim environments
\IfFileExists{upquote.sty}{\usepackage{upquote}}{}
% use microtype if available
\IfFileExists{microtype.sty}{\usepackage{microtype}}{}
\usepackage[margin=1in]{geometry}
\usepackage{color}
\usepackage{fancyvrb}
\newcommand{\VerbBar}{|}
\newcommand{\VERB}{\Verb[commandchars=\\\{\}]}
\DefineVerbatimEnvironment{Highlighting}{Verbatim}{commandchars=\\\{\}}
% Add ',fontsize=\small' for more characters per line
\newenvironment{Shaded}{}{}
\newcommand{\AlertTok}[1]{\textcolor[rgb]{1.00,0.00,0.00}{\textbf{#1}}}
\newcommand{\AnnotationTok}[1]{\textcolor[rgb]{0.38,0.63,0.69}{\textbf{\textit{#1}}}}
\newcommand{\AttributeTok}[1]{\textcolor[rgb]{0.49,0.56,0.16}{#1}}
\newcommand{\BaseNTok}[1]{\textcolor[rgb]{0.25,0.63,0.44}{#1}}
\newcommand{\BuiltInTok}[1]{\textcolor[rgb]{0.00,0.50,0.00}{#1}}
\newcommand{\CharTok}[1]{\textcolor[rgb]{0.25,0.44,0.63}{#1}}
\newcommand{\CommentTok}[1]{\textcolor[rgb]{0.38,0.63,0.69}{\textit{#1}}}
\newcommand{\CommentVarTok}[1]{\textcolor[rgb]{0.38,0.63,0.69}{\textbf{\textit{#1}}}}
\newcommand{\ConstantTok}[1]{\textcolor[rgb]{0.53,0.00,0.00}{#1}}
\newcommand{\ControlFlowTok}[1]{\textcolor[rgb]{0.00,0.44,0.13}{\textbf{#1}}}
\newcommand{\DataTypeTok}[1]{\textcolor[rgb]{0.56,0.13,0.00}{#1}}
\newcommand{\DecValTok}[1]{\textcolor[rgb]{0.25,0.63,0.44}{#1}}
\newcommand{\DocumentationTok}[1]{\textcolor[rgb]{0.73,0.13,0.13}{\textit{#1}}}
\newcommand{\ErrorTok}[1]{\textcolor[rgb]{1.00,0.00,0.00}{\textbf{#1}}}
\newcommand{\ExtensionTok}[1]{#1}
\newcommand{\FloatTok}[1]{\textcolor[rgb]{0.25,0.63,0.44}{#1}}
\newcommand{\FunctionTok}[1]{\textcolor[rgb]{0.02,0.16,0.49}{#1}}
\newcommand{\ImportTok}[1]{\textcolor[rgb]{0.00,0.50,0.00}{\textbf{#1}}}
\newcommand{\InformationTok}[1]{\textcolor[rgb]{0.38,0.63,0.69}{\textbf{\textit{#1}}}}
\newcommand{\KeywordTok}[1]{\textcolor[rgb]{0.00,0.44,0.13}{\textbf{#1}}}
\newcommand{\NormalTok}[1]{#1}
\newcommand{\OperatorTok}[1]{\textcolor[rgb]{0.40,0.40,0.40}{#1}}
\newcommand{\OtherTok}[1]{\textcolor[rgb]{0.00,0.44,0.13}{#1}}
\newcommand{\PreprocessorTok}[1]{\textcolor[rgb]{0.74,0.48,0.00}{#1}}
\newcommand{\RegionMarkerTok}[1]{#1}
\newcommand{\SpecialCharTok}[1]{\textcolor[rgb]{0.25,0.44,0.63}{#1}}
\newcommand{\SpecialStringTok}[1]{\textcolor[rgb]{0.73,0.40,0.53}{#1}}
\newcommand{\StringTok}[1]{\textcolor[rgb]{0.25,0.44,0.63}{#1}}
\newcommand{\VariableTok}[1]{\textcolor[rgb]{0.10,0.09,0.49}{#1}}
\newcommand{\VerbatimStringTok}[1]{\textcolor[rgb]{0.25,0.44,0.63}{#1}}
\newcommand{\WarningTok}[1]{\textcolor[rgb]{0.38,0.63,0.69}{\textbf{\textit{#1}}}}
\ifxetex
  \usepackage[setpagesize=false, % page size defined by xetex
              unicode=false, % unicode breaks when used with xetex
              xetex]{hyperref}
\else
  \usepackage[unicode=true]{hyperref}
\fi
\hypersetup{breaklinks=true,
            bookmarks=true,
            pdfauthor={Justin Le},
            pdftitle={Adjunctions in the wild: foldl},
            colorlinks=true,
            citecolor=blue,
            urlcolor=blue,
            linkcolor=magenta,
            pdfborder={0 0 0}}
\urlstyle{same}  % don't use monospace font for urls
% Make links footnotes instead of hotlinks:
\renewcommand{\href}[2]{#2\footnote{\url{#1}}}
\setlength{\parindent}{0pt}
\setlength{\parskip}{6pt plus 2pt minus 1pt}
\setlength{\emergencystretch}{3em}  % prevent overfull lines
\setcounter{secnumdepth}{0}

\title{Adjunctions in the wild: foldl}
\author{Justin Le}
\date{January 13, 2020}

\begin{document}
\maketitle

\emph{Originally posted on
\textbf{\href{https://blog.jle.im/entry/foldl-adjunction.html}{in Code}}.}

\documentclass[]{}
\usepackage{lmodern}
\usepackage{amssymb,amsmath}
\usepackage{ifxetex,ifluatex}
\usepackage{fixltx2e} % provides \textsubscript
\ifnum 0\ifxetex 1\fi\ifluatex 1\fi=0 % if pdftex
  \usepackage[T1]{fontenc}
  \usepackage[utf8]{inputenc}
\else % if luatex or xelatex
  \ifxetex
    \usepackage{mathspec}
    \usepackage{xltxtra,xunicode}
  \else
    \usepackage{fontspec}
  \fi
  \defaultfontfeatures{Mapping=tex-text,Scale=MatchLowercase}
  \newcommand{\euro}{€}
\fi
% use upquote if available, for straight quotes in verbatim environments
\IfFileExists{upquote.sty}{\usepackage{upquote}}{}
% use microtype if available
\IfFileExists{microtype.sty}{\usepackage{microtype}}{}
\usepackage[margin=1in]{geometry}
\ifxetex
  \usepackage[setpagesize=false, % page size defined by xetex
              unicode=false, % unicode breaks when used with xetex
              xetex]{hyperref}
\else
  \usepackage[unicode=true]{hyperref}
\fi
\hypersetup{breaklinks=true,
            bookmarks=true,
            pdfauthor={},
            pdftitle={},
            colorlinks=true,
            citecolor=blue,
            urlcolor=blue,
            linkcolor=magenta,
            pdfborder={0 0 0}}
\urlstyle{same}  % don't use monospace font for urls
% Make links footnotes instead of hotlinks:
\renewcommand{\href}[2]{#2\footnote{\url{#1}}}
\setlength{\parindent}{0pt}
\setlength{\parskip}{6pt plus 2pt minus 1pt}
\setlength{\emergencystretch}{3em}  % prevent overfull lines
\setcounter{secnumdepth}{0}

\textbackslash begin\{document\}

I recently made a few connections that linked some different concepts in Haskell
that I hadn't realized before. They deal with one of my favorite ``practical''
libraries in Haskell, and also one of the more ``profound'' category
theory-inspired abstractions in Haskell. In the process, it made the library a
bit more useful to me, and also made the concept a bit more concrete and
understandable to me.

This post mainly goes through my thought process in finding this out --- it's
very much a ``how I think through this'' sort of thing --- in the end, the goal
is to show how much this example made me further appreciate the conceptual idea
of adjunctions and how they can pop up in interesting places in practical
libraries. Unlike most of my other posts, it's not about necessarily about how
practically useful an abstraction is, but rather what insight it gives us to
understanding its instances.

The audience of this post is Haskellers with an understanding/appreciation of
abstractions like \texttt{Applicative}, but be aware that the final section is
separately considered as a fun aside for those familiar with some of Haskell's
more esoteric types. The code samples used here (along with exercise solutions)
are
\href{https://github.com/mstksg/inCode/tree/master/code-samples/adjunctions/foldl.hs}{available
on github}.

\section{foldl}\label{foldl}

The first concept is the great
\emph{\href{http://hackage.haskell.org/package/foldl}{foldl}} library, which
provides a nice ``stream processor'' type called \texttt{Fold}, where
\texttt{Fold\ r\ a} is a stream processor that takes a stream of \texttt{r}s and
produces an \texttt{a}:

\begin{Shaded}
\begin{Highlighting}[]
\KeywordTok{import}           \DataTypeTok{Control.Foldl}\NormalTok{   (}\DataTypeTok{Fold}\NormalTok{(..))}
\KeywordTok{import} \KeywordTok{qualified} \DataTypeTok{Control.Foldl}   \KeywordTok{as} \DataTypeTok{F}

\NormalTok{F.sum}\OtherTok{  ::} \DataTypeTok{Num}\NormalTok{ a        }\OtherTok{=\textgreater{}} \DataTypeTok{Fold}\NormalTok{ a a}
\NormalTok{F.mean}\OtherTok{ ::} \DataTypeTok{Fractional}\NormalTok{ a }\OtherTok{=\textgreater{}} \DataTypeTok{Fold}\NormalTok{ a a}
\NormalTok{F.elem}\OtherTok{ ::} \DataTypeTok{Eq}\NormalTok{ a         }\OtherTok{=\textgreater{}}\NormalTok{ a }\OtherTok{{-}\textgreater{}} \DataTypeTok{Fold}\NormalTok{ a }\DataTypeTok{Bool}

\NormalTok{F.fold F.sum  [}\DecValTok{1}\NormalTok{,}\DecValTok{2}\NormalTok{,}\DecValTok{3}\NormalTok{,}\DecValTok{4}\NormalTok{]}
\PreprocessorTok{\#   =\textgreater{} 10}
\NormalTok{F.fold F.mean [}\DecValTok{1}\NormalTok{,}\DecValTok{2}\NormalTok{,}\DecValTok{3}\NormalTok{,}\DecValTok{4}\NormalTok{]}
\PreprocessorTok{\#   =\textgreater{} 2.5}
\NormalTok{F.fold (F.elem }\DecValTok{3}\NormalTok{) [}\DecValTok{1}\NormalTok{,}\DecValTok{2}\NormalTok{,}\DecValTok{3}\NormalTok{,}\DecValTok{4}\NormalTok{]}
\PreprocessorTok{\#   =\textgreater{} True}
\NormalTok{F.fold (F.elem }\DecValTok{5}\NormalTok{) [}\DecValTok{1}\NormalTok{,}\DecValTok{2}\NormalTok{,}\DecValTok{3}\NormalTok{,}\DecValTok{4}\NormalTok{]}
\PreprocessorTok{\#   =\textgreater{} False}
\end{Highlighting}
\end{Shaded}

The most useful thing about the library is that it treats the folds as
first-class objects, so you can create more complex folds by combining simpler
folds (for example, with \texttt{-XApplicativeDo})

\begin{Shaded}
\begin{Highlighting}[]
\CommentTok{{-}{-} source: https://github.com/mstksg/inCode/tree/master/code{-}samples/adjunctions/foldl.hs\#L22{-}L29}

\OtherTok{variance ::} \DataTypeTok{Fractional}\NormalTok{ a }\OtherTok{=\textgreater{}} \DataTypeTok{Fold}\NormalTok{ a a}
\NormalTok{variance }\OtherTok{=} \KeywordTok{do}
\NormalTok{    x  }\OtherTok{\textless{}{-}}\NormalTok{ F.mean}
\NormalTok{    x2 }\OtherTok{\textless{}{-}}\NormalTok{ lmap (}\OperatorTok{\^{}}\DecValTok{2}\NormalTok{) F.mean     }\CommentTok{{-}{-} the mean of squared items}
    \FunctionTok{pure}\NormalTok{ (x2 }\OperatorTok{{-}}\NormalTok{ x}\OperatorTok{*}\NormalTok{x)}

\OtherTok{varianceTooBig ::}\NormalTok{ (}\DataTypeTok{Fractional}\NormalTok{ a, }\DataTypeTok{Ord}\NormalTok{ a) }\OtherTok{=\textgreater{}} \DataTypeTok{Fold}\NormalTok{ a }\DataTypeTok{Bool}
\NormalTok{varianceTooBig }\OtherTok{=}\NormalTok{ (}\OperatorTok{\textgreater{}} \DecValTok{3}\NormalTok{) }\OperatorTok{\textless{}$\textgreater{}}\NormalTok{ variance}
\end{Highlighting}
\end{Shaded}

Most importantly, \texttt{Fold\ r} is an instance of both \texttt{Functor} and
\texttt{Applicative}, so you can map over and combine the results of different
folds.

To me, \emph{foldl} is one of the shining examples of how well Haskell works for
data and stream processing, and a library I often show to people when they ask
what the big deal is about Haskell abstractions like \texttt{Applicative},
purity, and lists --- this technique is often described as
``\href{https://www.google.com/search?q=beautiful+folds}{beautiful folds}''.

\section{Adjunctions}\label{adjunctions}

The second concept is the idea of
\emph{\href{https://en.wikipedia.org/wiki/Adjoint_functors}{adjoint functors}}
(see also \href{https://bartoszmilewski.com/2016/04/18/adjunctions/}{Bartosz
Milewski's introduction} and
\href{https://ncatlab.org/nlab/show/adjoint+functor}{nlab}'s description, as
well as
\href{https://www.math3ma.com/blog/what-is-an-adjunction-part-1}{Tai-Danae
Bradley's motivation}), represented in Haskell by the
\emph{\href{https://hackage.haskell.org/package/adjunctions/docs/Data-Functor-Adjunction.html}{adjunctions
library and typeclass}}
(\href{https://chrispenner.ca/posts/adjunction-battleship}{Chris Penner} has a
nice article with an example of using the typeclass's utility functions to
simplify programs).

For some functors, we can think of a ``conceptual inverse''. We can ask ``I have
a nice functor \texttt{F}. Conceptually, what functor represents the opposite
idea/spirit of \texttt{F}?'' The concept of an adjunction is one way to
formalize what this means.

In Haskell\footnote{Note that the intuition we are going to be going into is
  specifically for adjunctions between \texttt{Functor} functors --- functors
  that the \texttt{Functor} typeclass models (aka, endofunctors in Hask). For a
  more general view of adjunctions in general, see the links above.}, with the
\texttt{Adjunctions} typeclass (specifically, \texttt{Functor} functors), this
manifests as this: if \texttt{F\ -\textbar{}\ U} (\texttt{F} is left adjoint to
\texttt{U}, and \texttt{U} is right adjoint to \texttt{F}), then all the ways of
going ``out of'' \texttt{F\ a} to \texttt{b} are the same as all the ways of
going ``into'' \texttt{U\ b} from \texttt{a}. Ways of going out can be encoded
as ways of going in, and vice versa. They represent opposite ideas.

\begin{Shaded}
\begin{Highlighting}[]
\CommentTok{{-}{-} | The class saying you can always convert between:}
\CommentTok{{-}{-}}
\CommentTok{{-}{-} * \textasciigrave{}f a {-}\textgreater{} b\textasciigrave{} (the ways to go out of \textasciigrave{}f\textasciigrave{})}
\CommentTok{{-}{-} * \textasciigrave{}a {-}\textgreater{} u b\textasciigrave{} (the ways to go into \textasciigrave{}g\textasciigrave{})}
\KeywordTok{class} \DataTypeTok{Adjunction}\NormalTok{ f u }\KeywordTok{where}
\NormalTok{    leftAdjunct}
\OtherTok{        ::}\NormalTok{ (f a }\OtherTok{{-}\textgreater{}}\NormalTok{ b)       }\CommentTok{{-}{-} \^{} the ways of going "out of" \textasciigrave{}f\textasciigrave{}}
        \OtherTok{{-}\textgreater{}}\NormalTok{ (a }\OtherTok{{-}\textgreater{}}\NormalTok{ u b)       }\CommentTok{{-}{-} \^{} the ways of going "into" \textasciigrave{}u\textasciigrave{}}

\NormalTok{    rightAdjunct}
\OtherTok{        ::}\NormalTok{ (a }\OtherTok{{-}\textgreater{}}\NormalTok{ u b)       }\CommentTok{{-}{-} \^{} the ways of going "into" u}
        \OtherTok{{-}\textgreater{}}\NormalTok{ (f a }\OtherTok{{-}\textgreater{}}\NormalTok{ b)       }\CommentTok{{-}{-} \^{} the ways of going "out of" f}
\end{Highlighting}
\end{Shaded}

\subsection{Examples}\label{examples}

For example, one of the more famous adjunctions in Haskell is the adjunction
between \texttt{(,)\ r} and \texttt{(-\textgreater{})\ r}. ``Tupling''
represents some sort of ``opposite'' idea to ``parameterizing''.

The ways to get ``out'' of a tuple is \texttt{(r,\ a)\ -\textgreater{}\ b}. The
ways to go ``into'' a function is
\texttt{a\ -\textgreater{}\ (r\ -\textgreater{}\ b)}. Haskellers will recognize
that these two types are the ``same'' (isomorphic) --- any
\texttt{(a,\ b)\ -\textgreater{}\ c} can be re-written as
\texttt{a\ -\textgreater{}\ (b\ -\textgreater{}\ c)} (currying), and vice versa
(uncurrying).

Another common pair is with same-typed either and tuple:

\begin{Shaded}
\begin{Highlighting}[]
\CommentTok{{-}{-} source: https://github.com/mstksg/inCode/tree/master/code{-}samples/adjunctions/foldl.hs\#L31{-}L34}

\KeywordTok{newtype} \DataTypeTok{SameEither}\NormalTok{ a }\OtherTok{=} \DataTypeTok{SE}\NormalTok{ (}\DataTypeTok{Either}\NormalTok{ a a)}

\KeywordTok{newtype} \DataTypeTok{SameTuple}\NormalTok{  a }\OtherTok{=} \DataTypeTok{ST}\NormalTok{ (a, a)}
\end{Highlighting}
\end{Shaded}

People familiar with \texttt{Either} (sums) and \texttt{(,)} (products) in
Haskell will recognize them as ``opposite'' ideas --- one is ``or'', and the
other is ``and'' (depending on if you are talking about using them or making
them).

We can formalize this idea of opposites using adjunctions: Going ``out of''
\texttt{Either\ a\ a} into \texttt{b} can be encoded as going ``into''
\texttt{(b,\ b)} from \texttt{a}, and vice versa:
\texttt{Either\ a\ a\ -\textgreater{}\ b} can be encoded as
\texttt{a\ -\textgreater{}\ (b,\ b)}, which can be encoded as
\texttt{Either\ a\ a\ -\textgreater{}\ b} --- the two types are isomorphic. This
is because to go out of \texttt{Either\ a\ a}, you have to handle the situation
of getting a \texttt{Left} and the situation of getting a \texttt{Right}. To go
into \texttt{(b,\ b)}, you have to able to ask what goes in the first field, and
what goes in the right field. Both \texttt{Either\ a\ a\ -\textgreater{}\ b} and
\texttt{a\ -\textgreater{}\ (b,\ b)} have to answer the same questions. (A fun
exercise would be to write the functions to convert between the two ---
\href{https://github.com/mstksg/inCode/tree/master/code-samples/adjunctions/foldl.hs\#L55-L56}{one
solution is here})

\subsection{Big Picture}\label{big-picture}

Aside from being an interesting curiosity (formalizing the idea of ``opposite
idea'' is pretty neat), hunting for adjunctions can be useful in figuring out
``why'' a functor is useful, what you can do with it, and also what functors are
intimately connected with it. There's also the helper functions in the
\href{https://hackage.haskell.org/package/adjunctions/docs/Data-Functor-Adjunction.html}{Data.Functor.Adjunction}
module that implement some nice helper functions on your types if an adjoint
happens to exist --- you can do some neat things by going ``back and forth''
between adjoint functors.

\section{Hunting for Adjunctions}\label{hunting-for-adjunctions}

So, from the build-up, you've probably guessed what we're going to do next: find
a functor that is adjoint to \texttt{Fold\ r}. What's the ``conceptual
opposite'' of \texttt{Fold\ r}? Let's go adjunction hunting!

Important note --- the rest of this section is not a set of hard rules, but
rather an intuitive process of heuristics to search for candidates that would be
adjoint to a given functor of interest. There are no hard and fast rules, and
the adjoint might not always exist (it usually doesn't). But when it does, it
can be a pleasant surprise.

\subsection{Patterns to look for}\label{patterns-to-look-for}

Now, on to the hunting. Let's say we have functor \texttt{Q} and we want to
identify any adjoints. We want to spot functions that use both \texttt{Q\ a} and
\texttt{a} with some other value, in
\href{https://www.foldl.io/posts/pos-neg-functions/}{opposite positions}.

(Of course, this is only the case if we are using a functor that comes from a
library. If we are writing our own functor from scratch, and want to hunt for
adjunctions there, we have to instead \emph{think} of ways to use \texttt{Q\ a}
and \texttt{a})

One common pattern is functions for ``converting between'' the going-in and
going-out functions. In
\href{https://hackage.haskell.org/package/adjunctions/docs/Data-Functor-Adjunction.html}{Data.Functor.Adjunctions},
these are called \texttt{leftAdjunct} and \texttt{rightAdjunct}:

\begin{Shaded}
\begin{Highlighting}[]
\OtherTok{leftAdjunct  ::} \DataTypeTok{Adjunction}\NormalTok{ f u }\OtherTok{=\textgreater{}}\NormalTok{ (f a }\OtherTok{{-}\textgreater{}}\NormalTok{ b) }\OtherTok{{-}\textgreater{}}\NormalTok{ (a }\OtherTok{{-}\textgreater{}}\NormalTok{ u b)}
\OtherTok{rightAdjunct ::} \DataTypeTok{Adjunction}\NormalTok{ f u }\OtherTok{=\textgreater{}}\NormalTok{ (a }\OtherTok{{-}\textgreater{}}\NormalTok{ u b) }\OtherTok{{-}\textgreater{}}\NormalTok{ (f a }\OtherTok{{-}\textgreater{}}\NormalTok{ b)}
\end{Highlighting}
\end{Shaded}

This pair is significant because it is the adjunctions ``in practice'': Sure, an
\texttt{(r,\ a)\ -\textgreater{}\ b} is useful, but ``using'' the adjunction
means that you can convert \emph{between} \texttt{(r,\ a)\ -\textgreater{}\ b}
and \texttt{a\ -\textgreater{}\ r\ -\textgreater{}\ b}

Another common pattern that you can spot are ``indexing'' and ``tabulating''
functions, in the case that you have a right-adjoint:

\begin{Shaded}
\begin{Highlighting}[]
\OtherTok{indexAdjunction    ::} \DataTypeTok{Adjunction}\NormalTok{ f u }\OtherTok{=\textgreater{}}\NormalTok{ u b }\OtherTok{{-}\textgreater{}}\NormalTok{ f () }\OtherTok{{-}\textgreater{}}\NormalTok{ b}
\OtherTok{tabulateAdjunction ::} \DataTypeTok{Adjunction}\NormalTok{ f u }\OtherTok{=\textgreater{}}\NormalTok{ (f () }\OtherTok{{-}\textgreater{}}\NormalTok{ b) }\OtherTok{{-}\textgreater{}}\NormalTok{ u b}
\end{Highlighting}
\end{Shaded}

\texttt{indexAdjunction} means: if it's possible to ``extract'' from
\texttt{u\ b} to \texttt{b} using only an \texttt{f\ ()} as extra information,
then \texttt{u} might be right-adjoint to \texttt{f}.

\texttt{tabulateAdjunction} means: if it's possible to ``generate'' a
\texttt{u\ b} based on a function that ``builds'' a \texttt{b} from
\texttt{f\ ()}, then \texttt{u} might right-adjoint to \texttt{f}.

This pair is equivalent in power --- you can implement \texttt{rightAdjunct} in
terms of \texttt{indexAdjunction} and \texttt{leftAdjunct} in terms of
\texttt{tabulateAdjunction} and vice versa. This comes from the fact that all
Adjunctions in Haskell \texttt{Functor}s arise from some idea of
``indexability''. We'll go into more detail later, but this is the general
intuition.

\subsection{\texorpdfstring{Adjoints to
\texttt{Fold}}{Adjoints to Fold}}\label{adjoints-to-fold}

Now, let's look out for examples of these functions for \texttt{Fold}! In the
case of \texttt{Fold}, there is actually only one function I can find that
directly takes a \texttt{Fold\ r\ a} and returns an \texttt{a}:

\begin{Shaded}
\begin{Highlighting}[]
\OtherTok{fold ::} \DataTypeTok{Fold}\NormalTok{ r b }\OtherTok{{-}\textgreater{}}\NormalTok{ [r] }\OtherTok{{-}\textgreater{}}\NormalTok{ b}
\end{Highlighting}
\end{Shaded}

(the type has been simplified and re-labeled, for illustration's sake)

You ``give'' a \texttt{Fold\ r\ b} and ``get'' an \texttt{b} (and so they have
opposite polarities/positions). This sort of function would make
\texttt{Fold\ r} a \emph{right adjoint}, since the naked type \texttt{b} (the
final parameter of \texttt{Fold\ r\ b}) is the final result, not the input.

Of our common patterns, this one looks a looooot like
\texttt{indexAdjunction}.\footnote{As it so happens, \texttt{fold} is actually
  exactly \texttt{index} for \texttt{Representable\ (Fold\ r)}, from
  \emph{\href{https://hackage.haskell.org/package/adjunctions/docs/Data-Functor-Rep.html}{Data.Functor.Rep}}.
  Here we are utilizing the fact that a representable Functor gives rise to a
  left-adjoint for free --- the two ideas are equivalent in Haskell. We go into
  this in more detail in an upcoming aside.}

\begin{Shaded}
\begin{Highlighting}[]
\OtherTok{fold            ::} \DataTypeTok{Fold}\NormalTok{ r b }\OtherTok{{-}\textgreater{}}\NormalTok{ [r]  }\OtherTok{{-}\textgreater{}}\NormalTok{ b}
\OtherTok{indexAdjunction ::} \DataTypeTok{Fold}\NormalTok{ r b }\OtherTok{{-}\textgreater{}}\NormalTok{ f () }\OtherTok{{-}\textgreater{}}\NormalTok{ b}
\end{Highlighting}
\end{Shaded}

This means that \texttt{Fold\ r\ b} is right-adjoint to some functor \texttt{f}
where \texttt{f\ ()\ =\ {[}r{]}}. A good first guess (just a hunch?) would be to
just have \texttt{f\ a\ =\ ({[}r{]},\ a)}:

\begin{Shaded}
\begin{Highlighting}[]
\CommentTok{{-}{-} source: https://github.com/mstksg/inCode/tree/master/code{-}samples/adjunctions/foldl.hs\#L73{-}L74}

\KeywordTok{data} \DataTypeTok{EnvList}\NormalTok{ r a }\OtherTok{=} \DataTypeTok{EnvList}\NormalTok{ [r] a}
  \KeywordTok{deriving}\NormalTok{ (}\DataTypeTok{Functor}\NormalTok{, }\DataTypeTok{Show}\NormalTok{, }\DataTypeTok{Eq}\NormalTok{, }\DataTypeTok{Ord}\NormalTok{)}
\end{Highlighting}
\end{Shaded}

\texttt{EnvList\ r}\footnote{The name here is inspired by the
  \href{https://hackage.haskell.org/package/comonad/docs/Control-Comonad-Trans-Env.html}{\texttt{Env}
  comonad} --- \texttt{EnvList\ r} is \texttt{Env\ {[}r{]}}.} adds a \emph{list}
of \texttt{r}s to a type. It is now also our suspect for a potential
left-adjoint to \texttt{Fold\ r}: a ``conceptual opposite''.

\begin{Shaded}
\begin{Highlighting}[]
\CommentTok{{-}{-} source: https://github.com/mstksg/inCode/tree/master/code{-}samples/adjunctions/foldl.hs\#L76{-}L77}

\OtherTok{indexFold ::} \DataTypeTok{Fold}\NormalTok{ r b }\OtherTok{{-}\textgreater{}} \DataTypeTok{EnvList}\NormalTok{ r () }\OtherTok{{-}\textgreater{}}\NormalTok{ b}
\NormalTok{indexFold fld (}\DataTypeTok{EnvList}\NormalTok{ rs \_) }\OtherTok{=}\NormalTok{ F.fold fld rs}
\end{Highlighting}
\end{Shaded}

To seal the deal, let's find its pair, \texttt{tabulateAdjunction}. That means
we are looking for:

\begin{Shaded}
\begin{Highlighting}[]
\OtherTok{tabulateFold ::}\NormalTok{ (}\DataTypeTok{EnvList}\NormalTok{ r () }\OtherTok{{-}\textgreater{}}\NormalTok{ b) }\OtherTok{{-}\textgreater{}} \DataTypeTok{Fold}\NormalTok{ r b}
\end{Highlighting}
\end{Shaded}

Or, to simplify the type by expanding the definition of \texttt{EnvList\ r\ ()}:

\begin{Shaded}
\begin{Highlighting}[]
\OtherTok{tabulateFold ::}\NormalTok{ ([r] }\OtherTok{{-}\textgreater{}}\NormalTok{ b) }\OtherTok{{-}\textgreater{}} \DataTypeTok{Fold}\NormalTok{ r b}
\end{Highlighting}
\end{Shaded}

This tells us that, given any list processor
\texttt{{[}r{]}\ -\textgreater{}\ b}, we can write a fold \texttt{Fold\ r\ b}
representing that list processor. Scanning things more, we can see that this
actually looks a lot like \texttt{foldMap} from the library:

\begin{Shaded}
\begin{Highlighting}[]
\KeywordTok{import} \KeywordTok{qualified} \DataTypeTok{Control.Foldl} \KeywordTok{as} \DataTypeTok{F}

\NormalTok{F.foldMap}
\OtherTok{    ::} \DataTypeTok{Monoid}\NormalTok{ w}
    \OtherTok{=\textgreater{}}\NormalTok{ (r }\OtherTok{{-}\textgreater{}}\NormalTok{ w)}
    \OtherTok{{-}\textgreater{}}\NormalTok{ (w }\OtherTok{{-}\textgreater{}}\NormalTok{ b)}
    \OtherTok{{-}\textgreater{}} \DataTypeTok{Fold}\NormalTok{ r b}

\CommentTok{{-}{-} or}

\NormalTok{F.foldMap (\textbackslash{}r }\OtherTok{{-}\textgreater{}}\NormalTok{ [r])}
\OtherTok{    ::}\NormalTok{ ([r] }\OtherTok{{-}\textgreater{}}\NormalTok{ b)}
    \OtherTok{{-}\textgreater{}} \DataTypeTok{Fold}\NormalTok{ r b}
\end{Highlighting}
\end{Shaded}

So:

\begin{Shaded}
\begin{Highlighting}[]
\CommentTok{{-}{-} source: https://github.com/mstksg/inCode/tree/master/code{-}samples/adjunctions/foldl.hs\#L79{-}L80}

\OtherTok{tabulateFold ::}\NormalTok{ (}\DataTypeTok{EnvList}\NormalTok{ r () }\OtherTok{{-}\textgreater{}}\NormalTok{ b) }\OtherTok{{-}\textgreater{}} \DataTypeTok{Fold}\NormalTok{ r b}
\NormalTok{tabulateFold f }\OtherTok{=}\NormalTok{ F.foldMap (}\OperatorTok{:}\NormalTok{[]) (\textbackslash{}rs }\OtherTok{{-}\textgreater{}}\NormalTok{ f (}\DataTypeTok{EnvList}\NormalTok{ rs ()))}
\end{Highlighting}
\end{Shaded}

The fact the have \emph{both} of these gives us a pretty strong footing to claim
that \texttt{EnvList\ r} is the left-adjoint of \texttt{Fold\ r}. Proof by
hunch, for now.

Note that if we had missed \texttt{fold} during our adjunction hunt, we might
have also lucked out by noticing \texttt{F.foldMap\ (:{[}{]})} fitting the
criteria for a candidate for \texttt{tabulateAdjunction}, instead.

\section{Opposite Concepts}\label{opposite-concepts}

We've identified a likely candidate for a left-adjoint to \texttt{Fold\ r}! But
\ldots{} does any of this make any sense? Does this make sense as a
left-adjoint, conceptually \ldots{} and did we gain anything?

Let's think about this from the beginning: What is the conceptual opposite of
``something that folds a list''?

Well, what other thing is more naturally an opposite than ``a list to be
folded''!

\begin{itemize}
\tightlist
\item
  \texttt{EnvList\ r}: A list of \texttt{r}
\item
  \texttt{Fold\ r}: Consumes a list of \texttt{r}
\end{itemize}

Or, in terms of the result of the functor application:

\begin{itemize}
\tightlist
\item
  \texttt{EnvList\ r\ a}

  \begin{itemize}
  \tightlist
  \item
    An \texttt{a}
  \item
    \ldots{} tupled with a list of \texttt{r}
  \end{itemize}
\item
  \texttt{Fold\ r\ a}

  \begin{itemize}
  \tightlist
  \item
    An \texttt{a}
  \item
    \ldots{} parameterized by consumption of a list of \texttt{r}
  \end{itemize}
\end{itemize}

It seems to ``flip'' the idea of ``list vs.~list consumer'', and \emph{also} the
idea of ``tupled vs.~parameterizing'' (which was our first example of an
adjunction earlier, as well).

In addition, lists seem to be at the heart of how to create and consume a
\texttt{Fold\ r}.

\texttt{fold} can be thought of as the fundamental way to \emph{consuming} a
\texttt{Fold\ r}. This makes the adjunction with \texttt{EnvList\ r} make sense:
what good is the \emph{ability} to fold \ldots{} if there is nothing \emph{to
fold}? \texttt{EnvList\ r} (a list of \texttt{{[}r{]}}) is intimately related to
\texttt{Fold\ r}: they are the yin and yang, peanut butter and jelly, night and
day. Their fates are intertwined from their very inception. You cannot have one
without the other.

In addition, \texttt{F.foldMap} is arguably a fundamental (although maybe
inefficient) way to \emph{specify} a \texttt{Fold\ r}. A \texttt{Fold\ r} is,
fundamentally, a list processor --- which is what
\texttt{EnvList\ r\ a\ -\textgreater{}\ b} literally is (an
\texttt{{[}r{]}\ -\textgreater{}\ b}). \texttt{Fold\ r} and \texttt{EnvList\ r}
--- \href{https://starwars.fandom.com/wiki/Dyad_in_the_Force}{dyads in the
force}. (Or, well\ldots literally \emph{monads}, since all adjunctions give rise
to monads, as we will see later.)

The fact that \texttt{EnvList\ r} and \texttt{Fold\ r} form an adjunction
together formalizes the fact that they are conceptually ``opposite'' concepts,
and also that they are bound together by destiny in a close and fundamental way.

\begin{center}\rule{0.5\linewidth}{0.5pt}\end{center}

\textbf{A Note on Representable}

Note that in this case, a lot of what we are concluding simply stems from the
fact that we can ``index'' a \texttt{Fold\ r\ a} using an \texttt{{[}r{]}}. This
actually is more fundamentally associated with the concept of a
\href{https://hackage.haskell.org/package/adjunctions/docs/Data-Functor-Rep.html}{Representable
Functor}.

\begin{Shaded}
\begin{Highlighting}[]
\CommentTok{{-}{-} source: https://github.com/mstksg/inCode/tree/master/code{-}samples/adjunctions/foldl.hs\#L87{-}L90}

\KeywordTok{instance} \DataTypeTok{Representable}\NormalTok{ (}\DataTypeTok{Fold}\NormalTok{ r) }\KeywordTok{where}
    \KeywordTok{type} \DataTypeTok{Rep}\NormalTok{ (}\DataTypeTok{Fold}\NormalTok{ r) }\OtherTok{=}\NormalTok{ [r]}
\NormalTok{    tabulate }\OtherTok{=}\NormalTok{ F.foldMap (}\OperatorTok{:}\NormalTok{[])}
    \FunctionTok{index}    \OtherTok{=}\NormalTok{ F.fold}
\end{Highlighting}
\end{Shaded}

As it turns out, in Haskell, a functor being representable is \emph{equivalent}
to it having a left adjoint. So thinking of \texttt{Fold\ r} as a representable
functor and thinking of it as a right adjoint are equivalent ideas. This article
chooses to analyze it from the adjunctions perspective because we get to imagine
the adjoint \texttt{Functor}, which can sometimes reveal some extra insight over
just looking at some index \emph{value}.

\begin{center}\rule{0.5\linewidth}{0.5pt}\end{center}

\section{The Helper Functions}\label{the-helper-functions}

Let's take a look at some of the useful helper functions that an instance of
\texttt{Adjunction} gives us for \texttt{Fold\ r}, to see how their existence
can better help us understand \texttt{Fold}. For all of these, I'm going to
write them first as \texttt{EnvList\ r\ a}, and then also as
\texttt{({[}r{]},\ a)}, to help make things clearer.

\begin{Shaded}
\begin{Highlighting}[]
\OtherTok{unit ::}\NormalTok{ a }\OtherTok{{-}\textgreater{}} \DataTypeTok{Fold}\NormalTok{ r (}\DataTypeTok{EnvList}\NormalTok{ r a)}
\OtherTok{unit ::}\NormalTok{ a }\OtherTok{{-}\textgreater{}} \DataTypeTok{Fold}\NormalTok{ r ([r], a)}

\OtherTok{counit ::} \DataTypeTok{EnvList}\NormalTok{ r (}\DataTypeTok{Fold}\NormalTok{ r a) }\OtherTok{{-}\textgreater{}}\NormalTok{ a}
\OtherTok{counit ::}\NormalTok{ [r] }\OtherTok{{-}\textgreater{}} \DataTypeTok{Fold}\NormalTok{ r a }\OtherTok{{-}\textgreater{}}\NormalTok{ a}

\OtherTok{leftAdjunct ::}\NormalTok{ (}\DataTypeTok{EnvList}\NormalTok{ r a }\OtherTok{{-}\textgreater{}}\NormalTok{ b) }\OtherTok{{-}\textgreater{}}\NormalTok{ (a }\OtherTok{{-}\textgreater{}} \DataTypeTok{Fold}\NormalTok{ r b)}
\OtherTok{leftAdjunct ::}\NormalTok{ ([r] }\OtherTok{{-}\textgreater{}}\NormalTok{ a }\OtherTok{{-}\textgreater{}}\NormalTok{ b   ) }\OtherTok{{-}\textgreater{}}\NormalTok{ (a }\OtherTok{{-}\textgreater{}} \DataTypeTok{Fold}\NormalTok{ r b)}

\OtherTok{rightAdjunct ::}\NormalTok{ (a }\OtherTok{{-}\textgreater{}} \DataTypeTok{Fold}\NormalTok{ r b) }\OtherTok{{-}\textgreater{}}\NormalTok{ (}\DataTypeTok{EnvList}\NormalTok{ r a }\OtherTok{{-}\textgreater{}}\NormalTok{ b)}
\OtherTok{rightAdjunct ::}\NormalTok{ (a }\OtherTok{{-}\textgreater{}} \DataTypeTok{Fold}\NormalTok{ r b) }\OtherTok{{-}\textgreater{}}\NormalTok{ ([r] }\OtherTok{{-}\textgreater{}}\NormalTok{ a }\OtherTok{{-}\textgreater{}}\NormalTok{ b   )}

\OtherTok{tabulateAdjunction ::}\NormalTok{ (}\DataTypeTok{EnvList}\NormalTok{ r () }\OtherTok{{-}\textgreater{}}\NormalTok{ b) }\OtherTok{{-}\textgreater{}} \DataTypeTok{Fold}\NormalTok{ r b}
\OtherTok{tabulateAdjunction ::}\NormalTok{ ([r] }\OtherTok{{-}\textgreater{}}\NormalTok{ b)          }\OtherTok{{-}\textgreater{}} \DataTypeTok{Fold}\NormalTok{ r b}

\OtherTok{indexAdjunction ::} \DataTypeTok{Fold}\NormalTok{ r b }\OtherTok{{-}\textgreater{}} \DataTypeTok{EnvList}\NormalTok{ r a }\OtherTok{{-}\textgreater{}}\NormalTok{ b}
\OtherTok{indexAdjunction ::} \DataTypeTok{Fold}\NormalTok{ r b }\OtherTok{{-}\textgreater{}}\NormalTok{ [r]         }\OtherTok{{-}\textgreater{}}\NormalTok{ b}

\OtherTok{zipR ::} \DataTypeTok{Fold}\NormalTok{ r a }\OtherTok{{-}\textgreater{}} \DataTypeTok{Fold}\NormalTok{ r b }\OtherTok{{-}\textgreater{}} \DataTypeTok{Fold}\NormalTok{ r (a, b)}
\end{Highlighting}
\end{Shaded}

\begin{enumerate}
\def\labelenumi{\arabic{enumi}.}
\item
  \texttt{unit\ ::\ a\ -\textgreater{}\ Fold\ r\ ({[}r{]},\ a)}, when we
  specialize \texttt{a\ \textasciitilde{}\ ()}, becomes:

\begin{Shaded}
\begin{Highlighting}[]
\OtherTok{unit ::} \DataTypeTok{Fold}\NormalTok{ r [r]}
\end{Highlighting}
\end{Shaded}

  This means that \texttt{unit} for \texttt{Fold\ r} folds a list
  \texttt{{[}r{]}} into ``itself'', while also tagging on a value

\begin{Shaded}
\begin{Highlighting}[]
\NormalTok{F.fold (unit }\DataTypeTok{True}\NormalTok{) [}\DecValTok{1}\NormalTok{,}\DecValTok{2}\NormalTok{,}\DecValTok{3}\NormalTok{]}
\PreprocessorTok{\#   =\textgreater{} EnvList [1,2,3] True}
\end{Highlighting}
\end{Shaded}
\item
  \texttt{counit\ ::\ {[}r{]}\ -\textgreater{}\ Fold\ r\ a\ -\textgreater{}\ a}
  is essentially just \texttt{F.fold} when we expand it. Neat!
\item
  \texttt{leftAdjunct\ ::\ ({[}r{]}\ -\textgreater{}\ a\ -\textgreater{}\ b)\ -\textgreater{}\ (a\ -\textgreater{}\ Fold\ r\ b)}
  \ldots{} if we write it as
  \texttt{leftAdjunct\ ::\ a\ -\textgreater{}\ (a\ -\textgreater{}\ {[}r{]}\ -\textgreater{}\ b)\ -\textgreater{}\ Fold\ r\ b},
  and feed the \texttt{a} into the first function, we get:

\begin{Shaded}
\begin{Highlighting}[]
\OtherTok{leftAdjunct\textquotesingle{} ::}\NormalTok{ ([r] }\OtherTok{{-}\textgreater{}}\NormalTok{ b) }\OtherTok{{-}\textgreater{}} \DataTypeTok{Fold}\NormalTok{ r b}
\end{Highlighting}
\end{Shaded}

  which is just \texttt{tabulateAdjunction}, or \texttt{F.foldMap\ (:{[}{]})}!
  It encodes our list processor \texttt{{[}r{]}\ -\textgreater{}\ b} into a
  \texttt{Fold\ r\ b.}
\item
  \texttt{rightAdjunct\ ::\ (a\ -\textgreater{}\ Fold\ r\ b)\ -\textgreater{}\ ({[}r{]}\ -\textgreater{}\ a\ -\textgreater{}\ b)}
  -- if we again rewrite as
  \texttt{rightAdjunct\ ::\ a\ -\textgreater{}\ (a\ -\textgreater{}\ Fold\ r\ b)\ -\textgreater{}\ {[}r{]}\ -\textgreater{}\ b},
  and again feed the \texttt{a} into the first function, becomes:

\begin{Shaded}
\begin{Highlighting}[]
\OtherTok{rightAdjunct\textquotesingle{} ::} \DataTypeTok{Fold}\NormalTok{ r b }\OtherTok{{-}\textgreater{}}\NormalTok{ [r] }\OtherTok{{-}\textgreater{}}\NormalTok{ b}
\end{Highlighting}
\end{Shaded}

  which happens to just be \texttt{fold}, or \texttt{counit}!
\item
  \texttt{tabulateAdjunction} and \texttt{indexAdjunction} we went over earlier,
  seeing them as \texttt{F.foldMap\ (:{[}{]})} and \texttt{fold}
\item
  \texttt{zipR\ ::\ Fold\ r\ a\ -\textgreater{}\ Fold\ r\ b\ -\textgreater{}\ Fold\ r\ (a,\ b)}
  takes two \texttt{Fold\ r}s and combines them into a single fold. This is
  exactly the ``combining fold'' behavior that makes \texttt{Fold}s so useful!
  The implementation of \texttt{zipR} is less efficient than the implementation
  of \texttt{\textless{}*\textgreater{}}/\texttt{liftA2} for \texttt{Fold\ r},
  but knowing that \texttt{zipR} exists means that we know \texttt{Fold\ r}s can
  be combined.
\end{enumerate}

Seeing how these functions all fit together, we can write a full instance of
\texttt{Adjunction}. We can choose to provide \texttt{unit} and \texttt{counit},
or \texttt{leftAdjunct} and \texttt{rightAdjunct}\footnote{We should also be
  able to define it using \texttt{tabulateAdjunction} and
  \texttt{indexAdjunction} \ldots{} but this isn't allowed for some reason?};
the \texttt{unit}/\texttt{counit} definitions are the easiest to conceptualize,
for me, but the other pair isn't much tricker to write.

\begin{Shaded}
\begin{Highlighting}[]
\CommentTok{{-}{-} source: https://github.com/mstksg/inCode/tree/master/code{-}samples/adjunctions/foldl.hs\#L92{-}L97}

\KeywordTok{instance} \DataTypeTok{Adjunction}\NormalTok{ (}\DataTypeTok{EnvList}\NormalTok{ r) (}\DataTypeTok{Fold}\NormalTok{ r) }\KeywordTok{where}
\NormalTok{    unit x }\OtherTok{=}\NormalTok{ F.foldMap (}\OperatorTok{:}\NormalTok{[]) (}\OtherTok{\textasciigrave{}EnvList\textasciigrave{}}\NormalTok{ x)}
\NormalTok{    counit (}\DataTypeTok{EnvList}\NormalTok{ rs fld) }\OtherTok{=}\NormalTok{ F.fold fld rs}

\NormalTok{    leftAdjunct f x }\OtherTok{=}\NormalTok{ F.foldMap (}\OperatorTok{:}\NormalTok{[]) (\textbackslash{}rs }\OtherTok{{-}\textgreater{}}\NormalTok{ f (}\DataTypeTok{EnvList}\NormalTok{ rs x))}
\NormalTok{    rightAdjunct f (}\DataTypeTok{EnvList}\NormalTok{ rs x) }\OtherTok{=}\NormalTok{ F.fold (f x) rs}
\end{Highlighting}
\end{Shaded}

\subsection{Induced Monad and Comonad}\label{induced-monad-and-comonad}

Another interesting thing we might want to look at is the monad and comonad that
our adjunction defines.
\href{https://bartoszmilewski.com/2016/12/27/monads-categorically/}{All
adjunctions} \href{http://www.stephendiehl.com/posts/adjunctions.html}{define a
monad}, so what does our new knowledge of the \texttt{Fold\ r} adjunction give
us?

\subsubsection{Induced Monad}\label{induced-monad}

If we have \texttt{F\ -\textbar{}\ U}, then \texttt{U\ .\ F} is a monad. In this
case, we have \texttt{FoldEnv}:

\begin{Shaded}
\begin{Highlighting}[]
\CommentTok{{-}{-} source: https://github.com/mstksg/inCode/tree/master/code{-}samples/adjunctions/foldl.hs\#L99{-}L100}

\KeywordTok{newtype} \DataTypeTok{FoldEnv}\NormalTok{ r a }\OtherTok{=} \DataTypeTok{FE}\NormalTok{ \{}\OtherTok{ getFE ::} \DataTypeTok{Fold}\NormalTok{ r (}\DataTypeTok{EnvList}\NormalTok{ r a) \}}
  \KeywordTok{deriving} \DataTypeTok{Functor}

\CommentTok{{-}{-} or}
\KeywordTok{type} \DataTypeTok{FoldEnv}\NormalTok{ r }\OtherTok{=} \DataTypeTok{Fold}\NormalTok{ r ([r], a)}
\end{Highlighting}
\end{Shaded}

As it turns out, this is essentially equivalent to the famous
\href{https://hackage.haskell.org/package/transformers/docs/Control-Monad-Trans-State-Lazy.html}{State
Monad}! More specifically, it's the
\href{https://hackage.haskell.org/package/adjunctions/docs/Control-Monad-Representable-State.html}{\emph{Representable}
State Monad}.

\begin{Shaded}
\begin{Highlighting}[]
\KeywordTok{type} \DataTypeTok{FoldEnv}\NormalTok{ r }\OtherTok{=} \DataTypeTok{State}\NormalTok{ [r]}

\CommentTok{{-}{-} or, more literally, from Control.Monad.Representable.State}
\KeywordTok{type} \DataTypeTok{FoldEnv}\NormalTok{ r }\OtherTok{=} \DataTypeTok{State}\NormalTok{ (}\DataTypeTok{Fold}\NormalTok{ r)}
\end{Highlighting}
\end{Shaded}

So the induced monad from the adjunction we just found is essentially the same
as the \texttt{State} monad over a list --- except with some potentially
different performance characteristics.

In the end, finding this adjunction gives us a neat way to represent stateful
computations on lists, which is arguably an extension of what \texttt{Fold} was
really meant for in the first place.

\subsubsection{Induced Comonad}\label{induced-comonad}

If we have \texttt{F\ -\textbar{}\ U}, then \texttt{F\ .\ U} is a comonad. In
this case, we have \texttt{EnvFold}:

\begin{Shaded}
\begin{Highlighting}[]
\CommentTok{{-}{-} source: https://github.com/mstksg/inCode/tree/master/code{-}samples/adjunctions/foldl.hs\#L111{-}L112}

\KeywordTok{newtype} \DataTypeTok{EnvFold}\NormalTok{ r a }\OtherTok{=} \DataTypeTok{EF}\NormalTok{ \{}\OtherTok{ getEF ::} \DataTypeTok{EnvList}\NormalTok{ r (}\DataTypeTok{Fold}\NormalTok{ r a) \}}
  \KeywordTok{deriving} \DataTypeTok{Functor}

\CommentTok{{-}{-} or}
\KeywordTok{type} \DataTypeTok{EnvFold}\NormalTok{ r }\OtherTok{=}\NormalTok{ ([r], }\DataTypeTok{Fold}\NormalTok{ r a)}
\end{Highlighting}
\end{Shaded}

This one is exactly the
\href{https://hackage.haskell.org/package/adjunctions/docs/Control-Comonad-Representable-Store.html}{Representable
Store comonad}. It's essentially the normal
\href{https://hackage.haskell.org/package/comonad/docs/Control-Comonad-Store.html}{\texttt{Store}
comonad}:

\begin{Shaded}
\begin{Highlighting}[]
\KeywordTok{type} \DataTypeTok{EnvFold}\NormalTok{ r }\OtherTok{=} \DataTypeTok{Store}\NormalTok{ [r]}

\CommentTok{{-}{-} or, more literally, from Control.Comonad.Representable.Store}
\KeywordTok{type} \DataTypeTok{EnvFold}\NormalTok{ r }\OtherTok{=} \DataTypeTok{Store}\NormalTok{ (}\DataTypeTok{Fold}\NormalTok{ r)}
\end{Highlighting}
\end{Shaded}

This comonad ``stores'' an \texttt{{[}r{]}} as well as a way to produce an
\texttt{a} from an \texttt{{[}r{]}}. All of the utility of this induced comonad
basically is the same as the utility of \texttt{Store}, except with potentially
different performance profiles.

In the end, finding this adjunction gives us a neat way to define a comonadic
contextual projection on lists, which I would say is also an extension of the
original purpose of \texttt{Fold}.

\section{Conclusion}\label{conclusion}

In the end, in a ``practical'' sense, we got some nice helper functions, as well
as a new way to extend the conceptual idea of \texttt{Fold} using the induced
monad and comonad.

Admittedly, the selection of helper functions that \texttt{Adjunction} gives us
pales in comparison to abstractions like \texttt{Monoid}, \texttt{Applicative},
\texttt{Traversable}, \texttt{Monad}, etc., which makes \texttt{Adjunction} (in
my opinion) nowhere as practical when compared to them. A lot of these helper
functions (like the induced state monad and store comonad) actually also just
exist if we only talk about \texttt{Representable}.

However, to me (and, as how I've seen other people use it), \texttt{Adjunction}
is most useful as a conceptual tool in Haskell. The idea of ``opposites'' or
``duals'' show up a lot in Haskell, starting from the most basic level -- sums
and products, products and functions. From day 1, Haskellers are introduced to
natural pairs and opposites in concepts. The idea of opposites and how they
interact with each other is always on the mind of a Haskeller, and close to
their heart.

So, what makes \texttt{Adjunction} so useful to me is that it actually is able
to formalize what we mean by ``opposite concepts''. The process of identifying a
functor's ``opposite concept'' (if it exists) will only help is better
understand the functor we're thinking about, in terms of how it works and how it
is used.

Hopefully this blog post helps you appreciate both \texttt{Fold} in a new way,
and also the fundamental ``idea'' of adjunctions in Haskell.

\section{The Algebraic Way}\label{the-algebraic-way}

This article is done! Our first guess for an adjunction seems to be morally
correct. But as an aside \ldots{} let's see if we can take this idea further.

In this section we're going to get a bit mathy and look at the definition of
\texttt{Fold}, to see if we can \emph{algebraically} find an adjunction of
\texttt{Fold}, instead of just trying to hunt for API functions like before. In
practice you don't often have to make algebraic deductions like this, but it's
at least nice to know that something like this possible from a purely algebraic
and logical sense. You never \emph{need} all this fancy math to be able to write
Haskell \ldots{} but many feel like it can make things a lot more fun! :)

Be warned that this method \emph{does} require some familiarity (or at least
awareness) of certain types that appear often in the more \ldots{} esoteric
corners of Haskelldom :)

The game plan here is to start with the definition of \texttt{Fold}, and then
rearrange it using algebraic substitutions until it matches something that
already has an \texttt{Adjunction} instance in the \emph{adjunctions} library.

First, the actual definition of \texttt{Fold} in the \emph{foldl} library itself
is:

\begin{Shaded}
\begin{Highlighting}[]
\KeywordTok{data} \DataTypeTok{Fold}\NormalTok{ r a }\OtherTok{=} \KeywordTok{forall}\NormalTok{ x}\OperatorTok{.} \DataTypeTok{Fold}\NormalTok{ (x }\OtherTok{{-}\textgreater{}}\NormalTok{ r }\OtherTok{{-}\textgreater{}}\NormalTok{ x) x (x }\OtherTok{{-}\textgreater{}}\NormalTok{ a)}
\end{Highlighting}
\end{Shaded}

Maybe not the friendliest definition at first! But something in this looks a
little familiar, maybe. Let's do some re-arranging:

\begin{Shaded}
\begin{Highlighting}[]
\KeywordTok{data} \DataTypeTok{Fold}\NormalTok{ r a }\OtherTok{=} \KeywordTok{forall}\NormalTok{ x}\OperatorTok{.} \DataTypeTok{Fold}\NormalTok{ (x }\OtherTok{{-}\textgreater{}}\NormalTok{ r }\OtherTok{{-}\textgreater{}}\NormalTok{ x) x (x }\OtherTok{{-}\textgreater{}}\NormalTok{ a)}
              \OtherTok{=} \KeywordTok{forall}\NormalTok{ x}\OperatorTok{.} \DataTypeTok{Fold}\NormalTok{ x (x }\OtherTok{{-}\textgreater{}}\NormalTok{ r }\OtherTok{{-}\textgreater{}}\NormalTok{ x) (x }\OtherTok{{-}\textgreater{}}\NormalTok{ a)}
              \OtherTok{=} \KeywordTok{forall}\NormalTok{ x}\OperatorTok{.} \DataTypeTok{Fold}\NormalTok{ x (x }\OtherTok{{-}\textgreater{}}\NormalTok{ (r }\OtherTok{{-}\textgreater{}}\NormalTok{ x, a))}
\end{Highlighting}
\end{Shaded}

Ah, this looks a \emph{lot} like the constructor \texttt{Nu} for some
\texttt{f}:

\begin{Shaded}
\begin{Highlighting}[]
\KeywordTok{data} \DataTypeTok{Nu}\NormalTok{ f }\OtherTok{=} \KeywordTok{forall}\NormalTok{ x}\OperatorTok{.} \DataTypeTok{Nu}\NormalTok{ (x }\OtherTok{{-}\textgreater{}}\NormalTok{ f x) x}
\end{Highlighting}
\end{Shaded}

\texttt{Nu} is one of the three main famous
\href{https://hackage.haskell.org/package/recursion-schemes/docs/Data-Functor-Foldable.html}{fixed-point
type combinators} in Haskell. The other two are \texttt{Mu} and \texttt{Fix}:

\begin{Shaded}
\begin{Highlighting}[]
\KeywordTok{data}    \DataTypeTok{Nu}\NormalTok{  f }\OtherTok{=} \KeywordTok{forall}\NormalTok{ x}\OperatorTok{.} \DataTypeTok{Nu}\NormalTok{ (x }\OtherTok{{-}\textgreater{}}\NormalTok{ f x) x}
\KeywordTok{newtype} \DataTypeTok{Fix}\NormalTok{ f }\OtherTok{=} \DataTypeTok{Fix}\NormalTok{ (f (}\DataTypeTok{Fix}\NormalTok{ f))}
\KeywordTok{newtype} \DataTypeTok{Mu}\NormalTok{  f }\OtherTok{=} \DataTypeTok{Mu}\NormalTok{ (}\KeywordTok{forall}\NormalTok{ x}\OperatorTok{.}\NormalTok{ (f x }\OtherTok{{-}\textgreater{}}\NormalTok{ x) }\OtherTok{{-}\textgreater{}}\NormalTok{ x)}
\end{Highlighting}
\end{Shaded}

In Haskell these are all equivalent\footnote{They are only equivalent in Haskell
  because of laziness --- in strict languages, they are different.}, but they
have very different performance profiles for certain operations. \texttt{Nu} is
easy to ``build up'', and \texttt{Mu} is easy to ``tear down'' -- and they exist
sort of opposite to each other. \texttt{Fix} exists in opposite to \ldots{}
itself. Sorry, \texttt{Fix}.

Anyway, looking at \texttt{Fold\ r\ a}:

\begin{Shaded}
\begin{Highlighting}[]
\KeywordTok{data} \DataTypeTok{Fold}\NormalTok{ r a }\OtherTok{=} \KeywordTok{forall}\NormalTok{ x}\OperatorTok{.} \DataTypeTok{Fold}\NormalTok{ x (x }\OtherTok{{-}\textgreater{}}\NormalTok{ (r }\OtherTok{{-}\textgreater{}}\NormalTok{ x, a))}
\KeywordTok{data} \DataTypeTok{Nu}\NormalTok{   f   }\OtherTok{=} \KeywordTok{forall}\NormalTok{ x}\OperatorTok{.} \DataTypeTok{Nu}\NormalTok{   x (x }\OtherTok{{-}\textgreater{}}\NormalTok{ f x)}
\end{Highlighting}
\end{Shaded}

it seems like we can pick an \texttt{F} such that
\texttt{Nu\ (F\ r\ a)\ =\ Fold\ r\ a}. Let's try\ldots{}

\begin{Shaded}
\begin{Highlighting}[]
\KeywordTok{newtype}\NormalTok{ (f }\OperatorTok{:.:}\NormalTok{ g) x }\OtherTok{=} \DataTypeTok{Comp}\NormalTok{ (f (g x))}

\KeywordTok{type} \DataTypeTok{Fold}\NormalTok{ r a }\OtherTok{=} \DataTypeTok{Nu}\NormalTok{ (((,) a) }\OperatorTok{:.:}\NormalTok{ ((}\OtherTok{{-}\textgreater{}}\NormalTok{) r))}
\end{Highlighting}
\end{Shaded}

From here, some might recognize the fixed point of \texttt{(,)\ a\ :.:\ f} as
\texttt{Cofree}, from
\href{https://hackage.haskell.org/package/free/docs/Control-Comonad-Cofree.html}{Control.Comonad.Cofree}
--- one of the more commonly used fixed points.

\begin{Shaded}
\begin{Highlighting}[]
\KeywordTok{type} \DataTypeTok{Cofree}\NormalTok{ f a }\OtherTok{=} \DataTypeTok{Nu}\NormalTok{ ((,) a }\OperatorTok{:.:}\NormalTok{ f     )}

\KeywordTok{type} \DataTypeTok{Fold}\NormalTok{ r a   }\OtherTok{=} \DataTypeTok{Nu}\NormalTok{ ((,) a }\OperatorTok{:.:}\NormalTok{ (}\OtherTok{{-}\textgreater{}}\NormalTok{) r)}
\KeywordTok{type} \DataTypeTok{Fold}\NormalTok{ r     }\OtherTok{=} \DataTypeTok{Cofree}\NormalTok{ ((}\OtherTok{{-}\textgreater{}}\NormalTok{) r)}
\end{Highlighting}
\end{Shaded}

It looks like \texttt{Fold\ r} is just \texttt{Cofree\ ((-\textgreater{})\ r)}
\footnote{Some might recognize \texttt{Cofree\ ((-\textgreater{})\ r)} as a
  common way of implementing a
  \href{https://en.wikipedia.org/wiki/Moore_machine}{Moore machine} in Haskell.
  In fact, our derivation here is basically a backwards version of
  \href{https://www.schoolofhaskell.com/user/edwardk/moore/for-less}{the process
  described here by Edward Kmett}.} \ldots{} and now we've hit the jackpot!
That's because \texttt{Cofree\ f} has an instance of \texttt{Adjunction}!

\begin{Shaded}
\begin{Highlighting}[]
\KeywordTok{instance} \DataTypeTok{Adjunction}\NormalTok{ f u }\OtherTok{=\textgreater{}} \DataTypeTok{Adjunction}\NormalTok{ (}\DataTypeTok{Free}\NormalTok{ f) (}\DataTypeTok{Cofree}\NormalTok{ u)}
\end{Highlighting}
\end{Shaded}

This means that \texttt{Cofree\ u} is right-adjoint to \texttt{Free\ f}, if
\texttt{f} is right-adjoint to \texttt{u}. Well, our \texttt{u} here is
\texttt{(-\textgreater{})\ r}, which was actually our very first example of a
right-adjoint functor --- it's right-adjoint to \texttt{(,)}. So,
\texttt{Fold\ r} is apparently a right-adjoint, like we guessed previously! More
specifically, it looks like like \texttt{Fold\ r} is right-adjoint to
\texttt{Free\ ((,)\ r)}.

At least, we've reached our goal! We found an adjunction for \texttt{Fold\ r} in
a purely algebraic way, and deduced it to be right-adjunct to
\texttt{Free\ ((,)\ r)}.

At this point we have our answer, so we can stop here. But it's possible to go a
little further, to find a true ``perfect companion'' for \texttt{Fold\ r}. Its
perfect match and conceptual opposite, as the adjunction mythos claims.

We know that \texttt{Free\ f} is, itself, a fixed-point -- it's the fixed point
of \texttt{Sum\ (Const\ a)\ f} (from
\emph{\href{https://hackage.haskell.org/package/base/docs/Data-Functor-Sum.html}{Data.Functor.Sum}}).
So \texttt{Free\ ((,)\ r)\ a} is the fixed-point of
\texttt{Sum\ (Const\ a)\ ((,)\ r)}. Since we are looking at conceptual
opposites, maybe let's try using the \texttt{Mu} fixed-point operator, to be
opposite of the \texttt{Nu} that \texttt{Fold\ r} is. This also makes sense
because this is something we're going to ``tear down'' with a \texttt{Fold}, and
\texttt{Mu} is good at being torn down.

\begin{Shaded}
\begin{Highlighting}[]
\KeywordTok{newtype} \DataTypeTok{Mu}\NormalTok{ f }\OtherTok{=} \DataTypeTok{Mu}\NormalTok{ (}\KeywordTok{forall}\NormalTok{ x}\OperatorTok{.}\NormalTok{ (f x }\OtherTok{{-}\textgreater{}}\NormalTok{ x) }\OtherTok{{-}\textgreater{}}\NormalTok{ x)}

\KeywordTok{type} \DataTypeTok{EL}\NormalTok{ r a }\OtherTok{=} \DataTypeTok{Mu}\NormalTok{ (}\DataTypeTok{Sum}\NormalTok{ (}\DataTypeTok{Const}\NormalTok{ a) ((,) r))}

\KeywordTok{newtype} \DataTypeTok{EL}\NormalTok{ r a }\OtherTok{=} \DataTypeTok{EL}\NormalTok{ \{}
\OtherTok{    runEL ::} \KeywordTok{forall}\NormalTok{ x}\OperatorTok{.}\NormalTok{ (}\DataTypeTok{Either}\NormalTok{ a (r, x) }\OtherTok{{-}\textgreater{}}\NormalTok{ x) }\OtherTok{{-}\textgreater{}}\NormalTok{ x}
\NormalTok{  \}}

\CommentTok{{-}{-} or, with some shuffling around, recognizing that \textasciigrave{}Either a b {-}\textgreater{} c\textasciigrave{} is}
\CommentTok{{-}{-} equivalent to \textasciigrave{}(a {-}\textgreater{} c, b {-}\textgreater{} c)\textasciigrave{}}
\KeywordTok{newtype} \DataTypeTok{EL}\NormalTok{ r a }\OtherTok{=} \DataTypeTok{EL}\NormalTok{ \{}
\OtherTok{    runEL ::} \KeywordTok{forall}\NormalTok{ x}\OperatorTok{.}\NormalTok{ (a }\OtherTok{{-}\textgreater{}}\NormalTok{ x) }\OtherTok{{-}\textgreater{}}\NormalTok{ (x }\OtherTok{{-}\textgreater{}}\NormalTok{ r }\OtherTok{{-}\textgreater{}}\NormalTok{ x) }\OtherTok{{-}\textgreater{}}\NormalTok{ x}
\NormalTok{  \}}
\end{Highlighting}
\end{Shaded}

The new \texttt{EL} is actually isomorphic to the \texttt{EnvList} one we wrote
earlier (as long as the list is finite), meaning that one can encode the other,
and they have identical structure. Writing functions to convert between the two
can be fun;
\href{https://github.com/mstksg/inCode/tree/master/code-samples/adjunctions/foldl-algebraic.hs\#L31-L39}{here
is one solution}, and there's a
\href{https://github.com/mstksg/inCode/tree/master/code-samples/adjunctions/foldl-algebraic.hs\#L73-L79}{bonus
solution} if you can write it using only the
\href{https://github.com/mstksg/inCode/tree/master/code-samples/adjunctions/foldl-algebraic.hs\#L58-L71}{\emph{new}
instance} for \texttt{Adjunction\ (EL\ r)\ (Fold\ r)} and \texttt{F.foldMap},
since it can be shown that all adjuncts are unique up to isomorphism.

And\ldots this looks pretty neat, I think. In the end we discover that these two
types are adjoints to each other:\footnote{The instance is
  \href{https://github.com/mstksg/inCode/tree/master/code-samples/adjunctions/foldl-algebraic.hs\#L58-L71}{written
  out here}.}

\begin{Shaded}
\begin{Highlighting}[]
\KeywordTok{data} \DataTypeTok{Fold}\NormalTok{ r a }\OtherTok{=} \KeywordTok{forall}\NormalTok{ x}\OperatorTok{.} \DataTypeTok{Fold}\NormalTok{            (x }\OtherTok{{-}\textgreater{}}\NormalTok{ a)    (x }\OtherTok{{-}\textgreater{}}\NormalTok{ r }\OtherTok{{-}\textgreater{}}\NormalTok{ x)    x}
\KeywordTok{data} \DataTypeTok{EL}\NormalTok{   r a }\OtherTok{=}           \DataTypeTok{EL}\NormalTok{   (}\KeywordTok{forall}\NormalTok{ x}\OperatorTok{.}\NormalTok{ (a }\OtherTok{{-}\textgreater{}}\NormalTok{ x) }\OtherTok{{-}\textgreater{}}\NormalTok{ (x }\OtherTok{{-}\textgreater{}}\NormalTok{ r }\OtherTok{{-}\textgreater{}}\NormalTok{ x) }\OtherTok{{-}\textgreater{}}\NormalTok{ x)}
\end{Highlighting}
\end{Shaded}

They look superficially syntactically similar and I don't really know what to
make of that \ldots{} but a lot of ``opposites'' seem to be paired here. The
existential \texttt{x} in \texttt{Fold} becomes a Rank2 universal in
\texttt{EList}, and the \texttt{x\ -\textgreater{}\ a} in \texttt{Fold} becomes
an \texttt{a\ -\textgreater{}\ x} in \texttt{EList}. Neat neat.

Adjunctions: take an idea and just make everything opposite.

One nice thing about this representation is that writing the fundamental
operation of \texttt{Fold} (that is, \texttt{fold}) becomes really
clean:\footnote{Implementing a \texttt{tabulate} equivalent
  (\href{https://github.com/mstksg/inCode/tree/master/code-samples/adjunctions/foldl-algebraic.hs\#L54-L56}{solution
  here}) reveals that this refactoring is only really useful for \texttt{index}
  (to ``consume'' a \texttt{Fold}) \ldots{} using \texttt{tabulate} or
  \texttt{leftAdjunct} to ``produce'' a \texttt{Fold} reveals how inefficient
  this is.}

\begin{Shaded}
\begin{Highlighting}[]
\CommentTok{{-}{-} source: https://github.com/mstksg/inCode/tree/master/code{-}samples/adjunctions/foldl{-}algebraic.hs\#L51{-}L52}

\OtherTok{foldEL ::} \DataTypeTok{Fold}\NormalTok{ r b }\OtherTok{{-}\textgreater{}} \DataTypeTok{EL}\NormalTok{ r a }\OtherTok{{-}\textgreater{}}\NormalTok{ b}
\NormalTok{foldEL (}\DataTypeTok{Fold}\NormalTok{ step }\FunctionTok{init}\NormalTok{ extr) el }\OtherTok{=}\NormalTok{ extr (runEL el (}\FunctionTok{const} \FunctionTok{init}\NormalTok{) step)}
\end{Highlighting}
\end{Shaded}

And this is, maybe, the real treasure all along.

\section{Special Thanks}\label{special-thanks}

I am very humbled to be supported by an amazing community, who make it possible
for me to devote time to researching and writing these posts. Very special
thanks to my supporter at the ``Amazing'' level on
\href{https://www.patreon.com/justinle/overview}{patreon}, Josh Vera! :)

\section{Signoff}\label{signoff}

Hi, thanks for reading! You can reach me via email at
\href{mailto:justin@jle.im}{\nolinkurl{justin@jle.im}}, or at twitter at
\href{https://twitter.com/mstk}{@mstk}! This post and all others are published
under the \href{https://creativecommons.org/licenses/by-nc-nd/3.0/}{CC-BY-NC-ND
3.0} license. Corrections and edits via pull request are welcome and encouraged
at \href{https://github.com/mstksg/inCode}{the source repository}.

If you feel inclined, or this post was particularly helpful for you, why not
consider \href{https://www.patreon.com/justinle/overview}{supporting me on
Patreon}, or a \href{bitcoin:3D7rmAYgbDnp4gp4rf22THsGt74fNucPDU}{BTC donation}?
:)

\textbackslash end\{document\}

\end{document}
