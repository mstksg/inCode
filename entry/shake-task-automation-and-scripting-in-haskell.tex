\documentclass[]{article}
\usepackage{lmodern}
\usepackage{amssymb,amsmath}
\usepackage{ifxetex,ifluatex}
\usepackage{fixltx2e} % provides \textsubscript
\ifnum 0\ifxetex 1\fi\ifluatex 1\fi=0 % if pdftex
  \usepackage[T1]{fontenc}
  \usepackage[utf8]{inputenc}
\else % if luatex or xelatex
  \ifxetex
    \usepackage{mathspec}
    \usepackage{xltxtra,xunicode}
  \else
    \usepackage{fontspec}
  \fi
  \defaultfontfeatures{Mapping=tex-text,Scale=MatchLowercase}
  \newcommand{\euro}{€}
\fi
% use upquote if available, for straight quotes in verbatim environments
\IfFileExists{upquote.sty}{\usepackage{upquote}}{}
% use microtype if available
\IfFileExists{microtype.sty}{\usepackage{microtype}}{}
\usepackage[margin=1in]{geometry}
\ifxetex
  \usepackage[setpagesize=false, % page size defined by xetex
              unicode=false, % unicode breaks when used with xetex
              xetex]{hyperref}
\else
  \usepackage[unicode=true]{hyperref}
\fi
\hypersetup{breaklinks=true,
            bookmarks=true,
            pdfauthor={Justin Le},
            pdftitle={Shake: Task Automation and Scripting in Haskell},
            colorlinks=true,
            citecolor=blue,
            urlcolor=blue,
            linkcolor=magenta,
            pdfborder={0 0 0}}
\urlstyle{same}  % don't use monospace font for urls
% Make links footnotes instead of hotlinks:
\renewcommand{\href}[2]{#2\footnote{\url{#1}}}
\setlength{\parindent}{0pt}
\setlength{\parskip}{6pt plus 2pt minus 1pt}
\setlength{\emergencystretch}{3em}  % prevent overfull lines
\setcounter{secnumdepth}{0}

\title{Shake: Task Automation and Scripting in Haskell}
\author{Justin Le}
\date{September 17, 2013}

\begin{document}
\maketitle

\emph{Originally posted on
\textbf{\href{https://blog.jle.im/entry/shake-task-automation-and-scripting-in-haskell.html}{in
Code}}.}

\documentclass[]{}
\usepackage{lmodern}
\usepackage{amssymb,amsmath}
\usepackage{ifxetex,ifluatex}
\usepackage{fixltx2e} % provides \textsubscript
\ifnum 0\ifxetex 1\fi\ifluatex 1\fi=0 % if pdftex
  \usepackage[T1]{fontenc}
  \usepackage[utf8]{inputenc}
\else % if luatex or xelatex
  \ifxetex
    \usepackage{mathspec}
    \usepackage{xltxtra,xunicode}
  \else
    \usepackage{fontspec}
  \fi
  \defaultfontfeatures{Mapping=tex-text,Scale=MatchLowercase}
  \newcommand{\euro}{€}
\fi
% use upquote if available, for straight quotes in verbatim environments
\IfFileExists{upquote.sty}{\usepackage{upquote}}{}
% use microtype if available
\IfFileExists{microtype.sty}{\usepackage{microtype}}{}
\usepackage[margin=1in]{geometry}
\ifxetex
  \usepackage[setpagesize=false, % page size defined by xetex
              unicode=false, % unicode breaks when used with xetex
              xetex]{hyperref}
\else
  \usepackage[unicode=true]{hyperref}
\fi
\hypersetup{breaklinks=true,
            bookmarks=true,
            pdfauthor={},
            pdftitle={},
            colorlinks=true,
            citecolor=blue,
            urlcolor=blue,
            linkcolor=magenta,
            pdfborder={0 0 0}}
\urlstyle{same}  % don't use monospace font for urls
% Make links footnotes instead of hotlinks:
\renewcommand{\href}[2]{#2\footnote{\url{#1}}}
\setlength{\parindent}{0pt}
\setlength{\parskip}{6pt plus 2pt minus 1pt}
\setlength{\emergencystretch}{3em}  % prevent overfull lines
\setcounter{secnumdepth}{0}


\begin{document}

As someone who comes from a background in ruby and *rake*, I'm used to powerful
task management systems with expressive dependency. *Make* is a favorite tool of
mine when I'm working on projects with people who don't use ruby, and when I'm
working on ruby projects I never go far without starting a good Rakefile. The
two tools provided a perfect DSL for setting up systems of tasks that had
complicated file and task dependencies.

As I was starting to learn Haskell and building larger-scale Haskell projects, I
began to look for alternatives in Haskell. Was there a Haskell counterpart to
Ruby's [*rake*](http://rake.rubyforge.org/), Node's
[*jake*](https://github.com/mde/jake)? (Not to mention the tools of slightly
different philosophy [*grunt*](http://gruntjs.com/) and
[*ant*](http://ant.apache.org/))

It turns out that by far the most established answer is a library known as
[*Shake*](http://hackage.haskell.org/package/shake) (maintained by the prolific
Neil Mitchell of [*hoogle*](http://haskell.org/hoogle) fame and much more). So
far it's served me pretty well. Its documentation is written from the
perspective of chiefly using it as a build tool (more "make" than "rake"), so if
you're looking to use it as a task management system, you might have to do some
digging. Hopefully this post can help you get started.

I also go over the core concepts of a task management system, so I assume no
knowledge of *make*; this post therefore should also be a good introduction to
starting with any sort of task management system.

## Our Sample Project

Our sample project is going to be a report build system that builds reports
written in markdown with [pandoc](http://johnmacfarlane.net/pandoc/) into html,
pdf, and doc formats. This is honestly one of my most common use cases for
*make*, so porting it all to *shake* will be something useful for me.

The final directory structure will look like this:

> -   img
>     -   img1.jpg
>     -   img2.jpg
> -   out
>     -   report.doc
>     -   report.html
>     -   report.pdf
> -   src
>     -   report.md
> -   css
>     -   report.css
> -   Shakefile

When we run `shake`, we want to build `report.doc` and `report.pdf` if
`report.md` or any of the images have changed, and `report.html` if `report.md`,
`report.css`, or any of the images have changed.

Furthermore, `img2.jpg` actually comes from online, and requires us to
re-download it every time we compile to make sure it is up to date.

## Setup

### Installing Shake

Installing *shake* is as simple as installing any other cabal package:

``` bash
$ cabal update
$ cabal install shake
```

I'll will be using `shake-0.10.6` for this post.

### Setting up the Shakefile

We set up our Shakefile with a simple scaffold:

``` haskell
-- Shakefile

import Development.Shake

opts = shakeOptions { shakeFiles    = ".shake/" }        -- 1

(~>) = phony                                             -- 2
                                                         -- (obsolete)

main :: IO ()
main = shakeArgs opts $ do
    want []

    "clean" ~> removeFilesAfter ".shake" ["//*"]
```

On my machine I've set this up to be generated by a [bash
script](https://gist.github.com/mstksg/6588764) called "shakeup", so I can start
a project up on a Shakefile by simply typing `shakeup` at the project root.

Some notes:

1.  Store shake's metadata files to the folder `.shake/`. This differs from the
    default behavior, where all files would be saved to the root directory with
    `.shake` as a filename prefix.

2.  I've aliased the operator `~>` for `phony` to allow for a more expressive
    infix notation --- more on this later. I've submitted a patch to the project
    and it should be included in the next cabal release.

    **Edit**: As of the 0.10.7 release of *Shake*, this is no longer needed, as
    `~>` is included in the library.

## What is a Rule?

If you haven't used *make* before, it is important that you understand the key
concepts before moving on.

A task management system/build system is a system that works to ensure that all
files in the project are "up to date". In our case, our system will ensure that
the files in the `out` directory are up to date.

In order to do this, files are given "rules". Rules specify:

1.  What other files/rules this file "depends" on

2.  Instructions to execute to make this file up to date (or to create the
    file), if it is not already up to date or created.

A file or rule is out of date if any of its dependencies are out of date **or**
if the file it indicates is either not created or has been updated since the
last time the task management system has run. When this happens, the guilty
dependencies are updated using their own rules. Afterwards, the file's own
instructions are executed.

If a file has no rule, "out of date" simply means that it has been updated or
changed since the last time the task management system has run, or it does not
exist. If it has, then all files or rules that depend on it are also out of
date.

A good task management system is smart enough to keep track of what is up to
date and what isn't. If multiple rules all have one dependency, that dependency
might be checked and updated every single time. For example, all of our builds
in this sample project require `img2.jpg` to be downloaded afresh from online. A
naive build system might re-download `img2.jpg` for every single build, instead
of once for all three.

## File Rules

Let's set up `src/report.md` with a simple markdown document on our new project:

``` markdown
<!-- src/report.md -->

Report
======

This is a report.  Render me!

![first image](img/img1.jpg)
![second image](img/img2.jpg)
```

Our project tree should look like this at this point:

> -   img
>     -   img1.jpg
>     -   img2.jpg
> -   out
> -   src
>     -   report.md
> -   template
> -   Shakefile

Let's set up our first rule -- rendering `out/report.doc` if `report.md` has
changed.

``` haskell
"out/report.doc" *> \f -> do
    need ["src/report.md","img/img1.jpg","img/img2.jpg"]
    cmd "pandoc" [ "src/report.md", "-o", f ]
```

This is equivalent to the Makefile rule:

``` makefile
out/report.doc: src/report.md
	pandoc src/report.md -o out/report.doc
```

The operator `*>` attaches an
[`Action`](http://hackage.haskell.org/packages/archive/shake/0.10.6/doc/html/Development-Shake.html#t:Action)
(with a parameter) to a
[`FilePattern`](http://hackage.haskell.org/packages/archive/shake/0.10.6/doc/html/Development-Shake.html#t:FilePattern)
(a string) -- that is, when *shake* decides that it needs that specified file on
the left hand side to be up to date, it runs the action on the right hand side
with that filename as a parameter.

To be clear, the right hand side is of type:

``` haskell
rightHandSide :: FilePattern -> Action ()
```

where the `FilePattern` is the filename of the file that is being "needed".

The `need` function specifies all of the dependencies of that action. If *shake*
decides it needs `out/report.doc` to be up to date, `need` tells it that it
first needs `src/report.md` and the images to be up to date -- or rather, that
`out/report.doc` is only out of date if `src/report.md` or the images are out of
date, or have changed since the last build.

With this in mind, let us write the rest of our file rules:

``` haskell
-- Shakefile

"out/report.doc" *> \f -> do
    need ["src/report.md","img/img1.jpg","img/img2.jpg"]
    cmd "pandoc" [ "src/report.md", "-o", f ]

"out/report.pdf" *> \f -> do
    need ["src/report.md","img/img1.jpg","img/img2.jpg"]
    cmd "pandoc" [ "src/report.md", "-o", f, "-V", "links-as-notes" ]

"out/report.html" *> \f -> do
    need [ "src/report.md"
         , "img/img1.jpg"
         , "img/img2.jpg"
         , "css/report.css" ]
    cmd "pandoc" [ "src/report.md", "-o", f, "-c", "css/report.css", "-S" ]

"img/img2.jpg" *> \f -> do
    cmd "wget" [ "http://example.com/img2.jpg", "-O", f ]
```

And that is it!

## Running Shake

How do we tell *shake* what file it is that we want to be up to date? We specify
this by modifying the line `want []`:

``` haskell
want ["out/report.doc","out/report.pdf","out/report.html"]
```

That tells *shake* that when we run `main` with no arguments, we want those
three files to be checked to be up to date.

Now, to wrap it all together, we run:

``` bash
$ runhaskell Shakefile
```

And let the magic happen!

I run this enough times that I like to alias this:

``` bash
# in ~/.bashrc
alias shake=runhaskell Shakefile
```

Note that `want` specifies the **default** "wants". You can specify your own
collection by passing a parameter:

``` bash
$ runhaskell Shakefile out/report.doc
```

## Wildcards

You may have noticed that even though we had multiple images in the `img`
folder, we required them all explicitly. This could cause problems. What if in
the future, our documents used more images?

We can define wildcards using *shake*'s `getDirectoryFiles`, which returns
results of a wildcard search in an `Action` monad. `getDirectoryFiles` takes a
directory base and a list of wildcards.

``` haskell
-- Shakefile

srcFiles :: Action [FilePath]
srcFiles = getDirectoryFiles ""
    [ "src/report.md"
    , "img/*.jpg" ]

main :: IO ()
main = shakeArgs opts $ do
    want ["out/report.doc","out/report.pdf","out/report.html"]

    "out/report.doc" *> \f -> do
        deps <- srcFiles
        need deps
        cmd "pandoc" [ "src/report.md", "-o", f ]

    "out/report.pdf" *> \f -> do
        deps <- srcFiles
        need deps
        cmd "pandoc" [ "src/report.md", "-o", f, "-V", "links-as-notes" ]

    "out/report.html" *> \f -> do
        deps <- srcFiles
        need $ "css/report.css" : deps
        cmd "pandoc" [ "src/report.md", "-o", f, "-c", "css/report.css", "-S" ]

    "img/img2.jpg" *> \f -> do
        cmd "wget" [ "http://example.com/img2.jpg", "-O", f ]

    "clean" ~> removeFilesAfter ".shake" ["//*"]
```

If you are comfortable with monadic operators, you can make it all happen on one
line:

``` haskell
"out/report.doc" *> \f -> do
    need =<< srcFiles
    cmd "pandoc" [ "src/report.md", "-o", f ]
```

## Phony Rules

Now, you might sometimes want rules that are "just tasks" that don't relate to
creating a specific file. That is, they still depend on other files or rules and
are triggered to update when their dependencies are out of date, but they just
aren't about building files.

For example, what if you wanted a task `build-some`, which builds only
`report.pdf` and `report.doc`, and outputs a proverb to the command line?

One thing you can do is to simply use a rule with a name that does not
correspond to any file:

``` haskell
-- Bad
"build-some" *> \_ -> do
    need ["out/report.pdf","out/report.doc"]
    cmd "fortune" [""]
```

However, this is kind of an inelegant solution. There really actually is not a
file `build-some`. Also, if someone ever decides to create a file called
`build-some`, you'll find that this rule never gets run.

The best way is to create a "phony" rule, which is a rule that is not tied to a
file. This is the reason for the alias I specified at the beginning of the post:

``` haskell
-- Good
"build-some" ~> do
    need ["out/report.pdf","out/report.doc"]
    cmd "fortune" [""]
```

And voilà!

### Cleanup

You might have noticed the phony rule in the scaffold Shakefile:

``` haskell
"clean" ~> removeFilesAfter ".shake" ["//*"]
```

If you run `shake clean`, it will remove all files in the `.shake/` directory
after the rule has completed its execution. `removeFilesAfter` removes the files
in the given base directory (`.shake`) matching the given wildcards (`["//*"]`)
after all rules have completed their course.

This is useful for cleaning up *shake*'s metadata files after you are done with
your build, or if you want to run the task management system on a clean start.

## Completed File

``` haskell
-- Shakefile
{-# OPTIONS_GHC -fno-warn-wrong-do-bind #-}

import Control.Applicative ((<$>))
import Development.Shake

opts = shakeOptions { shakeFiles    = ".shake/" }

main :: IO ()
main = shakeArgs opts $ do
    want ["out/report.doc","out/report.pdf","out/report.html"]

    "build-some" ~> do
        need ["out/report.pdf","out/report.doc"]
        cmd "fortune" [""]

    "out/report.doc" *> \f -> do
        need <$> srcFiles
        cmd "pandoc" [ "src/report.md", "-o", f ]

    "out/report.pdf" *> \f -> do
        need <$> srcFiles
        cmd "pandoc" [ "src/report.md", "-o", f, "-V", "links-as-notes" ]

    "out/report.html" *> \f -> do
        deps <- srcFiles
        need $ "css/report.css" : deps
        cmd "pandoc" [ "src/report.md", "-o", f, "-c", "css/report.css", "-S" ]

    "img/img2.jpg" *> \f -> do
        cmd "wget" [ "http://example.com/img2.jpg", "-O", f ]

    "clean" ~> removeFilesAfter ".shake" ["//*"]

srcFiles :: Action [FilePath]
srcFiles = getDirectoryFiles ""
    [ "src/report.md"
    , "img/*.jpg" ]
```

## Wrapping Up

If you look at the [Shake
Documentation](http://hackage.haskell.org/packages/archive/shake/0.10.6/doc/html/Development-Shake.html),
you will find a lot of ways you can build complex networks of dependencies.

Hopefully there are enough use cases here to be useful in general applications.

### Monadic Tricks

Because everything is Haskell, you can easily generate rules using your basic
monad iterators by taking advantage of Haskell's extensive standard library of
monad functions. For example, if you want to generate multiple reports, you can
use `forM_`:

``` haskell
let reports = ["report1", "report2", "report3"]

want $ (\s f -> "out/" ++ s ++ "." ++ f) <$>
    ["report1","report2","report3"] <*> ["doc","pdf","html"]

forM_ ["report1","report2","report3"] $ \reportName -> do
    let
        outBase = "out/" ++ reportName
        srcName = "src/" ++ reportName ++ ".md"

    outBase ++ ".doc" *> \f -> do
        need <$> srcFiles
        cmd "pandoc" [ srcName, "-o", f ]

    outBase ++ ".pdf" *> \f -> do
        need <$> srcFiles
        cmd "pandoc" [ srcName, "-o", f, "-V", "links-as-notes" ]

    outBase ++ ".html" *> \f -> do
        deps <- srcFiles
        need $ "css/report.css" : deps
        cmd "pandoc" [ srcName, "-o", f, "-c", "css/report.css", "-S" ]
```

Note however that you can get the same thing by just using wildcards (with
`takeFileName`). But this is just an example, feel free to let your imagination
roam!

### Looking Forward

We've seen how *Shake* is good at setting up systems for managing and executing
dependencies. This is good for running simple system commands. However, there is
a lot more about scripting and task automation than managing dependencies.

For example, almost everything we've done can be done with a simple Makefile.
What does Haskell offer to the scripting scene?

#### Strong Typing

As you'll know, one of the magical things about Haskell is that because of its
expressive strong typing system, you leave the debugging to the compiler. If it
compiles, it works exactly the way you want!

This is pretty lacking in the bare-bones system we have in place now. Right now
we are just firing off arbitrary system commands that are basically specified in
strings with no type of typing. We can compile anything, whether there are bugs
in it or not.

Luckily *Shake* is very good at integrating seamlessly with any kind of
framework. We can leave this up to other frameworks.

One popular framework for this that is gaining in maturity is
[*Shelly*](http://hackage.haskell.org/package/shelly) (A fork of an older
project that is an
[ongoing](http://www.yesodweb.com/blog/2012/03/shelly-for-shell-scripts) Yesod
Project [experiment](http://www.yesodweb.com/blog/2012/07/shelly-update)), but
you are welcome to using your own. At the present Haskell is still developing
and growing in this aspect. I hope to eventually write an article about *Shelly*
integration with *Shake*.

#### Other

These are just some ways to think about using *Shake* in new more creative ways.
Let me know if you think of any clever integrations in the comments!

# Signoff

Hi, thanks for reading! You can reach me via email at <justin@jle.im>, or at
twitter at [\@mstk](https://twitter.com/mstk)! This post and all others are
published under the [CC-BY-NC-ND
3.0](https://creativecommons.org/licenses/by-nc-nd/3.0/) license. Corrections
and edits via pull request are welcome and encouraged at [the source
repository](https://github.com/mstksg/inCode).

If you feel inclined, or this post was particularly helpful for you, why not
consider [supporting me on Patreon](https://www.patreon.com/justinle/overview),
or a [BTC donation](bitcoin:3D7rmAYgbDnp4gp4rf22THsGt74fNucPDU)? :)

\end{document}

\end{document}
